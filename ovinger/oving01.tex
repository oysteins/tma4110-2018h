\documentclass[11pt, a4paper, norsk]{NTNUoving}

\usepackage[utf8]{inputenc}
\usepackage[T1]{fontenc}
\usepackage{amssymb,amsmath}
\usepackage{systeme}
\usepackage[dvipsnames,table]{xcolor}
\usepackage{tikz}
\usepackage{pgfornament}
\usepackage{wrapfig}
\usepackage[all]{xy}
\usepackage{systeme}

\newcommand{\abs}[1]{|#1|}

\newcommand{\V}[1]{\mathbf{#1}}
\newcommand{\vv}[2]{\begin{bmatrix} #1 \\ #2 \end{bmatrix}}
\newcommand{\vvv}[3]{\begin{bmatrix} #1 \\ #2 \\ #3 \end{bmatrix}}
\newcommand{\vvvv}[4]{\begin{bmatrix} #1 \\ #2 \\ #3 \\ #4 \end{bmatrix}}
\newcommand{\vn}[2]{\vvvv{#1_1}{#1_2}{\vdots}{#1_#2}}

\semester{Høst 2018}
\fag{TMA4110 Matematikk 3}
\institutt{Institutt for matematiske fag}
\ovingnr{1}

\begin{document}


\begin{oppgave}
hvilke av likningene er lineære?
\begin{punkt}
...
\end{punkt}
\begin{punkt}
...
\end{punkt}
\end{oppgave}


\begin{oppgave}
hvilke av matrisene er på trappeform?  på redusert trappeform?
\begin{punkt}
...
\end{punkt}
\begin{punkt}
...
\end{punkt}
\end{oppgave}


\begin{oppgave}
løs likningssystemer (3-4 likninger/ukjente)
\begin{punkt}
(én løsning)
\end{punkt}
\begin{punkt}
(uendelig mange løsninger)
\end{punkt}
\begin{punkt}
(ingen løsninger)
\end{punkt}
\end{oppgave}


\begin{oppgave}
Lag et lineært likningssystem med to likninger og to ukjente som
\begin{punkt}
\ldots\ har entydig løsning.
\end{punkt}
\begin{punkt}
\ldots\ ikke har noen løsning.
\end{punkt}
\begin{punkt}
\ldots\ har uendelig mange løsninger.
\end{punkt}
I hver deloppgave: Tegn grafene til de to likningene i systemet ditt.
\end{oppgave}


\begin{oppgave}
to vektorer $\V{u}$ og $\V{v}$

regn ut $a \V{u} + b \V{v}$ for noen ulike $a$ og~$b$,
tegn opp i planet
\end{oppgave}


\begin{oppgave}
er $\V{b}$ en lineærkombinasjon av ...?
\end{oppgave}


\begin{oppgave}
gitt vektorer u, v, w (to eller tre), finn vektor som ikke er lineærkombinasjon av disse
\end{oppgave}


\begin{oppgave}
en lineær likning med to ukjente kan tegnes som en rett linje i $x$-$y$-planet.
hvordan kan vi på tilsvarende måte se for oss en lineær likning med tre ukjente?
\end{oppgave}


\begin{oppgave}
system med $m$ likninger, $n$ ukjente

ni tilfeller:\\
($m < n$, $m = n$, $m > n$) $\times$ (ingen løsninger, én løsning, uendelig mange løsninger)

hvilke tilfeller er mulige?
\end{oppgave}


\begin{oppgave}
Se på likningssystemet
\[
\systeme[xy]{
  ax + by = m,
  cx + dy = n
}
\]
der $a$, $b$, $c$, $d$, $m$ og~$n$ er konstanter, og vi antar at $ad \ne bc$.

Hvor mange løsninger har systemet?  Finn løsningen(e) uttrykt ved $a$,
$b$, $c$, $d$, $m$ og~$n$.
\end{oppgave}


\begin{oppgave}
tre punkter i planet, vil finne andregradspolynom $ax^2 + bx + c$ slik
at grafen går gjennom de tre punktene
\begin{punkt}
Sett opp et lineært likningssystem for $a$, $b$ og~$c$.
\end{punkt}
\begin{punkt}
Løs systemet, og finn andregradspolynomet som går gjennom alle punktene.
\end{punkt}
\end{oppgave}


\begin{oppgave}
to polynomer
\begin{align*}
p(x) &= ax^2 + bx + c \\
q(x) &= dx^2 + ex + f
\end{align*}
(sett inn tall for a,b,c,d,e,f)

tredje polynom $r(x) = ...$.  Kan det skrives som lineærkombinasjon av $p$ og~$q$?

vis at alle andregradspolynomer er lineærkombinasjoner av $p$, $q$ og et tredje polynom
\end{oppgave}



\end{document}
