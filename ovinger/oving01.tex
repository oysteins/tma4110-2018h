\documentclass[11pt, a4paper, norsk]{NTNUoving}

\usepackage[utf8]{inputenc}
\usepackage[T1]{fontenc}
\usepackage{amssymb,amsmath}
\usepackage{systeme}
\usepackage[dvipsnames,table]{xcolor}
\usepackage{tikz}
\usepackage{pgfornament}
\usepackage{wrapfig}
\usepackage[all]{xy}
\usepackage{systeme}

\newcommand{\R}{\mathbb{R}}

\newcommand{\abs}[1]{|#1|}

\newcommand{\V}[1]{\mathbf{#1}}
\newcommand{\vv}[2]{\begin{bmatrix} #1 \\ #2 \end{bmatrix}}
\newcommand{\vvv}[3]{\begin{bmatrix} #1 \\ #2 \\ #3 \end{bmatrix}}
\newcommand{\vvvv}[4]{\begin{bmatrix} #1 \\ #2 \\ #3 \\ #4 \end{bmatrix}}
\newcommand{\vn}[2]{\vvvv{#1_1}{#1_2}{\vdots}{#1_#2}}

\newcommand{\roweq}{\sim}

\newenvironment{amatrix}[1]{% "augmented matrix"
  \left[\begin{array}{*{#1}{c}|c}
}{%
  \end{array}\right]
}

\semester{Høst 2018}
\fag{TMA4110 Matematikk 3}
\institutt{Institutt for matematiske fag}
\ovingnr{1}

\begin{document}


\begin{oppgave}
Hvilke av disse likningene er lineære?
\begin{punkt}
$14x + 3y = 2x + 1 - 5z$
\end{punkt}
\begin{punkt}
$x + 2xy + y = 1$
\end{punkt}
\begin{punkt}
$\frac{x + y}{2} = z$
\end{punkt}
\end{oppgave}


\begin{oppgave}
Hvilke av disse matrisene er på trappeform?  Hvilke av dem er på
redusert trappeform?
\begin{punkt}
$
\begin{bmatrix}
1 & 5 & 0 & 0 \\
0 & 0 & 0 & 1
\end{bmatrix}
$
\end{punkt}
\begin{punkt}
$
\begin{bmatrix}
1 & 0 \\
0 & 1 \\
0 & -1
\end{bmatrix}
$
\end{punkt}
\begin{punkt}
$
\begin{bmatrix}
0 & 2 & 1 \\
0 & 0 & 4 \\
0 & 0 & 0
\end{bmatrix}
$
\end{punkt}
\begin{punkt}
$
\begin{bmatrix}
0 & 0 & 0 \\
0 & 0 & 0 \\
0 & 0 & 0
\end{bmatrix}
$
\end{punkt}
\end{oppgave}


\begin{oppgave}
Løs likningssystemene.
\begin{punkt}
$
\systeme{
  2x - 4y + 9z = -38,
  4x - 3y + 8z = -26,
 -2x + 4y - 2z =  17
}
$
\end{punkt}
\begin{punkt}
(TODO: system med uendelig mange løsninger)
\end{punkt}
\begin{punkt}
(TODO: system med ingen løsninger)
\end{punkt}
\end{oppgave}


\begin{oppgave}
\begin{punkt}
Er disse to likningssystemene ekvivalente?
(TODO: to likningssystemer som er ekvivalente)
\end{punkt}
\begin{punkt}
Er disse to matrisene radekvivalente?
(TODO: to totalmatriser som ikke er radekvivalente)
\end{punkt}
\end{oppgave}


\begin{oppgave}
Lag et lineært likningssystem med to likninger og to ukjente som
\begin{punkt}
\ldots\ har entydig løsning.
\end{punkt}
\begin{punkt}
\ldots\ ikke har noen løsning.
\end{punkt}
\begin{punkt}
\ldots\ har uendelig mange løsninger.
\end{punkt}
I hver deloppgave: Tegn grafene til de to likningene i systemet ditt.
\end{oppgave}


\begin{oppgave}
La $\V{u} = \vv{3}{2}$ og~$\V{v} = \vv{-1}{1}$ være to vektorer
i~$\R^2$.

\begin{punkt}
Regn ut $\V{u} + \V{v}$ og $\frac{1}{2} \V{u} - 2 \V{v}$.
\end{punkt}

\begin{punkt}
Tegn en figur som viser vektorene $\V{u}$, $\V{v}$, $\V{u} + \V{v}$ og
$\frac{1}{2} \V{u} - 2 \V{v}$ i planet.
\end{punkt}
\end{oppgave}


\begin{oppgave}
(TODO: noen gitte vektorer, finn ut om en av dem er lineærkombinasjon av de andre)
\end{oppgave}


\begin{oppgave}
(TODO: gitt vektorer u, v, w (to eller tre), finn vektor som ikke er lineærkombinasjon av disse)
\end{oppgave}


\begin{oppgave}
\begin{punkt}
En lineær likning med to ukjente kan tegnes som en rett linje i $x$--$y$-planet.
Hvordan kan vi på tilsvarende måte se for oss en lineær likning med tre ukjente?
\end{punkt}
\begin{punkt}
Se på følgende likningssystem:
(TODO: system med to likninger og tre ukjente)

Tegn en figur som illustrerer løsningene av hver av disse likningene
og løsningene av systemet.
\end{punkt}
\end{oppgave}


\begin{oppgave}
Anta at vi har et likningssystem med $m$~likninger og~$n$ ukjente.
Hvilke av de ni forskjellige tilfellene i følgende tabell er mulige?
\[
\begin{array}{r|c|c|c|}
                                & m < n & m = n & m > n \\ \hline
\text{ingen løsninger}          &       &       &       \\ \hline
\text{én løsning}               &       &       &       \\ \hline
\text{uendelig mange løsninger} &       &       &       \\ \hline
\end{array}
\]
\end{oppgave}


\begin{oppgave}
Se på likningssystemet
\[
\systeme[xy]{
  ax + by = m,
  cx + dy = n
}
\]
der $a$, $b$, $c$, $d$, $m$ og~$n$ er konstanter, og vi antar at $ad \ne bc$.

Hvor mange løsninger har systemet?  Finn løsningen(e) uttrykt ved $a$,
$b$, $c$, $d$, $m$ og~$n$.
\end{oppgave}


\begin{oppgave}
(TODO: velg tre passende punkter og skriv ferdig oppgaveteksten)

tre punkter i planet, vil finne andregradspolynom $ax^2 + bx + c$ slik
at grafen går gjennom de tre punktene
\begin{punkt}
Sett opp et lineært likningssystem for $a$, $b$ og~$c$.
\end{punkt}
\begin{punkt}
Løs systemet, og finn andregradspolynomet som går gjennom alle punktene.
\end{punkt}
\end{oppgave}


\begin{oppgave}
(TODO: velg passende polynomer (sett inn tall for a,b,c,d,e,f)
og skriv ferdig oppgaveteksten.  Ikke bruk ordet «lineærkombinasjon».)

to polynomer
\begin{align*}
p(x) &= ax^2 + bx + c \\
q(x) &= dx^2 + ex + f
\end{align*}

tredje polynom $r(x) = ...$.  Kan det skrives som lineærkombinasjon av $p$ og~$q$?

vis at alle andregradspolynomer er lineærkombinasjoner av $p$, $q$ og et tredje polynom
\end{oppgave}


\begin{oppgave}
Vis at følgende påstander er sanne for alle matriser $M$, $N$ og~$L$:
\begin{punkt}
$M \roweq M$.
\end{punkt}
\begin{punkt}
Hvis $M \roweq N$, så: $N \roweq M$.
\end{punkt}
\begin{punkt}
Hvis $M \roweq L$ og $L \roweq N$, så: $M \roweq N$.
\end{punkt}
\end{oppgave}


\end{document}
