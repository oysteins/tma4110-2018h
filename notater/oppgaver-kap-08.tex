% -*- TeX-master: "oving06"; -*-
\oppgaver{8}


\linje
TODO --
vil ha oppgaver med:
\begin{itemize}
\item aksiomene -- noe som er og noe som ikke er vektorrom
\item underrom -- både av $\R^n$ og av andre
\item basis for kolonnerom, nullrom, (radrom), egenrom
\item basis for andre rom
\item koordinater med hensyn på en basis
\item basisskifte?  (usikker på om vi vil bruke notasjonen fra boka)
\item dimensjon, kanskje noe om uendeligdimensjonale rom
\item kanskje noe knyttet til beviset for at alle vektorrom har basis?
\end{itemize}
\linje


\begin{oppgave}
\begin{punkt}
Finn en basis for vektorrommet $\M_{m \times n}$.  Hva er dimensjonen?
\end{punkt}
\begin{punkt}
Se på følgende delmengder av~$\M_n$:
\begin{align*}
U \colon &\text{alle diagonalmatriser} \\
V \colon &\text{alle inverterbare matriser} \\
W \colon &\text{alle matriser~$A$ slik at $A = A\tr$}
\end{align*}
Hvilke av disse mengdene er underrom av~$\M_n$?
\end{punkt}
\begin{punkt}
For de mengdene i del~\textbf{b)} som er underrom, hva er dimensjonen?
\end{punkt}
\end{oppgave}

\begin{losning}
TODO
\end{losning}


\begin{oppgave}
Se på funksjonene $\sin$, $\cos$ og~$\tan$ som vektorer i~$\C(\R)$.
\begin{punkt}
Er de lineært uavhengige?
\end{punkt}
\begin{punkt}
Kan du få et annet svar ved å isteden se på dem som vektorer i
$\C(D)$, der $D$ er en delmengde av~$\R$?
\end{punkt}
\end{oppgave}

\begin{losning}
TODO
\end{losning}


\begin{oppgave}
Hvis $V$ er et vektorrom som er en endelig mengde, hva kan du da si om
antall elementer i~$V$?
\\
Hint: Kan $V$ ha null elementer?  Ett element?  To elementer?  Flere
enn to?
\end{oppgave}

\begin{losning}
TODO
\end{losning}


\begin{oppgave}
La
\[
V = \left\{\, \boks{r} \;\middle|\; r \in \R \,\right\}
\]
være mengden av alle bokser som inneholder et reelt tall, slik at for
eksempel
\[
\boks{5}\,,\quad
\boks{\frac{3}{4}}\,,\quad
\boks{\pi}\quad\text{og}\quad
\boks{-328}
\]
er elementer i~$V$.  Definer vektoraddisjon og skalarmultiplikasjon
for~$V$ slik:
\begin{align*}
\boks{r} + \boks{s} &= \boks{rs} \\
c \cdot \boks{r}    &= \boks{r^c}
\end{align*}
Er $V$ et vektorrom?
\end{oppgave}

\begin{losning}
Ja, $V$ er et vektorrom.

TODO: forklar hvorfor hvert aksiom holder.  Få med hva som er additiv
identitet ($\boks{1}$) og invers ($\boks{1/a}$).
\end{losning}


\begin{oppgave}
La $V$ være et vektorrom, og la $U_1$ og~$U_2$ være to underrom
av~$V$.  Hvilke av følgende påstander kan vi da konkludere med?
\begin{punkt}
Snittet $U_1 \intersect U_2$ er et underrom av~$V$.
\end{punkt}
\begin{punkt}
Unionen $U_1 \union U_2$ er et underrom av~$V$.
\end{punkt}
\end{oppgave}

\begin{losning}
Snittet er et underrom; unionen er ikke nødvendigvis et underrom.
TODO: detaljer.
\end{losning}


\begin{oppgave}
\end{oppgave}

\begin{losning}
TODO
\end{losning}


