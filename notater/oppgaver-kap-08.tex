% -*- TeX-master: "oving06"; -*-
\oppgaver{8}


\begin{oppgave}
La $A$ og~$B$ være følgende matriser:
\[
A =
\begin{bmatrix}
\;0 & 1 & 1 & 0 & 0 & 0 & 1\;\\
\;0 & 0 & 0 & 0 & 1 & 0 & 1\;\\
\;0 & 0 & 0 & 0 & 0 & 1 & 1\;\\
\;0 & 0 & 0 & 0 & 0 & 0 & 0\;
\end{bmatrix}
\qquad
B =
\begin{bmatrix}
 1 & 2 & 3\\
 2 & 3 & 4\\
 3 & 4 & 5\\
 4 & 5 & 6
\end{bmatrix}
\]
\begin{punkt}
Finn en basis for kolonnerommet, nullrommet og radrommet til~$A$, og
finn dimensjonen til hvert av disse rommene.
\end{punkt}
\begin{punkt}
Gjør det samme for matrisen~$B$.
\end{punkt}
\begin{punkt}
Ligger vektoren $(0, 1, -2, 3, -1, -1, 1)$ i nullrommet til~$A$?
\end{punkt}
\begin{punkt}
Ligger vektoren $(-1, -1, -1, -1)$ i kolonnerommet til~$A$?  Ligger den
i kolonnerommet til~$B$?
\end{punkt}
% I hver del: Finn en basis for kolonnerommet, nullrommet og radrommet
% til~$A$, finn dimensjonen til hvert av disse rommene, og sjekk om
% vektoren
% $\begin{bmatrix}
% 0\\
% 1\\
% -2\\
% 3\\
% -1\\
% -1\\
% 1
% \end{bmatrix}$ ligger i nullrommet og om $\begin{bmatrix}
% -1\\
% -1\\
% -1\\
% -1
% \end{bmatrix}$ ligger i kolonnerommet -- hvis spørsmålet gir mening -- for matrisen \ldots

% \begin{punkt}
% \ldots$\begin{bmatrix}
% \;0 & 1 & 1 & 0 & 0 & 0 & 1\;\\
% \;0 & 0 & 0 & 0 & 1 & 0 & 1\;\\
% \;0 & 0 & 0 & 0 & 0 & 1 & 1\;\\
% \;0 & 0 & 0 & 0 & 0 & 0 & 0\;
% \end{bmatrix}$ 
% \end{punkt}

% \begin{punkt}
% \ldots$\begin{bmatrix}
% 	1 & 2 & 3\\
% 	2 & 3 & 4\\
% 	3 & 4 & 5\\
% 	4 & 5 & 6
% 	\end{bmatrix}$ 
% \end{punkt}


\end{oppgave}

\begin{losning}

\begin{punkt}
Matrisen har fire frie variabler og tre pivotelement. Dimensjonen på nullrommet er derfor lik fire, mens dimensjonen på kolonnerommet (og derfor også radrommet) er lik tre. For å finne en basis holder det å finne fire lineært uavhengige vektorer i nullrommet og tre lineært uavhengige vektorer i kolonnerommet (og radrommet).


\noindent
Eksempel på valg av basis: Du kan ta kolonnene og radene som svarer til pivotelementene for å finne basis for kolonne- og radrommet; kolonnene som svarer til pivotelement er alltid lineært uavhengige, og vi har tre slike kolonner. Basis for kolonnerommet: kolonne 2, 5 og 6. Basis for radrommet: rad 1, 2 og 3. Matrisen er allerede radredusert, og ligningene for nullrommet er $$x_1+x_2+x_7=0$$ $$x_5+x_7=0$$ $$x_6+x_7=0.$$ Vi kan ta $x_1$, $x_3$, $x_4$ og~$x_7$ som frie variabler. Da blir en parametrisering av nullrommet
$$\begin{bmatrix}
x_1\\
x_2\\
x_3\\
x_4\\
x_5\\
x_6\\
x_7
\end{bmatrix}=\begin{bmatrix}
x_1\\
-x_1-x_7\\
x_3\\
x_4\\
-x_7\\
-x_7\\
x_7
\end{bmatrix}=x_1\begin{bmatrix}
1\\
-1\\
0\\
0\\
0\\
0\\
0
\end{bmatrix}+x_3\begin{bmatrix}
0\\
0\\
1\\
0\\
0\\
0\\
0
\end{bmatrix}+x_4\begin{bmatrix}
0\\
0\\
0\\
1\\
0\\
0\\
0
\end{bmatrix}+x_7\begin{bmatrix}
0\\
0\\
0\\
0\\
-1\\
-1\\
1
\end{bmatrix}.$$ Vi kan f. eks ta $$\begin{bmatrix}
1\\
-1\\
0\\
0\\
0\\
0\\
0
\end{bmatrix},\quad \begin{bmatrix}
0\\
0\\
1\\
0\\
0\\
0\\
0
\end{bmatrix}, \quad \begin{bmatrix}
0\\
0\\
0\\
1\\
0\\
0\\
0
\end{bmatrix}, \quad \begin{bmatrix}
0\\
0\\
0\\
0\\
-1\\
-1\\
1
\end{bmatrix}$$ som basis.
\end{punkt}

\begin{punkt}

Ved radredusering ser vi at det er en fri variabel og to pivotelement. Derfor er dimensjonen på kolonnerommet (radrommet) lik to og dimensjonen på nullromet lik en. Du kan finne en basis for alle underrommene på samme måte som i del $\textbf{a)}$.

\noindent
Husk: Du kan sjekke om svaret ditt er riktig: Du trenger to lineært uavhengige vektorer i kolonnerommet (radrommet) og en ikke-null vektor i nullrommet. Ligger svaret ditt i ønsket underrom? Er vektorene i hvert underrom lineært uavhengige?
\end{punkt}

\begin{punkt}
Vektoren ligger i nullrommet fordi 
$$\begin{bmatrix}
\;0 & 1 & 1 & 0 & 0 & 0 & 1\;\\
\;0 & 0 & 0 & 0 & 1 & 0 & 1\;\\
\;0 & 0 & 0 & 0 & 0 & 1 & 1\;\\
\;0 & 0 & 0 & 0 & 0 & 0 & 0\;
\end{bmatrix}
\begin{bmatrix}
0\\
1\\
-2\\
3\\
-1\\
-1\\
1
\end{bmatrix}=
\begin{bmatrix}
0\\
0\\
0\\
0
\end{bmatrix}.
 $$
\end{punkt}

\begin{punkt}
Vektoren ligger ikke i kolonnerommet til~$A$ fordi systemet med totalmatrise 
$$
\begin{bmatrix}
\;0 & 1 & 1 & 0 & 0 & 0 & 1 & -1 \;\\
\;0 & 0 & 0 & 0 & 1 & 0 & 1 & -1 \;\\
\;0 & 0 & 0 & 0 & 0 & 1 & 1 & -1 \;\\
\;0 & 0 & 0 & 0 & 0 & 0 & 0 & -1 \;
\end{bmatrix}
$$
umulig kan ha løsning; den siste raden svarer til ligningen $0=-1$.

Vektoren ligger i kolonnerommet til~$B$; du kan sjekke at ligningssystemet med totalmatrise 
$$
\begin{bmatrix}
	1 & 2 & 3 & -1\\
	2 & 3 & 4 & -1\\
	3 & 4 & 5 & -1\\
	4 & 5 & 6 & -1
	\end{bmatrix}
$$ har løsning.
\end{punkt}

\end{losning}


\begin{oppgave}

\begin{punkt}
Finn en basis for $\P_2$. Vis at det faktisk er en basis.
\end{punkt}

\begin{punkt}
Hva er koordinatene til $1+2x+3x^2$ i basisen du fant for $\P_2$?
\end{punkt}

\begin{punkt}
Finn en basis for $\P_n$, der $n \ge 0$.
\end{punkt}

\end{oppgave}

\begin{losning}


\begin{punkt}
Vi kan f. eks velge $1$, $x$ og $x^2$. En basis er -- per definisjon -- en samling som spenner ut og er lineært uavhengig.

\noindent
Spenner ut: En vilkårlig vektor $a+bx+cx^2$ er trivielt en lineærkombinasjon $a\cdot 1+b\cdot x+c\cdot x^2$ av $1$, $x$ og $x^2$.

\noindent
Lineær uavhengig: Gitt en ligning $a\cdot 1+b\cdot x+c\cdot x^2=0$, så må vi vise at vi kun har triviell løsning. Men et andregradspolynom kan maksimalt ha to nullpunkt, altså kan ligningen umulig holde hvis ikke $a=0$, $b=0$ og~$c=0$; vi har kun triviell løsning.
\end{punkt}


\begin{punkt}
I koordinatene til basisen vi valgte i $\textbf{a)}$ svarer et polynom $a_0+a_1x+a_2x^2$ til vektoren $\begin{bmatrix}
a_0\\
a_1\\
a_2
\end{bmatrix}.$ Spesielt blir $1+2x+3x^3$ vektoren $\begin{bmatrix}
1\\
2\\
3
\end{bmatrix}.$


\noindent
Merk: Du innførte koordinater for å løse oppgave $\textbf{3.9.}$
\end{punkt}

\end{losning}


\begin{oppgave}
La $\V{a}_1$, $\V{a}_2$, $\u$ og~$\v$ være følgende vektorer i~$\R^3$:
\[
\V{a}_1 = \vvv{1}{2}{3}
\quad
\V{a}_2 = \vvv{2}{3}{4}
\quad
\u = \vvv{1}{-1}{1}
\quad
\v = \vvv{1}{1}{1}
\]
\begin{punkt}
Se på planet i~$\R^3$ som består av alle vektorer på formen
\[
\u + s \cdot \V{a}_1 + t \cdot \V{a}_2
\]
der $s$ og~$t$ er vilkårlige tall.  Er dette planet et underrom
av~$\R^3$?
\end{punkt}
\begin{punkt}
Er planet som består av alle vektorer på formen
\[
\v + s \cdot \V{a}_1 + t \cdot \V{a}_2
\]
et underrom av~$\R^3$?
\end{punkt}
\begin{punkt}
La $A = \begin{bmatrix} \V{a}_1 & \V{a}_2 \end{bmatrix}$
være matrisen som har $\V{a}_1$ og~$\V{a}_2$ som kolonner.
Ligger vektoren~$\u$ i kolonnerommet til denne matrisen?
Hva med~$\v$?
Sammenlign med det du fant ut i del~$\textbf{a)}$ og~$\textbf{b)}$.
\end{punkt}
\end{oppgave}

\begin{losning}

\begin{punkt}
Ikke underrom; du kan f. eks sjekke at planet ikke går gjennom origo.
\end{punkt}

\begin{punkt}
Underrom; du kan f. eks sjekke at $\begin{bmatrix}
1\\
1\\
1
\end{bmatrix}$ er en lineærkombinasjon av de to andre.
\end{punkt}

\begin{punkt}
$\begin{bmatrix}
1\\
-1\\
1
\end{bmatrix}$ ligger ikke i kolonnerommet (vektoren løfter kolonnerommet, som er et plan, opp fra origo); $\begin{bmatrix}
1\\
1\\
1
\end{bmatrix}$ ligger i kolonnerommet (vektoren løfter ikke kolonnerommet, som er et plan, opp fra origo). 

\noindent
Prøv gjerne å tegne situasjonen i denne oppgaven.
\end{punkt}

\end{losning}

\begin{oppgave}
% Vi kan si at et vektorrom er \emph{trivielt} hvis det kun består av
% nullvektoren.
La $A$ være en $m\times n$-matrise hvor $m<n$.
Hvilke av følgende påstander kan vi da konkludere med?
\begin{punkt}
%Kolonnerommet til~$A$ er ikke trivielt.
$\dim \Col A > 0$
\end{punkt}

\begin{punkt}
%Nullrommet til~$A$ er ikke trivielt.
$\dim \Null A > 0$
\end{punkt}

\end{oppgave}

\begin{losning}

\begin{punkt}
Usant.  For eksempel får vi $\dim \Col A = 0$ hvis $A$ er nullmatrisen.
\end{punkt}

\begin{punkt}
Sant. Matrisen $A$ består av $n$ kolonnevektorer i $\R^m$ hvor $n>m$.
Kolonnevektorene må derfor være lineært avhengige.
\end{punkt}

\end{losning}


\begin{oppgave}
La $V$ være et vektorrom, og la $U_1$ og~$U_2$ være to underrom
av~$V$.  Hvilke av følgende påstander kan vi da konkludere med?
\begin{punkt}
Snittet $U_1 \intersect U_2$ er et underrom av~$V$.
\end{punkt}
\begin{punkt}
Unionen $U_1 \union U_2$ er et underrom av~$V$.
\end{punkt}
\end{oppgave}

\begin{losning}
Snittet er et underrom; unionen er ikke nødvendigvis et underrom.


\noindent
Union: Det er mange moteksempler. Du kan f. eks ta to linjer i $\R^2$ som kun krysser hverandre i origo.

\noindent
Snitt: $U_1$ og $U_2$ inneholder null og er lukket under vektorromsoperasjonene. Husk at $U_1 \cap U_2$ betyr $U_1$ og $U_2$. Null-vektoren ligger i $U_1$ og $U_2$ og derfor i $U_1\cap U_2$; summen av to vektorer i $U_1\cap U_2$ ligger i både $U_1$ og $U_2$ igjen ($U_1$ og $U_2$ er underrom); et skalarmultiplum av en vektor i $U_1\cap U_2$ ligger i både $U_1$ og $U_2$ igjen ($U_1$ og $U_2$ er underrom).
\end{losning}


\begin{oppgave}

\begin{punkt}
Finn en basis for vektorrommet $\M_{m \times n}$.  Hva er dimensjonen?
\end{punkt}

\begin{punkt}
Se på følgende delmengder av~$\M_n$:
\begin{align*}
U \colon &\text{alle diagonalmatriser} \\
V \colon &\text{alle inverterbare matriser} \\
W \colon &\text{alle matriser~$A$ slik at $A = A\tr$}
\end{align*}
Hvilke av disse mengdene er underrom av~$\M_n$?
\end{punkt}

\begin{punkt}
For de mengdene i del~\textbf{b)} som er underrom, hva er dimensjonen?
\end{punkt}
\end{oppgave}

\begin{losning}
\begin{punkt}
La $M_{i,j}$ være $m\times n$-matrisen som har 1 i posisjon $(i,j)$ og 0 ellers. Samlingen $M_{i,j}$, $i=1,\dots m$, $j=1,\dots n$ er en basis. Dimensjonen er antall element i en basis: $mn$.
\end{punkt}

\begin{punkt}
$U$ og $W$ er underrom; $V$ er ikke.
\end{punkt}

\begin{punkt}
\noindent
Basis for $U$: Matrisene $M_{i,i}$ for $i=1,\dots,n$.

\noindent
Dimensjonen til $U$: $n$

\noindent
Basis for $W$: Matrisene $M_{i,j}+M_{j,i}$ for $i\neq j$ og $M_{i,i}$ for $i=j$.

\noindent
Hint: Element $(i,j)$ må være likt som element $(j,i)$ for symmetriske matriser, derfor inneholder $M_{i,j}+M_{j,i}$ akkurat den informasjonen du trenger.


\noindent 
Dimensjonen til $W$: $1+2+3+\dots +n=\frac{n(n+1)}{2}$ (vi trenger bare å telle elementene langs og under diagonalen).
\end{punkt}


\end{losning}



\begin{oppgave}

\begin{punkt}
Forklar hvilke av vektorrommene
\begin{align*}
& \P_n \quad \text{(for forskjellige~$n$)} \\
& \P \\
& \C(\R) \\
& \C^n(\R) \quad \text{(for forskjellige~$n$)} \\
& \C^\infty(\R)
\end{align*}
som er underrom av hverandre.
\end{punkt}

\begin{punkt}
Hvilke av vektorrommene i $\textbf{a)}$ er endeligdimensjonale?
Hvilke er uendeligdimensjonale?
\end{punkt}


\end{oppgave}

\begin{losning}

\begin{punkt}
Vektorrommene~$\P_n$, for forskjellige~$n$, er underrom av hverandre:
\[
\P_0 \subset \P_1 \subset \P_2 \subset \cdots
\]
På samme måte er vektorrommene~$\C^n(\R)$, for forskjellige~$n$, også
underrom av hverandre:
\[
\C^1(\R) \subset \C^2(\R) \subset \C^3(\R) \subset \cdots
\]
For hver~$n$ har vi:
\[
\P_n \subset \P \subset \C^\infty(\R) \subset \C^n(\R) \subset \C(\R)
\]
Alle $n$-tegradspolynomer er polynomer;
alle polynomer er uendelig mange ganger deriverbare;
alle uendelig mange ganger deriverbare funksjoner er deriverbare $n$ ganger;
alle $n$ ganger deriverbare funksjoner er kontinuerlige (deriverbar impliserer kontinuitet).
\end{punkt}

\begin{punkt}
Du har vist at $\P_n$ har en endelig basis tidligere, og det er derfor endeligdimensjonalt.
Vi har sett at $\P$ er uendeligdimensjonalt i notatet.
Derfor er resten av vektorrommene uendeligdimensjonale; alle har et uendeligdimensjonalt underrom. 
\end{punkt}


\end{losning}



\begin{oppgave}
La
\[
V = \left\{\, \boks{r} \;\middle|\; \text{$r \in \R$ og $r > 0$} \,\right\}
\]
være mengden der hvert element er en boks som inneholder et positivt reelt tall,
slik at for eksempel
\[
\boks{5}\,,\quad
\boks{\frac{3}{4}}\,,\quad
\boks{\pi}\quad\text{og}\quad
\boks{9328}
\]
er elementer i~$V$.  Definer vektoraddisjon og skalarmultiplikasjon
for~$V$ slik:
\begin{align*}
\boks{r} + \boks{s} &= \boks{rs} \\
c \cdot \boks{r}    &= \boks{r^c}
\end{align*}
Er $V$ et vektorrom?
\end{oppgave}

\begin{losning}
Ja, $V$ er et vektorrom.

Additiv identitet: $\boks{1}$, Additiv invers: $\boks{1/a}$. Du kan nå sjekke aksiomene (V1)--(V8)

Hint: Parantes betyr hva man skal regne ut først.
\end{losning}

\begin{oppgave}
Vi skriver
\[
\Z=\{\dots,-2,-1,0,1,2,\dots\}
\]
for mengden av heltall.
% I denne oppgaven skal vi se på to
% forskjellige måter man kan prøve å gjøre $\Z$ til et vektorrom.  I
% hvert tilfelle skal du finne ut om det faktisk blir et vektorrom eller
% ikke.

\begin{punkt}
Hvis vi tenker på elementene i vektorrommet~$\R^1$ som bare tall,
%(istedenfor kolonnevektorer med ett tall i),
kan vi se på $\Z$ som en delmengde av~$\R^1$.
Er $\Z$ et underrom av $\R^1$?
\end{punkt}

\begin{punkt}
Nå prøver vi å gjøre~$\Z$ til et vektorrom ved å definere
vektoraddisjon og skalarmultiplikasjon direkte.  Definer
vektoraddisjon som vanlig addisjon av tall, og definer
skalarmultiplikasjon av et reelt tall~$r$ og et heltall~$n$ ved
\[
r \ast n = \lfloor rn \rfloor,
\]
der uttrykket $\lfloor rn \rfloor$ betyr tallet~$rn$ rundet ned til et
heltall.  (Vi bruker symbolet~$\ast$ for skalarmultiplikasjonen her
for å ikke blande den sammen med vanlig multiplikasjon av tall.)

% Definer addisjon i $\mathbb{Z}$ på vanlig måte, og definer en
% skalarmultiplikasjon $$r\ast n=\lfloor rn\rfloor$$ hvor
% $\lfloor rn \rfloor$ runner ned til nærmeste heltall.

Er $\Z$ -- med disse operasjonene -- et vektorrom?
\end{punkt}

\end{oppgave}


\begin{losning}

\begin{punkt}
Nei. 

\noindent
$\mathbb{Z}$ er ikke lukket under skalarmultiplikasjon: Hvis du multipliserer et heltall med et reelt tall får du ikke nødvendigvis et heltall. Eksempel: $\frac{1}{2}\cdot 3=\frac{3}{2}$ er ikke et heltall.
\end{punkt}

\begin{punkt}
Nei.

\noindent
Du kan for eksempel sjekke at $(ab)\ast n \neq a\ast (b\ast n)$ for alle valg av relle tall $a$ og $b$, og heltall $n$. 

\noindent
Eksempel: $\lfloor \frac{1}{2}\cdot 1 \rfloor =0$ slik at $2 \ast(\frac{1}{2}\ast 1)=0$. Men $$(2\cdot \frac{1}{2})\ast 1=1 \ast 1=1.$$


\noindent
Merk: Vi viste altså at $\mathbb{Z}$ med denne skalarmultiplikasjonen umulig kan være et vektorrom. Mer generelt kan man vise at det ikke finnes noen vektorromstruktur på $\mathbb{Z}$. Dette henger sammen med at noen typer uendelig er større enn andre.
\end{punkt}

\end{losning}

\begin{oppgave}
Hvis $V$ er et vektorrom som er en endelig mengde, hva kan du da si om
antall elementer i~$V$?
\\
Hint: Kan $V$ ha null elementer?  Ett element?  To elementer?  Flere
enn to?
\end{oppgave}

\begin{losning}
Vektorrommet må bestå av ett element; null-vektoren. Med en gang det finnes en ikke-null vektor $\V{v}$ får vi uendelige mange vektorer; $t\V{v}$ hvor $t$ er et tall.
\end{losning}


\begin{oppgave}
La $\lambda$ være en egenverdi til en $n\times n$-matrise $A$. Vis at egenrommet til $\lambda$ er et underrom av $\R^n$.
\end{oppgave}

\begin{losning}
Det holder å vise at egenrommet til $\lambda$ er lukket under vektorromsoperasjonene og inneholder null-vektoren.

\noindent
Null-vektoren: Dette følger direkte fra definisjonen (egenrommet til $\lambda$ består av egenvektorene og null-vektoren).


\noindent
Addisjon: Hvis $\V{x}$ og $\V{y}$ er to vektorer i egenrommet til $\lambda$ har vi per definisjon at $A\V{x}=\lambda \V{x}$ og $A\V{y}=\lambda \V{y}$ (i spesialtilfellet $\V{0}$ har vi alltid $A\V{0}=\V{0}=\lambda \V{0}$). Vi må vise at $\V{x}+\V{y}$ er en egenvektor til $\lambda$: $$A(\V{x}+\V{y})=A\V{x}+A\V{y}=\lambda \V{x}+ \lambda \V{y}=\lambda(\V{x}+\V{y}),$$ $\V{x}+\V{y}$ er altså i egenrommet.

\noindent
Skalarmultiplikasjon: Hvis $\V{x}$ er i egenrommet til $\lambda$ og $c$ er en konstant har vi at $$A(c\V{x})=c(A \V{x})=c(\lambda \V{x})=\lambda(c \V{x}),$$ som betyr at $c\V{x}$ ligger i egenrommet til $\lambda$.
\end{losning}





\begin{oppgave}
La mengden $D$ være det åpne intervallet mellom $-\pi/2$ og~$\pi/2$:
\[
D = \left( - \frac{\pi}{2}, \frac{\pi}{2} \right)
\]
Se på funksjonene $\sin$, $\cos$ og~$\tan$ som vektorer i~$\C(D)$.
\begin{punkt}
Er de lineært uavhengige?
\end{punkt}
\begin{punkt}
Kan du få et annet svar ved å isteden se på dem som vektorer i
$\C(E)$, der $E$ er en delmengde av~$D$?
\end{punkt}
\end{oppgave}

\begin{losning}
\begin{punkt}
For å svare på spørsmålet må vi utforske om det finnes konstanter $a$, $b$ og $c$ slik at $$a\cos(x)+b\sin(x)+c\tan(x)=0$$ for alle $x$ i $D$. Velg $x=0$ for å se at $a=0$ ettersom $\sin(0)=0$ og $\tan(0)=0$. Vi har $\tan(x)=\frac{\sin(x)}{\cos(x)}$ som gir ligningen $$b\sin(x)=-c\frac{\sin(x)}{\cos(x)}$$ for alle $x$ i $D$ . Sinus er aldri null for $x\neq 0$ i $D$ slik at vi kan stryke $\sin(x)$ -- på denne delen av $D$ -- og få ligningen $$\cos(x)=\frac{-c}{b}.$$ Men Cosinus er helt klart ikke konstant for alle $x\neq 0$ i $D$, så det kan ikke finnes noen ikke-trivielle løsninger. Vektorene er altså lineært uavhengige.
\end{punkt}

\begin{punkt}
Ja. 

\noindent
Eksempel på løsning:La $E$ være ett punkt i $D$ (du kan velge dette punktet vilkårlig). En funksjon fra ett punkt til $\R$ er jo bare et tall i $\R$. Vektorrommet $\C(E)$ er altså bare vektorrommet $\R$. Tre vektorer (tall) i $\R$ er selvfølgelig lineært avhengige (vektorrommet er endimensjonalt).
\end{punkt}

\end{losning}


\begin{oppgave}
La $V$ være et vektorrom.
Vis at følgende påstander følger fra vektorromsaksiomene.
% [Alternativt:]
% Finn ut om følgende påstander må være sanne
% eller ikke, ved å enten vise at de følger fra aksiomene eller at det
% finnes et vektorrom der de ikke stemmer.
\begin{punkt}
Det additive identitetselementet er entydig.  Det finnes altså
nøyaktig én vektor~$\V{0}$ i~$V$ som er slik at
$\V{u} + \V{0} = \V{u}$ for alle vektorer~$\V{u}$.
\end{punkt}
\begin{punkt}
Hvis $\V{u} + \V{v} = \V{u} + \V{w}$ for tre vektorer $\V{u}$, $\V{v}$
og~$\V{w}$ i~$V$, så følger det at $\V{v} = \V{w}$.
\end{punkt}
\begin{punkt}
Additive inverser er entydige.  For hver vektor~$\V{u}$ i~$V$ finnes
det altså kun én vektor~$-\V{u}$ i~$V$ som er slik at
$\V{u} + (-\V{u}) = \V{0}$.
\end{punkt}
\end{oppgave}

\begin{losning}
\begin{punkt}
Hvis $\V{0}_1$ og~$\V{0}_2$ er identitetselementer, så har vi:
\begin{align*}
\V{0}_1
 &= \V{0}_1 + \V{0}_2 &&\text{(V3), $\V{0}_2$ er identitetselement} \\
 &= \V{0}_2 + \V{0}_1 &&\text{(V2)} \\
 &= \V{0}_2           &&\text{(V3), $\V{0}_1$ er identitetselement}
\end{align*}
\end{punkt}
\begin{punkt}
Fra likheten $\V{u} + \V{v} = \V{u} + \V{w}$ får du, ved å bruke
aksiom~(V2) på begge sider:
\[
\V{v} + \V{u} = \V{w} + \V{u}
\]
Vi vet fra aksiom~(V4) at $\V{u}$ har en additiv invers~$-\V{u}$.
Legg til denne på hver side av likheten over; da får du:
\[
(\V{v} + \V{u}) + (-\V{u}) = (\V{w} + \V{u}) + (-\V{u})
\]
% Ved å først bruke aksiom~(V3), deretter (V2), (V4), (V2), (V1), får
% du at $\V{v}$ kan omskrives slik:
% \begin{align*}
% \V{v}
%  &\stackrel{\text{(V3)}}{=} \V{v} + \V{0}
%   \stackrel{\text{(V2)}}{=} \V{0} + \V{v}
%   \stackrel{\text{(V4)}}{=} (\V{u} + (-\V{u})) + \V{v} \\
%  &\stackrel{\text{(V2)}}{=} (-\V{u} + \V{u}) + \V{v}
%   \stackrel{\text{(V1)}}{=} -\V{u} + (\V{u} + \V{v})
% \end{align*}
% På samme måte får du:
% \begin{align*}
% \V{w}
%  &\stackrel{\text{(V3)}}{=} \V{w} + \V{0}
%   \stackrel{\text{(V2)}}{=} \V{0} + \V{w}
%   \stackrel{\text{(V4)}}{=} (\V{u} + (-\V{u})) + \V{w} \\
%  &\stackrel{\text{(V2)}}{=} (-\V{u} + \V{u}) + \V{w}
%   \stackrel{\text{(V1)}}{=} -\V{u} + (\V{u} + \V{w})
% \end{align*}
%---
%v = v + 0 = 0 + v = (u + -u) + v = (-u + u) + v = -u + (u + v)
%w = w + 0 = 0 + w = (u + -u) + w = (-u + u) + w = -u + (u + w)
%----
Bruk aksiom~(V1) på begge sider:
\[
\V{v} + (\V{u} + (-\V{u})) = \V{w} + (\V{u} + (-\V{u}))
\]
Bruk aksiom~(V4):
\[
\V{v} + \V{0} = \V{w} + \V{0}
\]
Bruk aksiom~(V3):
\[
\V{v} = \V{w}
\]
\end{punkt}
\begin{punkt}
Bruk resultatet vist i del~\textbf{b)}.  Hvis to vektorer $\V{v}$
og~$\V{w}$ begge er additive inverser til~$\V{u}$, så har vi
\[
\V{u} + \V{v} = \V{0} = \V{u} + \V{w},
\]
og da gir resultatet fra del~\textbf{b)} at
\[
\V{v} = \V{w}.
\]
\end{punkt}
\end{losning}

\begin{oppgave} \textbf{Utfordring}

\noindent
Vi har kun definert hva en basis er for endeligdimensjonale
vektorrom. For å definere en basis til et
uendeligdimensjonalt vektorrom, må vi først forstå hva i) spenne ut
og ii) lineær uavhengighet skal bety for uendelige mengder.

\smallskip
\noindent
Hvis $S$ er en uendelig mengde av vektorer i et vektorrom~$V$,
så definerer vi følgende:

\noindent
i) Vi sier at $S$ \emph{spenner ut} $V$ dersom enhver vektor $\V{v}$ i
$V$ kan skrives som en lineærkombinasjon av et endelig antall vektorer
i~$S$.

\noindent
ii) Vi sier at $S$ er \emph{lineært uavhengig} dersom enhver
ligning
\[
x_1\V{s}_1+\dots x_n\V{s}_n=\V{0}
\]
(hvor $\V{s}_i$-ene er vektorer i $S$) kun har den trivielle løsningen
$x_i=0$ for alle $i$.

\smallskip
\noindent
Nå kan vi definere en basis:

\noindent
En \emph{basis} $\beta$ til et vektorrom $V$, er en undermengde som spenner ut og er lineært uavhengig.

\begin{punkt}
Vis at vi får tilbake den gamle definisjonen av en basis for endeligdimensjonale vektorrom.
\end{punkt}

\begin{punkt}
Foreslå en basis for $\P$.
\end{punkt}

\begin{punkt}
Vis at forslaget ditt er en basis.
\end{punkt}

\end{oppgave}


\begin{losning}


\begin{punkt}
Hvis $V$ er endeligdimensjonalt ser vi at definisjonen for å spenne ut er den samme. For lineært uavhengig hadde vi at $\V{v}_1,\dots,\V{v}_n$ er lineært uavhengige dersom $$x_1\V{v}_1+\dots+x_n\V{v}_n=\V{0}$$ kun har triviell løsning. I den nye definisjonen har vi at alle ligninger $$x_1\V{v}_{i_1}+\dots+x_k\V{v}_{i_k}=\V{0},$$ hvor $\{\V{v}_{i_1},\dots,\V{v}_{i_k}\}$ er en endelig delmengde av $\{\V{v}_1,\dots,\V{v}_n\}$, kun har triviell løsning. Vi kan spesielt ta $\{\V{v}_1,\dots,\V{v}_n\}$ som en delmengde av seg selv for å få tilbake den gamle definisjonen.
\end{punkt}

\begin{punkt}
En naturlig basis for $\P_n$ er gitt av $1,x\dots,x^n$. Dette er akkurat hva vi trenger for å få alle $n$-gradspolynom. Basert på dette virker det rimelig at $\beta=\{1,x,x^2,x^3,\dots\}$ er en basis for $\P$ (da burde vi akkurat ha det vi trenger for å få alle polynomer av vilkårlig grad). 
\end{punkt}

\begin{punkt}
Spenner ut: Et vilkårlig polynom kan skrives på formen $a_0x^{i_0}+a_1x^{i_1}+\dots +a_kx^{i_k}$ hvor $$i_0<i_1<\dots<i_k$$ er naturlige tall. Men dette polynomet er automatisk en linieærkombinasjon av $x^{i_0},x^{i_1},\dots,x^{i_k}$ som er en delmengde av $\beta$.

\noindent
Lineært uavhengig: Gitt en ligning på formen $$a_0x^{i_0}+a_1x^{i_1}+\dots +a_kx^{i_k}=0,$$ må vi vise at den kun har triviell løsning. Men dette er et $i_k$-gradspolynom, og har derfor maksimalt $i_k$ nullpunkt; polynomet kan umulig være null for alle tall $x$. Dette viser at vi kun har triviell løsning.

\noindent
Dette viser at $\beta$ spenner ut og er lineært uavhengig, som er definisjonen på en basis.
\end{punkt}

\end{losning}






