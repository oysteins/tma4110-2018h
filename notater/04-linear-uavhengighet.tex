\ifx\inkludert\undefined
\documentclass[norsk,a4paper,twocolumn,oneside]{memoir}

\usepackage[utf8]{inputenc}
\usepackage{babel}
\usepackage{amsmath,amssymb,amsthm}
\usepackage[total={17cm,27cm}]{geometry}
\usepackage[table]{xcolor}
%\usepackage{tabularx}
\usepackage{systeme}
%\usepackage{hyperref}
%\usepackage{enumerate}

%\usepackage{sectsty}
\setsecheadstyle{\bfseries\large}
%\subsectionfont{\bf\normalsize}

\usepackage{tikz}
\usetikzlibrary{arrows.meta}

\newcommand{\defterm}[1]{\emph{#1}}

\newcommand{\N}{\mathbb{N}}
\newcommand{\Z}{\mathbb{Z}}
\newcommand{\Q}{\mathbb{Q}}
\newcommand{\R}{\mathbb{R}}

\newcommand{\abs}[1]{|#1|}

\newcommand{\roweq}{\sim}
\DeclareMathOperator{\Span}{Span}

\newcommand{\V}[1]{\mathbf{#1}}
\newcommand{\vv}[2]{\begin{bmatrix} #1 \\ #2 \end{bmatrix}}
\newcommand{\vvv}[3]{\begin{bmatrix} #1 \\ #2 \\ #3 \end{bmatrix}}
\newcommand{\vvvv}[4]{\begin{bmatrix} #1 \\ #2 \\ #3 \\ #4 \end{bmatrix}}
\newcommand{\vn}[2]{\vvvv{#1_1}{#1_2}{\vdots}{#1_#2}}

\newenvironment{amatrix}[1]{% "augmented matrix"
  \left[\begin{array}{*{#1}{c}|c}
}{%
  \end{array}\right]
}

% \newcounter{notatnr}
% \newcommand{\notatnr}[2]
% {\setcounter{notatnr}{#1}%
%  \setcounter{page}{#2}%
% }

\newtheorem{thm}{Teorem}[chapter]
\newtheorem*{thm-nn}{Teorem}
\newtheorem{cor}[thm]{Korollar}
\newtheorem{lem}[thm]{Lemma}
\newtheorem{prop}[thm]{Proposisjon}
\theoremstyle{definition}
\newtheorem{exx}[thm]{Eksempel}
\newtheorem*{defnx}{Definisjon}
\newtheorem*{oppg}{Oppgave}
\newtheorem*{merkx}{Merk}
\newtheorem*{spmx}{Spørsmål}

\newenvironment{defn}
  {\pushQED{\qed}\renewcommand{\qedsymbol}{$\triangle$}\defnx}
  {\popQED\enddefnx}
\newenvironment{ex}
  {\pushQED{\qed}\renewcommand{\qedsymbol}{$\triangle$}\exx}
  {\popQED\endexx}
\newenvironment{merk}
  {\pushQED{\qed}\renewcommand{\qedsymbol}{$\triangle$}\merkx}
  {\popQED\endmerkx}
\newenvironment{spm}
  {\pushQED{\qed}\renewcommand{\qedsymbol}{$\triangle$}\spmx}
  {\popQED\endspmx}

\setlength{\columnsep}{26pt}

\newcommand{\Tittel}[2]{%
\twocolumn[
\begin{center}
\Large
\begin{tabularx}{\textwidth}{cXr}
\cellcolor{black}\color{white}%
\bf {#1} &
#2
\hfill &
\footnotesize TMA4110 høsten 2018
\\ \hline
\end{tabularx}
\end{center}
]}

\newcommand{\tittel}[1]{\Tittel{\arabic{notatnr}}{#1}}

\newcommand{\linje}{%
\begin{center}
\rule{.8\linewidth}{0.4pt}
\end{center}
}


\newcommand{\chapternumber}{}

\makechapterstyle{tma4110}{%
 \renewcommand*{\chapterheadstart}{}
 \renewcommand*{\printchaptername}{}
 \renewcommand*{\chapternamenum}{}
 \renewcommand*{\printchapternum}{\renewcommand{\chapternumber}{\thechapter}}
 \renewcommand*{\afterchapternum}{}
 \renewcommand*{\printchapternonum}{\renewcommand{\chapternumber}{}}
 \renewcommand*{\printchaptertitle}[1]{
\LARGE
\begin{tabularx}{\textwidth}{cXr}
\cellcolor{black}\color{white}%
\textbf{\chapternumber} &
\textbf{##1}
\hfill &
%\footnotesize TMA4110 høsten 2018
\\ \hline
\end{tabularx}%
}
 \renewcommand*{\afterchaptertitle}{\par\nobreak\vskip \afterchapskip}
 % \newcommand{\chapnamefont}{\normalfont\huge\bfseries}
 % \newcommand{\chapnumfont}{\normalfont\huge\bfseries}
 % \newcommand{\chaptitlefont}{\normalfont\Huge\bfseries}
 \setlength{\beforechapskip}{0pt}
 \setlength{\midchapskip}{0pt}
 \setlength{\afterchapskip}{10pt}
}


\newcounter{oppgnr}[chapter]
\newcounter{punktnr}[oppgnr]
\newenvironment{oppgave}
 {\par\noindent\stepcounter{oppgnr}\textbf{{\arabic{oppgnr}}.}}
 {\par\bigskip}
\newenvironment{punkt}
 {\par\smallskip\noindent\stepcounter{punktnr}\textbf{\alph{punktnr})} }
 {\par}

\newcommand{\oppgaver}{\linje\section*{Oppgaver}}

\usepackage{xr}
\externaldocument{tma4110-2018h}
\newcommand{\kapittel}[2]{\setcounter{chapter}{#1}\addtocounter{chapter}{-1}\chapter{#2}}
\newcommand{\kapittelslutt}{\enddocument}
\begin{document}
\chapterstyle{tma4110}
\pagestyle{plain}
\fi


\kapittel{4}{Lineær uavhengighet}
\label{ch:linear-uavhengighet}


\section*{Definisjonen av lineær uavhengighet}

Vi starter med et eksempel:

\begin{ex}
La $\V{u}$, $\V{v}$ og~$\V{w}$ være følgende tre vektorer i~$\R^3$:
\[
\V{u} = \vvv{4}{4}{9}
\qquad
\V{v} = \vvv{1}{1}{0}
\qquad
\V{w} = \vvv{1}{1}{1}
\]
Disse vektorene oppfyller følgende likhet (det er lett å sjekke):
\[
\V{u} = -5 \cdot \V{v} + 9 \cdot \V{w}
\]
En slik lineær likhet som knytter sammen vektorer tenker vi på som en
«avhengighet» mellom vektorene, og vi sier at vektorene $\V{u}$,
$\V{v}$ og~$\V{w}$ er lineært avhengige fordi det finnes en slik
sammenheng mellom dem.

Vi kan også skrive likheten vår på følgende måte ved å sette alle
vektorene på samme side av likhetstegnet:
\[
\V{u} + 4 \cdot \V{v} + 9 \cdot \V{w} = \V{0}
\]
Det vi har gjort nå er å skrive nullvektoren som en lineærkombinasjon
av $\V{u}$, $\V{v}$ og~$\V{w}$.  
\end{ex}

Hvis vi har en liste $\V{v}_1$, $\V{v}_2$, \ldots, $\V{v}_n$ med
vektorer, så er det klart at nullvektoren~$\V{0}$ er en
lineærkombinasjon av disse, fordi vi har likheten
\[
0 \cdot \V{v}_1 + 0 \cdot \V{v}_2 + \cdots + 0 \cdot \V{v}_n = \V{0}
\]
der vi har satt alle vektene til å være~$0$.

Men spørsmålet vi kan stille er: Kan vi også skrive $\V{0}$ som en
lineærkombinasjon av vektorene våre på en annen måte, der ikke alle
vektene er~$0$?  I eksempelet over kunne vi det, men i andre tilfeller
er det ikke mulig.  Dette er det vi vil bruke som definerende egenskap
for å si at noen gitte vektorer enten er lineært avhengige eller
lineært uavhengige.

\begin{defn}
La $\V{v}_1$, $\V{v}_2$, \ldots, $\V{v}_n$ være vektorer i $\R^m$.
Disse vektorene er \defterm{lineært uavhengige} dersom likningen
\[
x_1 \V{v}_1 + x_2 \V{v}_2 + \cdots + x_n \V{v}_n = \V{0}
\]
ikke har andre løsninger enn den trivielle løsningen
$x_1 = x_2 = \cdots = x_n = 0$.

I motsatt tilfelle kalles de \defterm{lineært avhengige}.
\end{defn}

\begin{ex}
% TODO: vektorer som det er lett å se at er lineært uavhengige (uten å gausseliminere)
\end{ex}


\section*{Lineær uavhengighet for to vektorer}

\begin{ex}
Er disse vektorene lineært uavhengige?
\[
\V{v}_1 = \vvv{8}{2}{-12}\quad
\V{v}_2 = \vvv{4}{1}{-6}
\]

\[
1 \cdot \V{v}_1 - 2 \cdot \V{v}_2 = \V{0}
\]
Vektorene $\V{v}_1$ og $\V{v}_2$ er lineært avhengige.
\end{ex}

\begin{thm}
To vektorer $\V{v}_1$ og $\V{v}_2$ er lineært uavhengige hvis og bare
hvis ingen av dem er lik en skalar ganger den andre.
\end{thm}

Med andre ord: De er lineært uavhengige hvis de ikke ligger på en rett
linje gjennom origo.

For to vektorer i $\R^2$ er det to muligheter: Enten er de lineært
avhengige, eller så utspenner de hele~$\R^2$.





\section*{Lineær uavhengighet og matriser}

\begin{ex}
% TODO: omformuler eksempelet, bruk som introduksjon til teoremet
Er disse vektorene lineært uavhengige?
\[
\V{v}_1 = \vvvv{3}{9}{3}{3}\quad
\V{v}_2 = \vvvv{2}{7}{2}{4}\quad
\V{v}_3 = \vvvv{8}{31}{12}{22}
\]
\[
\begin{bmatrix}
3 & 2 & 8 \\
9 & 7 & 31 \\
3 & 2 & 12 \\
3 & 4 & 22
\end{bmatrix}
\sim
\begin{bmatrix}
3 & 2 & 8 \\
0 & 1 & 7 \\
0 & 0 & 4 \\
0 & 0 & 0
\end{bmatrix}
\]
Det er lederelementer i alle kolonner.  Dermed er $\V{v}_1$,
$\V{v}_2$ og $\V{v}_3$ lineært uavhengige.
\end{ex}


\begin{thm}
La $A$ være en matrise.  Følgende påstander er ekvivalente:
\begin{enumerate}
\item Kolonnene i $A$ er lineært uavhengige.
\item Likningen $A \V{x} = \V{0}$ har kun den trivielle løsningen.
\item Vi får ingen frie variabler når vi løser $A \V{x} = \V{0}$.
\item Når vi gausseliminerer $A$, får vi et lederelement i hver kolonne.
\end{enumerate}
\end{thm}

Gitt vektorer $\V{v}_1$, $\V{v}_2$, \ldots, $\V{v}_n$ i $\R^m$.  For å
sjekke om de er lineært uavhengige:
\begin{enumerate}
\item Lag en matrise
$A = \begin{bmatrix} \V{v}_1 & \V{v}_2 & \ldots &
\V{v}_n \end{bmatrix}$ med disse vektorene som kolonner.
\item Gausseliminer $A$ til trappeform.
\item Hvis hver kolonne inneholder et lederelement, er vektorene lineært uavhengige.
Ellers er de lineært avhengige.
\end{enumerate}


\begin{ex}
Er disse vektorene lineært uavhengige?
\[
\V{v}_1 = \vvvv{5}{10}{5}{0}\quad
\V{v}_2 = \vvvv{3}{7}{5}{4}\quad
\V{v}_3 = \vvvv{2}{6}{6}{8}
\]

\[
\begin{bmatrix}
5 & 3 & 2 \\
10 & 7 & 6 \\
5 & 5 & 6 \\
0 & 4 & 8
\end{bmatrix}
\sim
\begin{bmatrix}
5 & 3 & 2 \\
0 & 1 & 2 \\
0 & 0 & 0 \\
0 & 0 & 0
\end{bmatrix}
\]
Siste kolonne har ikke noe lederelement.  Dermed er $\V{v}_1$,
$\V{v}_2$ og $\V{v}_3$ lineært avhengige.
\end{ex}


\begin{ex}
Er disse vektorene lineært uavhengige?
\[
\V{v}_1 = \vvv{8}{7}{4}\quad
\V{v}_2 = \vvv{14}{-2}{5}\quad
\V{v}_3 = \vvv{3}{1}{0}\quad
\V{v}_4 = \vvv{7}{5}{11}\quad
\]

\[
\begin{bmatrix}
8 & 14 & 3 & 7 \\
7 & -2 & 1 & 5 \\
4 & 5 & 0 & 11
\end{bmatrix}
\sim
\begin{bmatrix}
? & ? & ? & ? \\
? & ? & ? & ? \\
? & ? & ? & ?
\end{bmatrix}
\]
Kan ikke få lederelement i alle kolonnene.  Dermed er
$\V{v}_1$, $\V{v}_2$, $\V{v}_3$ og $\V{v}_4$ lineært avhengige.
\end{ex}




\begin{thm}
Gitt $n$ vektorer  $\V{v}_1$, $\V{v}_2$, \ldots, $\V{v}_n$ i $\R^m$.  Hvis
\begin{enumerate}
\item en av vektorene er en lineærkombinasjon av de andre, eller
\item en av vektorene er $\V0$, eller
\item $n > m$,
\end{enumerate}
så er vektorene lineært avhengige.
\end{thm}



\begin{thm}
Gitt $n$ vektorer $\V{v}_1$, $\V{v}_2$, \ldots, $\V{v}_n$ i $\R^n$.

Da er de lineært uavhengige hvis og bare hvis de utspenner hele $\R^n$,
altså hvis og bare hvis
\[
\Sp \{ \V{v}_1, \V{v}_2, \ldots, \V{v}_n \} = \R^n.
\]
\end{thm}




\kapittelslutt
