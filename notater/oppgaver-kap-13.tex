% -*- TeX-master: "oving11"; -*-
\oppgaver{13}


\begin{oppgave}
\begin{punkt}
La $\lambda\neq \mu$ være to reelle tall, begge ulik null. Forklar hvorfor matrisen
$$A=
\begin{bmatrix}
\lambda & \mu & 0\\
\mu & \lambda & 0\\
0 & 0 & \lambda-\mu
\end{bmatrix}$$ har tre lineært uavhengige egenvektorer.
\end{punkt}
\begin{punkt}
Finn egenverdiene til $A$. Hva er den algebraiske- og geometriske multiplisiteten til egenverdiene?
\end{punkt}
\end{oppgave}

\begin{losning}


\begin{punkt}
Matrisen er symmetrisk. Spektralteoremet.
\end{punkt}

\begin{punkt}
$\lambda-\mu$: geometrisk multiplisitet=algebraisk multiplisitet=2 (fordi egenvektorene skal kunne danne en basis).

\noindent
$\lambda-\mu$: geometrisk multiplisitet=algebraisk multiplisitet=1.

\end{punkt}


\end{losning}
