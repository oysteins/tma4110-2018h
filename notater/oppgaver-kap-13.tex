% -*- TeX-master: "oving11"; -*-
\oppgaver{13}


\begin{oppgave}
Anta at $\V{v}_1,\dots,\V{v}_5$ er lineært uavhengige egenvektorer til en matrise $A$ med egenverdier henholdsvis $1,2,3,5,5$.  Hva blir $A$ i basisen gitt av egenvektorene?

\end{oppgave}

\begin{losning}
$A=\begin{bmatrix}
1 & 0 & 0 & 0 & 0\\
0 & 2 & 0 & 0 & 0\\
0 & 0 & 3 & 0 & 0\\
0 & 0 & 0 & 5 & 0\\
0 & 0 & 0 & 0 & 5
\end{bmatrix}$
\end{losning}



\begin{oppgave}
La $A=\begin{bmatrix}
0 & 4\\
0 & 0
\end{bmatrix}$. 

\begin{punkt}
Kan du avgjøre om det finnes en basis av egenvektorer til $A$ uten å regne?
\end{punkt}

\begin{punkt}
Finn egenverdiene og egenvektorene til $A$. Hva er den algebraiske- og geometriske multiplisiteten til egenverdiene? Har vi en basis ev egenvektorer?
\end{punkt}
\end{oppgave}

\begin{losning}
\begin{punkt}
Nei.
\end{punkt}

\begin{punkt}
Egenverier: 0

\noindent
Egenvektorer til  0: linjen utspent av $\vv{1}{0}$. Geometrisk multiplisitet=1, mens algebraisk multiplisitet=2.

\noindent
Dette utgjør ikke en basis for $\mathbb{R}^2$.
\end{punkt}

\end{losning}

\begin{oppgave}
La $A=\begin{bmatrix}
2 & 3 & 5\\
0 & 3 & 5\\
0 & 0 & 5
\end{bmatrix}$. 

\begin{punkt}
Kan du avgjøre om det finnes en basis av egenvektorer til $A$ uten å regne?
\end{punkt}

\begin{punkt}
Finn egenverdiene og egenvektorene til $A$. Hva er den algebraiske- og geometriske multiplisiteten til egenverdiene? Har vi en basis ev egenvektorer?
\end{punkt}
\end{oppgave}


\begin{losning}
\begin{punkt}
Ja: Matrisen er triangulær, så vi kan lese av at $2,3,5$ er egenverdoiene til $A$. Egenvektorer til ulike egenverdier er lineært uavhengige. Derfor kan vi ta en egenvektor fra hvert egenrom for å lage en basis.
\end{punkt}

\begin{punkt}
Egenverier: 2,3,5

\noindent
Egenvektorer til  2: linjen utspent av $\vvv{1}{0}{0}$. Geometrisk multiplisitet=algebraisk multiplisitet=1.

\noindent
Egenvektorer til  3: linjen utspent av $\vvv{3}{1}{0}$. Geometrisk multiplisitet=algebraisk multiplisitet=1.

\noindent
Egenvektorer til  5: linjen utspent av $\vvv{25}{15}{6}$. Geometrisk multiplisitet=algebraisk multiplisitet=1.


\end{punkt}

\end{losning}


\begin{oppgave}

La $\lambda\neq \mu$ være to reelle tall, begge ulik null, og la
$$A=
\begin{bmatrix}
\lambda & \mu & 0\\
\mu & \lambda & 0\\
0 & 0 & \lambda-\mu
\end{bmatrix}.$$
\begin{punkt}
Kan du avgjøre om det finnes en basis av egenvektorer uten å regne?
\end{punkt}
\begin{punkt}
Finn egenverdiene til $A$. Hva er den algebraiske- og geometriske multiplisiteten til egenverdiene? Har vi en basis ev egenvektorer?
\end{punkt}
\end{oppgave}

\begin{losning}


\begin{punkt}
Ja. Matrisen er symmetrisk. Spektralteoremet.
\end{punkt}

\begin{punkt}
$\lambda-\mu$: geometrisk multiplisitet=algebraisk multiplisitet=2 (fordi egenvektorene skal kunne danne en basis).

\noindent
$\lambda-\mu$: geometrisk multiplisitet=algebraisk multiplisitet=1.

\end{punkt}


\end{losning}


\begin{oppgave}
La $A=\begin{bmatrix}
3 & -1 & 2\\
3 & -1 & 6\\
-2 & 2 & -2
\end{bmatrix}$. 

\begin{punkt}
Kan du avgjøre om det finnes en basis av egenvektorer til $A$ uten å regne?
\end{punkt}

\begin{punkt}
Vis at $2$ er en egenverdi til $A$.
\end{punkt}


\begin{punkt}
Finn egenverdiene og egenvektorene til $A$. Hva er den algebraiske- og geometriske multiplisiteten til egenverdiene? Har vi en basis ev egenvektorer?
\end{punkt}

\begin{punkt}
Hvorfor er ikke denne oppgaven et moteksempel til spektralteoremet?
\end{punkt}

\end{oppgave}


\begin{losning}
\begin{punkt}
Nei.
\end{punkt}

\begin{punkt}
Hint: Hva er determinanten til $A-2I$?
\end{punkt}

\begin{punkt}
Egenverier: 2,-4

\noindent
Egenvektorer til  2: planet utspent av $\vvv{1}{1}{0}$ og $\vvv{-2}{0}{1}$. Geometrisk multiplisitet=algebraisk multiplisitet=2.

\noindent
Egenvektorer til  -4: linjen utspent av $\vvv{1}{3}{-2}$. Geometrisk multiplisitet=algebraisk multiplisitet=1.

\noindent
Vi har en basis av egenvektorer; geometrisk multiplisitet samsvarer med algebraisk multiplisitet for hver egenverdi.


\end{punkt}


\begin{punkt}
Spektralteoremet: $A$ er symmetrisk impliserer at det finnes en basis av egenvektorer. Men vi kan fortsatt ha en basis av egenvektorer uten at matrisen vi startet med er symmetrisk.
\end{punkt}

\end{losning}