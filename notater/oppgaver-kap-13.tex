% -*- TeX-master: "oving11"; -*-
\oppgaver{13}




%\begin{oppgave}
%Anta at $\V{v}_1,\dots,\V{v}_5$ er lineært uavhengige egenvektorer til en matrise $A$ med egenverdier henholdsvis $1,2,3,5,5$.  Hva blir $A$ i basisen gitt av egenvektorene?
%
%\end{oppgave}
%
%\begin{losning}
%$A=\begin{bmatrix}
%1 & 0 & 0 & 0 & 0\\
%0 & 2 & 0 & 0 & 0\\
%0 & 0 & 3 & 0 & 0\\
%0 & 0 & 0 & 5 & 0\\
%0 & 0 & 0 & 0 & 5
%\end{bmatrix}$
%\end{losning}



\begin{oppgave}
Finn matrisenes egenverdier og egenvektorer, og avgjør om matrisene er diagonaliserbare. 
\begin{punkt}
$\begin{bmatrix}
0 & 4\\
0 & 0
\end{bmatrix}$
\end{punkt}
%
%\begin{punkt}
%Kan du avgjøre om det finnes en basis av egenvektorer til $A$ uten å regne?
%\end{punkt}
%
%\begin{punkt}
%Finn egenverdiene og egenvektorene til $A$. Hva er den algebraiske- og geometriske multiplisiteten til egenverdiene? Har vi en basis ev egenvektorer?
%\end{punkt}
%
%\begin{punkt}
%Er $A$ diagonaliserbar? I så fall: Finn $P$ og $D$ slik at $A=PDP^{-1}$.
%\end{punkt}
%

%\begin{losning}
%\begin{punkt}
%Nei. Vi har kun en egenverdi, og vi må sjekke den geometriske multiplisiteten eksplisitt.
%\end{punkt}
%
%\begin{punkt}
%Egenverier: 0
%
%\noindent
%Egenvektorer til  0: linjen utspent av $\vv{1}{0}$. Geometrisk multiplisitet=1, mens algebraisk multiplisitet=2.
%
%\noindent
%Dette utgjør ikke en basis for $\mathbb{R}^2$.
%\end{punkt}
%
%Ikke diagonaliserbar.
%
%\end{losning}


%
%\begin{losning}
%\begin{punkt}
%Ja: Matrisen er triangulær, så vi kan lese av at $2,3,5$ er egenverdoiene til $A$. Egenvektorer til ulike egenverdier er lineært uavhengige. Derfor kan vi ta en egenvektor fra hvert egenrom for å lage en basis.
%\end{punkt}
%
%\begin{punkt}
%Egenverier: 2,3,5
%
%\noindent
%Egenvektorer til  2: linjen utspent av $\vvv{1}{0}{0}$. Geometrisk multiplisitet=algebraisk multiplisitet=1.
%
%\noindent
%Egenvektorer til  3: linjen utspent av $\vvv{3}{1}{0}$. Geometrisk multiplisitet=algebraisk multiplisitet=1.
%
%\noindent
%Egenvektorer til  5: linjen utspent av $\vvv{25}{15}{6}$. Geometrisk multiplisitet=algebraisk multiplisitet=1.
%
%
%\end{punkt}
%
%$P=\begin{bmatrix}
%1 & 3 & 25\\
%0 & 1 & 15\\
%0 & 0 & 6\\
%\end{bmatrix}$, $D=\begin{bmatrix}
%2 & 0 & 0\\
%0 & 3 & 0\\
%0 & 0 & 5\\
%\end{bmatrix}.$
%
%\end{losning}


\begin{punkt}
$
\begin{bmatrix}
2 & 2 & 5\\
0 & 2 & 5\\
0 & 0 & 5
\end{bmatrix}
$ 
\end{punkt}
%
%\begin{losning}
%\begin{punkt}
%Nei: Vi har kun to egenverdier $2$ og $5$, og vi vet ikke om den geometriske multiplisiteten til $2$ er lik den algebraiske multiplisiteten til $2$.
%\end{punkt}
%
%\begin{punkt}
%Egenverier: 2,5
%
%\noindent
%Egenvektorer til  2: linjen utspent av $\vvv{1}{0}{0}$. Geometrisk multiplisitet=2, algebraisk multiplisitet=1.
%
%
%\noindent
%Egenvektorer til  5: linjen utspent av $\vvv{25}{15}{9}$. Geometrisk multiplisitet=algebraisk multiplisitet=1.
%
%
%\end{punkt}
%
%Ikke diagonaliserbar.
%
%\end{losning}
%
\begin{punkt}
$\begin{bmatrix}
3 & -1 & 2\\
3 & -1 & 6\\
-2 & 2 & -2
\end{bmatrix}$
\end{punkt}

\begin{punkt}
$\begin{bmatrix}
0 & -1 & 0\\
1 & 0 & 0\\
0 & 0 & 3
\end{bmatrix}$
\end{punkt}



\begin{punkt}
$\begin{bmatrix}
0 & -1 & 1& 5\\
1 & 0 & 2& 6\\
0 & 0 & 3 & 0\\
0 & 0 & 4 & 0
\end{bmatrix}$
\end{punkt}


\begin{punkt}
$\begin{bmatrix}
0 & -1 & 1& 5\\
1 & 0 & 2& 6\\
0 & 0 & 3 & 7\\
0 & 0 & 4 & 0
\end{bmatrix}$
\end{punkt}

\end{oppgave}


\begin{losning}
\begin{punkt}
Matrisen har egenverdien~$0$, med tilhørende egenrom $\Sp \{ \vvS{1}{0} \}$.
Siden den ikke har to lineært uavhengige egenvektorer er den
ikke diagonaliserbar.
\end{punkt}

\begin{punkt}
Matrisen har egenverdiene $2$ og~$5$, med tilhørende egenrom
henholdsvis
\[
\Sp \left\{ \vvv{1}{0}{0} \right\}
\qquad\text{og}\qquad
\Sp \left\{ \vvv{25}{15}{9} \right\}.
\]
Siden den ikke har to lineært uavhengige egenvektorer er den
ikke diagonaliserbar.
\end{punkt}

\begin{punkt}
Det karakteristiske polynomet er
\[
\begin{vmatrix}
3-\lambda & -1         & 2          \\
3         & -1-\lambda & 6          \\
-2        & 2          & -2-\lambda
\end{vmatrix}
=
-\lambda^3 + 12\lambda - 16.
\]
Vi må altså løse likningen
\[
-\lambda^3 + 12\lambda - 16 = 0.
\]
Ved å prøve oss frem finner vi ganske raskt at $\lambda=2$ er en
løsning.  Det vil si at vi kan dele ut faktoren $(\lambda-2)$ fra det
karakteristiske polynomet ved polynomdivisjon.  Da får vi at
\[
-\lambda^3 + 12\lambda - 16 = (\lambda - 2) (-\lambda^2 - 2\lambda + 8).
\]
Vi finner dermed de andre løsningene av likningen ved å løse
\[
-\lambda^2 - 2\lambda + 8 = 0,
\]
som vi gjør ved å bruke den vanlige formelen for andregradslikninger:
\[
\lambda
 = \frac{-(-2) \pm \sqrt{(-2)^2 - 4 \cdot (-1) \cdot 8}}{2 (-1)}
 = -1 \pm 3
\]
Egenverdiene til matrisen er altså $2$ og~$-4$.

Vi finner egenrommene på vanlig måte.  Egenrommet til egenverdien~$2$ er
\[
\Sp \left\{ \vvv{1}{1}{0}, \vvv{-2}{0}{1} \right\},
\]
og egenrommet til egenverdien~$-4$ er
\[
\Sp \left\{ \vvv{1}{3}{-1} \right\}.
\]
Matrisen har altså tre lineært uavhengige egenvektorer, så den er
diagonaliserbar.
\end{punkt}

\begin{punkt}
Egenverdiene er $3$, $-i$ og~$i$, med tilhørende egenrom henholdsvis
\[
\Sp \left\{ \vvv{0}{0}{1} \right\},\qquad
\Sp \left\{ \vvv{-i}{1}{0} \right\},\qquad
\Sp \left\{ \vvv{i}{1}{0} \right\}.
\]
Matrisen har altså tre lineært uavhengige egenvektorer, så den er
diagonaliserbar.
\end{punkt}

\begin{punkt}
Egenverdiene er $0$, $3$, $-i$ og~$i$, med tilhørende egenrom
henholdsvis
\[
\Sp \left\{ \vvvv{-6}{5}{0}{1} \right\},\quad
\Sp \left\{ \vvvv{39}{113}{30}{40} \right\},\quad
\Sp \left\{ \vvvv{-i}{1}{0}{0} \right\},\quad
\Sp \left\{ \vvvv{i}{1}{0}{0} \right\}.
\]
Matrisen har altså fire lineært uavhengige egenvektorer, så den er
diagonaliserbar.
\end{punkt}

\begin{punkt}
Egenverdiene er $-4$, $7$, $-i$ og~$i$, med tilhørende egenrom
henholdsvis
\[
\Sp \left\{ \vvvv{20}{12}{17}{-17} \right\},\quad
\Sp \left\{ \vvvv{151}{293}{350}{200} \right\},\quad
\Sp \left\{ \vvvv{-i}{1}{0}{0} \right\},\quad
\Sp \left\{ \vvvv{i}{1}{0}{0} \right\}.
\]
Matrisen har altså fire lineært uavhengige egenvektorer, så den er
diagonaliserbar.
\end{punkt}
\end{losning}


\begin{oppgave}
La
$A=\begin{bmatrix}
2 & 3 & 5\\
0 & 3 & 5\\
0 & 0 & 5
\end{bmatrix}$. Beregn $A^{10}$. \\[5pt] (Hint: Se på $A=VDV^{-1}$. 
Hva blir $A^n$?)
\end{oppgave}


\begin{losning}
Vi starter med å finne en diagonalisering $A=VDV^{-1}$ av~$A$.  Siden
$A$ er øvre triangulær, er egenverdiene til~$A$ bare tallene på
diagonalen: $2$, $3$ og~$5$.  Vi lager matrisen $V$ ved å sette sammen
egenvektorer som hører til de tre egenverdiene, og vi lager
matrisen~$D$ ved å sette egenverdiene på diagonalen:
\begin{align*}
V &=
\begin{bmatrix}
1 & 3 & 25 \\
0 & 1 & 15 \\
0 & 0 &  6
\end{bmatrix}
\\
D &=
\begin{bmatrix}
2 & 0 & 0 \\
0 & 3 & 0 \\
0 & 0 & 5
\end{bmatrix}
\end{align*}
Da får vi (ved å regne ut inversen på vanlig måte) at
\[
V^{-1} =
\begin{bmatrix}
1 & -3 & 10/3 \\
0 &  1 & -5/2 \\
0 &  0 &  1/6 
\end{bmatrix}
\]
Dermed har vi diagonalisert matrisen~$A$.

Legg merke til at når $A = VDV^{-1}$, så har vi
\begin{align*}
A^2 &= (VDV^{-1})(VDV^{-1}) = VD (V^{-1}V) DV^{-1} \\
    &= V D^2 V^{-1}, \\
A^3 &= (V D^2 V^{-1})(VDV^{-1}) = V D^2 (V^{-1}V) D V^{-1} \\
    &= V D^3 V^{-1},
\end{align*}
og så videre.  Generelt:
\[
A^n = V D^n V^{-1}
\]
Det betyr at i vårt tilfelle kan vi beregne $A^{10}$ slik:
\begin{align*}
A^{10}
&= V D^{10} V^{-1} \\
&=
\begin{bmatrix}
1 & 3 & 25 \\
0 & 1 & 15 \\
0 & 0 &  6
\end{bmatrix}
\begin{bmatrix}
2^{10} & 0      & 0      \\
0      & 3^{10} & 0      \\
0      & 0      & 5^{10}
\end{bmatrix}
\begin{bmatrix}
1 & -3 & 10/3 \\
0 &  1 & -5/2 \\
0 &  0 &  1/6 
\end{bmatrix}
\\
&=
\begin{bmatrix}
1024 & 174075 & 40250650 \\
   0 & 59049  & 24266440 \\
   0 & 0      & 9765625
\end{bmatrix}
\end{align*}
\end{losning}


%\begin{losning}

%
%\begin{punkt}
%Ja. Matrisen er symmetrisk. Spektralteoremet.
%\end{punkt}
%
%\begin{punkt}
%$\lambda-\mu$: geometrisk multiplisitet=algebraisk multiplisitet=2 (fordi egenvektorene skal kunne danne en basis).
%
%\noindent
%$\lambda-\mu$: geometrisk multiplisitet=algebraisk multiplisitet=1.
%
%\end{punkt}
%
%$D=\begin{bmatrix}
%\lambda-\mu & 0 & 0\\
%0 & \lambda-\mu & 0\\
%0 & 0 & \lambda+\mu\\
%\end{bmatrix}.$
%
%
%\end{losning}


%\begin{oppgave}
%
%
%\begin{punkt}
%Vis at $2$ er en egenverdi til $A$.
%\end{punkt}
%
%
%\begin{punkt}
%Finn egenverdiene og egenvektorene til $A$. Hva er den algebraiske- og geometriske multiplisiteten til egenverdiene? Har vi en basis ev egenvektorer?
%\end{punkt}
%
%\begin{punkt}
%Er $A$ diagonaliserbar? I så fall: Finn $P$ og $D$ slik at $A=PDP^{-1}$.
%\end{punkt}
%
%\begin{punkt}
%Hvorfor er ikke denne oppgaven et moteksempel til spektralteoremet?
%\end{punkt}
%
%\end{oppgave}


%\begin{losning}
%\begin{punkt}
%Nei. Det er ikke klart hva egenverdiene til $A$ er.
%\end{punkt}
%
%\begin{punkt}
%Hint: Hva er determinanten til $A-2I$?
%\end{punkt}
%
%\begin{punkt}
%Egenverier: 2,-4
%
%\noindent
%Egenvektorer til  2: planet utspent av $\vvv{1}{1}{0}$ og $\vvv{-2}{0}{1}$. Geometrisk multiplisitet=algebraisk multiplisitet=2.
%
%\noindent
%Egenvektorer til  -4: linjen utspent av $\vvv{1}{3}{-2}$. Geometrisk multiplisitet=algebraisk multiplisitet=1.
%
%\noindent
%Vi har en basis av egenvektorer; geometrisk multiplisitet samsvarer med algebraisk multiplisitet for hver egenverdi.
%
%
%\end{punkt}
%
%\begin{punkt}
%$P=\begin{bmatrix}
%1 & -2 & 1\\
%1 & 0 & 3\\
%0 & 1 & -2\\
%\end{bmatrix}$, $D=\begin{bmatrix}
%2 & 0 & 0\\
%0 & 2 & 0\\
%0 & 0 & -4\\
%\end{bmatrix}.$
%\end{punkt}
%
%\begin{punkt}
%Spektralteoremet: $A$ er symmetrisk impliserer at det finnes en basis av egenvektorer. Men vi kan fortsatt ha en basis av egenvektorer uten at matrisen vi startet med er symmetrisk.
%\end{punkt}
%
%\end{losning}



%\begin{oppgave}
%\begin{punkt}
%Anta at $A$ er en diagonaliserbar. Hva er fordelen med å kunne velge en ortonormal basis av egenvektorer (hvis mulig) mtp. å regne ut $P$, $D$ og~$P^{-1}$ i dekomponeringen $A=PDP^{-1}$.
%\end{punkt}
%
%\begin{punkt}
%La $A$ være som i oppgave $\mathbf{5.}$ med $\lambda=2$ og $\mu=1$. Regn ut $P$, $D$ og $P^{-1}$.
%\end{punkt}
%
%\end{oppgave}
%
%\begin{losning}
%\begin{punkt}
%I forrige øving beviste vi at $U^{-1}=U\tr$ dersom kolonnene til $U$ er ortonormale. Derfor kan vi enkelt regne ut $P^{-1}=P\tr$ dersom vi velger en ortonormal egenvektorbasis. 
%\end{punkt}
%
%\begin{punkt}
%$$D=\begin{bmatrix}
%1 & 0 & 0\\
%0 & 1 & 0\\
%0 & 0 & 3
%\end{bmatrix}$$ $$P=\begin{bmatrix}
%-\frac{1}{\sqrt{2}} & 0 & \frac{1}{\sqrt{2}}\\
%\frac{1}{\sqrt{2}} & 0 & \frac{1}{\sqrt{2}}\\
%0 & 1 & 0
%\end{bmatrix}$$
%
%$$P^{-1}=P\tr=\begin{bmatrix}
%-\frac{1}{\sqrt{2}} & \frac{1}{\sqrt{2}} & 0\\
%0 & 0 & 1\\
%\frac{1}{\sqrt{2}} & \frac{1}{\sqrt{2}} & 0
%\end{bmatrix}$$
%\end{punkt}
%
%\end{losning}

%\begin{oppgave}
%La $A=\begin{bmatrix}
%0 & -1 & 0\\
%1 & 0 & 0\\
%0 & 0 & 3
%\end{bmatrix}$.
%
%\begin{punkt}
%Er $A$ diagonaliserbar som en reell matrise?
%\end{punkt}
%
%\begin{punkt}
%Er $A$ diagonaliserbar som en kompleks matrise?
%\end{punkt}
%
%\begin{punkt}
%Hvis $A$ er diagonaliserbar -- enten som en reell eller kompleks matrise -- finn diagonaliseringen.
%\end{punkt}
%
%\end{oppgave}
%
%\begin{losning}
%
%\begin{punkt}
%Det karakteristiske polynomet er $(\lambda^2+1)(3-\lambda)$. Vi har bare en reell egenverdi $\lambda=3$ med algebraisk multiplisitet=1. Den geometriske multiplisiteten er derfor mindre eller lik 1. Dermed har vi ikke en basis for $\mathbb{R}^3$ av egenvektorer; vi trenger tre lineær uavhengige egenvektorer. 
%\end{punkt}
%
%\begin{punkt}
%Nå har vi tre egenverdier: $3,i,-i$. Derfor har vi tre lineært uavhengige egenvektorer som utgjør en basis for $\mathbb{C}^3$.
%\end{punkt}
%
%\begin{punkt}
%$$P=\begin{bmatrix}
%0 & i & -i\\
%0 & 1 & 1\\
%1 & 0 & 0
%\end{bmatrix}$$
%$$D=\begin{bmatrix}
%3 & 0 & 0\\
%0 & i & 0\\
%0 & 0 & -i
%\end{bmatrix}$$
%\end{punkt}
%
%\end{losning}
%
%\begin{oppgave}
%La $A$ være matrisen fra oppgave $\textbf{2.}$ Får vi flere lineært uavhengige egenvektorer dersom vi betrakter $A$ som en kompleks matrise?
%\end{oppgave}
%
%\begin{losning}
%Nei. Vi har fortsatt en dobbel egenverdi $\lambda=2$ med et degenerert egenrom av dimensjon 1 (geometrisk multiplisitet = 1).
%\end{losning}


\begin{oppgave}
Finn $V$ og $D$ slik at $A=VDV^*$.
\begin{punkt}
$
A=\begin{bmatrix}
2 & 3-i\\
3+i & 4
\end{bmatrix}
$
\end{punkt}
\begin{punkt}
$
A=\begin{bmatrix}
1 & 1-i\\
1+i & -1
\end{bmatrix}
$
\end{punkt}
\begin{punkt}
$
A=
\begin{bmatrix}
1 & 2 & 2\\
2 & 6 & 2\\
2 & 2 & 6
\end{bmatrix}
$ 
\end{punkt}
\end{oppgave}


\begin{losning}
\begin{punkt}
Det karakteristiske polynomet er
\[
\begin{vmatrix}
2 - \lambda & 3 - i       \\
3 + i       & 4 - \lambda
\end{vmatrix}
= (2 - \lambda)(4 - \lambda) - (3 - i)(3 + i)
= \lambda^2 - 6\lambda - 2,
\]
og egenverdiene er $3 + \sqrt{11}$ og $3 - \sqrt{11}$.

Vi finner lett ut at
\[
\v_1 = \vv{3-i}{1+\sqrt{11}}
\]
er en egenvektor for egenverdien $3 + \sqrt{11}$, og at
\[
\v_2 = \vv{3-i}{1-\sqrt{11}}
\]
er en egenvektor for egenverdien $3 - \sqrt{11}$.  For å lage en
ortogonal diagonalisering må vi ha egenvektorer med lengde~$1$, så vi
deler hver av disse på lengden sin og får normaliserte egenvektorer
$\hat\v_1$ og~$\hat\v_2$:
\begin{align*}
\hat\v_1
&= \frac{1}{\| \v_1 \|} \v_1
 = \frac{1}{\sqrt{2 (11+\sqrt{11})}} \v_1 \\
&= \vv{(3-i) \Big/ \sqrt{2 (11+\sqrt{11})}}
      {(1+\sqrt{11}) \Big/ \sqrt{2 (11+\sqrt{11})}}
\\
\hat\v_2
&= \frac{1}{\| \v_2 \|} \v_2
 = \frac{1}{\sqrt{2 (11-\sqrt{11})}} \v_2 \\
&= \vv{(3-i) \Big/ \sqrt{2 (11-\sqrt{11})}}
      {(1-\sqrt{11}) \Big/ \sqrt{2 (11-\sqrt{11})}}
\end{align*}
Da kan vi sette
\[
V = \begin{bmatrix} \hat\v_1 & \hat\v_2 \end{bmatrix}
\quad\text{og}\quad
D =
\begin{bmatrix}
3+\sqrt{11} & 0           \\
0           & 3-\sqrt{11}
\end{bmatrix}
\]
\end{punkt}
\begin{punkt}
Det karakteristiske polynomet er
\[
\begin{vmatrix}
1-\lambda & 1-i\\
1+i & -1-\lambda
\end{vmatrix}
= \lambda^2 - 3.
\]
Egenverdiene er $\sqrt{3}$ og~$-\sqrt{3}$.  Vi finner tilhørende
egenvektorer
\[
\v_1 = \vv{-1+i}{1-\sqrt{3}}
\qquad\text{og}\qquad
\v_2 = \vv{-1+i}{1+\sqrt{3}}.
\]
Vi normaliserer disse:
\begin{align*}
\hat\v_1
&= \vv{(-1+i)/\sqrt{(6-2\sqrt{3})}}
      {(1-\sqrt{3})/\sqrt{(6-2\sqrt{3})}}
\\
\hat\v_1
&= \vv{(-1+i)/\sqrt{(6+2\sqrt{3})}}
      {(1+\sqrt{3})/\sqrt{(6+2\sqrt{3})}}
\end{align*}
Da kan vi sette
\[
V = \begin{bmatrix} \hat\v_1 & \hat\v_2 \end{bmatrix}
\quad\text{og}\quad
D =
\begin{bmatrix}
\sqrt{3} & 0         \\
0        & -\sqrt{3}
\end{bmatrix}
\]
\end{punkt}
\begin{punkt}
Det karakteristiske polynomet er
\[
\begin{vmatrix}
1 & 2 & 2\\
2 & 6 & 2\\
2 & 2 & 6
\end{vmatrix}
= -\lambda^3 + 13\lambda^2 - 36\lambda
\]
Egenverdiene er $0$, $4$ og~$9$.  Vi finner tilhørende egenvektorer
\[
\v_1 = \vvv{-4}{1}{1},\qquad
%\v_2 = \vvv{
\]

TODO
\end{punkt}
\end{losning}


\begin{oppgave}
La $A=\begin{bmatrix}
r_1 & z\\
\overline{z} & r_2
\end{bmatrix}$ være en symmetrisk $2 \times 2$-matrise. Utled en formel for egenverdiene til $A$.
\end{oppgave}


\begin{losning}
Det karakteristiske polynomet til~$A$ er:
\begin{align*}
\begin{vmatrix}
r_1 - \lambda & z             \\
\overline{z}  & r_2 - \lambda
\end{vmatrix}
&= (r_1 - \lambda)(r_2 - \lambda) - z \overline{z} \\
&= \lambda^2 - (r_1 + r_2) \lambda + r_1 r_2 - z \overline{z}
\end{align*}
Egenverdiene blir:
\[
\lambda = \frac{r_1 + r_2 \pm \sqrt{(r_1 + r_2)^2 - 4 (r_1 r_2 - z \overline{z})}}{2}
\]
\end{losning}


\begin{oppgave}
La $ a\neq b$ være to reelle tall, begge ulik null, og la
$$A=
\begin{bmatrix}
a & b & 0\\
b & a & 0\\
0 & 0 & a-b
\end{bmatrix}.$$
Avgjør om $A$ er diagonaliserbar, og finn egenverdier og egenvektorer.
%\begin{punkt}
%Kan du avgjøre om det finnes en basis av egenvektorer uten å regne?
%\end{punkt}
%\begin{punkt}
%Finn egenverdiene til $A$. Hva er den algebraiske- og geometriske multiplisiteten til egenverdiene? Forklar hvorfor vi har en basis av egenvektorer.
%\end{punkt}
%\begin{punkt}
%Finn $D$ slik at du vet at det finnes en $P$ og $A=PDP^{-1}$.
%\end{punkt}
\end{oppgave}


\begin{losning}
TODO
\end{losning}



\begin{oppgave}
En reell matrise er symmetrisk hvis og bare hvis den er ortogonalt diagonaliserbar. 
Dette står i kontrast til komplekse symmetriske matriser, hvor implikasjonen kun går den ene veien. 
Vis at dersom $A$ er reell, og $A=VDV^*$, må $A=A^T$.
\end{oppgave}


\begin{losning}
Anta at $A$ er en reell matrise slik at $A = VDV^*$, altså $A = VDV\tr$.
Da er
\[
A\tr = (VDV\tr)\tr.
\]
Regnereglene for transponering (teorem~\ref{thm:tr}) gir:
\[
(VDV\tr)\tr = (V\tr)\tr D\tr V\tr = V D\tr V\tr
\]
Siden $D$ er en diagonalmatrise, er den sin egen transponert:
\[
D\tr = D
\]
Dette betyr at vi har
\[
A\tr = (VDV\tr)\tr = V D\tr V\tr = VDV\tr = A.
\]

(Hvorfor fungerer ikke dette for komplekse matriser?  Da må vi
erstatte transponering med adjungering, og vi får ikke nødvendigvis at
$D^* = D$.)
\end{losning}


\begin{oppgave}
En kompleks matrise $A$ er \defterm{normal} dersom $A^*A=AA^*$. 
Vis at dersom $A$ er symmetrisk, er $A$ normal. 

\textbf{Fun fact:} En kompleks matrise er ortogonalt diagonaliserbar hvis og bare hvis den er normal.
\end{oppgave}


\begin{losning}
Anta at $A$ er en symmetrisk matrise, altså at $A = A^*$.  Da har vi
\[
A^* A = A A = A A^*,
\]
så $A$ er normal.
\end{losning}


\begin{oppgave}
La 
$
A=
\begin{bmatrix}
5 &-1-2i \\-1-2i & 5
\end{bmatrix}.
$
\begin{punkt}
Er $A$ symmetrisk?
\end{punkt}
\begin{punkt}
Er $A$ ortogonalt diagonaliserbar? 
\end{punkt}
\begin{punkt}
Er $A$ normal?
\end{punkt}
\end{oppgave}


\begin{losning}
La 
$
A=
\begin{bmatrix}
5 &-1-2i \\-1-2i & 5
\end{bmatrix}.
$
\begin{punkt}
Den adjungerte av~$A$ er:
\[
A^* =
\begin{bmatrix}
5     & -1+2i \\
-1+2i & 5
\end{bmatrix}
\]
Siden $A^* \ne A$ er $A$ ikke symmetrisk.
\end{punkt}
\begin{punkt}
Vi kan sjekke direkte om $A$ er ortogonalt diagonaliserbar, men det
kan være enklere å svare på del~\textbf{c)} først.
\end{punkt}
\begin{punkt}
Vi regner ut $A^* A$ og $A A^*$:
\begin{align*}
A^* A &=
\begin{bmatrix}
5     & -1+2i \\
-1+2i & 5
\end{bmatrix}
\begin{bmatrix}
5 &-1-2i \\
-1-2i & 5
\end{bmatrix}
=
\begin{bmatrix}
 30 & -10 \\
-10 & 30
\end{bmatrix}
\\
A A^* &=
\begin{bmatrix}
5 &-1-2i \\
-1-2i & 5
\end{bmatrix}
\begin{bmatrix}
5     & -1+2i \\
-1+2i & 5
\end{bmatrix}
=
\begin{bmatrix}
 30 & -10 \\
-10 & 30
\end{bmatrix}
\end{align*}
Vi har altså at $A^* A = A A^*$, så $A$ er en normal matrise.
\end{punkt}
\textbf{b)} Vi har funnet ut at $A$ er normal, og dermed må den være
ortogonalt diagonaliserbar.
\end{losning}


\begin{oppgave}
La $T:\mathcal{P}_2\rightarrow \mathcal{P}_2$ være lineærtransformasjonen som deriverer annengradspolynomer: $$T(ax^2+bx+c)=2ax+b.$$

\begin{punkt}
Finn matrisen $A$ til $T$ med hensyn på basisen $(1,x,x^2)$.
\end{punkt}

\begin{punkt}
Finn egenverdiene og egenvektorene til $A$. Er $A$ diagonaliserbar?
\end{punkt}

%\begin{punkt}
%Verifiser svaret ditt i del \textbf{b)} ved regning. 
%\end{punkt}
\end{oppgave}

\begin{losning}

\begin{punkt}
$A=\begin{bmatrix}
0 & 1 & 0\\
0 & 0 & 2\\
0 & 0 & 0
\end{bmatrix}$.
\end{punkt}

\begin{punkt}
Ikke diagonaliserbar: Vi ser at $\lambda=0$ er eneste egenverdien (triangulær matrise). Men det er kun konstante polynom som er egenvektorer til 0 (den deriverte av en konstant er lik null). Med andre ord, egenrommet til null er endimensjonalt.
\end{punkt}
\end{losning}



\begin{oppgave}
La $T:\mathcal{P}_2\rightarrow \mathcal{P}_2$ være lineærtransformasjonen mellom andregradspolynom gitt ved: $$T(f)=(x+1)f'(x)+f(x).$$

\begin{punkt}
Finn matrisen $A$ til $T$ med hensyn på basisen $(1,x,x^2)$.
\end{punkt}


\begin{punkt}
Finn egenverdiene og egenvektorene $A$. Er $A$ diagonaliserbar?
\end{punkt}
%\begin{punkt}
%Verifiser svaret ditt i del \textbf{b)} ved regning. 
%\end{punkt}
\end{oppgave}


\begin{losning}
\begin{punkt}
% T(1) = 1
% T(x) = (x+1) + x = 2x + 1
% T(x^2) = (x+1) 2x + x^2 = 3x^2 + 2x
$
A =
\begin{bmatrix}
1 & 1 & 0 \\
0 & 2 & 2 \\
0 & 0 & 3
\end{bmatrix}
$
\end{punkt}

\begin{punkt}
Egenverdiene er $1$, $2$ og~$3$, med tilhørende egenrom henholdsvis
\[
\Sp \left\{ \vvv{1}{0}{0} \right\},\qquad
\Sp \left\{ \vvv{1}{1}{0} \right\},\qquad
\Sp \left\{ \vvv{1}{2}{1} \right\}.
\]
Matrisen $A$ er diagonaliserbar siden den har tre lineært uavhengige
egenvektorer.
\end{punkt}
\end{losning}



\begin{oppgave}
Lineærtransformasjonen $T(\V x)=A\V x$, der $A$ er matrisen
\[
A=
\begin{bmatrix}
\frac{\sqrt{3}}{2} & -\frac{1}{2} \\ \frac{1}{2} & \frac{\sqrt{3}}{2}
\end{bmatrix}
\]
roterer vektorer i $\R^2$. 
\begin{punkt}
Hva er rotasjonsvinkelen?
\end{punkt}
\begin{punkt}
Finn egenverdiene og egenvektorene til matrisen. 
\end{punkt}
\begin{punkt}
Egenvektorene $\V v_1$ og $\V v_2$ danner en basis for $\C^2$. Hva er standardmatrisen til $T$ i denne basisen?
\end{punkt}

%\begin{punkt}
%Verifiser svaret ditt i del \textbf{b)} ved regning. 
%\end{punkt}
\end{oppgave}


\begin{losning}
\begin{punkt}
Rotasjonsvinkelen er $\pi/6$.
\end{punkt}
\begin{punkt}
Det karakteristiske polynomet er:
\[
\begin{vmatrix}
\frac{\sqrt{3}}{2} - \lambda & -\frac{1}{2}                 \\
\frac{1}{2}                  & \frac{\sqrt{3}}{2} - \lambda
\end{vmatrix}
=
(\frac{\sqrt{3}}{2} - \lambda)^2 + \frac{1}{4}
\]
% \sqrt{3}/2 - \lambda = \pm \sqrt{-1/4} = \pm i/2
% \lambda = \sqrt{3}/2 \pm i/2
Egenverdiene er
\[
\frac{\sqrt{3}}{2} + \frac{i}{2}
\qquad\text{og}\qquad
\frac{\sqrt{3}}{2} - \frac{i}{2},
\]
med tilhørende egenrom henholdsvis
% -i/2 -1/2
%  1/2 -i/2
% --
% i  1
% 1 -i
% --
% 1 -i
% 0  0
% -----------------
% i/2 -1/2
% 1/2  i/2
% --
% i -1
% 1  i
% --
% 1 i
% 0 0
\[
\Sp \left\{ \vv{i}{1} \right\}
\qquad\text{og}\qquad
\Sp \left\{ \vv{i}{-1} \right\}.
\]
\end{punkt}
\begin{punkt}
La
\[
\v_1 = \vv{i}{1}
\qquad\text{og}\qquad
\v_2 = \vv{i}{-1}
\]
være de to egenvektorene i fant i del~\textbf{b)}.  Vi skal finne
matrisen til~$T$ med hensyn på basisen $\B = (\v_1, \v_2)$.  Vi vet at
\[
T(\v_1) = (\frac{\sqrt{3}}{2} + \frac{i}{2}) \v_1
\qquad\text{og}\qquad
T(\v_2) = (\frac{\sqrt{3}}{2} - \frac{i}{2}) \v_2.
\]
Dermed får vi følgende matrise:
\[
\begin{bmatrix}
\frac{\sqrt{3}}{2} + \frac{i}{2} & 0 \\
0 & \frac{\sqrt{3}}{2} - \frac{i}{2}
\end{bmatrix}
\]
\end{punkt}
\end{losning}



%\begin{punkt}
%Verifiser svaret ditt i del \textbf{b)} ved regning. 
%\end{punkt}

%\begin{oppgave}
%Finn løsningen av systemet
%\[
%\systeme{
%3x +y = x',
%x +3y = y'
%}
%\]
%som tilfredsstiller $x(0)=y(0)=1$.
%\end{oppgave}


%\begin{oppgave}
%La $A$ være en kvadratisk matrise. Anta at $\V{v}_1$ og~$\V{v}_2$ er egenvektorer av $A$ med ulike tilhørende egenverdier.
%
%\begin{punkt}
%Finn et eksempel som illustrerer at $\V{v}_1$ og~$\V{v}_2$ ikke nødvendigvis er ortogonale.
%\end{punkt}
%
%\begin{punkt}
%Vis at hvis $A$ er symmetrisk, så er $\V{v}_1$ og~$\V{v}_2$ ortogonale.
%
%\noindent
%Hint: Steg 1: Hva vet du? Steg 2: Hva skal du vise? Steg 3: Utforsk uttrykket $(\lambda_1-\lambda_2)(\V{v}_1\boldsymbol{\cdot}\V{v}_2)$ hvor $\lambda$-ene er egenverdiene.
%\end{punkt}
%
%\end{oppgave}
%
%\begin{losning}
%La $A$ være en kvadratisk matrise. Anta at $\V{v}_1$ og~$\V{v}_2$ er egenvektorer av $A$ med ulike tilhørende egenverdier.
%
%\begin{punkt}
%Du har allerede funnet eksempler tidligere i øvingen.
%\end{punkt}
%
%\begin{punkt}
%La $\lambda_1$ være egenverdien til $\V{v}_1$, og la $\lambda_2$ være egenverdien til $\V{v}_2$. Vi har antatt at $\lambda_1\neq\lambda_2$ og $A=A\tr$.
%
%\noindent
%Forslag til løsning: Vi ønsker å vise at $\V{v}_1 
%\boldsymbol{\cdot}\V{v}_2=0$. Vi må bruke at de er 
%egenvektorer på en eller annen måte, så vi utforsker følgende 
%uttrykk (basert på at egenvektorer kun skalerer vektorer):
%$$(\lambda_1\V{v}_1 )\boldsymbol{\cdot}\V{v}_2=(A\V{v}_1 )
%\boldsymbol{\cdot}\V{v}_2=(A\V{v}_1 )\tr\V{v}_2=(\V{v}_1\tr A
%\tr)\V{v}_2.$$ Men nå kan vi bruke antagelsen om at $A$ er 
%symmetrisk til å se at $$(\lambda_1\V{v}_1 )\boldsymbol{\cdot}
%\V{v}_2=\V{v}_1\tr(A\V{v}_2).$$ Lignende regning som ovenfor 
%gir nå at $$(\lambda_1\V{v}_1 )\boldsymbol{\cdot}\V{v}_2=\V{v}
%_1 \boldsymbol{\cdot}(\lambda_2\V{v}_2).$$ Med gode gamle 
%flytte-bytte får vi altså $$(\lambda_1\V{v}_1 )
%\boldsymbol{\cdot}\V{v}_2-\V{v}_1 \boldsymbol{\cdot}
%(\lambda_2\V{v}_2)=0,$$ som -- siden prikkproduktet er lineært 
%-- kan skrives $$(\lambda_1-\lambda_2)(\V{v}_1 
%\boldsymbol{\cdot}\V{v}_2)=0.$$ Men første faktor er ikke lik 
%null fordi $\lambda_1\neq\lambda_2$, slik at vi kan konkludere 
%med at $$\V{v}_1 \boldsymbol{\cdot}\V{v}_2=0.$$ Egenvektorene 
%er ortogonale.
%\end{punkt}
%
%\end{losning}