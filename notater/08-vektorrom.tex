\ifx\inkludert\undefined
\documentclass[norsk,a4paper,twocolumn,oneside]{memoir}

\usepackage[utf8]{inputenc}
\usepackage{babel}
\usepackage{amsmath,amssymb,amsthm}
\usepackage[total={17cm,27cm}]{geometry}
\usepackage[table]{xcolor}
%\usepackage{tabularx}
\usepackage{systeme}
%\usepackage{hyperref}
%\usepackage{enumerate}

%\usepackage{sectsty}
\setsecheadstyle{\bfseries\large}
%\subsectionfont{\bf\normalsize}

\usepackage{tikz}
\usetikzlibrary{arrows.meta}

\newcommand{\defterm}[1]{\emph{#1}}

\newcommand{\N}{\mathbb{N}}
\newcommand{\Z}{\mathbb{Z}}
\newcommand{\Q}{\mathbb{Q}}
\newcommand{\R}{\mathbb{R}}

\newcommand{\abs}[1]{|#1|}

\newcommand{\roweq}{\sim}
\DeclareMathOperator{\Span}{Span}

\newcommand{\V}[1]{\mathbf{#1}}
\newcommand{\vv}[2]{\begin{bmatrix} #1 \\ #2 \end{bmatrix}}
\newcommand{\vvv}[3]{\begin{bmatrix} #1 \\ #2 \\ #3 \end{bmatrix}}
\newcommand{\vvvv}[4]{\begin{bmatrix} #1 \\ #2 \\ #3 \\ #4 \end{bmatrix}}
\newcommand{\vn}[2]{\vvvv{#1_1}{#1_2}{\vdots}{#1_#2}}

\newenvironment{amatrix}[1]{% "augmented matrix"
  \left[\begin{array}{*{#1}{c}|c}
}{%
  \end{array}\right]
}

% \newcounter{notatnr}
% \newcommand{\notatnr}[2]
% {\setcounter{notatnr}{#1}%
%  \setcounter{page}{#2}%
% }

\newtheorem{thm}{Teorem}[chapter]
\newtheorem*{thm-nn}{Teorem}
\newtheorem{cor}[thm]{Korollar}
\newtheorem{lem}[thm]{Lemma}
\newtheorem{prop}[thm]{Proposisjon}
\theoremstyle{definition}
\newtheorem{exx}[thm]{Eksempel}
\newtheorem*{defnx}{Definisjon}
\newtheorem*{oppg}{Oppgave}
\newtheorem*{merkx}{Merk}
\newtheorem*{spmx}{Spørsmål}

\newenvironment{defn}
  {\pushQED{\qed}\renewcommand{\qedsymbol}{$\triangle$}\defnx}
  {\popQED\enddefnx}
\newenvironment{ex}
  {\pushQED{\qed}\renewcommand{\qedsymbol}{$\triangle$}\exx}
  {\popQED\endexx}
\newenvironment{merk}
  {\pushQED{\qed}\renewcommand{\qedsymbol}{$\triangle$}\merkx}
  {\popQED\endmerkx}
\newenvironment{spm}
  {\pushQED{\qed}\renewcommand{\qedsymbol}{$\triangle$}\spmx}
  {\popQED\endspmx}

\setlength{\columnsep}{26pt}

\newcommand{\Tittel}[2]{%
\twocolumn[
\begin{center}
\Large
\begin{tabularx}{\textwidth}{cXr}
\cellcolor{black}\color{white}%
\bf {#1} &
#2
\hfill &
\footnotesize TMA4110 høsten 2018
\\ \hline
\end{tabularx}
\end{center}
]}

\newcommand{\tittel}[1]{\Tittel{\arabic{notatnr}}{#1}}

\newcommand{\linje}{%
\begin{center}
\rule{.8\linewidth}{0.4pt}
\end{center}
}


\newcommand{\chapternumber}{}

\makechapterstyle{tma4110}{%
 \renewcommand*{\chapterheadstart}{}
 \renewcommand*{\printchaptername}{}
 \renewcommand*{\chapternamenum}{}
 \renewcommand*{\printchapternum}{\renewcommand{\chapternumber}{\thechapter}}
 \renewcommand*{\afterchapternum}{}
 \renewcommand*{\printchapternonum}{\renewcommand{\chapternumber}{}}
 \renewcommand*{\printchaptertitle}[1]{
\LARGE
\begin{tabularx}{\textwidth}{cXr}
\cellcolor{black}\color{white}%
\textbf{\chapternumber} &
\textbf{##1}
\hfill &
%\footnotesize TMA4110 høsten 2018
\\ \hline
\end{tabularx}%
}
 \renewcommand*{\afterchaptertitle}{\par\nobreak\vskip \afterchapskip}
 % \newcommand{\chapnamefont}{\normalfont\huge\bfseries}
 % \newcommand{\chapnumfont}{\normalfont\huge\bfseries}
 % \newcommand{\chaptitlefont}{\normalfont\Huge\bfseries}
 \setlength{\beforechapskip}{0pt}
 \setlength{\midchapskip}{0pt}
 \setlength{\afterchapskip}{10pt}
}


\newcounter{oppgnr}[chapter]
\newcounter{punktnr}[oppgnr]
\newenvironment{oppgave}
 {\par\noindent\stepcounter{oppgnr}\textbf{{\arabic{oppgnr}}.}}
 {\par\bigskip}
\newenvironment{punkt}
 {\par\smallskip\noindent\stepcounter{punktnr}\textbf{\alph{punktnr})} }
 {\par}

\newcommand{\oppgaver}{\linje\section*{Oppgaver}}

\usepackage{xr}
\externaldocument{tma4110-2018h}
\newcommand{\kapittel}[2]{\setcounter{chapter}{#1}\addtocounter{chapter}{-1}\chapter{#2}}
\newcommand{\kapittelslutt}{\enddocument}
\begin{document}
\chapterstyle{tma4110}
\pagestyle{plain}
\fi


\kapittel{8}{Vektorrom}
\label{ch:vektorrom}

I de foregående kapitlene har vi tatt en lang vandring gjennom den
lineære algebraens jungel.  Nå skal vi gå opp på en fjelltopp og skue
ut over landskapet vi har vandret gjennom.

Det sentrale temaet i dette kapitlet og det neste er å generalisere og
abstrahere ideene vi har jobbet med til nå.  Vi skal definere begrepet
\emph{vektorrom}, som egentlig bare betyr en mengde med vektorer slik
som $\R^2$ eller~$\R^3$.  Men definisjonen vår av vektorrom kommer til
å være så generell at vi også kan ha vektorrom som ser helt annerledes
ut.  Dette betyr at vi skal bli vant til at vektorer kan være mye mer
enn kolonnevektorene vi kjenner til -- for eksempel kan en funksjon
eller en matrise også være en vektor.

I neste kapittel skal vi innføre det andre sentrale konseptet vi
trenger for å drive med lineær algebra på en generell måte, nemlig
\emph{lineærtransformasjoner}.  En lineærtransformasjon er en funksjon
fra et vektorrom til et annet.  Men den kan ikke være en hvilken som
helst funksjon -- vi krever at den skal \emph{bevare
  vektorromsstrukturen}.

Den generelle formuleringen av lineær algebra ved hjelp av vektorrom
og lineærtransformasjoner gir flere fordeler.  Den gir oss et språk og
en del teknikker som (etter hvert som vi blir vant til dem) gjør mange
problemer mye mer lettfattelige.  Og den gjør at vi kan anvende lineær
algebra innen andre områder av matematikk, for eksempel til å løse
differensiallikninger, som vi skal se på senere.

% Annen tekst som kunne vært brukt i kapittelintroduksjon:
% -------------
% Nå har vi lært ganske mye lineær algebra.  Vi startet med lineære
% likningssystemer, og så hvordan disse kan løses ved gausseliminasjon.
% Derfra har vi gått en lang vei.  Ved å innføre vektorer og matriser
% fikk vi fine måter å beskrive lineære likningssystemer på, noe som
% både ga oss en bedre forståelse av slike systemer og åpnet for at vi
% kunne se på andre relaterte problemer.

\medskip
Vi kan stille oss selv spørsmålet
\begin{quote}
\emph{Hva handler lineær algebra egentlig om?}
\end{quote}
Hva som er et naturlig svar på dette spørsmålet endrer seg etter
hvert som vi lærer mer lineær algebra.  Helt i begynnelsen ville vi
antagelig sagt at lineær algebra handler om å løse lineære
likningssystemer.  Litt senere kunne vi si at lineær algebra handler
om vektorer og matriser.  Etter at vi har kommet oss gjennom dette
kapitlet og det neste, vil vi antagelig besvare spørsmålet med:
\begin{quote}
\emph{Lineær algebra handler om vektorrom og\\lineærtransformasjoner.}
\end{quote}
Og da har vi en ganske god forståelse av lineær algebra.


\section*{Vektorer slik vi kjenner dem}

Vi er vant til vektorer i $\R^2$ og i $\R^3$, og mer generelt
i~$\R^n$.  Vi vet dessuten at hver av disse mengdene utgjør hver sin
separate verden av vektorer.  For eksempel kan vi alltid legge sammen
to vektorer som begge er i~$\R^2$, men det gir ikke mening å legge
sammen en vektor i~$\R^2$ med en vektor i~$\R^3$:
\[
\vv{5}{2} + \vvv{3}{1}{4}
\quad\text{er ikke definert.}
\]
Når vi legger sammen to vektorer i~$\R^2$, blir summen igjen en vektor
i~$\R^2$, og når vi ganger en vektor i~$\R^2$ med en skalar, får vi
også en vektor i~$\R^2$ som resultat.  Vi holder oss altså alltid
innenfor den samme verdenen når vi tar lineærkombinasjoner av
vektorer.  En slik verden av vektorer er det vi kaller et vektorrom.

\smallskip
For $\R^2$ og~$\R^3$ har vi en geometrisk tolkning av
vektorer: Vi kan tenke på en vektor som et punkt i planet/rommet,
eller som pilen som peker fra origo til dette punktet.  Men et punkt
har også koordinater, og det er de vi vanligvis bruker når vi skal
regne på vektorer.

\begin{center}
\begin{tikzpicture}[scale=.9]
\draw[->] (-3.5,0) -- (3.5,0);
\draw[->] (0,-1.2) -- (0,2.4);
\foreach \x in {-3,-2,-1,1,2,3}
\draw (\x cm,1pt) -- (\x cm,-1pt) node[anchor=north] {$\x$};
\foreach \y in {-1,1,2}
\draw (1pt,\y cm) -- (-1pt,\y cm) node[anchor=east] {$\y$};
\draw[-{>[length=7,width=5]},line width=1pt] (0,0) -- (2,1);
\filldraw (2,1) circle [radius=2pt];
\node[anchor=west] at (2,1) {$(2,1) = \vv{2}{1}$};
\end{tikzpicture}
\\
{\small \textit{Forskjellige aspekter av samme vektor:\\punkt -- pil -- koordinater}}
\end{center}

Antagelig er du innen nå blitt såpass komfortabel med dette litt
fleksible vektorbegrepet at du uten problemer går frem og tilbake
mellom å tenke på en vektor som en pil, et punkt, eller en
kolonnevektor.

Nå skal vi bli enda mer fleksible i vår forståelse av hva en vektor
kan være.  Faktisk vil vi si at det er fullstendig likegyldig hva en
vektor \emph{er}, det som betyr noe er hvordan den \emph{oppfører
  seg}.

% TODO: ta med dette?
% -------
% For $\R^n$ der $n > 3$ mangler vi den geometriske forståelsen, så her
% må vi nøye oss med å se på vektorer som koordinater, skrevet enten som
% en liste med tall eller en kolonnevektor:
% \[
% (v_1, v_2, \ldots, v_n) = \vn{v}{n}
% \]


% TODO om \R^n, bruk begrepet vektorrom om \R^n
% tegninger \R^0, \R^1, \R^2, ...


\section*{Hva kan vi gjøre med vektorer?}

Vi har to operasjoner for vektorer.  Vi kan legge sammen vektorer, og
vi kan gange en vektor med en skalar:
\begin{align*}
\text{addisjon av vektorer: }&\ \V{u} + \V{v} \\
\text{skalarmultiplikasjon: }&\ c \cdot \V{u}
\end{align*}
Begge operasjonene gir ut nye vektorer som resultat.

Når vi nå skal generalisere vektorbegrepet, vil vi si at alle slags
ting kan være vektorer, så lenge de kan adderes og
skalarmultipliseres, og oppfyller visse kriterier som gjør at vi kan
regne med dem på samme måte som vi er vant til å regne med vektorer.
Disse kriteriene kaller vi \emph{aksiomene} for et vektorrom.


\section*{Aksiomer for vektorrom}

\begin{figure*}[p]
% TODO: litt tilfeldig hvor figuren havner, men ser greit ut akkurat nå
\begin{center}
\begin{minipage}{.9\textwidth}
\begin{center}
\pgfornament[width=1.5cm]{39}
\hfill
\pgfornament[width=1.5cm]{40}

\vspace{10pt}

{\HUGE \textit{Vektorromsaksiomene}}
\large

\vspace{25pt}

\hbox{}\hspace{-20pt}
\begin{tabular}{rll}
\multirow{5}{*}{
\begin{tabular}{c}
aksiomer\\for\\addisjon
\end{tabular}
$\left\{\rule{0pt}{90pt}\right.$
}
\\[-4pt]
& {\large {\LARGE(}V1{\LARGE)}} &
$(\V{u} + \V{v}) + \V{w} = \V{u} + (\V{v} + \V{w})$
for alle vektorer $\V{u}$, $\V{v}$ og~$\V{w}$.
\\[4pt]
&&{\small (Vektoraddisjon er en \emph{assosiativ} operasjon.)}
\\[10pt]
& {\large {\LARGE(}V2{\LARGE)}} &
$\V{u} + \V{v} = \V{v} + \V{u}$
for alle vektorer $\V{u}$ og~$\V{v}$.
\\[4pt]
&&{\small (Vektoraddisjon er en \emph{kommutativ} operasjon.)}
\\[10pt]
& {\large {\LARGE(}V3{\LARGE)}} &
Det finnes en vektor $\V{0}$ slik at
$\V{u} + \V{0} = \V{u}$
for alle vektorer $\V{u}$.
\\[4pt]
&&{\small (Vektoraddisjon har et \emph{identitetselement}.)}
\\[10pt]
& {\large {\LARGE(}V4{\LARGE)}} &
For hver vektor~$\V{u}$ finnes en vektor $-\V{u}$ slik at
$\V{u} + (-\V{u}) = \V{0}$.
\\[4pt]
&&{\small (Vektoraddisjon har \emph{inverser} for alle elementer.)}
\\[10pt]
%%%%%%%%%%%%%%%%
\multirow{5}{*}{
\begin{tabular}{c}
aksiomer\\for\\skalar-\\multi-\\plikasjon
\end{tabular}
$\left\{\rule{0pt}{90pt}\right.$
}
\\[-4pt]
& {\large {\LARGE(}V5{\LARGE)}} &
$(ab) \cdot \V{u} = a \cdot (b \cdot \V{u})$
for alle vektorer $\V{u}$ og alle skalarer $a$ og~$b$.
\\[4pt]
&&{\small (Skalarmultiplikasjon er \emph{kompatibel} med multiplikasjon av skalarer.)}
\\[10pt]
& {\large {\LARGE(}V6{\LARGE)}} &
$1 \cdot \V{u} = \V{u}$
for alle vektorer $\V{u}$.
\\[4pt]
&&{\small (Tallet~$1$ er \emph{identitetselement} for skalarmultiplikasjon.)}
\\[10pt]
& {\large {\LARGE(}V7{\LARGE)}} &
$a \cdot (\V{u} + \V{v}) = a \V{u} + a \V{v}$
for alle vektorer $\V{u}$ og~$\V{v}$, og alle skalarer~$a$.
\\[4pt]
&&{\small (Skalarmultiplikasjon er \emph{distributiv} over addisjon av vektorer.)}
\\[10pt]
& {\large {\LARGE(}V8{\LARGE)}} &
$(a + b) \cdot \V{u} = a \V{u} + b \V{u}$
for alle vektorer $\V{u}$, og alle skalarer~$a$ og~$b$.
\\[4pt]
&&{\small (Skalarmultiplikasjon er \emph{distributiv} over addisjon av skalarer.)}
\end{tabular}
\hspace{10pt}\hbox{}

\vspace{30pt}
\pgfornament[width=1.5cm,symmetry=h]{39}
\hfill
\pgfornament[width=1.5cm,symmetry=h]{40}
\end{center}
\end{minipage}
\end{center}
\label{fig:aksiomene}
\end{figure*}

Aksiomene for vektorrom er åtte regneregler som holder for vektorene
i~$\R^n$, og som er essensielle for at vektorer skal oppføre seg slik
vi synes at vektorer skal oppføre seg.  Vi nummerer aksiomene (V1),
(V2), \ldots, (V8).  Du finner alle sammen i en fin liste på
side~\pageref{fig:aksiomene}.

Det første aksiomet, (V1), sier at det ikke har noe å si hvor vi
setter parentesene når vi legger sammen vektorer:
\[
(\V{u} + \V{v}) + \V{w} = \V{u} + (\V{v} + \V{w})
\]
Dette betyr at vi kan skrive en sum
\[
\V{u} + \V{v} + \V{w}
\]
uten parenteser, fordi vi får samme resultat om vi legger sammen
$\V{u}$ og~$\V{v}$ først, eller legger sammen $\V{v}$ og~$\V{w}$
først.  Mer generelt betyr det at vi kan skrive alle slags lengre
summer
\[
\u_1 + \u_2 + \cdots \u_n
\]
uten parenteser.  På fint matematikkspråk kalles dette at addisjonen
er en \emph{assosiativ} operasjon.

Dette ser kanskje så åpenbart ut at det ikke skulle være nødvendig å
gjøre noe stort nummer ut av det.  Men det er likevel nødvendig å ha
det med som et krav, fordi vi i utgangspunktet sier at vi kan definere
vektoraddisjonen til å gjøre akkurat hva vi vil, og det er fullt mulig
å definere en operasjon som ikke er slik at parenteser kan flyttes
fritt.

For å gjøre det helt klart at assosiativitet ikke er noe vi kan ta for
gitt, kan det være nyttig å tenke på at du kjenner godt til minst én
operasjon som ikke er assosiativ, nemlig opphøyd-i-operasjonen.  Med
den operasjonen har det en betydning hvor vi setter parentesene, for
\[
(a^b)^c
\quad\text{og}\quad
a^{(b^c)}
\]
er ikke det samme.

Aksiom~(V2) sier at rekkefølgen ikke spiller noen rolle når vi adderer
vektorer:
\[
\u + \v = \v + \u
\]
Det fine navnet på denne egenskapen er at addisjonen er
\emph{kommutativ}.

Aksiom~(V3) sier at det skal finnes en \emph{nullvektor}.  I~$\R^n$ er
vi vant til å si at nullvektoren er vektoren som består av bare
nuller.  Men nå vil vi definere nullvektoren ut fra hvordan den
oppfører seg med hensyn på addisjonsoperasjonen.  Den definerende
egenskapen for en nullvektor~$\0$ er at det å legge til nullvektoren
ikke endrer noe, altså at
\[
\u + \0 = \u
\]
for alle vektorer~$\u$.

Aksiom~(V4) sier at hver vektor har en \emph{additiv invers}.  Det
betyr at for hver vektor~$\u$ skal det være mulig å finne en
vektor~$\v$ slik at
\[
\u + \v = \0.
\]
Vektoren~$\v$ er da den additive inversen til~$\u$, og vi kaller den
for~$-\u$.

De fire første aksiomene handler bare om addisjonsoperasjonen.  De
fire siste handler om skalarmultiplikasjon.

Aksiom~(V5) sier at skalarmultiplikasjonen er \emph{kompatibel} med
det å gange sammen tall, i den forstand at å gange en vektor med ett
og ett tall er det samme som å gange sammen tallene først og så
multiplisere med vektoren:
\[
(ab) \cdot \u = a \cdot (b \cdot \u)
\]

Aksiom~(V6) sier at det å gange en vektor med tallet~$1$ alltid gir
oss den samme vektoren tilbake:
\[
1 \cdot \u = \u
\]
Vi kan altså si at tallet~$1$ er et \emph{identitetselement} for
skalarmultiplikasjonen.

Aksiomene (V7) og~(V8) sier at vi kan gange ut parenteser slik vi er
vant med, både når vi har en sum av vektorer og når vi har en sum av
tall:
\begin{align*}
a \cdot (\u + \v) &= a \u + a \v \\
(a + b) \cdot \u  &= a \u + b \u
\end{align*}
Dette kalles at skalarmultiplikasjonen er \emph{distributiv} over
addisjon.

% \FloatBarrier

\section*{Definisjonen av vektorrom}

De åtte aksiomene (V1)--(V8) er det vi vil kreve for at noe skal
kvalifisere til å være et vektorrom.  Nå er vi derfor klare for å
skrive ned definisjonen av vektorrom.

\begin{defn}
La $V$ være en mengde, og anta at vi har definert to operasjoner:
\begin{align*}
\text{addisjon av vektorer: } & \V{u} + \V{v} \\
\text{skalarmultiplikasjon: } & c \cdot \V{u}
\end{align*}
Addisjonen skal være definert for alle elementer $\V{u}$ og~$\V{v}$
i~$V$, og skalarmultiplikasjonen for alle skalarer~$c$ og alle $\V{u}$
i~$V$.  Resultatet av operasjonene skal alltid være et element i~$V$.

Dersom mengden~$V$ og de to operasjonene oppfyller vektorromsaksiomene
(V1)--(V8), så sier vi at $V$ er et \defterm{vektorrom}, og vi kaller
elementene i~$V$ for \defterm{vektorer}.
\end{defn}

Aksiomene for vektorrom er valgt ut som de egenskapene ved $\R^n$ som
anses som å være essensielle.  Derfor er selvsagt hver $\R^n$ et
vektorrom.  Men poenget med å gi en så generell definisjon er at det
også finnes flere vektorrom enn disse.

Som et første eksempel kan vi se at en delmengde av~$\R^n$ også kan
være et vektorrom.

\begin{ex}
\label{ex:underrom1}
Se på mengden
\[
U = \Sp \left\{ \vv{1}{1} \right\}
\]
utspent av vektoren $\vvS{1}{1}$ i~$\R^2$, altså denne linjen:
\begin{center}
\begin{tikzpicture}[scale=.42]
\draw[->] (-5,0) -- (5,0);
\draw[->] (0,-3) -- (0,3);
\draw (-3,-3) -- (3,3) node[anchor=north west] {$U$};
\end{tikzpicture}
\end{center}
Hvis vi lar addisjonen og skalarmultiplikasjonen være definert som
i~$\R^2$, så kan vi sjekke at mengden~$U$ i seg selv også blir et
vektorrom.

Resultatet av å addere eller skalarmultiplisere vektorer i~$U$ blir
alltid en vektor i~$U$, slik at det gir mening å definere operasjonene
på denne måten.

Nullvektoren~$\vvS{0}{0}$ i~$\R^2$ er med i~$U$ og fungerer også som
nullvektor for~$U$, slik at aksiom~(V3) er oppfylt.  For alle
vektorer~$\u$ i~$U$ er også den additive inversen~$-\u$ med i~$U$,
slik at aksiom~(V4) er oppfylt.  Vi ser lett at alle de andre
aksiomene også holder for~$U$, siden de holder for~$\R^2$.

Vektorrommet~$U$ er geometrisk sett en linje, akkurat som~$\R^1$.  Det
fungerer altså litt som å plassere en kopi av~$\R^1$ på skrå inne
i~$\R^2$.
\end{ex}


\section*{Eksempler på vektorrom}

La oss nå se på noen vektorrom som er virkelig forskjellige fra de
gamle og kjente rommene $\R^n$.  Vektorrommene vi definerer her vil vi
bruke senere, så det er lurt å prøve å bli kjent med dem.  For hvert
av disse vektorrommene er det lett (men litt tid- og plasskrevende) å
sjekke at alle vektorromsaksiomene holder.  Du bør prøve å sjekke det
selv, for i hvert fall ett av vektorrommene vi ser på.

\medskip\noindent\textbf{Polynomer av begrenset grad. }%
Vi skriver $\P_n$ for mengden av alle polynomer av grad~$n$ eller
lavere, altså alle funksjoner på formen
\begin{align*}
p(x)
&= a_n x^n + a_{n-1} x^{n-1} + \cdots + a_1 x + a_0 \\
&= \sum_{i_0}^n a_i x^i.
\end{align*}
Vi definerer addisjon og skalarmultiplikasjon av polynomer på den
åpenbare måten.  Hvis
\[
p(x) = \sum_{i_0}^n a_i x^i
\quad\text{og}\quad
q(x) = \sum_{i_0}^n b_i x^i
\]
er to polynomer i $\P_n$, så er summen $p + q$ polynomet der vi
summerer koeffisientene fra $p$ og~$q$:
\[
(p + q)(x) = \sum_{i_0}^n (a_i + b_i) \cdot x^i
\]
Skalarmultiplikasjonen definerer vi ved at $c \cdot p$ er polynomet
der vi ganger alle koeffisientene i $p$ med~$c$:
\[
(cp)(x) = \sum_{i_0}^n (c \cdot a_i) \cdot x^i
\]
Med disse operasjonene er $\P_n$ et vektorrom.

\medskip\noindent\textbf{Alle polynomer. }%
Vi skriver $\P$ for mengden av alle polynomer av vilkårlig grad.  Med
addisjon og skalarmultiplikasjon definert som i $\P_n$ blir $\P$ også
et vektorrom.

\begin{ex}
Vi ser på tre funksjoner $f$, $g$ og~$h$, gitt ved
\begin{align*}
f(x) &= x^2, \\
g(x) &= 3x^2 + x, \\
h(x) &= x^2.
\end{align*}
Disse er polynomer av grad~$2$, så de er vektorer i
vektorrommet~$\P_2$.  Dette betyr at vi kan regne med dem som vektorer
-- vi kan for eksempel ta lineærkombinasjoner av dem.
Lineærkombinasjonen $5f + g$ blir funksjonen gitt ved
\[
(5f + g)(x) = 5 \cdot f(x) + g(x) = 8x^2 + x.
\]
Siden $f$, $g$ og~$h$ er vektorer, kan vi også spørre om de er lineært
uavhengige.  Ved å prøve oss frem litt med lineærkombinasjoner av de
tre vektorene, finner vi ganske raskt ut at
\[
-4f + g + h = 0,
\]
som betyr at $f$, $g$ og~$h$ er lineært avhengige.
\end{ex}

\medskip\noindent\textbf{Kontinuerlige funksjoner. }%
Vi skriver $\C(D)$ for mengden av alle kontinuerlige funksjoner
definert på et område~$D$, der $D \subseteq \R$.  Mengden $\C(D)$
består altså av alle funksjoner som er på formen
\[
f \colon D \to \R
\]
og er kontinuerlige.

Vi kan definere addisjon og skalarmultiplikasjon av funksjoner på en
naturlig måte.  Summen $f + g$ av to funksjoner $f$ og~$g$ blir en
funksjon gitt ved
\[
(f + g)(x) = f(x) + g(x),
\]
og produktet $cf$ av en skalar~$c$ og en funksjon~$f$ blir en funksjon
gitt ved
\[
(cf)(x) = c \cdot (f(x)).
\]
Med disse operasjonene blir mengden $\C(D)$ et vektorrom.

\begin{ex}
\label{ex:sin-cos-lin-uavh}
La $f$ og~$g$ være funksjonene gitt ved
\[
f(x) = \sin x
\qquad\text{og}\qquad
g(x) = \cos x.
\]
Da kan vi se på $f$ og~$g$ som vektorer i vektorrommet~$\C(\R)$ av
kontinuerlige funksjoner fra $\R$ til~$\R$.  Er disse to vektorene
lineært uavhengige?

For å finne ut det, kan vi huske at vi vet at to vektorer er lineært
uavhengige hvis og bare hvis den ene kan skrives som en skalar ganger
den andre.  Men siden
\[
f(0) = \sin 0 = 0
\qquad\text{og}\qquad
g(0) = \cos 0 = 1,
\]
kan vi ikke ha at $g$ er en skalar ganger $f$, for da måtte $0$ ganger
denne skalaren vært~$1$.  Tilsvarende ser vi at siden
\[
f\left(\frac{\pi}{2}\right) = \sin \frac{\pi}{2} = 1
\qquad\text{og}\qquad
g\left(\frac{\pi}{2}\right) = \cos \frac{\pi}{2} = 0,
\]
kan vi ikke ha at $f$ er en skalar ganger $g$.  Altså er $f$ og~$g$
lineært uavhengige vektorer i~$\C(\R)$.
\end{ex}

\medskip\noindent\textbf{Deriverbare og glatte funksjoner. }%
Vi skriver $\C^1(D)$ for mengden av alle funksjoner
\[
f \colon D \to \R
\]
som er deriverbare, og mer generelt $\C^n(D)$ for mengden av alle
funksjoner som er deriverbare $n$~ganger.  Funksjoner som er
deriverbare uendelig mange ganger kalles \emph{glatte} funksjoner, og
vi skriver $\C^\infty(D)$ for mengden av alle glatte funksjoner fra
$D$ til~$\R$.

Med addisjon og skalarmultiplikasjon definert på samme måte som
i~$\C(D)$ blir mengdene $\C^n(D)$ og $\C^\infty(D)$ også vektorrom.

\medskip\noindent\textbf{Uendelige lister. }%
Vi skriver $\R^\N$ for mengden av alle uendelige lister
\[
(a_1, a_2, a_3, \ldots) = (a_i)_{i \in \N}
\]
av reelle tall.  Vi definerer addisjon og skalarmultiplikasjon for
slike lister ved:
\begin{align*}
(a_i)_{i \in \N} + (b_i)_{i \in \N} &= (a_i + b_i)_{i \in \N} \\
c \cdot (a_i)_{i \in \N} &= (c \cdot a_i)_{i \in \N}
\end{align*}
Da er $\R^\N$ et vektorrom.

\begin{ex}
Se på de tre vektorene
\begin{align*}
\u &= (1, 2, 3, 4, 5, \ldots) \\
\v &= (1, 1, 1, 1, 1, \ldots) \\
\w &= (1, 3, 5, 7, 9, \ldots)
\end{align*}
i vektorrommet~$\R^\N$.  Er vektoren~$\w$ en lineærkombinasjon av $\u$
og~$\v$?

Hvis vi ganger~$\u$ med to, så får vi:
\[
2 \u = (2, 4, 6, 8, 10, \ldots)
\]
Nå kan vi trekke fra~$\v$ og ende opp med:
\[
2 \u - \v = (1, 3, 5, 7, 9, \ldots)
\]
Vi ser altså at $\w = 2 \u - \v$, så vi kan konkludere med at
vektoren~$\w$ er en lineærkombinasjon av $\u$ og~$\v$.
\end{ex}

\medskip\noindent\textbf{Matriser. }%
Vi skriver $\M_{m \times n}$ for mengden av alle
$m \times n$-matriser.  Vi har allerede (i kapittel~\ref{ch:matriser})
definert hvordan vi legger sammen matriser, og hvordan vi ganger en
skalar med en matrise.  Med disse operasjonene er $\M_{m \times n}$ et
vektorrom.  Siden kvadratiske matriser er spesielt interessante,
definerer vi en egen notasjon, $\M_n$, for vektorrommet som består av
alle $n \times n$-matriser.


\section*{Underrom}

I eksempel~\ref{ex:underrom1} så vi at en linje i~$\R^2$ kunne være et
vektorrom i seg selv.  Et slikt vektorrom som ligger inni et annet
vektorrom kalles et underrom.

\begin{defn}
Et \defterm{underrom} av et vektorrom~$V$ er en delmengde
$U \subseteq V$ som i seg selv utgjør et vektorrom, med addisjon og
skalarmultiplikasjon definert på samme måte som i~$V$.
\end{defn}

\begin{ex}
\label{ex:underrom2}
Som i eksempel~\ref{ex:underrom1} lar vi $U$ være delmengden
\[
U = \Sp \left\{ \vv{1}{1} \right\}
\]
av~$\R^2$.  Da er $U$ et underrom av~$\R^2$.
\end{ex}

% eks \P_0 \subseteq \P_1 \subseteq \P_2 ...
%     \P_n \subseteq \P \subseteq \C(\R)
%     \C(D) \subseteq \R^D
% (gis som oppgave isteden)

\begin{ex}
\label{ex:ikke-underrom}
Se på mengden
\[
U = \left\{ \vv{a}{1} \;\middle|\ a \in \R \right\}
\]
av alle vektorer i~$\R^2$ på formen $\vvS{a}{1}$, altså den
horisontale linjen vist her:
\begin{center}
\begin{tikzpicture}[scale=.42]
\draw[->] (-5.5,0) -- (5.8,0);
\draw[->] (0,-1.3) -- (0,2.8);
\draw (-5.5,1) -- (5.5,1) node[anchor=south west] {$U$};
\end{tikzpicture}
\end{center}
Er $U$ et underrom av~$\R^2$?

Hvis $U$ skal være et underrom, må vi ha at $U$ i seg selv blir et
vektorrom når vi bruker vektoraddisjonen og skalarmultiplikasjonen
fra~$\R^2$.  Men det gir ikke fungerende operasjoner på~$U$.  For
eksempel har vi at $\vvS{0}{1}$ og~$\vvS{1}{1}$ er vektorer i~$U$, men
summen
\[
\vv{0}{1} + \vv{1}{1} = \vv{1}{2}
\]
er ikke i~$U$.  Operasjonene fra $\R^2$ fungerer altså ikke til å
gjøre $U$ til et vektorrom, så $U$ er ikke et underrom av~$\R^2$.
\end{ex}

I eksempel~\ref{ex:ikke-underrom} så vi at mengden~$U$ ikke ble et
underrom fordi vi ikke holder oss innenfor~$U$ når vi legger sammen
vektorer fra~$U$.  Vi sier da at mengden~$U$ ikke er \emph{lukket}
under addisjon.

Men hvis vi har en delmengde~$U$ av et vektorrom~$V$ som \emph{er}
lukket under både addisjon og skalarmultiplikasjon, og som inneholder
nullvektoren, så blir alle vektorromsaksiomene automatisk oppfylt
for~$U$ fordi de holder i~$V$.  Vi har dermed følgende teorem.

\begin{thm}
\label{thm:underrom}
La $V$ være et vektorrom.  En delmgende $U \subseteq V$ er et underrom
av~$V$ hvis og bare hvis følgende tre betingelser er oppfylt.
\begin{enumerate}
\item Nullvektoren $\V{0}$ i~$V$ ligger i~$U$.
\item For alle vektorer $\V{u}$ og~$\V{v}$ i~$U$ er også summen
$\V{u} + \V{v}$ i~$U$.
\item For alle vektorer~$\V{u}$ i~$U$ og alle skalarer~$c$ er også
produktet~$c\V{u}$ i~$U$.
\end{enumerate}
\end{thm}

Det er lett å se at en mengde utspent av en liste med vektorer
oppfyller disse tre betingelsene.  Vi skriver opp det også som et
teorem.

\begin{thm}
\label{thm:underrom-sp}
En mengde $\Sp \{ \V{v}_1, \V{v}_2, \ldots, \V{v}_t \}$ utspent av vektorer
i et vektorrom~$V$ er alltid et underrom av~$V$.
\end{thm}
\begin{proof}
Mengden $\Sp \{ \V{v}_1, \V{v}_2, \ldots, \V{v}_t \}$ består av alle
lineærkombinasjoner av vektorene $\V{v}_1$, $\V{v}_2$, \ldots,
$\V{v}_t$.  Uansett hva disse vektorene er, vet vi at nullvektoren er
en lineærkombinasjon av dem.  Videre ser vi enkelt at summen av to
lineærkombinasjoner blir en ny lineærkombinasjon av de samme
vektorene, og at en skalar ganger en lineærkombinasjon igjen blir en
lineærkombinasjon.  Dermed er alle betingelsene fra
teorem~\ref{thm:underrom} oppfylt.
\end{proof}

% TODO tekst?  eks?


\section*{Endeligdimensjonale vektorrom}

For vektorrommene $\R^n$ har vi et klart begrep om dimensjon.  Vi sier
at $\R^2$ er todimensjonalt, at $\R^3$ er tredimensjonalt, og mer
generelt at $\R^n$ er $n$-dimensjonalt.

For et vilkårlig vektorrom~$V$ er det ikke like klart hva dimensjonen
skal være.  Vi skal etter hvert komme frem til at vi har en
meningsfylt måte å snakke om dimensjon på også her, men først skal vi
foreta en veldig grov inndeling av alle vektorrommene.  Vi skiller
mellom vektorrom som er endeligdimensjonale (og for disse skal vi om
en stund se at vi kan definere en dimensjon) og de som ikke er det
(for disse må vi bare nøye oss med å si at dimensjonen er uendelig).

\begin{defn}
Et vektorrom~$V$ er \defterm{endeligdimensjonalt} hvis det finnes en
endelig mengde som utspenner~$V$.  Ellers er~$V$
\defterm{uendeligdimensjonalt}.
\end{defn}

Alle de «gode gamle» vektorrommene $\R^n$ som vi kjenner fra før er
endeligdimensjonale, siden $\R^n$ er utspent av den endelige mengden
\[
\left\{
\vvvv{1}{0}{\vdots}{0},
\vvvv{0}{1}{\vdots}{0},
\ldots,
\vvvv{0}{0}{\vdots}{1}
\right\}
\]
bestående av de~$n$ enhetsvektorene.

Men siden vi tar oss bryet med å definere begrepene
endeligdimensjonalt og uendeligdimensjonalt, bør det også finnes
eksempler på vektorrom som er uendeligdimensjonale.  Flere av
vektorrommene vi har sett tidligere i dette kapitlet er
uendeligdimensjonale.  Vi sjekker at ett av dem er
uendeligdimensjonalt nå; de andre skal du få finne ut av selv i
oppgavene.

\begin{ex}
Vektorrommet $\P$ av alle polynomer er uendeligdimensjonalt.  Hvordan
kan vi se det?  La
\[
\{ p_1, p_2, \ldots, p_t \}
\]
være en endelig mengde av polynomer i~$\P$.  Hvert av polynomene
$p_1$, $p_2$, \ldots, $p_t$ har en grad.  La $n$ være den høyeste av
disse gradene.  Da kan vi ikke skrive et polynom av grad $n+1$ som en
lineærkombinasjon av polynomene $p_1$, $p_2$, \ldots, $p_t$, så disse
utspenner ikke hele~$\P$.

Siden vi kan si dette om enhver endelig mengde, kan det ikke finnes
noen endelig mengde som utspenner~$\P$.  Dermed er $\P$ et
uendeligdimensjonalt vektorrom.
\end{ex}


\section*{Basis}

Vi sa at et vektorrom er endeligdimensjonalt hvis det er utspent av en
endelig mengde.  Nøkkelen til å definere dimensjonen til et vektorrom
er å lete etter en «best mulig» utspennende mengde, der «best» betyr
at den ikke inneholder noen overflødige vektorer.  En slik mengde
kalles en basis for vektorrommet.

Basiser er essensielle for at vi skal kunne definere dimensjonen til
et vektorrom, men er også nyttige til mye annet.  Et vektorrom som
ikke er~$\R^n$ kan være vanskelig å jobbe med, men om vi har en basis,
blir det mye mer håndterlig.

\begin{defn}
En \defterm{basis} for et vektorrom~$V$ er en liste
\[
\B = (\b_1, \b_2, \ldots, \b_n)
\]
av vektorer som både utspenner~$V$ og er lineært uavhengige.
\end{defn}

\begin{merk}
Det er to forskjellige måter å definere en basis på.  Enten sier man
at en basis er en \emph{mengde} med vektorer, eller så sier man (som
vi gjør) at en basis er en \emph{liste} med vektorer.  Forskjellen er
at i en mengde har ikke elementene noen bestemt rekkefølge, mens i en
liste er det ett bestemt element som er det første, ett som er det
andre, og så videre.

Det viktigste med en basis er å ha vektorer som utspenner det aktuelle
vektorrommet og er lineært uavhengige, og det har man uansett om man
velger å plassere dem i en mengde eller en liste.  Forskjellen dukker
opp når man vil bruke basisen til å innføre koordinater (som vi skal
gjøre ganske snart).  Da må elementene i basisen ha en rekkefølge.
Hvis vi hadde valgt å definere en basis som en mengde, ville vi fått
litt ekstra jobb for å få definert koordinater på en skikkelig måte.
\end{merk}

\begin{ex}
La oss se på vektorrommet~$\R^3$.
Vi ser lett at listen
\[
\left(
\vvv{1}{0}{0},\ %
\vvv{0}{1}{0},\ %
\vvv{0}{0}{1}
\right)
\]
bestående av de tre enhetsvektorene er en basis.  Men vi kan også
finne andre basiser.  Enhver liste med tre lineært uavhengige vektorer
blir en basis for~$\R^3$ (du husker fra teorem~\ref{thm:linuavhspan}
at tre vektorer i~$\R^3$ er lineært uavhengige hvis og bare hvis de
utspenner~$\R^3$).  Så for eksempel er listen
\[
\left(
\vvv{1}{2}{3},\ %
\vvv{5}{0}{0},\ %
\vvv{0}{0}{8}
\right)
\]
også en basis for~$\R^3$.
\end{ex}

Vi ser at vi alltid kan bruke enhetsvektorene til å lage en basis
for~$\R^n$.  Denne basisen,
\[
\left(
\vvvv{1}{0}{\vdots}{0},
\vvvv{0}{1}{\vdots}{0},
\ldots,
\vvvv{0}{0}{\vdots}{1}
\right),
\]
kalles \defterm{standardbasisen} for~$\R^n$.

Det som gjør $\R^n$ lettere å jobbe med enn andre vektorrom er at vi
har \emph{koordinater}.  Enhver vektor i~$\R^n$ er en kolonnevektor
\[
\vn{v}{n}
\]
der $v_1$ er vektorens første koordinat, $v_2$ er dens andre
koordinat, og så videre.  En av de viktigste egenskapene til en basis
er at den lar oss innføre koordinater.

\begin{thm}
\label{thm:koordinater}
La $V$ være et vektorrom med basis
\[
\B = (\b_1, \b_2, \ldots, \b_n).
\]
Da kan hver vektor~$\v$ i~$V$ skrives som en lineærkombinasjon
\[
\v = c_1 \b_1 + c_2 \b_2 + \cdots + c_n \b_n
\]
av basisvektorene i~$\B$, på en entydig måte.
\end{thm}
\begin{proof}
Det at hver vektor~$\v$ kan skrives som en lineærkombinasjon av
basisvektorene følger av at basisvektorene utspenner~$V$.  Det at
denne lineærkombinasjonen er entydig følger av at basisvektorene er
lineært uavhengige.
% TODO referanse til teorem som skulle vært med i kapitlet om lineær
% uavhengighet
\end{proof}

\begin{defn}
Tallene $c_1$, $c_2$, \ldots, $c_n$ i teorem~\ref{thm:koordinater}
kalles \defterm{koordinatene} til vektoren~$\v$ \defterm{med hensyn
  på} basisen~$\B$.  Vi definerer notasjonen $\koord{\v}{\B}$ for
vektoren i~$\R^n$ som består av koordinatene til~$\v$:
\[
\koord{\v}{\B} = \vn{c}{n}\qedhere
\]
\end{defn}

\begin{ex}
I eksempel~\ref{ex:sin-cos-lin-uavh} så vi at funksjonene $\sin$
og~$\cos$ er lineært uavhengige vektorer i vektorrommet~$\C(\R)$ av
kontinuerlige funksjoner fra~$\R$ til~$\R$.  Det betyr at hvis vi ser
på underrommet
\[
U = \Sp \{ \sin, \cos \}
\]
av $\C(\R)$ utspent av disse to funksjonene, så er
\[
\B = (\sin, \cos)
\]
en basis for~$U$.  La oss nå se på en vektor i~$U$, for eksempel
funksjonen~$f$ gitt ved
\[
f(x) = 4 \sin x - 7 \cos x.
\]
Koordinatene til~$f$ med hensyn på basisen~$\B$ er $4$ og~$-7$, så
koordinatvektoren til~$f$ blir vektoren
\[
\koord{f}{\B} = \vv{4}{-7}
\]
i~$\R^2$.  På denne måten kan vi gå fra å snakke om funksjoner i~$U$
til å snakke om vektorer i~$\R^2$.

La oss nå se på funksjonen~$2f$, som er lineærkombinasjonen
\[
2f = 8 \sin - 14 \cos
\]
av vektorene $\sin$ og~$\cos$.  Det betyr at den har koordinatvektor
\[
\koord{2f}{\B} = \vv{8}{-14}
\]
med hensyn på basisen~$\B$.  Vi ser at å gange vektoren med~$2$
tilsvarer å gange koordinatvektoren med~$2$.
\end{ex}

\begin{thm}
\label{thm:koordinater-lin-komb}
La $V$ være et vektorrom med basis~$\B$.  Koordinatene til en
lineærkombinasjon av vektorer er den tilsvarende lineærkombinasjonen
av koordinatene til hver vektor:
\begin{multline*}
\koord{c_1 \v_1 + c_2 \v_2 + \cdots + c_t \v_t}{\B}\\
= c_1 \cdot \koord{\v_1}{\B} + c_2 \cdot \koord{\v_2}{\B} + \cdots + c_t \cdot \koord{\v_t}{\B}
\end{multline*}
\end{thm}
% TODO bevis

Hvis vi ser på koordinater med hensyn på standardbasisen i~$\R^n$, så
tilsvarer det å lage et vanlig koordinatsystem.  Men hvis vi ser på
koordinater med hensyn på en annen basis for~$\R^n$, så tilsvarer det
å lage et «skrått» koordinatsystem.

\begin{ex}
Vi ser på~$\R^2$ med basisen
\[
\B
 = (\b_1, \b_2)
 = \left( \vv{4}{1},\ \vv{1}{2} \right).
\]
Det å bruke denne basisen for~$\R^2$ tilsvarer å regne i et skrått
koordinatsystem der $\b_1$ og~$\b_2$ tar rollene som enhetsvektorer:
\begin{center}
\begin{tikzpicture}[scale=.48]
\draw[->] (-3.8,0) -- (11.5,0);
\draw[->] (0,-1.5) -- (0,8.8);
\foreach \x in {-3,-2,-1,1,2,3,4,5,6,7,8,9,10,11}
\draw (\x,5pt) -- (\x,-5pt);
\foreach \y in {-1,1,2,3,4,5,6,7,8}
\draw (5pt,\y) -- (-5pt,\y);
\draw (-5,-3) -- (7,0);
\draw (-4,-1) -- (8,2);
\draw (-3,1) -- (9,4);
\draw (-2,3) -- (10,6);
\draw (-1,5) -- (11,8);
\draw (-5,-3) -- (-1,5);
\draw (-1,-2) -- (3,6);
\draw (3,-1) -- (7,7);
\draw (7,0) -- (11,8);
\filldraw (4,1) circle [radius=4pt] node[anchor=south east] {$\b_1$};
\filldraw (1,2) circle [radius=4pt] node[anchor=north west] {$\b_2$};
\filldraw (11,8) circle [radius=4pt] node[anchor=west] {$\v$};
\end{tikzpicture}
\end{center}
For eksempel har vektoren~$\v$ på tegningen koordinater
\[
\v = \vv{11}{8}
\]
med hensyn på standardbasisen, men koordinater
\[
\koord{\v}{\B} = \vv{2}{3}
\]
med hensyn på basisen~$\B$.
\end{ex}

Nå har vi sett noen eksempler på hva en basis kan brukes til.  Videre
vil vi vise at det alltid er mulig å finne en basis, forutsatt at
vektorrommet vårt er endeligdimensjonalt.

\begin{thm}
La $V$ være et endeligdimensjonalt vektorrom.  Da kan enhver endelig
mengde som utspenner $V$ reduseres til en basis for~$V$.  Mer presist:
Hvis $G$ er en endelig mengde av vektorer slik at $\Sp G = V$, så
finnes en delmengde $B \subseteq G$ slik at vektorene i~$B$ utgjør en
basis for~$V$.
\end{thm}
\begin{proof}
Det eneste som kan hindre oss fra å bare bruke vektorene i~$G$ som en
basis er at de kan være lineært avhengige.  Så hvis vektorene i~$G$ er
lineært uavhengige, kan vi bare sette~$B = G$, og vi er ferdige.

Anta nå at vektorene i~$G$ er lineært avhengige.  Da finnes en
vektor~$\v$ i~$G$ som er en lineærkombinasjon av de andre.  Vi lager
en ny mengde
\[
G_1 = G - \{ \v \}
\]
der vi har fjernet denne vektoren.  Siden $\v$ er en lineærkombinasjon
av vektorene i~$G_1$, får vi at $G_1$ utspenner det samme som~$G$,
altså hele vektorrommet~$V$.

Nå kan vi fortsette på samme måte med å fjerne ett og ett element så
lenge vektorene i mengden vår er lineært avhengige.  Da får vi stadig
nye delmengder
\[
G \supset G_1 \supset G_2 \supset \cdots% \supset G_{s-1} \supset G_s
\]
som alle utspenner hele~$V$.  Siden $G$ er en endelig mengde, kan vi
ikke fortsette slik som dette i det uendelige, og på et eller annet
punkt må vi derfor få en mengde av lineært uavhengige vektorer.  Disse
vektorene utgjør en basis for~$V$.
\end{proof}

Siden et endeligdimensjonalt vektorrom per definisjon er utspent av en
endelig mengde, viser dette teoremet at det alltid finnes en basis for
et slikt rom.  Vi skriver dette enda tydeligere i et nytt teorem.

\begin{thm}
Ethvert endeligdimensjonalt vektorrom har en basis.
\end{thm}

% TODO thm hver lin. uavh. mengde kan utvides til basis


\section*{Dimensjon}

Nå som vi vet at alle endeligdimensjonale vektorrom har basis, kan vi
bruke det til å definere dimensjonen til et vektorrom.  Vi vil si at
dimensjonen til et vektorrom er antall vektorer i basisen, men før vi
kan si det, må vi forsikre oss om at forskjellige basiser for det
samme rommet ikke kan ha forskjellig antall elementer.

Vi begynner med å generalisere et kjent resultat fra~$\R^n$ til et
vektorrom med basis.  Vi husker fra teorem~\ref{thm:linavhspan} at
hvis vi har en liste med mer enn~$n$ vektorer i~$\R^n$, så må
vektorene i listen være lineært avhengige.  Det tilsvarende utsagnet
formulert med utgangspunkt i en basis sier at hvis vi har en liste med
flere vektorer enn størrelsen på basisen, så må disse vektorene være
lineært avhengige.

\begin{thm}
\label{thm:flere-vektorer-enn-str-basis}
La $V$ være et vektorrom med en basis~$\B$ som består av~$n$ vektorer.
La $\v_1$, $\v_2$, \ldots, $\v_m$ være $m$ vektorer i~$V$, der $m > n$.
Da er disse vektorene lineært avhengige.
\end{thm}
\begin{proof}
La
\begin{align*}
\u_1 &= \koord{\v_1}{\B} \\
\u_2 &= \koord{\v_2}{\B} \\
     &\ \ \vdots \\
\u_m &= \koord{\v_m}{\B}
\end{align*}
være koordinatvektorene til vektorene vi ser på, med hensyn på
basisen~$\B$.  Da er $\u_1$, $\u_2$, \ldots, $\u_m$ en liste med $m$
vektorer i~$\R^n$, og siden $m > n$ vet vi da fra
teorem~\ref{thm:linavhspan} at de er lineært avhengige.  Det vil si at
det finnes skalarer $c_1$, $c_2$, \ldots, $c_m$ (som ikke alle er~$0$)
slik at
\[
c_1 \u_1 + c_2 \u_2 + \cdots c_m \u_m = \0.
\]
Uttrykket på venstresiden her er det samme som
\[
c_1 \koord{\v_1}{\B} + c_2 \koord{\v_2}{\B} + \cdots + c_m \koord{\v_m}{\B},
\]
og ved teorem~\ref{thm:koordinater-lin-komb} er dette igjen det samme
som
\[
\koord{c_1 \v_1 + c_2 \v_2 + \cdots c_m \v_m}{\B}.
\]
Vi har altså at
\[
\koord{c_1 \v_1 + c_2 \v_2 + \cdots c_m \v_m}{\B} = \0,
\]
og dermed må vi ha
\[
c_1 \v_1 + c_2 \v_2 + \cdots c_m \v_m = \0,
\]
Dette betyr at vektorene $\v_1$, $\v_2$, \ldots, $\v_m$ er lineært
avhengige.
\end{proof}

Ved hjelp av dette teoremet ser vi at alle basiser for samme vektorrom
må ha like mange elementer.

\begin{thm}
\label{thm:basis-str-invariant}
La $V$ være et endeligdimensjonalt vektorrom.  Da har enhver basis
for~$V$ samme størrelse.
\end{thm}
\begin{proof}
Anta at vi har to basiser
\begin{align*}
\B_1 &= (\u_1, \u_2, \ldots, \u_n) \\
\B_2 &= (\v_1, \v_2, \ldots, \v_m)
\end{align*}
for~$V$.  Hvis $m > n$, så sier
teorem~\ref{thm:flere-vektorer-enn-str-basis} at vektorene
\[
\v_1,\ \v_2,\ \ldots,\ \v_m
\]
er lineært avhengige, men det kan de ikke være siden $\B_2$ er en
basis.  Det er altså ikke mulig at $m > n$.  På samme måte viser vi at
det ikke er mulig at $n > m$.  Da er det bare én mulighet igjen,
nemlig at $m = n$, altså at basisene har samme størrelse.
\end{proof}

Nå som vi vet at alle basiser for samme vektorrom har samme størrelse,
kan vi trygt definere dimensjonen til et vektorrom som størrelsen til
en hvilken som helst basis for vektorrommet.

\begin{defn}
La $V$ være et endeligdimensjonalt vektorrom.  Vi definerer
\defterm{dimensjonen} til~$V$ som antall vektorer i en basis for~$V$.
Vi bruker notasjonen $\dim V$ for dimensjonen til~$V$.  Hvis $\B$ er
en basis for~$V$, har vi altså
\[
\dim V = |\B|.\qedhere
\]
\end{defn}

% TODO tekst? eks?

Det er en enkel og grei sammenheng mellom dimensjon og underrom: Et
underrom kan aldri ha større dimensjon enn vektorrommet det er
underrom av.  Vi formulerer dette som et teorem.

\begin{thm}
La $V$ være et vektorrom med et underrom~$U$.  Hvis $V$ er
endeligdimensjonalt, så er $U$ også endeligdimensjonalt, og
\[
\dim U \le \dim V.
\]
\end{thm}
% TODO bevis

% TODO tekst? eks?


\section*{Vektorrom tilknyttet en matrise}

Hittil i dette kapitlet har vi vært veldig generelle og abstrakte, og
ting har kanskje blitt litt høytflyvende.  Vi avslutter kapitlet med
noe litt mer håndfast, der vi ikke trenger å tenke på helt generelle
vektorrom, men bare på underrom av~$\R^n$.

Hvis $A$ er en $m \times n$-matrise, så er det visse underrom
av~$\R^n$ og av~$\R^m$ som er nært knyttet til~$A$, og det er noen
interessante sammenhenger mellom disse rommene.

\medskip\noindent\textbf{Nullrommet. }%
Vi definerer \defterm{nullrommet} til en $m \times n$-matrise~$A$ som
løsningsmengden til likningen $A \x = \0$, altså delmengden
\[
\Null A = \{ \x \in \R^n \mid A \x = \0 \}
\]
av $\R^n$.  I utgangspunktet er dette bare en mengde av vektorer
i~$\R^n$, men vi kan raskt finne ut at det faktisk er et underrom ved
å sjekke at det oppfyller de tre kriteriene i teorem~\ref{thm:underrom}:
\begin{enumerate}
\item Vi ser at nullvektoren er i $\Null A$, siden den er en løsning
av likningen $A \x = \0$.
\item Hvis $\u$ og~$\v$ er vektorer i $\Null A$, så har vi at
$A \u = \0$ og $A \v = \0$.  Da får vi
\[
A (\u + \v) = A \u + A \v = \0 + \0 = \0,
\]
som betyr at summen $\u + \v$ også er i $\Null A$.
\item Hvis $\u$ er i $\Null A$ og $c$ er en skalar, så får vi
\[
A \cdot (c \cdot \u) = c \cdot (A \u) = c \cdot \0 = \0,
\]
slik at $c \cdot \u$ også er i $\Null A$.
\end{enumerate}

\medskip\noindent\textbf{Kolonnerommet. }%
Vi definerer \defterm{kolonnerommet} til en $m \times n$-matrise
\[
A = \begin{bmatrix} \V{a}_1 & \V{a}_2 & \cdots & \V{a}_n \end{bmatrix}
\]
som underrommet av $\R^m$ utspent av kolonnene i~$A$:
\[
\Col A = \Sp \{ \V{a}_1, \V{a}_2, \ldots, \V{a}_n \}
\]
Kolonnerommet til~$A$ består altså av alle lineærkombinasjoner av
kolonnene i~$A$.  Siden et produkt $A \v$ av matrisen~$A$ og en
vektor~$\v$ i~$\R^n$ er definert til å være nettopp en
lineærkombinasjon av kolonnene i~$A$, kan vi også beskrive
kolonnerommet som alle vektorer som er på formen $A \v$:
\[
\Col A = \{ A \v \mid \v \in \R^n \}
\]

\medskip\noindent\textbf{Radrommet. }%
Vi definerer \defterm{radrommet} til en $m \times n$-matrise
\[
A = \begin{bmatrix} \V{r}_1\tr \\ \V{r}_2\tr \\ \vdots \\ \V{r}_n\tr \end{bmatrix}
\]
som underrommet av $\R^n$ utspent av radene i~$A$:
\[
\Row A = \Sp \{ \V{r}_1, \V{r}_2, \ldots, \V{r}_n \}
\]
Radrommet til~$A$ består altså av alle lineærkombinasjoner av radene
i~$A$ (der vi ser på radene som kolonnevektorer).  Dette er det samme
som kolonnerommet til den transponerte matrisen:
\[
\Row A = \Col A\tr
\]

Vi tar nå et ganske langt eksempel der vi ser på hva vi kan si om
nullrommet, kolonnerommet og rad\-rommet til en matrise.

\begin{ex}
La $A$ være følgende matrise:
\[
A =
\begin{bmatrix}
1 & 2 & 1 &  2 & 0 \\
2 & 4 & 3 &  7 & 1 \\
1 & 2 & 2 &  5 & 1 \\
3 & 6 & 6 & 15 & 3
\end{bmatrix}
\]
Vi vil prøve å beskrive nullrommet, kolonnerommet og radrommet
til~$A$.

For å finne nullrommet, må vi løse likningen
\[
A \x = \0.
\]
Det gjør vi ved å gausseliminere matrisen~$A$.  Da får vi (her er
mellomregningen utelatt):
\[
\begin{bmatrix}
1 & 2 & 1 &  2 & 0 \\
2 & 4 & 3 &  7 & 1 \\
1 & 2 & 2 &  5 & 1 \\
3 & 6 & 6 & 15 & 3
\end{bmatrix}
\roweq
\begin{bmatrix}
1 & 2 & 0 & -1 & -1 \\
0 & 0 & 1 &  3 & 1 \\
0 & 0 & 0 &  0 & 0 \\
0 & 0 & 0 &  0 & 0
\end{bmatrix}
\]
Vi får tre frie variabler, og den generelle løsningen blir:
% x1 = -2x2 + x4 + x5
% x2 fri
% x3 = -3x4 - x5
% x4 fri
% x5 fri
\[
\x = \vvvvv{-2}{1}{0}{0}{0} r +
     \vvvvv{1}{0}{-3}{1}{0} s +
     \vvvvv{1}{0}{-1}{0}{1} t
\]
Dette betyr at de tre vektorene
\[
\u = \vvvvv{-2}{1}{0}{0}{0},\quad
\v = \vvvvv{1}{0}{-3}{1}{0}\quad
\text{og}\quad
\w = \vvvvv{1}{0}{-1}{0}{1}
\]
utspenner nullrommet til~$A$.  De er dessuten lineært uavhengige: Legg
merke til at i posisjon to, fire og fem -- som tilsvarer de tre frie
variablene -- har én av vektorene tallet~$1$ og de andre to
tallet~$0$.  Dermed ser vi lett at ingen av dem kan være en
lineærkombinasjon av de to andre, slik at de må være lineært
uavhengige.  Dette betyr at
\[
(\u, \v, \w)
\]
er en basis for nullrommet~$\Null A$.

Når det gjelder kolonnerommet og radrommet, har vi direkte fra
definisjonene at dissee rommene kan beskrives slik:
\begin{align*}
\Col A &=
\Sp \left\{
\vvvv{1}{2}{1}{3},\ %
\vvvv{2}{4}{2}{6},\ %
\vvvv{1}{3}{2}{6},\ %
\vvvv{2}{7}{5}{15},\ %
\vvvv{0}{1}{1}{3}
\right\}
\\
\Row A &=
\Sp \left\{
\vvvvv{1}{2}{1}{ 2}{0},\ %
\vvvvv{2}{4}{3}{ 7}{1},\ %
\vvvvv{1}{2}{2}{ 5}{1},\ %
\vvvvv{3}{6}{6}{15}{3}
\right\}
\end{align*}
Men her har vi bare en utspennende mengde for hvert av rommene.  Den
beste måten å beskrive et vektorrom på er å gi en basis.  Det viser
seg at vi kan finne basiser for kolonnerommet og radrommet til~$A$ ved
å se på hva som skjer når vi gausseliminerer~$A$.

La oss ta kolonnerommet først.  Se på trappeformmatrisen vi endte opp
med.  Den har pivotelementer i første og tredje kolonne.  Hvis vi
stokker om på kolonnene i~$A$ slik at første og tredje kolonne kommer
først, og gjør det samme med trappeformmatrisen, så blir disse
matrisene også radekvivalente (fordi dette tilsvarer at vi bytter om
kolonnene på samme måte i hver matrise vi får underveis i
gausselimineringen):
\[
\left[
\begin{array}{cc|ccc}
1 & 1 & 2 &  2 & 0 \\
2 & 3 & 4 &  7 & 1 \\
1 & 2 & 2 &  5 & 1 \\
3 & 6 & 6 & 15 & 3
\end{array}
\right]
\roweq
\left[
\begin{array}{cc|ccc}
1 & 0 & 2 & -1 & -1 \\
0 & 1 & 0 &  3 & 1 \\
0 & 0 & 0 &  0 & 0 \\
0 & 0 & 0 &  0 & 0
\end{array}
\right]
\]
La oss nå lage en matrise
\[
C =
\begin{bmatrix}
1 & 1 \\
2 & 3 \\
1 & 2 \\
3 & 6
\end{bmatrix}
\]
av første og tredje kolonne i~$A$ (de to kolonnene som ender opp med
pivotelementer i trappeformmatrisen), og la oss kalle de tre andre
kolonnene i~$A$ for
\[
\b_1 = \vvvv{2}{4}{2}{6},\ %
\b_2 = \vvvv{2}{7}{5}{15},\ %
\b_3 = \vvvv{0}{1}{1}{3}.
\]
Matrisen vi fikk ved å stokke om kolonnene i~$A$ kan da beskrives som
\[
\left[
\begin{array}{c|ccc}
C & \b_1 & \b_2 & \b_3
\end{array}
\right].
\]
Vi ser fra trappeformmatrisen at kolonnene i~$C$ er lineært
uavhengige, og at systemene
\[
C \x = \b_1,\quad
C \x = \b_2\quad\text{og}\quad
C \x = \b_3
\]
har løsninger, slik at hver av vektorene $\b_1$, $\b_2$ og~$\b_3$ er
en lineærkombinasjon av kolonnene i~$C$.  Dette betyr at
\[
\left(
\vvvv{1}{2}{1}{3},\ %
\vvvv{1}{3}{2}{6}
\right)
\]
er en basis for kolonnerommet~$\Col A$.

Det er litt enklere å se hvordan gausseliminasjonen gir oss en basis
for radrommet.  Vi kan se at hvis to matriser er radekvivalente, så
har de samme radrom.  Når vi utfører en radoperasjon er det nemlig
slik at alle rader i den nye matrisen er lineærkombinasjoner av radene
i den gamle matrisen (dette kan du ganske enkelt sjekke selv).
Dessuten kan vi alltid finne en «omvendt» radoperasjon som tar oss
tilbake til den gamle matrisen, slik at alle rader i den gamle
matrisen er lineærkombinasjoner av radene i den nye.  Altså har
matrisene samme radrom.

Dette betyr at for å beskrive radrommet til~$A$ kan vi like godt se på
trappeformmatrisen vi fikk ved å gausseliminere~$A$.  Der ser vi lett
at alle radene som ikke er nullrader må være lineært uavhengige.  Vi
får dermed at
\[
\left(
\vvvvv{1}{2}{0}{-1}{-1},\ %
\vvvvv{0}{0}{1}{3}{1}
\right)
\]
er en basis for radrommet~$\Row A$.
\end{ex}

Metodene vi brukte i eksempelet for å finne basiser for nullrommet,
kolonnerommet og radrommet fungerer generelt for en hvilken som helst
matrise.  Vi ser dermed at vi kan beskrive dimensjonene til disse tre
rommene ved hjelp av antall frie variabler og antall pivotelementer i
trappeformmatrisen.

% TODO: burde beskrevet metodene skikkelig på en generell måte,
%       og bevist følgende teorem

\begin{thm}
\label{thm:dim-null-col-row}
La $A$ være en $m \times n$-matrise, og la $E$ være trappeformmatrisen
vi får når vi gausseliminerer~$A$.  Da har vi:
\begin{enumerate}
\item[(a)] Dimensjonen til nullrommet til~$A$ er lik antall frie
variabler vi får når vi løser likningssystemet $A \x = \b$, altså
antall kolonner uten pivotelement i~$E$.
\item[(b)] Dimensjonen til kolonnerommet til~$A$ er lik antall kolonner
med pivotelementer i~$E$.
\item[(c)] Dimensjonen til radrommet til~$A$ er lik antall rader som
ikke er null i~$E$.
\end{enumerate}
\end{thm}

Siden det er ett pivotelement i hver rad som ikke er null, får vi ved
å kombinere del~(b) og~(c) i dette teoremet at kolonnerommet og
radrommet har samme dimensjon.  Vi skriver opp dette også som et
teorem.

\begin{thm}
\label{thm:dim-row=dim-col}
La $A$ være en $m \times n$-matrise.  Da har kolonnerommet og
radrommet til~$A$ samme dimensjon:
\[
\dim \Col A = \dim \Row A
\]
\end{thm}

Dette ene tallet, som både er dimensjonen til kolonnerommet og
dimensjonen til radrommet, kalles \defterm{rangen} til matrisen.  Vi
skriver:
\[
\rank A = \dim \Col A = \dim \Row A
\]

Siden enhver kolonne i trappeformmatrisen enten inneholder et
pivotelement eller gir opphav til en fri variabel for likningen
$A \x = \0$, får vi følgende resultat ved å kombinere del~(a) og~(b)
fra teorem~\ref{thm:dim-null-col-row}.

\begin{thm}
La $A$ være en $m \times n$-matrise.  Da er
\[
\dim \Null A + \rank A = n.
\]
\end{thm}


\kapittelslutt
