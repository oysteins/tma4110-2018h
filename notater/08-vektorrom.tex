\ifx\inkludert\undefined
\documentclass[norsk,a4paper,twocolumn,oneside]{memoir}

\usepackage[utf8]{inputenc}
\usepackage{babel}
\usepackage{amsmath,amssymb,amsthm}
\usepackage{mathrsfs}
\usepackage[total={17cm,27cm}]{geometry}
\usepackage[table]{xcolor}
%\usepackage{tabularx}
\usepackage{systeme}
%\usepackage{hyperref}
%\usepackage{enumerate}
\usepackage{ifthen}
\usepackage{textgreek}
\usepackage{multirow}
\usepackage{placeins}

%\usepackage{sectsty}
\setsecheadstyle{\bfseries\large}
%\subsectionfont{\bf\normalsize}

\usepackage{tikz,pgfplots}
\usetikzlibrary{calc}
\usetikzlibrary{arrows.meta}
\def\centerarc[#1](#2)(#3:#4:#5)% Syntax: [draw options] (center) (initial angle:final angle:radius)
    { \draw[#1] ($(#2)+({#5*cos(#3)},{#5*sin(#3)})$) arc (#3:#4:#5); }
\usepackage{pgfornament}

\newcommand{\defterm}[1]{\emph{#1}}

\newcommand{\N}{\mathbb{N}}
\newcommand{\Z}{\mathbb{Z}}
\newcommand{\Q}{\mathbb{Q}}
\newcommand{\R}{\mathbb{R}}

\newcommand{\M}{\mathcal{M}} % vektorrom av matriser
\newcommand{\C}{\mathcal{C}} % vektorrom av kontinuerlige funksjoner
\renewcommand{\P}{\mathcal{P}} % vektorrom av polynomer
\newcommand{\B}{\mathscr{B}} % basis

\renewcommand{\Im}{\operatorname{Im}}
\renewcommand{\Re}{\operatorname{Re}}

\newcommand{\abs}[1]{|#1|}
\newcommand{\intersect}{\cap}
\newcommand{\union}{\cup}
\newcommand{\fcomp}{\circ}
\newcommand{\iso}{\cong}

\newcommand{\roweq}{\sim}
\DeclareMathOperator{\Sp}{Sp}
\DeclareMathOperator{\Null}{Null}
\DeclareMathOperator{\Col}{Col}
\DeclareMathOperator{\Row}{Row}
\DeclareMathOperator{\rank}{rank}
\DeclareMathOperator{\im}{im}
\DeclareMathOperator{\id}{id}
\DeclareMathOperator{\Hom}{Hom}
\newcommand{\tr}{^\top}
\newcommand{\koord}[2]{[\,{#1}\,]_{#2}} % koordinater mhp basis

\newcommand{\V}[1]{\mathbf{#1}}
\newcommand{\vv}[2]{\begin{bmatrix} #1 \\ #2 \end{bmatrix}}
\newcommand{\vvS}[2]{\left[ \begin{smallmatrix} #1 \\ #2 \end{smallmatrix} \right]}
\newcommand{\vvv}[3]{\begin{bmatrix} #1 \\ #2 \\ #3 \end{bmatrix}}
\newcommand{\vvvv}[4]{\begin{bmatrix} #1 \\ #2 \\ #3 \\ #4 \end{bmatrix}}
\newcommand{\vvvvv}[5]{\begin{bmatrix} #1 \\ #2 \\ #3 \\ #4 \\ #5 \end{bmatrix}}
\newcommand{\vn}[2]{\vvvv{#1_1}{#1_2}{\vdots}{#1_#2}}

\newcommand{\e}{\V{e}}
\renewcommand{\u}{\V{u}}
\renewcommand{\v}{\V{v}}
\newcommand{\w}{\V{w}}
\renewcommand{\b}{\V{b}}
\newcommand{\x}{\V{x}}
\newcommand{\0}{\V{0}}

\newenvironment{amatrix}[1]{% "augmented matrix"
  \left[\begin{array}{*{#1}{c}|c}
}{%
  \end{array}\right]
}

\newcommand{\boks}[1]{\framebox{\strut $#1$}}

% \newcounter{notatnr}
% \newcommand{\notatnr}[2]
% {\setcounter{notatnr}{#1}%
%  \setcounter{page}{#2}%
% }

\newtheorem{thm}{Teorem}[chapter]
\newtheorem*{thm-nn}{Teorem}
\newtheorem{cor}[thm]{Korollar}
\newtheorem{lem}[thm]{Lemma}
\newtheorem{prop}[thm]{Proposisjon}
\theoremstyle{definition}
\newtheorem{exx}[thm]{Eksempel}
\newtheorem*{defnx}{Definisjon}
\newtheorem*{oppg}{Oppgave}
\newtheorem*{merkx}{Merk}
\newtheorem*{spmx}{Spørsmål}

\newenvironment{defn}
  {\pushQED{\qed}\renewcommand{\qedsymbol}{$\triangle$}\defnx}
  {\popQED\enddefnx}
\newenvironment{ex}
  {\pushQED{\qed}\renewcommand{\qedsymbol}{$\triangle$}\exx}
  {\popQED\endexx}
\newenvironment{merk}
  {\pushQED{\qed}\renewcommand{\qedsymbol}{$\triangle$}\merkx}
  {\popQED\endmerkx}
\newenvironment{spm}
  {\pushQED{\qed}\renewcommand{\qedsymbol}{$\triangle$}\spmx}
  {\popQED\endspmx}

\setlength{\columnsep}{26pt}

\newcommand{\Tittel}[2]{%
\twocolumn[
\begin{center}
\Large
\begin{tabularx}{\textwidth}{cXr}
\cellcolor{black}\color{white}%
\bf {#1} &
#2
\hfill &
\footnotesize TMA4110 høsten 2018
\\ \hline
\end{tabularx}
\end{center}
]}

\newcommand{\tittel}[1]{\Tittel{\arabic{notatnr}}{#1}}

\newcommand{\linje}{%
\begin{center}
\rule{.8\linewidth}{0.4pt}
\end{center}
}


\newcommand{\kapittelemnenavn}{TMA4110 høsten 2018}
\newcommand{\chapternumber}{}

\makechapterstyle{tma4110}{%
 \renewcommand*{\chapterheadstart}{}
 \renewcommand*{\printchaptername}{}
 \renewcommand*{\chapternamenum}{}
 \renewcommand*{\printchapternum}{\renewcommand{\chapternumber}{\thechapter}}
 \renewcommand*{\afterchapternum}{}
 \renewcommand*{\printchapternonum}{\renewcommand{\chapternumber}{}}
 \renewcommand*{\printchaptertitle}[1]{
\LARGE
\begin{tabularx}{\textwidth}{cXr}
\cellcolor{black}\color{white}%
\textbf{\chapternumber} &
\textbf{##1}
\hfill &
\footnotesize\kapittelemnenavn
\\ \hline
\end{tabularx}%
}
 \renewcommand*{\afterchaptertitle}{\par\nobreak\vskip \afterchapskip}
 % \newcommand{\chapnamefont}{\normalfont\huge\bfseries}
 % \newcommand{\chapnumfont}{\normalfont\huge\bfseries}
 % \newcommand{\chaptitlefont}{\normalfont\Huge\bfseries}
 \setlength{\beforechapskip}{0pt}
 \setlength{\midchapskip}{0pt}
 \setlength{\afterchapskip}{10pt}
}
\chapterstyle{tma4110}
\pagestyle{plain}


\newboolean{vis-oppgaver}
\newboolean{vis-losninger}
\setboolean{vis-oppgaver}{true}
\setboolean{vis-losninger}{false}

\newcounter{oppg-kap} % kapittelnummerering for oppgaver
\newcounter{oppgnr}[oppg-kap]
\newcounter{punktnr}[oppgnr]

\newenvironment{oppgave}%
 {\ifthenelse{\boolean{vis-oppgaver}}%
             {\par\noindent\stepcounter{oppgnr}\textbf{\arabic{oppgnr}.}}%
             {\expandafter\comment}}%
 {\ifthenelse{\boolean{vis-oppgaver}}%
             {\par\bigskip}%
             {\expandafter\endcomment}}

\newenvironment{losning}%
 {\ifthenelse{\boolean{vis-losninger}}%
             {\par\noindent\stepcounter{oppgnr}\textbf{\arabic{oppg-kap}.\arabic{oppgnr}.}}%
             {\expandafter\comment}}%
 {\ifthenelse{\boolean{vis-losninger}}%
             {\par\bigskip}%
             {\expandafter\endcomment}}

\newenvironment{punkt}
 {\par\smallskip\noindent\stepcounter{punktnr}\textbf{\alph{punktnr})} }
 {\par}

\newcommand{\kap}[1]{\setcounter{oppg-kap}{#1}\addtocounter{oppg-kap}{-1}\stepcounter{oppg-kap}}

\newcommand{\oppgaver}[1]{%
  \kap{#1}%
  \ifthenelse{\boolean{vis-oppgaver}}%
             {\linje\section*{Oppgaver}}%
             {}}

\usepackage{xr}
\externaldocument{tma4110-2018h}
\newcommand{\kapittel}[2]{\setcounter{chapter}{#1}\addtocounter{chapter}{-1}\chapter{#2}}
\newcommand{\kapittelslutt}{\enddocument}
\begin{document}
\chapterstyle{tma4110}
\pagestyle{plain}
\fi


\kapittel{8}{Vektorrom}
\label{ch:vektorrom}



Nå har vi lært ganske mye lineær algebra.  Vi startet med lineære
likningssystemer, og så hvordan disse kan løses ved gausseliminasjon.

% TODO intro

\section*{Vektorer slik vi kjenner dem}

% TODO om \R^n
% tegninger \R^0, \R^1, \R^2, ...
% vektor: pil/punkt/koordinater (kolonnevektor)

\section*{Hva kan vi gjøre med vektorer?}

% TODO addisjon og skalarmultiplikasjon

\section*{Aksiomer for vektorrom}

% TODO beskriv alle aksiomene

\onecolumn % TODO: skulle vært flytende figur, som plasseres på egen side
\begin{center}
\textbf{Vektorromsaksiomene}

\begin{tabular}{rlll}
\multirow{4}{*}{aksiomer for addisjon $\left\{\rule{10pt}{0pt}\right.$} &
  (V1) & For alle vektorer $\V{u}$, $\V{v}$ og~$\V{w}$ er $(\V{u} + \V{v}) + \V{w} = \V{u} + (\V{v} + \V{w})$
       & \emph{Vektoraddisjon er assosiativ} \\
& (V2) & For alle vektorer $\V{u}$ og~$\V{v}$ er $\V{u} + \V{v} = \V{v} + \V{u}$
       & \emph{Vektoraddisjon er kommutativ} \\
\end{tabular}
\end{center}
\twocolumn

\section*{Definisjonen av vektorrom}

% TODO tekst

\begin{defn} % TODO skrives bedre
Et \defterm{vektorrom} er en mengde~$V$ sammen med to operasjoner:
\begin{align*}
\text{addisjon av vektorer: } & \V{u} + \V{v} \\
\text{skalarmultiplikasjon: } & c \cdot \V{u}
\end{align*}
Addisjonen skal være definert for alle elementer $\V{u}$ og~$\V{v}$
i~$V$, og skalarmultiplikasjonen for alle skalarer~$c$ og alle $\V{u}$
i~$V$.  Resultatet av operasjonene skal alltid være et element i~$V$.
Mengden~$V$ og de to operasjonene må oppfylle vektorromsaksiomene
(V1)--(V8).

Elementene i~$V$ kalles \defterm{vektorer}.
\end{defn}

% TODO tekst

\begin{ex}
For hver $n \ge 0$ har vi at $\R^n$ er et vektorrom.
\end{ex}


\section*{Eksempler på vektorrom}

% TODO eksempler på vektorrom:
% underrom av \R^2, \Sp (1,1)
% \C(D) (med oppg/eks)
% \C^\infty(D)?
% polynomfunksjoner?
% alle funksjoner D \to \R?
% \R^\N (med oppg/eks)
% \M_{m \times n}, \M_n


\section*{Underrom}

\begin{defn}
Et \defterm{underrom} av et vektorrom~$V$ er en delmengde
$U \subseteq V$ som i seg selv utgjør et vektorrom, med addisjon og
skalarmultiplikasjon definert på samme måte som i~$V$.
\end{defn}

% TODO eks \Sp(1,1) i \R^2

% TODO eks (a,1) i \R^2

\begin{thm}
\label{thm:underrom}
% TODO
\end{thm}

\begin{thm}
\label{thm:underrom-sp}
En mengde $\Sp \{ \V{v}_1, \V{v}_2, \ldots, \V{v}_t \}$ utspent av vektorer
i et vektorrom~$V$ er alltid et underrom av~$V$.
\end{thm}


\section*{Endeligdimensjonale og uendeligdimensjonale vektorrom}

\begin{defn}
% TODO endeligdimensjonal/uendeligdim
\end{defn}

% TODO eks \R^n

% TODO eks \R^\N (eller annet, for eksempel polynomer?)


\section*{Basis}

% TODO def basis

% TODO eks basiser for \R^3 eller \R^2

% TODO standardbasis \R^n

% TODO thm hver vektor kan skrives entydig som lineærkombinasjon av basisvektorene,
%      def koordinater mhp basis, med notasjon

% TODO eks, tegning med skrå koordinater i \R^2


% TODO thm redusere utspennende mengde til basis

% TODO thm alle end.dim. vektorrom har basis (følger direkte fra forrige thm)


% TODO thm underrom av end.dim. er end.dim.

% TODO thm hver lin. uavh. mengde kan utvides til basis


\section*{Dimensjon}

% TODO thm gitt basis B med |B|=n, da er hver mengde med mer enn n elementer lin. avhengig

% TODO thm hver basis for samme (end.dim.) vektorrom har samme størrelse (direkte fra forrige thm)

% TODO def dimensjon, notasjon dim V



% TODO: radrom, kolonnerom, nullrom: her eller i eget kapittel?


\kapittelslutt
