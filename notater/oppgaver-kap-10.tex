% -*- TeX-master: "oving08"; -*-
\oppgaver{10}

\begin{oppgave}
Finn $z$, der
\[
z^2-z+5=0 \\
\]
\[
z^3=2i \\
\]
\[
z^4=2 \\
\]
\[
z^5=2+2i \\
\]
\end{oppgave}

\begin{losning}
TODO
\end{losning}


\begin{oppgave}
Beregn $(1+2i)^3$, $\frac{5}{-3+4i}$ og $\left(\frac{2+i}{3-2i}\right)^2$.
\end{oppgave}

\begin{losning}
TODO
\end{losning}


\begin{oppgave}
Skriv $(1+2i)^3$, $\frac{5}{-3+4i}$ og $\left(\frac{2+i}{3-2i}\right)^2$ på polar form.
\end{oppgave}

\begin{losning}
TODO
\end{losning}


\begin{oppgave}
Merk av $(1+2i)^3$, $\frac{5}{-3+4i}$ og $\left(\frac{2+i}{3-2i}\right)^2$ i det komplekse planet.
\end{oppgave}

\begin{losning}
TODO
\end{losning}


\begin{oppgave}
Vis at $e^{i(x+y)}=e^{ix}e^{iy}$.
\end{oppgave}

\begin{losning}
TODO
\end{losning}


\begin{oppgave}
Vis at $(\overline z)^n=\overline{z^n}$. Hint: de Moivres formel.
\end{oppgave}


\begin{losning}
TODO
\end{losning}




\begin{oppgave}
\begin{punkt}
Skriv det komplekse tallet $-1+i\sqrt{3}$ på polarform.
\end{punkt}

\begin{punkt}
Vis at $-1+i\sqrt{3}$ er en sjetterot av 64 (uten å finne alle sjetterøttene til 64).
\end{punkt}

\begin{punkt}
Finn alle sjetterøttene til $-1+i\sqrt{3}$. Skissér dem i det komplekse planet.
\end{punkt}

\end{oppgave}

\begin{losning}
\begin{punkt}
$2e^{\frac{2\pi i}{3}}$
\end{punkt}

\begin{punkt}
$(-1+i\sqrt{3})^6=(2e^{\frac{2\pi i}{3}})^6=64e^{4\pi i}=64.$
\end{punkt}

\begin{punkt}
$2,2e^{\frac{\pi i}{3}},2e^{\frac{2\pi i}{3}},-2,2e^{\frac{4\pi i}{3}},2e^{\frac{5\pi i}{3}}$


\begin{center}
	\begin{tikzpicture}
	\draw[->] (-3,0) -- (3,0) node[right] {$x$};
	\draw[->] (0,-3) -- (0,3) node[above] {$y$};
	
	\node at (2,0) [] {$\boldsymbol{\cdot}$};
	\node at (1,1.732) [] {$\boldsymbol{\cdot}$};
	\node at (-1,1.732) [] {$\boldsymbol{\cdot}$};
	\node at (-2,0) [] {$\boldsymbol{\cdot}$};
	\node at (-1,-1.732) [] {$\boldsymbol{\cdot}$};
	\node at (1,-1.732) [] {$\boldsymbol{\cdot}$};
	
	\end{tikzpicture}
\end{center}


\end{punkt}

\end{losning}


\begin{oppgave}
\begin{punkt}
Finn alle løsninger av likningen $$z^3-3z^2+3z-1=0.$$ Skissér løsningene i det komplekse planet.
\end{punkt}
\begin{punkt}
Finn alle løsninger av likningen $$z^3-3z^2+3z-2=0$$ ved å bruke svaret du fant i $\textbf{a)}$. Skissér løsningene i det komplekse planet.
\end{punkt}
\end{oppgave}

\begin{losning}

\begin{punkt}
Vi har at $(z-1)^3=z^3-3z^2+3z-1$ slik at $z=1$ er en trippelrot.

\noindent
Aletrnativt kan du gjette på løsningen $z=1$ og deretter bruke polynomdivisjon.
\end{punkt}

\begin{punkt}


Fra \textbf{a)} har vi at likningen kan skrives på formen $$(z-1)^3=1=e^{2\pi i}.$$ Tredjerøttene til $1$ er $$e^{\frac{2\pi}{3}}, \quad e^{\frac{4\pi}{3}},\quad 1.$$ Dermed har vi tre løsninger for $z-1$ som igjen gir tre løsninger for $z$: $$z=1+e^{\frac{2\pi}{3}},1+e^{\frac{4\pi}{3}},2.$$

\noindent
Tredjerøttene til $1$ ligger uniformt fordelt på sirekelen sentrert i origo med radius 1, men er forskjøvet til å ligge uniformt fordelt på sirkelen sentrert i $(1,0)$ med radius 1.


\end{punkt}

\end{losning}


\begin{oppgave}

\begin{punkt}
Finn alle tredjerøttene til $1$. Tegn en rett strek mellom løsningene (etter økende vinkel). Hva slags geometrisk figur er dette?
\end{punkt}

\begin{punkt}
Repeter del $\textbf{a)}$ for alle fjerderøttene til $1$.
\end{punkt}


\begin{punkt}
Repeter del $\textbf{a)}$ for alle $n$-terøttene ($n\geq 3$) til $1$.
\end{punkt}

\begin{punkt}
Får vi samme geometriske figur i $\textbf{c)}$ dersom vi bytter ut $1$ med et reelt tall $r \neq 0$? Hva med et generelt kompleks tall $w\neq 0$?
\end{punkt}

\end{oppgave}


\begin{losning}

\begin{punkt}
Vi følger den vanlige metoden for å finne $n$-terøtter: Skriv $1$ på polarform; $1=e^{(2\pi k )i}$, $k$ et heltall. Ta tredjeroten for å få $\sqrt[3]{1}=e^{(\frac{2\pi k}{3} )i}$, $k$ heltall. Dette gir ulike komplekse tall for $k=0,1,2$: 
$$1,\quad e^{\frac{2\pi i}{3}}, \quad e^{\frac{4\pi i}{3}}.$$ 

\noindent
Dette er tre punkter uniformt fordelt på sirkelen sentrert i origo med radius 1. Dersom vi trekker rette linjer mellom punkter etter økende vinkel får vi en trekant med røttene som hjørner.
\end{punkt}

\begin{punkt}
Som i $\textbf{a)}$ gir $$1,\quad i,\quad -1, \quad -i. $$ Den geometriske figuren blir en firkant med hjørner i røttene.
\end{punkt}

\begin{punkt}
Som i $\textbf{a)}$ gir $$1,\quad e^{\frac{2\pi i}{n}},\quad e^{\frac{4\pi i}{n}},\dots,e^{\frac{(n-1)\pi i}{n}}.$$ Den geometriske figuren blir en $n$-kant med hjørner i røttene.
\end{punkt}

\begin{punkt}
Vi får fortsatt en $n$-kant med hjørner i røttene.

\noindent
Forklaring: Et komplekst tall kan skrives på formen $re^{i\theta}$ hvor $r>0$ er et reelt tall. Nå blir $n$-terøttene $\sqrt[n]{r}e^{(\theta+\frac{2\pi k}{n})i}$, $k=0,1,\dots, n-1$ (en hel runde i det komplekse planet). Dette er en uniform fordeling av $n$ punkter på sirkelen sentrert i origo med radius $\sqrt[n]{r}$.
\end{punkt}

\end{losning}

\begin{oppgave}
La $z$ og $w$ være to komplekse tall, begge ulik null. Er det mulig at $zw=0$?
\end{oppgave}

\begin{losning}
Nei.

\noindent
Løsning: Skriv tallene på polarform, $z=re^{i\theta}$ og $w=\rho e^{i\phi}$. Nå kan produktet uttrykkes $zw=r\rho e^{i(\theta+\phi)}$, hvor $r \rho>0$. Derfor er produktet forskjellig fra null.

\noindent
Alternativ løsning: Du kan også løse oppgaven ved å skrive $z$ og~$w$ på formen $a+ib$.
\end{losning}



\begin{oppgave}
La $n>0$ være et partall. Anta at $z$ er en $n$-terot av et reelt tall $r$ (betraktet som et komplekst tall med imaginær del lik null). Er den konjugerte $\bar{z}$ også en $n$-terot?
\end{oppgave}

\begin{losning}
Ja.

\noindent
Røttene til $r$ fordeles uniformt langs sirkelen om origo med radius $\sqrt[n]{r}$, og $\sqrt[n]{r}$ er en rot. Vi har sett eksempler på dette tidligere i øvingen.
\end{losning}

