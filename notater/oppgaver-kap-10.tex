% -*- TeX-master: "oving08"; -*-
\oppgaver{10}


\begin{oppgave}
Finn real- og imaginærdelen til
\begin{punkt}
$z^4$ 
\end{punkt}
\begin{punkt}
$\frac{1}{z}$ 
\end{punkt}
\begin{punkt}
$\frac{z-1}{z+1}$ 
\end{punkt}
\begin{punkt}
$\frac{1}{z^2}$ 
\end{punkt}
\end{oppgave}

\begin{losning}
TODO
\end{losning}


\begin{oppgave}
Beregn og merk av i det komplekse planet
\begin{punkt}
$(1+2i)^3$ 
\end{punkt}
\begin{punkt}
$\frac{5}{-3+4i}$  
\end{punkt}
\begin{punkt}
$\left(\frac{2+i}{3-2i}\right)^2$
\end{punkt}
\end{oppgave}

\begin{losning}
TODO
\end{losning}

\begin{oppgave}
Løs ligningene

\begin{punkt}
$z^2-z+5=0$ 
\end{punkt}
\begin{punkt}
$z^3=2i$
\end{punkt}
\begin{punkt}
$z^4=2 $
\end{punkt}
\begin{punkt}
$z^5=2+2i$
\end{punkt}
\end{oppgave}

\begin{losning}
TODO
\end{losning}



\begin{oppgave}
Vis at 
\[
|z+w|^2+|z-w|^2=2|z|^2+2|w|^2.
\]
\end{oppgave}



\begin{oppgave}
En variant av Eulers formel kalles de Moivres formel
\[
(\cos x + i\sin x)^n=\cos nx + i\sin nx.
\]
\begin{punkt}
Utled de Moivres formel.
\end{punkt}

\begin{punkt}
Vis at $(\overline z)^n=\overline{z^n}$.
\end{punkt}
\end{oppgave}


\begin{losning}
TODO
\end{losning}


\begin{oppgave}
La $z$ og~$w$ være følgende komplekse tall:
\[
z = \frac{3\pi}{4} i
\qquad\qquad
w = -\frac{3\pi}{4}i
\]
\begin{punkt}
% Skriv $e^{i\frac{3\pi}{4}}-e^{-i^\frac{3 \pi}{4}}$ på polarform.
Skriv tallet $e^z - e^w$ på polar form.
\end{punkt}

\begin{punkt}
% Skriv $\frac{e^{i\frac{3\pi}{4}}}{e^{-i^\frac{3 \pi}{4}}}$ på polarform.
Skriv tallet $e^z/e^w$ på polar form.
\end{punkt}

\begin{punkt}
Er det enkelt å dele komplekse tall på polar form?
Hva med addere?
Sammenlign med kartesisk form (altså formen $a + bi$).
\end{punkt}

\end{oppgave}

\begin{losning}

\begin{punkt}
Euler gir: $e^{i\frac{3\pi}{4}}=\cos(\frac{3\pi}{4})+i\sin (\frac{3\pi}{4})$. Bruk at $\cos$ er en like funksjon og $\sin$ er en odde funksjon for å se at $e^{-i\frac{3\pi}{4}}=\cos(\frac{3\pi}{4})-i\sin (\frac{3\pi}{4})$. Nå kan vi regne ut at $$e^{i\frac{3\pi}{4}}-e^{-i\frac{3 \pi}{4}}=-2i\sin \frac{3\pi}{4}=\sqrt{2}e^{i\frac{3\pi}{2}}.$$
\end{punkt}

\begin{punkt}
$\frac{e^{i\frac{3\pi}{4}}}{e^{-i\frac{3 \pi}{4}}}=e^{i(\frac{3\pi}{4}-(-\frac{3 \pi}{4}))}=e^{i\frac{3\pi}{2}}.$ 
\end{punkt}

\begin{punkt}
Polar Form: Veldig enkelt å multiplisere/addere som del $\textbf{b)}$ illustrerer; vanskelig å addere som del $\textbf{a)}$ illustrerer.

\noindent
Kartesisk: Her blir det motsatt: Enkelt å addere; vanskelig å dele.
\end{punkt}

\end{losning}



\begin{oppgave}
\begin{punkt}
Skriv det komplekse tallet $-1+i\sqrt{3}$ på polar form.
\end{punkt}

\begin{punkt}
Vis at $-1+i\sqrt{3}$ er en sjetterot av 64.
\end{punkt}

\begin{punkt}
Finn alle sjetterøttene til $64$. Skissér dem i det komplekse planet.
\end{punkt}

\end{oppgave}

\begin{losning}
\begin{punkt}
$2e^{\frac{2\pi i}{3}}$
\end{punkt}

\begin{punkt}
$(-1+i\sqrt{3})^6=(2e^{\frac{2\pi i}{3}})^6=64e^{4\pi i}=64.$
\end{punkt}

\begin{punkt}
$2,2e^{\frac{\pi i}{3}},2e^{\frac{2\pi i}{3}},-2,2e^{\frac{4\pi i}{3}},2e^{\frac{5\pi i}{3}}$


\begin{center}
	\begin{tikzpicture}
	\draw[->] (-3,0) -- (3,0) node[right] {$x$};
	\draw[->] (0,-3) -- (0,3) node[above] {$y$};
	
	\node at (2,0) [] {$\boldsymbol{\cdot}$};
	\node at (1,1.732) [] {$\boldsymbol{\cdot}$};
	\node at (-1,1.732) [] {$\boldsymbol{\cdot}$};
	\node at (-2,0) [] {$\boldsymbol{\cdot}$};
	\node at (-1,-1.732) [] {$\boldsymbol{\cdot}$};
	\node at (1,-1.732) [] {$\boldsymbol{\cdot}$};
	
	\end{tikzpicture}
\end{center}


\end{punkt}

\end{losning}


\begin{oppgave}
I denne oppgaven kan det være du bør huske på Pascals trekant.
\begin{punkt}
Finn alle løsninger av likningen $$z^3-3z^2+3z-1=0.$$ Skissér løsningene i det komplekse planet.
\end{punkt}
\begin{punkt}
Finn alle løsninger av likningen $$z^3-3z^2+3z-2=0$$ ved å bruke svaret du fant i $\textbf{a)}$. Skissér løsningene i det komplekse planet.
\end{punkt}
\end{oppgave}

\begin{losning}

\begin{punkt}
Vi har at $(z-1)^3=z^3-3z^2+3z-1$ slik at $z=1$ er en trippelrot.

\noindent
Aletrnativt kan du gjette på løsningen $z=1$ og deretter bruke polynomdivisjon.
\end{punkt}

\begin{punkt}


Fra \textbf{a)} har vi at likningen kan skrives på formen $$(z-1)^3=1=e^{2\pi i}.$$ Tredjerøttene til $1$ er $$e^{\frac{2\pi}{3}}, \quad e^{\frac{4\pi}{3}},\quad 1.$$ Dermed har vi tre løsninger for $z-1$ som igjen gir tre løsninger for $z$: $$z=1+e^{\frac{2\pi}{3}},1+e^{\frac{4\pi}{3}},2.$$

\noindent
Tredjerøttene til $1$ ligger uniformt fordelt på sirekelen sentrert i origo med radius 1, men er forskjøvet til å ligge uniformt fordelt på sirkelen sentrert i $(1,0)$ med radius 1.


\end{punkt}

\end{losning}

\begin{oppgave}

\begin{punkt}
Finn alle tredjerøttene til $1$. Tegn en rett strek mellom løsningene (etter økende vinkel). Hva slags geometrisk figur er dette?
\end{punkt}

\begin{punkt}
Repeter del $\textbf{a)}$ for alle fjerderøttene til $1$.
\end{punkt}


\begin{punkt}
Repeter del $\textbf{a)}$ for alle $n$-terøttene ($n\geq 3$) til $1$.
\end{punkt}

\begin{punkt}
Får vi samme geometriske figur i $\textbf{c)}$ dersom vi bytter ut $1$ med et reelt tall $r \neq 0$? Hva med et generelt kompleks tall $w\neq 0$?
\end{punkt}
\end{oppgave}


\begin{losning}

\begin{punkt}
Vi følger den vanlige metoden for å finne $n$-terøtter: Skriv $1$ på polar form; $1=e^{(2\pi k )i}$, $k$ et heltall. Ta tredjeroten for å få $\sqrt[3]{1}=e^{(\frac{2\pi k}{3} )i}$, $k$ heltall. Dette gir ulike komplekse tall for $k=0,1,2$: 
$$1,\quad e^{\frac{2\pi i}{3}}, \quad e^{\frac{4\pi i}{3}}.$$ 

\noindent
Dette er tre punkter uniformt fordelt på sirkelen sentrert i origo med radius 1. Dersom vi trekker rette linjer mellom punkter etter økende vinkel får vi en trekant med røttene som hjørner.
\end{punkt}

\begin{punkt}
Som i $\textbf{a)}$ gir $$1,\quad i,\quad -1, \quad -i. $$ Den geometriske figuren blir en firkant med hjørner i røttene.
\end{punkt}

\begin{punkt}
Som i $\textbf{a)}$ gir $$1,\quad e^{\frac{2\pi i}{n}},\quad e^{\frac{4\pi i}{n}},\dots,e^{\frac{(n-1)\pi i}{n}}.$$ Den geometriske figuren blir en $n$-kant med hjørner i røttene.
\end{punkt}

\begin{punkt}
Vi får fortsatt en $n$-kant med hjørner i røttene.

\noindent
Forklaring: Et komplekst tall kan skrives på formen $re^{i\theta}$ hvor $r>0$ er et reelt tall. Nå blir $n$-terøttene $\sqrt[n]{r}e^{(\theta+\frac{2\pi k}{n})i}$, $k=0,1,\dots, n-1$ (en hel runde i det komplekse planet). Dette er en uniform fordeling av $n$ punkter på sirkelen sentrert i origo med radius $\sqrt[n]{r}$.
\end{punkt}

\end{losning}


\begin{oppgave}
Vis at dersom koeffisentene $a_i$ i polynomligningen
\[
a_nz^n+a_{n-1}z^{n-1}+...+a_1z+a_0=0
\]
er reelle, kommer løsningene i konjugatpar, altså at dersom $w$ er en løsning, er også $\overline w$ det.
\end{oppgave}

\begin{losning}
Vi har at
\[
a_nw^n+a_{n-1}w^{n-1}+...+a_1w+a_0=0.
\]
Konjugering av hele ligningen gir 
\[
a_n(\overline w)^n+a_{n-1}(\overline w)^{n-1}+...+a_1(\overline w)+a_0=0,
\]
men tidligere har du vist at $(\overline w)^n=(\overline {w^n}$, slik at 
\[
a_n\overline w^n+a_{n-1}\overline w^{n-1}+...+a_1\overline w+a_0=0.
\]



\end{losning}



\begin{oppgave}
La $z$ og $w$ være to komplekse tall, begge ulik null. Er det mulig at $zw=0$?
\end{oppgave}

\begin{losning}
Nei.

\noindent
Løsning: Skriv tallene på polar form, $z=re^{i\theta}$ og $w=\rho e^{i\phi}$. Nå kan produktet uttrykkes $zw=r\rho e^{i(\theta+\phi)}$, hvor $r \rho>0$. Derfor er produktet forskjellig fra null.

\noindent
Alternativ løsning: Du kan også løse oppgaven ved å skrive $z$ og~$w$ på formen $a+ib$.
\end{losning}



\begin{oppgave}
La $n>0$ være et partall, og la $z$ være en $n$-terot av et reelt tall. Er $\bar{z}$ også en $n$-terot av dette tallet?
\end{oppgave}

\begin{losning}
Ja.

\noindent
Røttene til $r$ fordeles uniformt langs sirkelen om origo med radius $\sqrt[n]{r}$, og $\sqrt[n]{r}$ er en $n$-terot. Vi har sett eksempler på dette tidligere i øvingen.
\end{losning}

\begin{oppgave}
Vis at $\mathbb C$ er et vektorrom.
\end{oppgave}

\begin{oppgave}
Vis vektorrommet som består av alle matriser på formen 
\[
\begin{pmatrix}
a & b \\
-b & a \\
\end{pmatrix}
\]
er isomorft med $\mathbb C$. 
\end{oppgave}


%\begin{oppgave}
%La $\V{u}=\vvv{2}{-5}{1}$ og la $T:\mathbb{R}^3\rightarrow \mathbb{R}^3$ være den ortogonale projeksjonen ned på underrommet utspent av $\V{u}$,
%$$T(\V{x})=\frac{\V{x}\boldsymbol{\cdot} \V{u}}{\|\V{u} \|^2} \V{u}.$$
%
%\begin{punkt}
%Bruk definisjonen av en lineærtransformasjon for å vise at $T$ er en lineærtransformasjon.
%\end{punkt}
%
%\begin{punkt}
%Finn matrisen $A$ slik at $T(\V{x})=A\V{x}$ for alle vektorer $\V{x}$.
%\end{punkt}
%
%\end{oppgave}
%
%\begin{losning}
%
%\begin{punkt}
%Dette følger av at prikkproduktet er lineært.
%\end{punkt}
%
%\begin{punkt}
%Bruk $T$ på standardbasisen til $\mathbb{R}^3$ for å se at
%$$A=\frac{1}{30}\begin{bmatrix}
%4 & -10 & 2\\
%-10 & 25 & -5\\
%2 & -5 & 1
%\end{bmatrix}.$$
%\end{punkt}
%
%\end{losning}
