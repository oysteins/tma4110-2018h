\documentclass[norsk,a4paper,twocolumn,oneside]{memoir}

\usepackage[utf8]{inputenc}
\usepackage{babel}
\usepackage{amsmath,amssymb,amsthm}
\usepackage{mathrsfs}
\usepackage[total={17cm,27cm}]{geometry}
\usepackage[table]{xcolor}
%\usepackage{tabularx}
\usepackage{systeme}
%\usepackage{hyperref}
%\usepackage{enumerate}
\usepackage{ifthen}
\usepackage{textgreek}
\usepackage{multirow}
\usepackage{placeins}

%\usepackage{sectsty}
\setsecheadstyle{\bfseries\large}
%\subsectionfont{\bf\normalsize}

\usepackage{tikz,pgfplots}
\usetikzlibrary{calc}
\usetikzlibrary{arrows.meta}
\def\centerarc[#1](#2)(#3:#4:#5)% Syntax: [draw options] (center) (initial angle:final angle:radius)
    { \draw[#1] ($(#2)+({#5*cos(#3)},{#5*sin(#3)})$) arc (#3:#4:#5); }
\usepackage{pgfornament}

\newcommand{\defterm}[1]{\emph{#1}}

\newcommand{\N}{\mathbb{N}}
\newcommand{\Z}{\mathbb{Z}}
\newcommand{\Q}{\mathbb{Q}}
\newcommand{\R}{\mathbb{R}}

\newcommand{\M}{\mathcal{M}} % vektorrom av matriser
\newcommand{\C}{\mathcal{C}} % vektorrom av kontinuerlige funksjoner
\renewcommand{\P}{\mathcal{P}} % vektorrom av polynomer
\newcommand{\B}{\mathscr{B}} % basis

\renewcommand{\Im}{\operatorname{Im}}
\renewcommand{\Re}{\operatorname{Re}}

\newcommand{\abs}[1]{|#1|}
\newcommand{\intersect}{\cap}
\newcommand{\union}{\cup}
\newcommand{\fcomp}{\circ}
\newcommand{\iso}{\cong}

\newcommand{\roweq}{\sim}
\DeclareMathOperator{\Sp}{Sp}
\DeclareMathOperator{\Null}{Null}
\DeclareMathOperator{\Col}{Col}
\DeclareMathOperator{\Row}{Row}
\DeclareMathOperator{\rank}{rank}
\DeclareMathOperator{\im}{im}
\DeclareMathOperator{\id}{id}
\DeclareMathOperator{\Hom}{Hom}
\newcommand{\tr}{^\top}
\newcommand{\koord}[2]{[\,{#1}\,]_{#2}} % koordinater mhp basis

\newcommand{\V}[1]{\mathbf{#1}}
\newcommand{\vv}[2]{\begin{bmatrix} #1 \\ #2 \end{bmatrix}}
\newcommand{\vvS}[2]{\left[ \begin{smallmatrix} #1 \\ #2 \end{smallmatrix} \right]}
\newcommand{\vvv}[3]{\begin{bmatrix} #1 \\ #2 \\ #3 \end{bmatrix}}
\newcommand{\vvvv}[4]{\begin{bmatrix} #1 \\ #2 \\ #3 \\ #4 \end{bmatrix}}
\newcommand{\vvvvv}[5]{\begin{bmatrix} #1 \\ #2 \\ #3 \\ #4 \\ #5 \end{bmatrix}}
\newcommand{\vn}[2]{\vvvv{#1_1}{#1_2}{\vdots}{#1_#2}}

\newcommand{\e}{\V{e}}
\renewcommand{\u}{\V{u}}
\renewcommand{\v}{\V{v}}
\newcommand{\w}{\V{w}}
\renewcommand{\b}{\V{b}}
\newcommand{\x}{\V{x}}
\newcommand{\0}{\V{0}}

\newenvironment{amatrix}[1]{% "augmented matrix"
  \left[\begin{array}{*{#1}{c}|c}
}{%
  \end{array}\right]
}

\newcommand{\boks}[1]{\framebox{\strut $#1$}}

% \newcounter{notatnr}
% \newcommand{\notatnr}[2]
% {\setcounter{notatnr}{#1}%
%  \setcounter{page}{#2}%
% }

\newtheorem{thm}{Teorem}[chapter]
\newtheorem*{thm-nn}{Teorem}
\newtheorem{cor}[thm]{Korollar}
\newtheorem{lem}[thm]{Lemma}
\newtheorem{prop}[thm]{Proposisjon}
\theoremstyle{definition}
\newtheorem{exx}[thm]{Eksempel}
\newtheorem*{defnx}{Definisjon}
\newtheorem*{oppg}{Oppgave}
\newtheorem*{merkx}{Merk}
\newtheorem*{spmx}{Spørsmål}

\newenvironment{defn}
  {\pushQED{\qed}\renewcommand{\qedsymbol}{$\triangle$}\defnx}
  {\popQED\enddefnx}
\newenvironment{ex}
  {\pushQED{\qed}\renewcommand{\qedsymbol}{$\triangle$}\exx}
  {\popQED\endexx}
\newenvironment{merk}
  {\pushQED{\qed}\renewcommand{\qedsymbol}{$\triangle$}\merkx}
  {\popQED\endmerkx}
\newenvironment{spm}
  {\pushQED{\qed}\renewcommand{\qedsymbol}{$\triangle$}\spmx}
  {\popQED\endspmx}

\setlength{\columnsep}{26pt}

\newcommand{\Tittel}[2]{%
\twocolumn[
\begin{center}
\Large
\begin{tabularx}{\textwidth}{cXr}
\cellcolor{black}\color{white}%
\bf {#1} &
#2
\hfill &
\footnotesize TMA4110 høsten 2018
\\ \hline
\end{tabularx}
\end{center}
]}

\newcommand{\tittel}[1]{\Tittel{\arabic{notatnr}}{#1}}

\newcommand{\linje}{%
\begin{center}
\rule{.8\linewidth}{0.4pt}
\end{center}
}


\newcommand{\kapittelemnenavn}{TMA4110 høsten 2018}
\newcommand{\chapternumber}{}

\makechapterstyle{tma4110}{%
 \renewcommand*{\chapterheadstart}{}
 \renewcommand*{\printchaptername}{}
 \renewcommand*{\chapternamenum}{}
 \renewcommand*{\printchapternum}{\renewcommand{\chapternumber}{\thechapter}}
 \renewcommand*{\afterchapternum}{}
 \renewcommand*{\printchapternonum}{\renewcommand{\chapternumber}{}}
 \renewcommand*{\printchaptertitle}[1]{
\LARGE
\begin{tabularx}{\textwidth}{cXr}
\cellcolor{black}\color{white}%
\textbf{\chapternumber} &
\textbf{##1}
\hfill &
\footnotesize\kapittelemnenavn
\\ \hline
\end{tabularx}%
}
 \renewcommand*{\afterchaptertitle}{\par\nobreak\vskip \afterchapskip}
 % \newcommand{\chapnamefont}{\normalfont\huge\bfseries}
 % \newcommand{\chapnumfont}{\normalfont\huge\bfseries}
 % \newcommand{\chaptitlefont}{\normalfont\Huge\bfseries}
 \setlength{\beforechapskip}{0pt}
 \setlength{\midchapskip}{0pt}
 \setlength{\afterchapskip}{10pt}
}
\chapterstyle{tma4110}
\pagestyle{plain}


\newboolean{vis-oppgaver}
\newboolean{vis-losninger}
\setboolean{vis-oppgaver}{true}
\setboolean{vis-losninger}{false}

\newcounter{oppg-kap} % kapittelnummerering for oppgaver
\newcounter{oppgnr}[oppg-kap]
\newcounter{punktnr}[oppgnr]

\newenvironment{oppgave}%
 {\ifthenelse{\boolean{vis-oppgaver}}%
             {\par\noindent\stepcounter{oppgnr}\textbf{\arabic{oppgnr}.}}%
             {\expandafter\comment}}%
 {\ifthenelse{\boolean{vis-oppgaver}}%
             {\par\bigskip}%
             {\expandafter\endcomment}}

\newenvironment{losning}%
 {\ifthenelse{\boolean{vis-losninger}}%
             {\par\noindent\stepcounter{oppgnr}\textbf{\arabic{oppg-kap}.\arabic{oppgnr}.}}%
             {\expandafter\comment}}%
 {\ifthenelse{\boolean{vis-losninger}}%
             {\par\bigskip}%
             {\expandafter\endcomment}}

\newenvironment{punkt}
 {\par\smallskip\noindent\stepcounter{punktnr}\textbf{\alph{punktnr})} }
 {\par}

\newcommand{\kap}[1]{\setcounter{oppg-kap}{#1}\addtocounter{oppg-kap}{-1}\stepcounter{oppg-kap}}

\newcommand{\oppgaver}[1]{%
  \kap{#1}%
  \ifthenelse{\boolean{vis-oppgaver}}%
             {\linje\section*{Oppgaver}}%
             {}}

\usepackage{tikz}

\begin{document}

\notatnr{2}{3} % TODO sett riktig startsidetall ut fra lengden på forrige notat
\tittel{Lineære likningssystemer}

\noindent%
Grunnlaget for lineær algebra er \emph{lineære liknings\-systemer}.  I
dette notatet starter vi vår reise inn i den lineære algebraen ved å
først se på noen forskjellige måter å løse likningssystemer på, og så
forsikre oss om at vi er helt enige om nøyaktig hva det vil si at et
likningssystem er lineært.

\section*{Forskjellige fremgangsmåter}

Her er et eksempel på et lineært likningssystem:
\[
(\ast)
\systeme{
2x + 3y = 9,
-x + 6y = 3
}
\]
I dette systemet har vi to likninger og to ukjente.
En løsning av systemet består av to tall som vi kan sette inn
for $x$ og~$y$ slik at begge likningene er oppfylt samtidig.

Vi kjenner fra før til flere måter å løse et slikt system på.  La oss
løse systemet over med noen forskjellige metoder.

\begin{ex}
Den kanskje mest åpenbare metoden for å løse et likningssystem er å
først løse én likning med hensyn på én av de ukjente, og så sette inn
i den andre likningen (eller i \emph{de} andre likningene, hvis vi har
et system med mer enn to likninger).

For å løse systemet~$(\ast)$ med denne metoden kan vi først løse den
andre likningen med hensyn på~$x$; da får vi:
\[
x = 6y - 3
\]
Så setter vi dette inn i den første likningen og forenkler:
\begin{align*}
2 \cdot (6y - 3) + 3y &= 9 \\
12y - 6 + 3y &= 9 \\
15y &= 15 \\
y &= 1
\end{align*}
Til slutt setter vi denne $y$-verdien inn i uttrykket vi fant for~$x$,
og får:
\[
x = 6y - 3 = 6 - 3 = 3
\]

Vi har altså funnet ut at for at begge likningene skal være oppfylt,
må vi ha at $x = 3$ og $y = 1$.  Vi sjekker at dette virkelig er en
løsning av~$(\ast)$ ved å sette inn disse verdiene i begge likningene
i systemet:
\begin{align*}
2x + 3y &= 2 \cdot 3 + 3 \cdot 1 = 6 + 3 = 9 \\
-x + 6y &= -3 + 6 \cdot 1 = 3
\end{align*}
Vi har nå funnet ut at systemet~$(\ast)$ har nøyaktig én løsning,
nemlig $x = 3$ og $y = 1$.
\end{ex}

% Metoden i eksempelet over er enkel og grei, men kan bli temmelig
% tungvint å bruke hvis vi har mer enn to ukjente.  Vi 

Vi ser på en annen løsningsmetode for det samme systemet:

\begin{ex}
Vi ganger opp den andre likningen med~$2$, og deretter legger vi
sammen de to likningene:
\begin{align*}
2x + 3y &= 9   &&\text{(første likning)} \\
-2x + 12y &= 6 &&\text{(andre likning ganget med~$2$)} \\[4pt] \hline
\rule{0pt}{11pt}
15y &= 15      &&\text{(sum av likningene over)}
\end{align*}
På denne måten får vi $x$ til å forsvinne, og vi står igjen med en
likning med bare~$y$.

Den nye likningen $15y = 15$ kan vi forenkle til $y=1$.  Nå kan vi
gange opp denne med $-3$ og legge sammen med den første likningen fra
systemet for å få en likning der $y$ forsvinner:
\begin{align*}
2x + 3y &= 9 &&\text{(første likning fra $(\ast)$)}\\
-3y &= -3    &&\text{(ny likning ganget med $-3$)} \\[4pt] \hline
\rule{0pt}{11pt}
2x &= 6      &&\text{(sum av likningene over)}
\end{align*}
Nå har vi fått en likning med bare~$x$, og vi forenkler den til
$x = 3$.
Vi har dermed igjen funnet løsningen $x = 3$ og $y = 6$.
\end{ex}



% Når vi (i neste notat) skal beskrive en 

\begin{enumerate}
\item Gange en likning med et tall (ikke~$0$).
\item Legge sammen 
\end{enumerate}

\begin{ex}
Vi kan også løse systemet~$(\ast)$ grafisk.  Vi lager et
koordinatsystem med en $x$-akse og en $y$-akse.  Hver av de to
likningene $2x+3y=9$ og $-x+6y=3$ beskriver en rett linje:
\begin{center}
\begin{tikzpicture}[scale=.9]
%\clip (-0.6,-0.2) rectangle (0.6,1.51);
%\draw[step=1cm,help lines,gray] (-3.5,-3.5) grid (3.5,3.5);
%\filldraw[fill=green!20,draw=green!50!black] (0,0) -- (3mm,0mm)
%arc [start angle=0, end angle=30, radius=3mm] -- cycle;
\draw[->] (-1.5,0) -- (6.5,0);
\draw[->] (0,-1.5) -- (0,5.5);
%\draw (0,0) circle [radius=1cm];
\foreach \x in {-1,1,2,3,4,5,6}
\draw (\x cm,1pt) -- (\x cm,-1pt) node[anchor=north] {$\x$};
\foreach \y in {-1,1,2,3,4,5}
\draw (1pt,\y cm) -- (-1pt,\y cm) node[anchor=east] {$\y$};
\draw (-1.5,4) -- (6,-1);
\node[anchor=west] at (.4,3) {$2x + 3y = 9$};
\draw (-1.5,0.25) -- (6,1.5);
\node at (6,2) {$-x + 6y = 3$};
\filldraw (3,1) circle [radius=2pt];
\end{tikzpicture}
\end{center}
Løsningene av $2x + 3y = 9$ er altså alle punkter som ligger på den
ene linjen, mens løsningene av $-x + 6y = 3$ er alle punkter som
ligger på den andre linjen.  Den felles løsningen av begge likningene
er punktet der de to linjene møtes, nemlig $(3,1)$.  Løsningen er
altså $x = 3$ og $y = 1$.

Denne metoden er fin for å visualisere problemet og se løsningen på en
intuitiv måte, men ikke nødvendigvis den beste for å finne svaret
eksakt.  Dessuten blir det vanskelig å tegne hvis vi har mer enn to
ukjente (men det kan likevel være nyttig å prøve å se for seg
løsningene av for eksempel en likning med tre ukjente på en grafisk
måte).
\end{ex}


\section*{Hva er et lineært likningssystem?}

Et lineært likningssystem er et system av lineære likninger.  Men
nøyaktig hva mener vi med at en likning er lineær?

Ordet «lineær» kommer fra det latinske «linea», som betyr «linje».
Hvis vi har en likning med to ukjente, så kan vi tegne grafen til
denne likningen.  Vi sier at likningen er lineær hvis grafen dens er
en rett linje.  I eksempelet over så vi at grafene til likningene
$2x + 3y = 9$ og $-x + 6y = 3$ er rette linjer.  Men det er også lett
å finne likninger som er slik at grafen ikke blir en rett linje, for
eksempel $y = x^2$ eller $x^2 + y^2 = 4$.

Generelt er det slik at grafen til en likning er en rett linje hvis og
bare hvis likningen kan skrives på formen
\[
ax + by = c,
\]
der $a$, $b$ og~$c$ er konstanter.

Når vi vil se på likninger med mer enn to ukjente, gir det ikke lenger
mening å snakke om at grafen blir en rett linje.  Men det at likningen
kan skrives på formen $ax + by = c$ kan vi lett utvide til å ta med
flere ukjente, så det er dette vi bruker i den generelle definisjonen
av lineære likninger.

\begin{defn}
En likning med $n$ ukjente $x_1$, $x_2$, \ldots, $x_n$ kalles
\defterm{lineær} dersom den kan skrives på formen
\[
a_1 x_1 + a_2 x_2 + \cdots a_n x_n = b,
\]
der $a_1$, $a_2$, \ldots, $a_n$ og~$b$ er konstanter.  Et
\defterm{lineært likningssystem} er en samling av én eller flere
lineære likninger med de samme ukjente.
\end{defn}

% \begin{merk}
% Hvis vi har bare to eller tre ukjente i likningene våre, kaller vi dem
% gjerne $x$, $y$ og~$z$.  Hvis vi har mange ukjente, blir det tungvint
% å skulle finne en ny bokstav for hver av dem, så da kan vi for
% eksempel kalle dem $x_1$, $x_2$, $x_3$ og så videre, som i
% definisjonen over.  Men det spiller egentlig ingen rolle hva de ukjente heter.
% \end{merk}

\begin{ex}
Her er noen eksempler på lineære likninger:
\begin{align*}
5 x_1 + x_2 - 7 x_3 + x_4 &= 27 \\
\frac{2}{5} x - 5y &= \pi
\end{align*}
Og her er noen eksempler på likninger som ikke er lineære:
\begin{align*}
2 x^2 + y &= 3 \\
x_1 + 2 x_1 x_2 + x_3 &= 1
\qedhere
\end{align*}
\end{ex}


\section*{Ekvivalente systemer}

\begin{ex}
La oss løse det følgende lineære likningssystemet:
\[
\systeme{
  x + 0y + 2z = 21,
  -2x + 3y - 4z = -47,
  3x + 9y + 7z = 52
}
\]
Vi vil begynne med å eliminere $x$-en fra de to siste likningene.
Hvis vi ganger første likning med~$2$, får vi $2x + 4z = 42$, og hvis
vi legger til dette i andre likning, får vi den nye likningen
\[
3y = -5.
\]
Vi bytter ut den andre likningen i systemet med denne nye likningen:
\[
\systeme{
  x + 2z = 21,
  3y = -5,
  3x + 9y + 7z = 52
}
\]
På samme måte kan vi eliminere $x$ fra den siste likningen ved å
trekke fra første likning ganget med~$3$:
\[
\systeme{
  x + 2z = 21,
  3y = -5,
  9y + z = -11
}
\]
Videre kan vi eliminere $y$ fra den siste likningen ved å trekke fra
andre likning ganget med~$3$:
\[
\systeme{
  x + 2z = 21,
  3y = -5,
  z = 4
}
\]


\section*{Totalmatrisen til et system}

Generelt kan et lineært likningssystem (med $m$ likninger og $n$
ukjente) se slik ut:
\[
\left\{
\begin{aligned}
  a_{11} x_1 + a_{12} x_2 + \cdots + a_{1n} x_n &= b_1 \\
  a_{21} x_1 + a_{22} x_2 + \cdots + a_{2n} x_n &= b_2 \\
                                                &\ \ \vdots \\
  a_{m1} x_1 + a_{m2} x_2 + \cdots + a_{mn} x_n &= b_m
\end{aligned}
\right.
\]
Når vi skal løse et slikt system, vil vi skrive opp en rekke nye
systemer som er ekvivalent med dette, men som er stadig enklere.  Da
er det unødvendig tungvint å skrive alle de ukjente og alle $+$-ene og
$=$-tegnene hver eneste gang.  Den eneste informasjonen vi trenger å
ha med oss gjennom utregningen er hva 

\end{document}
