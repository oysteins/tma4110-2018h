\begin{oppgave}
Følgende tall er hentet fra SSB sin statistikk om folketall ved inngangen av 2. kvartal 2018:

\begin{table}[h]
\begin{tabular}{lll}
           & Trondheim & Steinkjer \\
Folketall & 194128    & 22081     \\
Utflytting & 22081     & 209      
\end{tabular}
\end{table}

\begin{punkt}
I en enkel modell antar vi at utflyttinger ikke endres fra år til år og at all utflytting fra Trondheim skjer til Steinkjer og vice versa. Gi en Markov-prosess som beskriver denne modellen.
\end{punkt}

\begin{punkt}
Hva er folketallet i Trondheim og Steinkjer i 2020 (under denne modellen)?
\end{punkt}

\begin{punkt}
Finnes det en likevektsvektor for denne prosessen? Hva er folketallet i Trondheim og Steinkjer langt inn i fremtiden?
\end{punkt}

\begin{punkt}
Virker modellen rimelig basert på svaret i del \textbf{c)}?
\end{punkt}

\end{oppgave}

\begin{losning}


\begin{punkt}
Modellen er beskrevet av den stokastiske matrisen $$P=\begin{bmatrix}
\frac{191706}{194128} & \frac{209}{22081}\\
\frac{2422}{194128} & \frac{21872}{22081}\\
\end{bmatrix}$$
\end{punkt}

\begin{punkt}
$\V{x}_2=P^2\V{x}_0\simeq \vv{189751}{26458.4}$.
\end{punkt}

\begin{punkt}
Ja, den finnes fordi vi har en regulær stokastisk matrise. Ved å bruke likevektsvektoren finner vi at folketallet er ca. 93262.7 i Trondheim og 122940 i Steinkjer.
\end{punkt}

\begin{punkt}
Nei: Det er mindre folk i Trondheim enn i Steinkjer.
\end{punkt}

\end{losning}

\begin{oppgave}
\begin{punkt}
La $P$ og $Q$ være stokastiske $2\times 2$-matriser. Vis at produktet $PQ$ er stokastisk.

\noindent
Hint: Skriv $P$ og $Q$ på elementform. Bruk dette for å uttrykke elementene til $PQ$ ved elementene til $P$ og $Q$. Hva må du vise at produktet tilfredsstiller? Hva vet vi basert på antagelsen?

\end{punkt}
\begin{punkt}
Kan man mer generelt si at produktet av to stokastiske $n\times n$-matriser er stokastisk?
\end{punkt}
\end{oppgave}

\begin{losning}

\begin{punkt}
Notasjon: $$P=\begin{bmatrix}
p_{11} & p_{12}\\
p_{21} & p_{22}
\end{bmatrix},\quad Q=\begin{bmatrix}
q_{11} & q_{12}\\
q_{21} & q_{22}
\end{bmatrix}$$

Produktet er $$PQ=\begin{bmatrix}
p_{11}q_{11}+p_{12}q_{21} & p_{11}q_{21}+p_{12}q_{22}\\
p_{21}q_{11}+p_{22}q_{21} & p_{21}q_{21}+p_{22}q_{22}
\end{bmatrix}$$
\end{punkt}

Vi må vise at sumen av elementene i kolonne 1 og 2 i $PQ$ er lik 1. Første kolonne: $$p_{11}q_{11}+p_{12}q_{21}+p_{21}q_{11}+p_{22}q_{21}=q_{11}(p_{11}+p_{21})+q_{21}(p_{12}+p_{22}).$$ Summene inni parantesene er begge lik en fordi $P$ er stokastisk. Derfor er altså summen av første kolonne $q_{11}+q_{21}$ som også er lik en siden $Q$ er stokastisk. Samme type argument kan brukes for andre kolonne.

\begin{punkt}
Ja. Samme triks som er illustrert ovenfor fungerer for $n\times n$-matriser (ved å faktorisere ut elementene fra $Q$ i hver sum bruker du at $P$ er stokastisk til å få en sum av en kolonne i $Q$ som også er lik en). 
\end{punkt}

\end{losning}



