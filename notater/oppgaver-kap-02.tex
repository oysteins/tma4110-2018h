% -*- TeX-master: "oving01"; -*-
\oppgaver{2}

\begin{oppgave}
Hvilke av disse matrisene er på trappeform?  Hvilke av dem er på
redusert trappeform?
\begin{punkt}
$
\begin{bmatrix}
1 & 5 & 0 & 0 \\
0 & 0 & 0 & 1
\end{bmatrix}
$
\end{punkt}
\begin{punkt}
$
\begin{bmatrix}
1 & 0 \\
0 & 1 \\
0 & -1
\end{bmatrix}
$
\end{punkt}
\begin{punkt}
$
\begin{bmatrix}
0 & 2 & 1 \\
0 & 0 & 4 \\
0 & 0 & 0
\end{bmatrix}
$
\end{punkt}
\begin{punkt}
$
\begin{bmatrix}
0 & 0 & 0 \\
0 & 0 & 0 \\
0 & 0 & 0
\end{bmatrix}
$
\end{punkt}
\end{oppgave}
\begin{losning}
Matrise (a), (b) og (d) er på trappeform; (a) er på redusert trappeform.
\end{losning}

\begin{oppgave}
Løs likningssystemene.
\begin{punkt}
$
\systeme{
  2x - 4y + 9z = -38,
  4x - 3y + 8z = -26,
 -2x + 4y - 2z =  17
}
$
\end{punkt}
\begin{punkt}
% 1 0 0 4
% 0 1 2 0
% 0 0 0 0
% ---
% 1 3 6 4
% 0 2 4 0
% 0 0 0 0
% ---
% 1 3  6 4
% 2 8 16 8
% 0 0  0 0
% ---
% 1 3  6 4
% 2 8 16 8
% 2 6 12 8
$
\systeme{
  x + 3y +  6z = 4,
 2x + 8y + 16z = 8,
 2x + 6y + 12z = 8
}
$
\end{punkt}
\begin{punkt}
%TODO: ingen løsninger
$
\systeme{
	x  + 2y - z = 1,
	2x + 3y - z = -1,
	3x + 4y - z = 1
}
$
\end{punkt}
\end{oppgave}

\begin{losning}
\begin{punkt}
$x=\frac{5}{2}$, $y=4$ og $z=-3$.
\end{punkt}
\begin{punkt}
Radredusering viser at systemet har en fri variabel. Dette kan også observeres direkte ettersom siste likning er første likning multiplisert med to. Husk at vi nå har mange valg for å løse oppgaven, og vi gir derfor bare en skisse til hvordan du kan komme frem til et endelig svar. Eksempelvis kan vi ende opp med det ekvivalente systemet 
$$
\systeme{
	x + 3y +  6z = 4,
	    y  +  2z = 0
}.\\
$$
Nå kan vi velge $y$ eller $z$ som fri variabel. Prøv gjerne ulike valg. Du kan sjekke om løsningene dine er korrekt ved å sette inn i likningene du opprinnelig skulle løse.
\end{punkt}
\begin{punkt}
Det finnes ingen løsning.
\end{punkt}
\end{losning}


\begin{oppgave}
Er følgende to likningssystemer ekvivalente? Begrunn svaret ditt.
\[
\systeme*{
	x = 1,
	x + y = 2,
	x + y + z = 3
}
\qquad\qquad
\systeme*{
	x = 1,
	y = 1,
	z = 1
}
\]
\end{oppgave}

\begin{losning}
Likningssystemene er ekvivalente fordi begge har entydig løsning $x=1$, $y=1$ og $z=1$.
\end{losning}


\begin{oppgave}
Er følgende to matriser radekvivalente? Begrunn svaret ditt.
\[
\begin{amatrix}{3}
1 & 0 & 0 & 1\\
1 & 1 & 0 & 1\\
1 & 1 & 1 & 1
\end{amatrix}
\qquad\qquad
\begin{amatrix}{3}
0 & 0 & 1 & 1\\
0 & 1 & 1 & 1\\
1 & 1 & 1 & 1
\end{amatrix}
\]
\end{oppgave}

\begin{losning}
Hint: Teorem \ref{thm:radekvivalens} sier at radekvivalente
totalmatriser alltid er ekvivalente som likningssystem, de har altså
like løsninger. Det holder derfor å vise at de korresponderende
likningssystemene har ulike løsninger.

Merk: Vi ser at den eneste forskjellen på matrisene er at første- og
tredje kolonne har byttet plass. Dette svarer til at første- og siste
variabel bytter plass i korresponderende likningssystem. Kan du
forklare dette?
\end{losning}



\begin{oppgave}
La $(1,2)$, $(2,3)$ og $(3,5)$ være tre punkter i planet. Vi skal
finne et andregradspolynom $ax^2 + bx + c$ slik at grafen går gjennom
de tre punktene.
\begin{center}
\begin{tikzpicture}[scale=.75]
\draw[->] (-1.5,0) -- (5.6,0);
\draw[->] (0,-0.5) -- (0,5.5);
\foreach \x in {-1,1,2,3,4,5}
\draw (\x cm,1pt) -- (\x cm,-1pt) node[anchor=north] {$\x$};
\foreach \y in {1,2,3,4,5}
\draw (1pt,\y cm) -- (-1pt,\y cm) node[anchor=east] {$\y$};
\filldraw (1,2) circle [radius=2pt] node[anchor=west] {$(1,2)$};
\filldraw (2,3) circle [radius=2pt] node[anchor=west] {$(2,3)$};
\filldraw (3,5) circle [radius=2pt] node[anchor=west] {$(3,5)$};
\end{tikzpicture}
\vspace{-5pt}
\end{center}
\begin{punkt}
Sett opp et lineært likningssystem for $a$, $b$ og~$c$.
\end{punkt}
\begin{punkt}
Løs systemet, og finn andregradspolynomet som går gjennom alle punktene.
\end{punkt}
\begin{punkt}
Sjekk at svaret ditt i \textbf{b)} er riktig.
\end{punkt}
\end{oppgave}

\begin{losning}
	\begin{punkt}
		Kravene $p(1)=2$, $p(2)=3$ og $p(3)=5$ gir følgende likningssystem:
		$$
		\systeme{
			a+b+c = 2,
			4a+2b+c = 3,
			9a+3b+c = 5
		}
		$$
		
	\end{punkt}
	\begin{punkt}
		Løsningen er $a=\frac{1}{2}$, $b=-\frac{1}{2}$ og~$c=2$.
	\end{punkt}
	
	\begin{punkt}
		Sett inn $1$, $2$ og~$3$ i polynomet $p(x)=\frac{1}{2}x^2-\frac{1}{2}x+2$ for å se at kravene i \textbf{a)} er oppfylt.  
	\end{punkt}
\end{losning}


\begin{oppgave}
Anta at vi har et likningssystem med $m$~likninger og~$n$ ukjente.
Hvilke av de ni forskjellige tilfellene i følgende tabell er mulige?
\[
\begin{array}{r|c|c|c|}
                                & m < n & m = n & m > n \\ \hline
\text{ingen løsninger}          &       &       &       \\ \hline
\text{én løsning}               &       &       &       \\ \hline
\text{uendelig mange løsninger} &       &       &       \\ \hline
\end{array}
\]
\end{oppgave}

\begin{losning}

La 1 betegne mulig og 0 umulig:
\[
\begin{array}{r|c|c|c|}
& m < n & m = n & m > n \\ \hline
\text{ingen løsninger}          &   1   &   1   &   1   \\ \hline
\text{én løsning}               &   0   &   1   &   1   \\ \hline
\text{uendelig mange løsninger} &   1   &   1   &   1   \\ \hline
\end{array}
\]
\end{losning}


\begin{oppgave}
Se på likningssystemet
\[
\systeme[xy]{
  ax + by = m,
  cx + dy = n
}
\]
der $a$, $b$, $c$, $d$, $m$ og~$n$ er konstanter, og vi antar at $ad \ne bc$.

Hvor mange løsninger har systemet?  Finn løsningen(e) uttrykt ved $a$,
$b$, $c$, $d$, $m$ og~$n$.

Hint: Start med å (i) multiplisere første rad med $d$ og andre rad med
$b$, eller (ii) multiplisere første rad med $c$ og andre rad med
$a$. Ta hensyn til at noen variabler kan være null.
\end{oppgave}
\begin{losning}
Løsningen er

$$x=\frac{1}{ad-bc}(dm-bn)$$
$$y=\frac{1}{ad-bc}(an-cm).$$

Merk at $ad-bc\neq 0$ ettersom vi har antatt $ad\neq bc$.

\end{losning}



\begin{oppgave}
Vis at følgende påstander er sanne for alle matriser $M$, $N$ og~$L$:
\begin{punkt}
$M \roweq M$.
\end{punkt}
\begin{punkt}
Hvis $M \roweq N$, så: $N \roweq M$.
\end{punkt}
\begin{punkt}
Hvis $M \roweq L$ og $L \roweq N$, så: $M \roweq N$.
\end{punkt}
\end{oppgave}

\begin{losning}
\begin{punkt}
	$M$ er trivielt radekvivalent med $M$.
\end{punkt}
\begin{punkt}
	Hint: Alle radoperasjoner er reversible; multiplisere en rad med et ikke-null tall $c$ kan reverseres ved å multiplisere samme rad med $\frac{1}{c}$; bytte om på to rader kan reverseres ved å bytte om på radene igjen; å legge til et multiplum av en rad til en annen kan reverseres ved å trekke fra det som ble lagt til. Dersom $M\roweq N$ betyr det at vi har gjort et endelig antall radoperasjoner $\text{O}_1,\dots,\text{O}_n$ for å lage $N$ fra $M$. Klarer du, ut ifra dette, å finne et endelig antall radoperasjoner som lager $M$ fra $N$?
\end{punkt}
\begin{punkt}
	Hint: Vi antar at det finnes et endelig antall radoperasjoner $\text{O}_1,\dots,\text{O}_n$ som lager $L$ fra $M$, og at det finnes et endelig antall radoperasjoner $\text{O}_{n+1},\dots,\text{O}_{n+m}$ som lager $N$ fra $L$. Klarer du, ut ifra dette, å finne et endelig antall radoperasjoner som lager $N$ fra $M$?
\end{punkt}
\end{losning}
