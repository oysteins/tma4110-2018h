% -*- TeX-master: "oving01"; -*-
\oppgaver{2}

\begin{oppgave}
Hvilke av disse matrisene er på trappeform?  Hvilke av dem er på
redusert trappeform?
\begin{punkt}
$
\begin{bmatrix}
1 & 5 & 0 & 0 \\
0 & 0 & 0 & 1
\end{bmatrix}
$
\end{punkt}
\begin{punkt}
$
\begin{bmatrix}
1 & 0 \\
0 & 1 \\
0 & -1
\end{bmatrix}
$
\end{punkt}
\begin{punkt}
$
\begin{bmatrix}
0 & 2 & 1 \\
0 & 0 & 4 \\
0 & 0 & 0
\end{bmatrix}
$
\end{punkt}
\begin{punkt}
$
\begin{bmatrix}
0 & 0 & 0 \\
0 & 0 & 0 \\
0 & 0 & 0
\end{bmatrix}
$
\end{punkt}
\end{oppgave}


\begin{oppgave}
Løs likningssystemene.
\begin{punkt}
$
\systeme{
  2x - 4y + 9z = -38,
  4x - 3y + 8z = -26,
 -2x + 4y - 2z =  17
}
$
\end{punkt}
\begin{punkt}
% 1 0 0 4
% 0 1 2 0
% 0 0 0 0
% ---
% 1 3 6 4
% 0 2 4 0
% 0 0 0 0
% ---
% 1 3  6 4
% 2 8 16 8
% 0 0  0 0
% ---
% 1 3  6 4
% 2 8 16 8
% 2 6 12 8
$
\systeme{
  x + 3y +  6z = 4,
 2x + 8y + 16z = 8,
 2x + 6y + 12z = 8
}
$
\end{punkt}
\begin{punkt}
(TODO: system med ingen løsninger)
\end{punkt}
\end{oppgave}


\begin{oppgave}
\begin{punkt}
Er disse to likningssystemene ekvivalente?
(TODO: to likningssystemer som er ekvivalente)
\end{punkt}
\begin{punkt}
Er disse to matrisene radekvivalente?
(TODO: to totalmatriser som ikke er radekvivalente)
\end{punkt}
\end{oppgave}


\begin{oppgave}
Anta at vi har et likningssystem med $m$~likninger og~$n$ ukjente.
Hvilke av de ni forskjellige tilfellene i følgende tabell er mulige?
\[
\begin{array}{r|c|c|c|}
                                & m < n & m = n & m > n \\ \hline
\text{ingen løsninger}          &       &       &       \\ \hline
\text{én løsning}               &       &       &       \\ \hline
\text{uendelig mange løsninger} &       &       &       \\ \hline
\end{array}
\]
\end{oppgave}


\begin{oppgave}
Se på likningssystemet
\[
\systeme[xy]{
  ax + by = m,
  cx + dy = n
}
\]
der $a$, $b$, $c$, $d$, $m$ og~$n$ er konstanter, og vi antar at $ad \ne bc$.

Hvor mange løsninger har systemet?  Finn løsningen(e) uttrykt ved $a$,
$b$, $c$, $d$, $m$ og~$n$.
\end{oppgave}


\begin{oppgave}
(TODO: velg tre passende punkter og skriv ferdig oppgaveteksten)

tre punkter i planet, vil finne andregradspolynom $ax^2 + bx + c$ slik
at grafen går gjennom de tre punktene
\begin{punkt}
Sett opp et lineært likningssystem for $a$, $b$ og~$c$.
\end{punkt}
\begin{punkt}
Løs systemet, og finn andregradspolynomet som går gjennom alle punktene.
\end{punkt}
\end{oppgave}


\begin{oppgave}
Vis at følgende påstander er sanne for alle matriser $M$, $N$ og~$L$:
\begin{punkt}
$M \roweq M$.
\end{punkt}
\begin{punkt}
Hvis $M \roweq N$, så: $N \roweq M$.
\end{punkt}
\begin{punkt}
Hvis $M \roweq L$ og $L \roweq N$, så: $M \roweq N$.
\end{punkt}
\end{oppgave}
