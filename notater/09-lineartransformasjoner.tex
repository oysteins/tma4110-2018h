\ifx\inkludert\undefined
\documentclass[norsk,a4paper,twocolumn,oneside]{memoir}

\usepackage[utf8]{inputenc}
\usepackage{babel}
\usepackage{amsmath,amssymb,amsthm}
\usepackage[total={17cm,27cm}]{geometry}
\usepackage[table]{xcolor}
%\usepackage{tabularx}
\usepackage{systeme}
%\usepackage{hyperref}
%\usepackage{enumerate}

%\usepackage{sectsty}
\setsecheadstyle{\bfseries\large}
%\subsectionfont{\bf\normalsize}

\usepackage{tikz}
\usetikzlibrary{arrows.meta}

\newcommand{\defterm}[1]{\emph{#1}}

\newcommand{\N}{\mathbb{N}}
\newcommand{\Z}{\mathbb{Z}}
\newcommand{\Q}{\mathbb{Q}}
\newcommand{\R}{\mathbb{R}}

\newcommand{\abs}[1]{|#1|}

\newcommand{\roweq}{\sim}
\DeclareMathOperator{\Span}{Span}

\newcommand{\V}[1]{\mathbf{#1}}
\newcommand{\vv}[2]{\begin{bmatrix} #1 \\ #2 \end{bmatrix}}
\newcommand{\vvv}[3]{\begin{bmatrix} #1 \\ #2 \\ #3 \end{bmatrix}}
\newcommand{\vvvv}[4]{\begin{bmatrix} #1 \\ #2 \\ #3 \\ #4 \end{bmatrix}}
\newcommand{\vn}[2]{\vvvv{#1_1}{#1_2}{\vdots}{#1_#2}}

\newenvironment{amatrix}[1]{% "augmented matrix"
  \left[\begin{array}{*{#1}{c}|c}
}{%
  \end{array}\right]
}

% \newcounter{notatnr}
% \newcommand{\notatnr}[2]
% {\setcounter{notatnr}{#1}%
%  \setcounter{page}{#2}%
% }

\newtheorem{thm}{Teorem}[chapter]
\newtheorem*{thm-nn}{Teorem}
\newtheorem{cor}[thm]{Korollar}
\newtheorem{lem}[thm]{Lemma}
\newtheorem{prop}[thm]{Proposisjon}
\theoremstyle{definition}
\newtheorem{exx}[thm]{Eksempel}
\newtheorem*{defnx}{Definisjon}
\newtheorem*{oppg}{Oppgave}
\newtheorem*{merkx}{Merk}
\newtheorem*{spmx}{Spørsmål}

\newenvironment{defn}
  {\pushQED{\qed}\renewcommand{\qedsymbol}{$\triangle$}\defnx}
  {\popQED\enddefnx}
\newenvironment{ex}
  {\pushQED{\qed}\renewcommand{\qedsymbol}{$\triangle$}\exx}
  {\popQED\endexx}
\newenvironment{merk}
  {\pushQED{\qed}\renewcommand{\qedsymbol}{$\triangle$}\merkx}
  {\popQED\endmerkx}
\newenvironment{spm}
  {\pushQED{\qed}\renewcommand{\qedsymbol}{$\triangle$}\spmx}
  {\popQED\endspmx}

\setlength{\columnsep}{26pt}

\newcommand{\Tittel}[2]{%
\twocolumn[
\begin{center}
\Large
\begin{tabularx}{\textwidth}{cXr}
\cellcolor{black}\color{white}%
\bf {#1} &
#2
\hfill &
\footnotesize TMA4110 høsten 2018
\\ \hline
\end{tabularx}
\end{center}
]}

\newcommand{\tittel}[1]{\Tittel{\arabic{notatnr}}{#1}}

\newcommand{\linje}{%
\begin{center}
\rule{.8\linewidth}{0.4pt}
\end{center}
}


\newcommand{\chapternumber}{}

\makechapterstyle{tma4110}{%
 \renewcommand*{\chapterheadstart}{}
 \renewcommand*{\printchaptername}{}
 \renewcommand*{\chapternamenum}{}
 \renewcommand*{\printchapternum}{\renewcommand{\chapternumber}{\thechapter}}
 \renewcommand*{\afterchapternum}{}
 \renewcommand*{\printchapternonum}{\renewcommand{\chapternumber}{}}
 \renewcommand*{\printchaptertitle}[1]{
\LARGE
\begin{tabularx}{\textwidth}{cXr}
\cellcolor{black}\color{white}%
\textbf{\chapternumber} &
\textbf{##1}
\hfill &
%\footnotesize TMA4110 høsten 2018
\\ \hline
\end{tabularx}%
}
 \renewcommand*{\afterchaptertitle}{\par\nobreak\vskip \afterchapskip}
 % \newcommand{\chapnamefont}{\normalfont\huge\bfseries}
 % \newcommand{\chapnumfont}{\normalfont\huge\bfseries}
 % \newcommand{\chaptitlefont}{\normalfont\Huge\bfseries}
 \setlength{\beforechapskip}{0pt}
 \setlength{\midchapskip}{0pt}
 \setlength{\afterchapskip}{10pt}
}


\newcounter{oppgnr}[chapter]
\newcounter{punktnr}[oppgnr]
\newenvironment{oppgave}
 {\par\noindent\stepcounter{oppgnr}\textbf{{\arabic{oppgnr}}.}}
 {\par\bigskip}
\newenvironment{punkt}
 {\par\smallskip\noindent\stepcounter{punktnr}\textbf{\alph{punktnr})} }
 {\par}

\newcommand{\oppgaver}{\linje\section*{Oppgaver}}

\usepackage{xr}
\externaldocument{tma4110-2018h}
\newcommand{\kapittel}[2]{\setcounter{chapter}{#1}\addtocounter{chapter}{-1}\chapter{#2}}
\newcommand{\kapittelslutt}{\enddocument}
\begin{document}
\chapterstyle{tma4110}
\pagestyle{plain}
\fi


\kapittel{9}{Lineærtransformasjoner}
\label{ch:lineartransformasjoner}

I forrige kapittel begynte vi å formulere lineær algebra på en
generell måte, ved å gi en abstrakt definisjon av vektorrom.  For å
beskrive sammenhenger mellom forskjellige vektorer og vektorrom
trenger vi også å se på funksjoner som tar inn vektorer og gir ut
vektorer.  For to vektorrom $V$ og~$W$ er vi interessert i de
funksjonene fra $V$ til~$W$ som \emph{bevarer vektorromsstrukturen}.
Slike funksjoner kaller vi lineærtransformasjoner.


\section*{Funksjoner}

Siden lineærtransformasjoner er en spesiell type funksjoner, begynner
vi med å ta en gjennomgang av en del generelle ting om funksjoner som
vi kommer til å få bruk for.  Aller først definerer vi presist hva en
funksjon er for noe.  En \defterm{funksjon} består av tre ting:
\begin{enumerate}
\item En mengde som kalles funksjonens \defterm{domene}.
\item En mengde som kalles funksjonens \defterm{kodomene}.
\item En regel som til hvert element i domenet tilordner et element i
kodomenet.
\end{enumerate}
Vi bruker notasjonen $f \colon A \to B$ for å angi at $f$ er en
funksjon med mengden $A$ som domene og mengden $B$ som kodomene.

(Mengdene som er tilknyttet en funksjon er også kjent under andre
navn.  Domenet kan også kalles \emph{definisjonsmengden} til
funksjonen, og kodomenet kan også kalles \emph{verdimengden}.)

Fra før er du antagelig mest vant til funksjoner som har mengden~$\R$
av reelle tall som både domene og kodomene, og der regelen for
funksjonen er gitt ved et aritmetisk uttrykk.  Her er et eksempel på
en slik funksjon.

\begin{ex}
\label{ex:funksjon1}
La $f \colon \R \to \R$ være funksjonen definert ved
\[
f(x) = 3x^2 + 1
\quad\text{for alle $x$ i~$\R$.}
\]
Da har vi for eksempel at
\[
f(5) = 3 \cdot 5^2 + 1 = 76.
\]
Vi kan tegne grafen til funksjonen:
\begin{center}
\begin{tikzpicture}[scale=.5]
\begin{axis}[
axis y line=center,
axis x line=middle,
% axis equal,
%grid=both,
xmax=9,xmin=-9,
ymin=-2,ymax=78,
% xlabel=$x$,ylabel=$y$,
% xtick={-10,...,10},
% ytick={-10,...,10},
width=15cm,
anchor=center,
]
\addplot[black] {3*x^2 + 1} ;
\end{axis}
% \draw[->] (-5,0) -- (5,0);
% \draw[->] (0,-3) -- (0,100);
% \addplot {3*x^2 + 1};
\end{tikzpicture}
\qedhere
\end{center}
\end{ex}

Men det er verdt å merke seg at både domenet og kodomenet kan være
hvilke som helst mengder, og regelen kan vi spesifisere akkurat slik
vi vil.

\begin{ex}
\label{ex:funksjon2}
La $A$ være mengden bestående av de tre fruktene eple, banan og
ananas, og la $B$ være mengden bestående av heltallene $0$, $1$,
\ldots, $5$:
\begin{align*}
A &= \{ \text{eple}, \text{banan}, \text{ananas} \} \\
B &= \{ 0, 1, 2, 3, 4, 5 \}
\end{align*}
Nå kan vi lage en funksjon
\[
g \colon A \to B
\]
ved å bestemme hva hvert av de tre elementene i mengden~$A$ skal
sendes til.  Hvis vi for eksempel bestemmer at
\begin{align*}
g(\text{eple}) &= 4, \\
g(\text{banan}) &= 3, \\
g(\text{ananas}) &= 0,
\end{align*}
så har vi beskrevet funksjonen~$g$ fullstendig.
\end{ex}

\begin{ex}
\label{ex:funksjon3}
Vi definerer en funksjon
\[
h \colon \R^3 \to \R^2
\]
ved regelen
\[
h\left( \vvv{x_1}{x_2}{x_3} \right)
= \vv{x_1 + x_2}{x_2 + x_3}.
\qedhere
\]
\end{ex}

% TODO tekst

\begin{defn}
La $f \colon A \to B$ være en funksjon.

Vi sier at $f$ er \defterm{injektiv} (eller \defterm{en-til-en}) hvis
det for hver $b$ i~$B$ er maksimalt én $a$ i~$A$ slik at $f(a) = b$.

Vi sier at $f$ er \defterm{surjektiv} (eller \defterm{på}) hvis
det for hver $b$ i~$B$ finnes en $a$ i~$A$ slik at $f(a) = b$.

\defterm{Bildet} til~$f$ er mengden av alle elementer i kodomenet som
blir truffet av~$f$, altså delmengden
\[
\im f = \{ f(a) \mid a \in A \}
\]
av~$B$.
\end{defn}

Det følger umiddelbart fra definisjonen at en funksjon
$f \colon A \to B$ er surjektiv hvis og bare hvis bildet til
funksjonen er hele kodomenet: $\im f = B$.

\begin{ex}
Vi finner bildene til hver av funksjonene $f$, $g$ og~$h$ fra
eksempel~\ref{ex:funksjon1}--\ref{ex:funksjon3}, og finner ut om
funksjonene er injektive og/eller surjektive.

For funksjonen~$f$ ser vi at vi kan få $f(x)$ til å bli alle tall fra
$1$ og oppover ved å variere~$x$, slik at
\[
\im f = [ 1, \infty ).
\]
Funksjonen er ikke injektiv, siden den sender flere elementer til det
samme -- for eksempel har vi:
\[
f(1) = 4 = f(-1)
\]
Funksjonen er heller ikke surjektiv, siden bildet ikke er hele
kodomenet.

For funksjonen~$g$ ser vi at bildet blir mengden bestående av de tre
tallene vi har valgt å sende fruktene til:
\[
\im g = \{ 0, 3, 4 \}
\]
Funksjonen er injektiv, siden den sender alle fruktene til
forskjellige tall.  Den er ikke surjektiv, siden bildet ikke er hele
kodomenet.

For å finne bildet til funksjonen~$h$ må vi kanskje tenke litt.  Men
når vi prøver oss litt frem, ser vi ganske raskt at den treffer
hele~$\R^2$, siden vi for enhver vektor $\vvS{v_1}{v_2}$ i~$\R^2$ får:
\[
h\left( \vvv{v_1}{0}{v_2} \right) = \vv{v_1}{v_2}
\]
Dette betyr at
\[
\im h = \R^2.
\]
Dermed har vi også vist at $h$ er surjektiv.  Men vi kan lett finne
flere vektorer i~$\R^3$ som $h$ sender til samme vektor, for eksempel:
\[
h\left(\vvv{1}{0}{1}\right) = \vv{1}{1} = h\left(\vvv{0}{1}{0}\right)
\]
Dette vil si at $h$ ikke er injektiv.
\end{ex}

Vi definerer to konsepter til som har med funksjoner å gjøre.

\begin{defn}
For enhver mengde~$A$ finnes en \defterm{identitetsfunksjon}
$\id_A \colon A \to A$, definert ved
\[
\id_A(a) = a \quad\text{for alle $a$ i~$A$.}
\qedhere
\]
\end{defn}

\begin{defn}
Hvis $f \colon A \to B$ og $g \colon B \to C$ er funksjoner, så er
\defterm{sammensetningen} av $g$ og~$f$ en funksjon
$g \fcomp f \colon A \to C$ definert ved
\[
(g \fcomp f)(a) = g(f(a)).
\qedhere
\]
\end{defn}

\begin{ex}
La $f \colon \R \to \R$ og $g \colon \R \to \R$ være funksjoner
definert ved:
\begin{align*}
f(x) &= \sin x + 1 \\
g(x) &= x^2 + 5x
\end{align*}
Da er sammensetningene $f \fcomp g$ og $g \fcomp f$ også funksjoner
fra $\R$ til~$\R$, og de er gitt ved:
\begin{align*}
(f \fcomp g)(x) &= \sin (x^2 + 5x) + 1 \\
(g \fcomp f)(x) &= (\sin x + 1)^2 + 5(\sin x + 1)\qedhere
\end{align*}
\end{ex}

% TODO tekst


\section*{Definisjon av lineærtransformasjoner}

Nå som vi har det grunnleggende om funksjoner på plass, går vi videre
til å definere lineærtransformasjoner.  En lineærtransformasjon er en
funksjon mellom vektorrom som bevarer vektorromsstrukturen.  Det vil
si at vi kan utføre addisjon eller skalarmultiplikasjon før vi
anvender funksjonen eller etterpå, og resultatet skal bli det samme.
Vi gjør dette presist i følgende definisjon.

\begin{defn}
La $V$ og~$W$ være vektorrom.  En funksjon $T \colon V \to W$ er en
\defterm{lineærtransformasjon} hvis den oppfyller følgende to
kriterier:
\begin{enumerate}
\item $T(\u + \v) = T(\u) + T(\v)$
      for alle $\u$ og~$\v$ i~$V$.
\item $T(c \u) = c \cdot T(\u)$
      for alle vektorer $\u$ i~$V$ og alle skalarer~$c$.
\qedhere
\end{enumerate}
\end{defn}

Vi kan illustrere de to kravene til en lineærtransformasjon slik:
\begin{center}
\begin{tikzpicture}[scale=0.27,baseline=(O)]
\coordinate (O) at (0,0);
\draw[->] (-1,0) -- (10,0);
\draw[->] (0,-1) -- (0,10);
\coordinate (u) at (2,4);
\coordinate (v) at (7,1);
\draw[->] (0,0) -- (u) node [anchor=south] {$\u$};
\draw[->] (0,0) -- (v) node [anchor=west] {$\v$};
\draw[dashed] (u) -- ++(v);
\draw[dashed] (v) -- ++(u);
\draw[->] (0,0) -- ($ (u) + (v) $) node[anchor=south] {$\u + \v$};
\end{tikzpicture}
\hfill
\begin{tikzpicture}[scale=0.27,baseline=(O)]
\coordinate (O) at (0,0);
\path[->] (0,5) edge[bend left=20] node[anchor=south] {$T$}  (3,5);
\end{tikzpicture}
\hfill
\begin{tikzpicture}[scale=0.27,baseline=(O)]
\coordinate (O) at (0,0);
\draw[->] (-1,0) -- (10,0);
\draw[->] (0,-1) -- (0,10);
\coordinate (Tu) at (2,7);
\coordinate (Tv) at (6,2);
\draw[->] (0,0) -- (Tu) node [anchor=south] {$T(\u)$};
\draw[->] (0,0) -- (Tv) node [anchor=west] {$T(\v)$};
\draw[dashed] (Tu) -- ++(Tv);
\draw[dashed] (Tv) -- ++(Tu);
\draw[->] (0,0) -- ($ (Tu) + (Tv) $) node[anchor=south] {$T(\u + \v)$};
\end{tikzpicture}
\\[15pt]
\begin{tikzpicture}[scale=0.27,baseline=(O)]
\coordinate (O) at (0,0);
\draw[->] (-1,0) -- (10,0);
\draw[->] (0,-1) -- (0,10);
\coordinate (u) at (3,0.7);
\coordinate (cu) at ($ 2.5*(u) $);
\draw[->] (0,0) -- (u) node [anchor=south] {$\u$};
\draw[->] (0,0) -- (cu) node [anchor=south] {$c \cdot \u$};
\end{tikzpicture}
\hfill
\begin{tikzpicture}[scale=0.27,baseline=(O)]
\coordinate (O) at (0,0);
\path[->] (0,5) edge[bend left=20] node[anchor=south] {$T$}  (3,5);
\end{tikzpicture}
\hfill
\begin{tikzpicture}[scale=0.27,baseline=(O)]
\coordinate (O) at (0,0);
\draw[->] (-1,0) -- (10,0);
\draw[->] (0,-1) -- (0,10);
\coordinate (Tu) at (3,3);
\coordinate (Tcu) at ($ 2.5*(Tu) $);
\draw[->] (0,0) -- (Tu) node [anchor=north west] {$T(\u)$};
\draw[->] (0,0) -- (Tcu) node [anchor=east,inner sep=6pt] {$T(c \cdot \u)$};
\end{tikzpicture}
\\[8pt]
{\small \textit{Lineærtransformasjonen~$T$ bevarer\\addisjon og skalarmultiplikasjon}}
\end{center}

\begin{ex}
Vi definerer $T \colon \R^3 \to \R^2$ ved:
\[
T\left( \vvv{x_1}{x_2}{x_3} \right) = \vv{2x_3}{x_1-3x_2}
\]
La oss nå sjekke om denne funksjonen er en lineærtransformasjon.  Vi
regner ut:
\begin{align*}
T\left( \vvv{u_1}{u_2}{u_3} + \vvv{v_1}{v_2}{v_3} \right)
&= T\left( \vvv{u_1 + v_1}{u_2 + v_2}{u_3 + v_3} \right) \\[4pt]
&= \vv{2(u_3 + v_3)}{(u_1 + v_1) - 3(u_2 + v_2)} \\
&= \vv{2u_3}{u_1-3u_2} + \vv{2v_3}{v_1-3v_2} \\
&= T\left( \vvv{u_1}{u_2}{u_3} \right) + T\left( \vvv{v_1}{v_2}{v_3} \right)
\end{align*}
Funksjonen $T$ oppfyller altså kravet om å bevare addisjon.  Vi
sjekker at den også oppfyller kravet om å bevare skalarmultiplikasjon:
\begin{align*}
T\left( c \cdot \vvv{u_1}{u_2}{u_3} \right)
&= T\left( \vvv{c u_1}{c u_2}{c u_3} \right)
 = \vv{2(c u_3)}{c u_1 - 3(c u_2)}\\
&= c \vv{2 u_3}{u_1 - 3 u_2}
 = c \cdot T\left( \vvv{u_1}{u_2}{u_3} \right)
\end{align*}
Vi har nå sjekket at funksjonen~$T$ oppfyller begge kravene i
definisjonen, så den er en lineærtransformasjon.
\end{ex}

\begin{ex}
Vi definerer $T \colon \R^2 \to \R^2$ ved:
\[
T\left( \vv{x_1}{x_2} \right) = \vv{x_1 + 2 x_2}{x_1 + 1}
\]
Nå kan vi for eksempel legge merke til at
\[
T\left( \vv{1}{1} \right) = \vv{3}{2},
\quad\text{men}\quad
T\left( 2 \cdot \vv{1}{1} \right) = \vv{6}{3}.
\]
Vi har altså
\[
T\left( 2 \cdot \vv{1}{1} \right) \ne 2 \cdot T\left( \vv{1}{1} \right),
\]
så $T$ er ikke en lineærtransformasjon.
\end{ex}

Her er noen egenskaper ved lineærtransformasjoner som følger ganske
enkelt fra definisjonen, kombinert med aksiomene for vektorrom:

\begin{thm}
\label{thm:lin-tr-lin-komb}
Hvis $T \colon V \to W$ er en lineærtransformasjon, så oppfyller den
følgende.
\begin{enumerate}
\item[(a)] En lineærkombinasjon i~$V$ sendes til den tilsvarende
lineærkombinasjonen i~$W$:
\begin{multline*}
T( c_1 \v_1 + c_2 \v_2 + \cdots + c_r \v_r )
\\
= c_1 \cdot T(\v_1) + c_2 \cdot T(\v_2) + \cdots + c_2 \cdot T(\v_r)
\end{multline*}
\item[(b)] Nullvektoren i~$V$ sendes til nullvektoren i~$W$:
\[
T(\0) = \0
\]
\end{enumerate}
\end{thm}
% TODO bevis

\begin{ex}
Anta at $T \colon \R^2 \to \R^2$ er en lineærtransformasjon slik at
\[
T\left( \vv{2}{3} \right) = \vv{1}{1}
\quad\text{og}\quad
T\left( \vv{0}{5} \right) = \vv{3}{1}.
\]
Kan vi ut fra dette finne ut hva
\[
T\left( \vv{8}{2} \right)
\]
må være?  Vi ser at vi kan skrive $\vvS{8}{2}$ som en
lineærkombinasjon av $\vvS{2}{3}$ og~$\vvS{0}{5}$:
\[
\vv{8}{2} = 4 \cdot \vv{2}{3} - 2 \cdot \vv{0}{5}
\]
Ved å bruke teorem~\ref{thm:lin-tr-lin-komb}~(a) får vi nå:
\begin{align*}
T\left( \vv{8}{2} \right)
&= T\left( 4 \cdot \vv{2}{3} - 2 \cdot \vv{0}{5} \right) \\
&= 4 \cdot T\left( \vv{2}{3} \right) - 2 \cdot T\left( \vv{0}{5} \right) \\
&= 4 \cdot \vv{1}{1} - 2 \cdot \vv{3}{1}
 = \vv{-2}{2}
\end{align*}
Mer generelt kan vi se at siden de to vektorene
\[
\vv{2}{3}
\quad\text{og}\quad
\vv{0}{5}
\]
utspenner~$\R^2$, er det nok å vite hva $T$ gjør med hver av disse for
å vite hva den gjør med en hvilken som helst vektor.
\end{ex}

Nå har vi sett noen eksempler på lineærtransformasjon mellom vektorrom
på formen~$\R^n$.  Vi tar med ett eksempel på en lineærtransformasjon
der domenet er et litt annerledes vektorrom.

\begin{ex}
Vi husker fra forrige kapittel at $\Cf(\R)$ er vektorrommet som består
av alle kontinuerlige funksjoner fra $\R$ til~$\R$.  La
$T \colon \Cf(\R) \to \R^2$ være funksjonen gitt ved:
\[
T(f) = \vv{f(0)}{f(1)}
\]
Hvis vi for eksempel ser på en funksjon $f \colon \R \to \R$ i
$\Cf(\R)$ gitt ved
\[
f(x) = 3x^2 + 1,
\]
så har vi:
\[
T(f) = \vv{f(0)}{f(1)} = \vv{1}{4}
\]
Vi sjekker at $T$ er en lineærtransformasjon:
\begin{align*}
T(f + g)
&= \vv{(f+g)(0)}{(f+g)(1)}
 = \vv{f(0) + g(0)}{f(1) + g(1)} \\
&= \vv{f(0)}{f(1)} + \vv{g(0)}{g(1)}
 = T(f) + T(g) \\[8pt]
T(cf)
&= \vv{(cf)(0)}{(cf)(1)}
 = c \cdot \vv{f(0)}{f(1)}
 = c \cdot T(f)
\end{align*}
Vi kan observere at for enhver vektor~$\vvS{a}{b}$ i~$\R^2$ kan vi
definere en funksjon~$f$ i~$\Cf(\R)$ ved
\[
f(x) = (b - a) x + a,
\]
og da får vi:
\[
T(f) = \vv{f(0)}{f(1)} = \vv{a}{b}
\]
Dette betyr at funksjonen~$T$ treffer alle vektorene i~$\R^2$, slik at
$\im T = \R^2$, og $T$ er surjektiv.

Men $T$ er ikke injektiv: Vi kan for eksempel se på to funksjoner $f$
og~$g$ i~$\Cf(\R)$ gitt ved:
\[
f(x) = 0
\quad\text{og}\quad
g(x) = x^2 - x
\]
Da har vi at
\[
T(f) = \vv{0}{0} = T(g),
\]
men $f \ne g$, så $T$ er ikke injektiv.
\end{ex}


\section*{Kjerne og bilde}

For enhver funksjon $f \colon A \to B$ har vi definert bildet $\im f$,
som er delmengden av kodomenet~$B$ bestående av alle elementer
funksjonen treffer.  For en lineærtransformasjon har vi også en
delmengde av domenet som det er naturlig å knytte til
lineærtransformasjonen, nemlig mengden av alle vektorer som sendes til
nullvektoren.

\begin{defn}
La $T \colon V \to W$ være en lineærtransformasjon.  \defterm{Kjernen}
til $T$ er mengden av alle vektorer i~$V$ som blir sendt til
nullvektoren i~$W$:
\[
\ker T = \{ \v \in V \mid T(\v) = \0 \}\qedhere
\]
\end{defn}

\begin{ex}
La $T \colon \R^2 \to \R^2$ være lineærtransformasjonen gitt ved:
\[
T \left( \vv{x_1}{x_2} \right) = \vv{x_1-x_2}{x_2-x_1}
\]
Vi vil finne kjernen og bildet til~$T$.

Vi ser at en vektor $\vvS{x_1}{x_2}$ blir sendt til nullvektoren hvis
og bare hvis de to komponentene $x_1$ og~$x_2$ er samme tall, så
kjernen blir mengden
\[
\ker T = \left\{ \vv{a}{a}\ \middle|\ a \in \R \right\}.
\]
Siden
\[
x_2 - x_1 = - (x_1 - x_2),
\]
ser vi at alle vektorer vi når ved å anvende~$T$ må være på formen
\[
\vv{a}{-a}.
\]
Vi ser dessuten at vi kan nå alle slike vektorer, siden vi for hvert
tall~$a$ har at
\[
T \left( \vv{a}{0} \right) = \vv{a}{-a}.
\]
Det betyr at
\[
\im T = \left\{ \vv{a}{-a}\ \middle|\ a \in \R \right\}.
\]
Vi har altså at både kjernen og bildet er rette linjer i~$\R^2$:
\begin{center}
\begin{tikzpicture}[scale=.16]
\draw[->] (-10,0) -- (10,0);
\draw[->] (0,-10) -- (0,10);
\draw (-10,-10) -- (10,10) node[anchor=north west] {$\ker T$};
\draw (-10,10) -- (10,-10) node[anchor=south west] {$\im T$};
\end{tikzpicture}
\end{center}
Spesielt betyr dette at både bildet og kjernen er underrom av~$\R^2$.
\end{ex}

I eksempelet hadde vi en lineærtransformasjon med $\R^2$ som både
domene og kodomene, og vi så at både bildet og kjernen ble underrom
av~$\R^2$.  Dette var ingen tilfeldighet, for vi kan vise generelt at
kjernen til en lineærtransformasjon alltid må være et underrom av
domenet, og bildet alltid et underrom av kodomenet.

\begin{thm}
La $T \colon V \to W$ være en lineærtransformasjon.
\begin{enumerate}
\item[(a)] Kjernen $\ker T$ er et underrom av~$V$.
\item[(b)] Bildet $\im T$ er et underrom av~$W$.
\end{enumerate}
\end{thm}
% TODO bevis

Vi vet at en lineærtransformasjon $T \colon V \to W$ er surjektiv hvis
og bare hvis $\im T = W$ (dette holder generelt for alle funksjoner,
ikke bare lineærtransformasjoner).  Vi skal nå se at det på samme måte
er en nær sammenheng mellom kjernen til~$T$ og hvorvidt $T$ er
injektiv.

Hvis $T$ er injektiv, så er det maksimalt én vektor i~$V$ som $T$
sender til nullvektoren i~$W$.  Men vi vet jo at $T$ må sende
nullvektoren i~$V$ til nullvektoren i~$W$.  Dermed får vi at
$\ker T = \{ \0 \}$.  Vi kan også vise at den motsatte implikasjonen
holder, og da får vi følgende teorem.

\begin{thm}
En lineærtransformasjon $T \colon V \to W$ er injektiv hvis og bare
hvis $\ker T = \{ \0 \}$.
\end{thm}
\begin{proof}
Vi har allerede vist at hvis $T$ er injektiv, så er
$\ker T = \{ \0 \}$.  Da gjenstår det å vise at hvis
$\ker T = \{ \0 \}$, så er $T$ injektiv.

Vi antar derfor at $\ker T = \{ \0 \}$, og vi ser på to vektorer $\u$
og~$\v$ i~$V$ som er slik at
\[
T(\u) = T(\v).
\]
Vi vil vise at dette medfører at $\u$ og~$\v$ må være den samme
vektoren.  Vi kan flytte over $T(\v)$ til venstre side og få:
\[
T(\u) - T(\v) = \0.
\]
Men $T(\u) - T(\v)$ er det samme som $T(\u-\v)$, siden $T$ er en
lineærtransformasjon.  Dermed har vi
\[
T(\u - \v) = \0,
\]
som betyr at $\u - \v$ ligger i kjernen til~$T$.  Antagelsen vi
startet med var at den eneste vektoren i kjernen til~$T$ er
nullvektoren, så dette vil si at
\[
\u - \v = \0,
\]
altså at $\u = \v$.  Vi har altså vist at hvis $T(\u) = T(\v)$, så er
$\u = \v$, og det vil si at $T$ er injektiv.
\end{proof}

Dette teoremet forteller oss at det er lettere å finne ut om en
lineærtransformasjon~$T$ er injektiv enn om en vilkårlig funksjon er
injektiv.  Det eneste vi trenger å sjekke er hva kjernen er, altså
hvilke vektorer $T$ sender til nullvektoren.


% \section*{Lineær utvidelse fra basis}

% \begin{thm}
% La $V$ være et vektorrom 
% \end{thm}

% TODO
% gitt vektorrom V, W og en basis B for V
% enhver funksjon B -> W utvides entydig til en lin.tr. V -> W

% flytte til senere i kapitlet?  droppe?


\section*{Lineærtransformasjoner gitt ved matriser}

La $A$ være en $m \times n$-matrise.  Da kan vi definere en funksjon
$T \colon \R^n \to R^m$ ved
\[
T(\x) = A \x.
\]
Dette blir en lineærtransformasjon, siden vi (ved å bruke regneregler
for matriser) får at
\begin{align*}
T(\u + \v) &= A \cdot (\u + \v) = A\u + A\v = T(\u) + T(\v) \\
T(c\u) &= A \cdot (c\u) = c \cdot (A\u) = c \cdot T(\u)
\end{align*}
for alle vektorer $\u$ og~$\v$ i~$V$, og alle skalarer~$c$.

\begin{ex}
La $A$ være $3 \times 2$-matrisen
\[
A =
\begin{bmatrix}
1 & -3 \\
3 & 5 \\
-1 & 7
\end{bmatrix},
\]
og definer en lineærtransformasjon $T \colon \R^2 \to \R^3$ ved
\[
T(\x) = A \x.
\]
Det å spesifisere $T$ på denne måten gjør at vi kan besvare spørsmål
om~$T$ ved å benytte regneteknikkene vi kjenner for matriser.

Hvis vi for eksempel lurer på hva $T(\vvS{2}{-1})$ blir, så er det
bare å regne ut:
\[
T(\vv{2}{-1})
 =
\begin{bmatrix}
1 & -3 \\
3 & 5 \\
-1 & 7
\end{bmatrix}
\vv{2}{-1}
 = \vvv{5}{1}{-9}
\]

Hvis vi lurer på om det finnes noen vektor~$\x$ i~$\R^2$ slik at
\[
T(\x) = \vvv{3}{2}{-5},
\]
så er det bare å løse likningssystemet
\[
A\x = \vvv{3}{2}{-5}
\]
på vanlig måte med gausseliminasjon.  Svaret blir ja, det finnes en
slik~$\x$, nemlig
\[
\x = \vv{3/2}{-1/2}.
\qedhere
\]
\end{ex}

Som vi så i eksempelet, kan vi alltid få til å besvare spørsmål av
typen «hva er $T(\v)$?» og «finnes det noen $x$ slik at $T(\x) = \b$?»
når lineærtransformasjonen~$T$ er definert ved en matrise~$A$.  Vi kan
også bruke matrisen til å regne ut kjernen og bildet til~$T$.

Kjernen $\ker T$ er definert som mengden av alle vektorer~$\v$ slik at
$T(\v) = \0$.  Men når $T(\x) = A\x$ for alle~$\x$, blir dette det
samme som mengden av alle vektorer~$\v$ slik at $A\v = \0$, og det er
nullrommet til~$A$.  Vi får altså at $\ker T = \Null A$.

Bildet $\im T$ er alle vektorer som kan skrives som $T(\v)$.  Når
$T(\x) = A\x$, blir dette det samme som alle vektorer som kan skrives
som $A\v$.  Det er det samme som alle lineærkombinasjoner av kolonnene
i~$A$, altså kolonnerommet til~$A$.  Vi får altså at $\im T = \Col A$.

Vi oppsummerer det vi har vist nå i et teorem.

\begin{thm}
La $A$ være en $m \times n$-matrise, og la $T \colon \R^n \to \R^m$
være lineærtransformasjonen gitt ved $T(\x) = A\x$.  Da er
\[
\ker T = \Null A
\qquad\text{og}\qquad
\im T = \Col A.
\]
\end{thm}

Det er altså mange grunner til at det er fordelaktig å ha
lineærtransformasjonene våre gitt ved matriser.  Hvis vi har en
lineærtransformasjon $T \colon \R^n \to \R^m$ som ikke er gitt ved en
matrise, kan det derfor være nyttig å prøve å finne en matrise~$A$
slik at $T(\x) = A\x$ for alle vektorer~$\x$ i~$\R^n$.  I det neste
eksempelet gjør vi nettopp dette.

\begin{ex}
La $\e_1 = \vvS{1}{0}$ og $\e_2 = \vvS{0}{1}$ være enhetsvektorene
i~$\R^2$, og la $T \colon \R^2 \to \R^3$ være en lineærtransformasjon
slik at
\[
T(\e_1) = \vvv{5}{-7}{2}
\quad\text{og}\quad
T(\e_2) = \vvv{-3}{8}{0}
\]
Basert på dette kan vi finne ut hva $T(\x)$ er for en vilkårlig vektor
$\x$ i~$\R^2$.  Vi kan nemlig skrive
\[
\x = \vv{x_1}{x_2} = x_1 \e_1 + x_2 \e_2,
\]
og da får vi:
\begin{align*}
T(\x)
&= T(x_1 \e_1 + x_2 \e_2)
 = x_1 \cdot T(\e_1) + x_2 \cdot T(\e_2) \\
&= x_1 \cdot \vvv{5}{-7}{2} + x_2 \cdot \vvv{-3}{8}{0} \\
&=
\begin{bmatrix}
 5 & -3 \\
-7 &  8 \\
 2 &  0
\end{bmatrix}
\x
\end{align*}
Her har vi endt opp med å skrive lineærtransformasjonen ved hjelp av
en matrise.  La $A$ være denne matrisen:
\[
A = \begin{bmatrix} \ T(\e_1) & T(\e_2)\;\ \end{bmatrix}
  =
\begin{bmatrix}
 5 & -3 \\
-7 &  8 \\
 2 &  0
\end{bmatrix}
\]
Da har vi altså at $T(\x) = A\x$ for alle vektorer~$\x$ i~$\R^2$.
\end{ex}

På samme måte som i dette eksempelet kan vi skrive enhver
lineærtransformasjon $T \colon \R^n \to \R^m$ på matriseform ved å
lage en matrise av vektorene som $T$ sender enhetsvektorene i~$\R^n$
til.  Vi beskriver det generelle tilfellet i et teorem.

\begin{thm}
\label{thm:standardmatrise}
La $T \colon \R^n \to \R^m$ være en lineærtransformasjon.
Da finnes en $m \times n$-matrise $A$ slik at
\[
T(\x) = A\x
\quad\text{for alle $\x$ i $\R^n$.}
\]
Matrisen $A$ er entydig bestemt av~$T$, og er gitt ved
\[
\begin{bmatrix} \ T(\e_1) & T(\e_2) & \cdots & T(\e_n) \;\ \end{bmatrix},
\]
der $(\e_1, \e_2, \ldots, \e_n)$ er standardbasisen for~$\R^n$.
\end{thm}
% TODO bevis?

\begin{defn}
Matrisen $A$ i teorem~\ref{thm:standardmatrise} kalles
\defterm{standardmatrisen} til lineærtransformasjonen~$T$.
\end{defn}

% TODO eks?

Faktisk kan vi gjøre teorem~\ref{thm:standardmatrise} mer generelt.
Så lenge vektorrommene våre er endeligdimensjonale, kan enhver
lineærtransformasjon beskrives ved en matrise.  Men det krever at vi
velger en basis for hvert vektorrom.
% TODO tekst

\begin{thm}
La $V$ og~$W$ være endeligdimensjonale vektorrom, og la $\B$ og
$\mathscr{C}$ være basiser for henholdsvis $V$ og~$W$.  La
$T \colon V \to W$ være en lineærtransformasjon.  Da finnes en
matrise~$A$ slik at
\[
\koord{T(\x)}{\mathscr{C}} = A \cdot \koord{\x}{\B}
\]
for alle vektorer~$\x$ i~$V$.
\end{thm}

Egentlig er det et valg av basis involvert i
teorem~\ref{thm:standardmatrise} også.  Der har vi valgt å bruke
standardbasisene for $\R^n$ og~$\R^m$.  For et vilkårlig vektorrom~$V$
har vi ikke nødvendigvis noen slik basis som er det åpenbare valget.

% TODO forklaring, eks?


% TODO sammensetning av lineærtransformasjoner er matrisemultiplikasjon


\section*{Egenverdier og egenvektorer}

Vi er vant til å se på egenverdier og egenvektorer for kvadratiske
matriser.  Nå vet vi at en $n \times n$-matrise~$A$ gir opphav til en
lineærtransformasjon $T \colon \R^n \to \R^n$ definert ved
\[
T(\x) = A\x,
\]
og da kan vi beskrive egenverdiene og egenvektorene til~$A$ ved hjelp
av denne lineærtransformasjonen.  Dersom $A$ har en
egenverdi~$\lambda$ med tilhørende egenvektor~$\v$, så betyr det at
\[
T(\v) = \lambda \v.
\]
Dette kan vi generalisere til mer generelle lineærtransformasjoner.

Hvis vi har en lineærtransformasjon $T \colon V \to V$ der domenet og
kodomenet er det samme vektorrommet~$V$, så kan vi definere
egenverdier og egenvektorer for~$T$ på tilsvarende måte som for
matriser.  Vi sier at $\lambda$ er en \defterm{egenverdi} for~$T$, og
$\v$ en tilhørende \emph{egenvektor}, dersom
\[
T(\v) = \lambda \v
\qquad\text{og}\qquad
\v \ne \0.
\]
Hvis $\lambda$ er en egenverdi for~$T$, så har $\lambda$ et tilhørende
\defterm{egenrom}, nemlig underrommet
\[
\{ \v \in V \mid T(\v) = \lambda \v \}
\]
av~$V$.


\section*{Isomorfi}

Til slutt i dette kapitlet ser vi på hvordan vi kan bruke
lineærtransformasjoner til å beskrive at to vektorrom er «strukturelt
like».  Med dette mener vi at de oppfører seg på akkurat samme måte
som vektorrom, selv om de kan bestå av helt forskjellige elementer.
Da vil vi si at de to vektorrommene er \emph{isomorfe}.  For å kunne
definere dette, trenger vi først et begrep om inverser for
lineærtransformasjoner.

\begin{defn}
La $T \colon V \to W$ være en lineærtransformasjon.  En
\defterm{invers} til~$T$ er en lineærtransformasjon $S \colon W \to V$
som er slik at
\begin{align*}
S(T(\v)) &= \v &&\text{for alle $\v$ i~$V$, og} \\
T(S(\w)) &= \w &&\text{for alle $\w$ i~$W$.}
\qedhere
% TODO formulere ved at sammensetningene er \id_V, \id_W
\end{align*}
\end{defn}

Vi vil si at to vektorrom er isomorfe hvis det er mulig å bevege seg
frem og tilbake mellom dem ved hjelp av lineærtransformasjoner som
bevarer all informasjonen om vektorrommene.  Vi vil altså ha en
situasjon slik som dette, der $T$ og~$S$ er hverandres inverser:
\begin{center}
\begin{tikzpicture}
\node(V) at (0,0) {$V$};
\node(W) at (1,0) {$W$};
\draw[->] (V) to[bend left=20] node[anchor=south] {$T$} (W);
\draw[->] (W) to[bend left=20] node[anchor=north] {$S$} (V);
\end{tikzpicture}
\end{center}
\begin{defn}
Hvis $T \colon V \to W$ er en lineærtransformasjon som har en invers,
så er $T$ en \defterm{isomorfi}.  Da sier vi dessuten at vektorrommene
$V$ og~$W$ er \defterm{isomorfe}, og vi skriver $V \iso W$.
\end{defn}

\begin{ex}
Vi lar $V$ være underrommet
\[
V = \Sp \left\{ \vv{3}{2} \right\}
\]
i~$\R^2$ utspent av vektoren~$\vvS{3}{2}$.  Da er $V$ et
endimensjonalt vektorrom, og geometrisk sett er det en linje.
Vektorrommet~$V$ ser ut og oppfører seg akkurat som
vektorrommet~$\R^1$.  Forskjellen er bare at elementene ser
forskjellige ut.  Hvert element i $V$ er en vektor på formen
\[
\vv{3t}{2t}
\]
mens hvert element i $\R^1$ er bare et tall.
\begin{center}
\begin{tikzpicture}[scale=.5,baseline]
\draw[->] (-3,0) -- (3,0);
\draw[->] (0,-2) -- (0,2);
\draw (-3,-2) -- (3,2) node[anchor=north] {$V$};
\end{tikzpicture}
\qquad
\begin{tikzpicture}[scale=.5,baseline]
\draw[->] (-3,0) -- (3,0) node[anchor=north] {$\R^1$};
\draw (0,.1) -- (0,-.1);
\coordinate (O) at (0,0);
\end{tikzpicture}
\\[4pt]
{\small \textit{De to vektorrommene $V$ og~$\R^1$}}
\end{center}

Det at $V$ og $\R^1$ «ser like ut» kan vi gjøre mer presist ved å vise
at de er isomorfe.  Vi definerer lineærtransformasjoner
\[
T \colon V \to \R^1
\quad\text{og}\quad
S \colon \R^1 \to V
\]
ved:
\[
T \left( \vv{3t}{2t} \right) = t
\qquad\quad
S(x) = \vv{3x}{2x}
\]
Vi kan lett sjekke at disse faktisk er lineærtransformasjoner, og vi
ser at de er hverandres inverser.  Det betyr at de er isomorfier, og
de viser dermed at $V \iso \R^1$.
\end{ex}

\begin{ex}
Vi husker at $\P_1$ er vektorrommet som består av alle polynomer av
grad~$1$ eller lavere, altså alle funksjoner på formen
\[
p(x) = a_1 x + a_0.
\]
Polynomet~$p$ er entydig bestemt av de to tallene $a_1$ og~$a_0$, og
vi vet at addisjon og skalarmultiplikasjon av polynomer foregår ved å
addere eller skalarmultiplisere hver koeffisient.  Hvis vi bare ser på
hva som skjer med koeffisientene, så ligner altså vektorrommet~$\P_1$
veldig på~$\R^2$.

Vi definerer to lineærtransformasjoner
\[
T \colon \P_1 \to \R^2
\quad\text{og}\quad
S \colon \R^2 \to \P_1
\]
på følgende måte:
\begin{align*}
T(p) &= \vv{a_1}{a_0}
 &&\text{for et polynom~$p$ definert} \\[-8pt]
&&&\text{ved $p(x) = a_1x + a_0$,} \\
S\left( \vv{v_1}{v_2} \right) &= q,
 &&\text{der $q$ er polynomet definert} \\[-8pt]
&&&\text{ved $q(x) = v_1 x + v_2$.}
\end{align*}
Det er lett å sjekke at $T$ og~$S$ er lineærtransformasjoner, og at de
er hverandres inverser.  Dermed er de isomorfier, og vi får at
$\P_1 \iso \R^2$.
\end{ex}

I dette eksempelet viste vi at det todimensjonale vektorrommet~$\P_1$
er isomorft med~$\R^2$.  På tilsvarende måte kan vi vise at ethvert
todimensjonalt vektorrom er isomorft med~$\R^2$, og mer generelt at
ethvert $n$-dimensjonalt vektorrom er isomorft med~$\R^n$.

\begin{thm}
Hvis $V$ er $n$-dimensjonalt vektorrom, så er $V$ isomorft med~$\R^n$.
\end{thm}

Vi kan også merke oss at det å være en isomorfi kan beskrives ved
hjelp av injektivitet og surjektivitet.

\begin{thm}
En lineærtransformasjon er en isomorfi hvis og bare hvis den er både
injektiv og surjektiv.
\end{thm}

\kapittelslutt
