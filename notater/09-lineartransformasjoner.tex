\ifx\inkludert\undefined
\documentclass[norsk,a4paper,twocolumn,oneside]{memoir}

\usepackage[utf8]{inputenc}
\usepackage{babel}
\usepackage{amsmath,amssymb,amsthm}
\usepackage[total={17cm,27cm}]{geometry}
\usepackage[table]{xcolor}
%\usepackage{tabularx}
\usepackage{systeme}
%\usepackage{hyperref}
%\usepackage{enumerate}

%\usepackage{sectsty}
\setsecheadstyle{\bfseries\large}
%\subsectionfont{\bf\normalsize}

\usepackage{tikz}
\usetikzlibrary{arrows.meta}

\newcommand{\defterm}[1]{\emph{#1}}

\newcommand{\N}{\mathbb{N}}
\newcommand{\Z}{\mathbb{Z}}
\newcommand{\Q}{\mathbb{Q}}
\newcommand{\R}{\mathbb{R}}

\newcommand{\abs}[1]{|#1|}

\newcommand{\roweq}{\sim}
\DeclareMathOperator{\Span}{Span}

\newcommand{\V}[1]{\mathbf{#1}}
\newcommand{\vv}[2]{\begin{bmatrix} #1 \\ #2 \end{bmatrix}}
\newcommand{\vvv}[3]{\begin{bmatrix} #1 \\ #2 \\ #3 \end{bmatrix}}
\newcommand{\vvvv}[4]{\begin{bmatrix} #1 \\ #2 \\ #3 \\ #4 \end{bmatrix}}
\newcommand{\vn}[2]{\vvvv{#1_1}{#1_2}{\vdots}{#1_#2}}

\newenvironment{amatrix}[1]{% "augmented matrix"
  \left[\begin{array}{*{#1}{c}|c}
}{%
  \end{array}\right]
}

% \newcounter{notatnr}
% \newcommand{\notatnr}[2]
% {\setcounter{notatnr}{#1}%
%  \setcounter{page}{#2}%
% }

\newtheorem{thm}{Teorem}[chapter]
\newtheorem*{thm-nn}{Teorem}
\newtheorem{cor}[thm]{Korollar}
\newtheorem{lem}[thm]{Lemma}
\newtheorem{prop}[thm]{Proposisjon}
\theoremstyle{definition}
\newtheorem{exx}[thm]{Eksempel}
\newtheorem*{defnx}{Definisjon}
\newtheorem*{oppg}{Oppgave}
\newtheorem*{merkx}{Merk}
\newtheorem*{spmx}{Spørsmål}

\newenvironment{defn}
  {\pushQED{\qed}\renewcommand{\qedsymbol}{$\triangle$}\defnx}
  {\popQED\enddefnx}
\newenvironment{ex}
  {\pushQED{\qed}\renewcommand{\qedsymbol}{$\triangle$}\exx}
  {\popQED\endexx}
\newenvironment{merk}
  {\pushQED{\qed}\renewcommand{\qedsymbol}{$\triangle$}\merkx}
  {\popQED\endmerkx}
\newenvironment{spm}
  {\pushQED{\qed}\renewcommand{\qedsymbol}{$\triangle$}\spmx}
  {\popQED\endspmx}

\setlength{\columnsep}{26pt}

\newcommand{\Tittel}[2]{%
\twocolumn[
\begin{center}
\Large
\begin{tabularx}{\textwidth}{cXr}
\cellcolor{black}\color{white}%
\bf {#1} &
#2
\hfill &
\footnotesize TMA4110 høsten 2018
\\ \hline
\end{tabularx}
\end{center}
]}

\newcommand{\tittel}[1]{\Tittel{\arabic{notatnr}}{#1}}

\newcommand{\linje}{%
\begin{center}
\rule{.8\linewidth}{0.4pt}
\end{center}
}


\newcommand{\chapternumber}{}

\makechapterstyle{tma4110}{%
 \renewcommand*{\chapterheadstart}{}
 \renewcommand*{\printchaptername}{}
 \renewcommand*{\chapternamenum}{}
 \renewcommand*{\printchapternum}{\renewcommand{\chapternumber}{\thechapter}}
 \renewcommand*{\afterchapternum}{}
 \renewcommand*{\printchapternonum}{\renewcommand{\chapternumber}{}}
 \renewcommand*{\printchaptertitle}[1]{
\LARGE
\begin{tabularx}{\textwidth}{cXr}
\cellcolor{black}\color{white}%
\textbf{\chapternumber} &
\textbf{##1}
\hfill &
%\footnotesize TMA4110 høsten 2018
\\ \hline
\end{tabularx}%
}
 \renewcommand*{\afterchaptertitle}{\par\nobreak\vskip \afterchapskip}
 % \newcommand{\chapnamefont}{\normalfont\huge\bfseries}
 % \newcommand{\chapnumfont}{\normalfont\huge\bfseries}
 % \newcommand{\chaptitlefont}{\normalfont\Huge\bfseries}
 \setlength{\beforechapskip}{0pt}
 \setlength{\midchapskip}{0pt}
 \setlength{\afterchapskip}{10pt}
}


\newcounter{oppgnr}[chapter]
\newcounter{punktnr}[oppgnr]
\newenvironment{oppgave}
 {\par\noindent\stepcounter{oppgnr}\textbf{{\arabic{oppgnr}}.}}
 {\par\bigskip}
\newenvironment{punkt}
 {\par\smallskip\noindent\stepcounter{punktnr}\textbf{\alph{punktnr})} }
 {\par}

\newcommand{\oppgaver}{\linje\section*{Oppgaver}}

\usepackage{xr}
\externaldocument{tma4110-2018h}
\newcommand{\kapittel}[2]{\setcounter{chapter}{#1}\addtocounter{chapter}{-1}\chapter{#2}}
\newcommand{\kapittelslutt}{\enddocument}
\begin{document}
\chapterstyle{tma4110}
\pagestyle{plain}
\fi


\kapittel{9}{Lineærtransformasjoner}
\label{ch:lineartransformasjoner}

I forrige kapittel begynte vi å formulere lineær algebra på en
generell måte, ved å gi en abstrakt definisjon av vektorrom.  For å
beskrive sammenhenger mellom forskjellige vektorer og vektorrom
trenger vi også å se på funksjoner som tar inn vektorer og gir ut
vektorer.  For to vektorrom $V$ og~$W$ er vi interessert i de
funksjonene fra $V$ til~$W$ som \emph{bevarer vektorromsstrukturen}.
Slike funksjoner kaller vi lineærtransformasjoner.


\section*{Funksjoner}

En \defterm{funksjon} består av tre ting:
\begin{enumerate}
\item En mengde som kalles funksjonens \defterm{domene}.
\item En mengde som kalles funksjonens \defterm{kodomene}.
\item En regel som til hvert element i domenet tilordner et element i
kodomenet.
\end{enumerate}
Vi bruker notasjonen $f \colon A \to B$ for å angi at $f$ er en
funksjon med mengden $A$ som domene og mengden $B$ som kodomene.

(Mengdene som er tilknyttet en funksjon er også kjent under andre
navn.  Domenet kan også kalles \emph{definisjonsmengden} til
funksjonen, og kodomenet kan også kalles \emph{verdimengden}.)

Fra før er du antagelig mest vant til funksjoner som har mengden~$\R$
av reelle tall som både domene og kodomene, og der regelen for
funksjonen er gitt ved et aritmetisk uttrykk.  Her er et eksempel på
en slik funksjon.

\begin{ex}
\label{ex:funksjon1}
La $f \colon \R \to \R$ være funksjonen definert ved
\[
f(x) = 3x^2 + 1
\quad\text{for alle $x$ i~$\R$.}
\]
% TODO
\end{ex}

Men det er verdt å merke seg at både domenet og kodomenet kan være
hvilke som helst mengder, og regelen kan vi spesifisere akkurat slik
vi vil.

\begin{ex}
\label{ex:funksjon2}
La $A$ være mengden bestående av de tre fruktene eple, banan og
ananas, og la $B$ være mengden bestående av heltallene $0$, $1$,
\ldots, $5$:
\begin{align*}
A &= \{ \text{eple}, \text{banan}, \text{ananas} \} \\
B &= \{ 0, 1, 2, 3, 4, 5 \}
\end{align*}
Nå kan vi lage en funksjon
\[
g \colon A \to B
\]
ved å bestemme hva hvert av de tre elementene i mengden~$A$ skal
sendes til.  Hvis vi for eksempel bestemmer at
\begin{align*}
g(\text{eple}) &= 4, \\
g(\text{banan}) &= 3, \\
g(\text{ananas}) &= 0,
\end{align*}
så har vi beskrevet funksjonen~$g$ fullstendig.
\end{ex}

\begin{ex}
\label{ex:funksjon3}
Vi definerer en funksjon
\[
h \colon \R^3 \to \R^2
\]
ved regelen
\[
h\left( \vvv{x_1}{x_2}{x_3} \right)
= \vv{x_1 + x_2}{x_2 + x_3}.
\qedhere
\]
\end{ex}

% TODO tekst

\begin{defn}
La $f \colon A \to B$ være en funksjon.

Vi sier at $f$ er \defterm{injektiv} (eller \defterm{en-til-en}) hvis
det for hver $b$ i~$B$ er maksimalt én $a$ i~$A$ slik at $f(a) = b$.

Vi sier at $f$ er \defterm{surjektiv} (eller \defterm{på}) hvis
det for hver $b$ i~$B$ finnes en $a$ i~$A$ slik at $f(a) = b$.

\defterm{Bildet} til~$f$ er mengden av alle elementer i kodomenet som
blir truffet av~$f$, altså delmengden
\[
\im f = \{ f(a) \mid a \in A \}
\]
av~$B$.
\end{defn}

Det følger umiddelbart fra definisjonen at en funksjon
$f \colon A \to B$ er surjektiv hvis og bare hvis $\im f = B$.

\begin{ex}
Vi finner bildene til hver av funksjonene $f$, $g$ og~$h$ fra
eksempel~\ref{ex:funksjon1}--\ref{ex:funksjon3}, og finner ut om
funksjonene er injektive og/eller surjektive.

For funksjonen~$f$ ser vi at vi kan få $f(x)$ til å bli alle tall fra
$1$ og oppover ved å variere~$x$, slik at
\[
\im f = [ 1, \infty ).
\]
Funksjonen er ikke injektiv, siden den sender flere elementer til det
samme -- for eksempel har vi:
\[
f(1) = 4 = f(-1)
\]
Funksjonen er heller ikke surjektiv, siden bildet ikke er hele
kodomenet.

For funksjonen~$g$ ser vi at bildet blir mengden bestående av de tre
tallene vi har valgt å sende fruktene til:
\[
\im g = \{ 0, 3, 4 \}
\]
Funksjonen er injektiv, siden den sender alle fruktene til
forskjellige tall.  Den er ikke surjektiv, siden bildet ikke er hele
kodomenet.

For å finne bildet til funksjonen~$h$ må vi kanskje tenke litt.  Men
når vi prøver oss litt frem, ser vi ganske raskt at den treffer
hele~$\R^2$, siden vi for enhver vektor $\vvS{v_1}{v_2}$ i~$\R^2$ får:
\[
h\left( \vvv{v_1}{0}{v_2} \right) = \vv{v_1}{v_2}
\]
Dette betyr at
\[
\im h = \R^2.
\]
Dermed har vi også vist at $h$ er surjektiv.  Men vi kan lett finne
flere vektorer i~$\R^3$ som $h$ sender til samme vektor, for eksempel:
\[
h\left(\vvv{1}{0}{1}\right) = \vv{1}{1} = h\left(\vvv{0}{1}{0}\right)
\]
Dette vil si at $h$ ikke er injektiv.
\end{ex}

% TODO tekst

\begin{defn}
For enhver mengde~$A$ finnes en \defterm{identitetsfunksjon}
$\id_A \colon A \to A$, definert ved
\[
\id_A(a) = a \quad\text{for alle $a$ i~$A$.}
\qedhere
\]
\end{defn}

\begin{defn}
Hvis $f \colon A \to B$ og $g \colon B \to C$ er funksjoner, så er
\defterm{sammensetningen} av $g$ og~$f$ en funksjon
$g \fcomp f \colon A \to C$ definert ved
\[
(g \fcomp f)(a) = g(f(a)).
\qedhere
\]
\end{defn}

% TODO tekst


\section*{Definisjon av lineærtransformasjoner}

% TODO tekst

\begin{defn}
La $V$ og~$W$ være vektorrom.  En funksjon $T \colon V \to W$ er en
\defterm{lineærtransformasjon} hvis den oppfyller følgende to
kriterier:
\begin{enumerate}
\item $T(\u + \v) = T(\u) + T(\v)$
      for alle $\u$ og~$\v$ i~$V$.
\item $T(c \u) = c \cdot T(\u)$
      for alle vektorer $\u$ i~$V$ og alle skalarer~$c$.
\qedhere
\end{enumerate}
\end{defn}

Strukturen til et vektorrom er gitt ved de to operasjonene addisjon og
skalarmultiplikasjon.  Vi sier at en lineærtransformasjon
$T \colon V \to W$ bevarer vektorromsstrukturen siden vi alltid får
samme resultat om vi %TODO skriv ferdig

% TODO forklare «bevarer vektorromsstrukturen»

Vi kan illustrere de to kravene til en lineærtransformasjon slik:
\begin{center}
\begin{tikzpicture}[scale=0.27,baseline=(O)]
\coordinate (O) at (0,0);
\draw[->] (-1,0) -- (10,0);
\draw[->] (0,-1) -- (0,10);
\coordinate (u) at (2,4);
\coordinate (v) at (7,1);
\draw[->] (0,0) -- (u) node [anchor=south] {$\u$};
\draw[->] (0,0) -- (v) node [anchor=west] {$\v$};
\draw[dashed] (u) -- ++(v);
\draw[dashed] (v) -- ++(u);
\draw[->] (0,0) -- ($ (u) + (v) $) node[anchor=south] {$\u + \v$};
\end{tikzpicture}
\hfill
\begin{tikzpicture}[scale=0.27,baseline=(O)]
\coordinate (O) at (0,0);
\path[->] (0,5) edge[bend left=20] node[anchor=south] {$T$}  (3,5);
\end{tikzpicture}
\hfill
\begin{tikzpicture}[scale=0.27,baseline=(O)]
\coordinate (O) at (0,0);
\draw[->] (-1,0) -- (10,0);
\draw[->] (0,-1) -- (0,10);
\coordinate (Tu) at (2,7);
\coordinate (Tv) at (6,2);
\draw[->] (0,0) -- (Tu) node [anchor=south] {$T(\u)$};
\draw[->] (0,0) -- (Tv) node [anchor=west] {$T(\v)$};
\draw[dashed] (Tu) -- ++(Tv);
\draw[dashed] (Tv) -- ++(Tu);
\draw[->] (0,0) -- ($ (Tu) + (Tv) $) node[anchor=south] {$T(\u + \v)$};
\end{tikzpicture}
\\[15pt]
\begin{tikzpicture}[scale=0.27,baseline=(O)]
\coordinate (O) at (0,0);
\draw[->] (-1,0) -- (10,0);
\draw[->] (0,-1) -- (0,10);
\coordinate (u) at (3,0.7);
\coordinate (cu) at ($ 2.5*(u) $);
\draw[->] (0,0) -- (u) node [anchor=south] {$\u$};
\draw[->] (0,0) -- (cu) node [anchor=south] {$c \cdot \u$};
\end{tikzpicture}
\hfill
\begin{tikzpicture}[scale=0.27,baseline=(O)]
\coordinate (O) at (0,0);
\path[->] (0,5) edge[bend left=20] node[anchor=south] {$T$}  (3,5);
\end{tikzpicture}
\hfill
\begin{tikzpicture}[scale=0.27,baseline=(O)]
\coordinate (O) at (0,0);
\draw[->] (-1,0) -- (10,0);
\draw[->] (0,-1) -- (0,10);
\coordinate (Tu) at (3,3);
\coordinate (Tcu) at ($ 2.5*(Tu) $);
\draw[->] (0,0) -- (Tu) node [anchor=north west] {$T(\u)$};
\draw[->] (0,0) -- (Tcu) node [anchor=east,inner sep=6pt] {$T(c \cdot \u)$};
\end{tikzpicture}
\\[8pt]
{\small \textit{Lineærtransformasjonen~$T$ bevarer\\addisjon og skalarmultiplikasjon}}
\end{center}

\begin{ex}
Vi definerer en funksjon $T \colon \R^3 \to \R^2$ ved:
\[
T\left( \vvv{x_1}{x_2}{x_3} \right) = \vv{2x_3}{x_1-3x_2}
\]
La oss nå sjekke om denne funksjonen er en lineærtransformasjon.  Vi
regner ut:
\begin{align*}
T\left( \vvv{u_1}{u_2}{u_3} + \vvv{v_1}{v_2}{v_3} \right)
&= T\left( \vvv{u_1 + v_1}{u_2 + v_2}{u_3 + v_3} \right) \\[4pt]
&= \vv{2(u_3 + v_3)}{(u_1 + v_1) - 3(u_2 + v_2)} \\
&= \vv{2u_3}{u_1-3u_2} + \vv{2v_3}{v_1-3v_2} \\
&= T\left( \vvv{u_1}{u_2}{u_3} \right) + T\left( \vvv{v_1}{v_2}{v_3} \right)
\end{align*}
Funksjonen $T$ oppfyller altså kravet om å bevare addisjon.  Vi
sjekker at den også oppfyller kravet om å bevare skalarmultiplikasjon:
\begin{align*}
T\left( c \cdot \vvv{u_1}{u_2}{u_3} \right)
&= T\left( \vvv{c u_1}{c u_2}{c u_3} \right)
 = \vv{2(c u_3)}{c u_1 - 3(c u_2)}\\
&= c \vv{2 u_3}{u_1 - 3 u_2}
 = c \cdot T\left( \vvv{u_1}{u_2}{u_3} \right)
\end{align*}
Vi har nå sjekket at funksjonen~$T$ oppfyller begge kravene i
definisjonen, så den er en lineærtransformasjon.
\end{ex}

\begin{ex}
Vi definerer en funksjon $T \colon \R^2 \to \R^2$ ved:
\[
T\left( \vv{x_1}{x_2} \right) = \vv{x_1 + 2 x_2}{x_1 + 1}
\]
% TODO
\end{ex}

% TODO to eksempler

\begin{thm}
Hvis $T \colon V \to W$ er en lineærtransformasjon, så oppfyller den
følgende.
\begin{enumerate}
\item[(a)] En lineærkombinasjon i~$V$ sendes til den tilsvarende
lineærkombinasjonen i~$W$:
\begin{multline*}
T( c_1 \v_1 + c_2 \v_2 + \cdots + c_r \v_r )
\\
= c_1 \cdot T(\v_1) + c_2 \cdot T(\v_2) + \cdots + c_2 \cdot T(\v_r)
\end{multline*}
\item[(b)] Nullvektoren i~$V$ sendes til nullvektoren i~$W$:
\[
T(\0) = \0
\]
\end{enumerate}
\end{thm}
% TODO bevis

% TODO eks

% TODO sum og skalarprodukt lineærtransformasjon?


\section*{Lineær utvidelse fra basis}

% TODO
% gitt vektorrom V, W og en basis B for V
% enhver funksjon B -> W utvides entydig til en lin.tr. V -> W


\section*{Kjerne og bilde}

\begin{defn}
La $T \colon V \to W$ være en lineærtransformasjon.  \defterm{Kjernen}
til $T$ er mengden av alle vektorer i~$V$ som blir sendt til
nullvektoren i~$W$:
\[
\ker T = \{ \v \in V \mid T(\u) = \0 \}\qedhere
\]
\end{defn}

% TODO eksempel i \R^2, med tegning av im og ker

\begin{thm}
La $T \colon V \to W$ være en lineærtransformasjon.
\begin{enumerate}
\item[(a)] Kjernen $\ker T$ er et underrom av~$V$.
\item[(b)] Bildet $\im T$ er et underrom av~$W$.
\end{enumerate}
\end{thm}
% TODO bevis

Vi vet at en lineærtransformasjon $T \colon V \to W$ er surjektiv hvis
og bare hvis $\im T = W$ (dette holder generelt for alle funksjoner,
ikke bare lineærtransformasjoner).  Vi skal nå se at det på samme måte
er en nær sammenheng mellom kjernen til~$T$ og hvorvidt $T$ er
injektiv.

Hvis $T$ er injektiv, så er det maksimalt én vektor i~$V$ som $T$
sender til nullvektoren i~$W$.  Men vi vet jo at $T$ må sende
nullvektoren i~$V$ til nullvektoren i~$W$.  Dermed får vi at
$\ker T = \{ \0 \}$.
% TODO tekst

\begin{thm}
En lineærtransformasjon $T \colon V \to W$ er injektiv hvis og bare
hvis $\ker T = \{ \0 \}$.
\end{thm}
% TODO bevis


\section*{Lineærtransformasjoner gitt ved matriser}

La $A$ være en $m \times n$-matrise.  Da kan vi definere en funksjon
$T \colon \R^n \to R^m$ ved
\[
T(\x) = A \x.
\]
Dette blir en lineærtransformasjon, siden vi (ved å bruke regneregler
for matriser) får at
\begin{align*}
T(\u + \v) &= A \cdot (\u + \v) = A\u + A\v = T(\u) + T(\v) \\
T(c\u) &= A \cdot (c\u) = c \cdot (A\u) = c \cdot T(\u)
\end{align*}
for alle vektorer $\u$ og~$\v$ i~$V$, og alle skalarer~$c$.

\begin{ex}
% TODO
\end{ex}

% TODO tekst

\begin{thm}
La $A$ være en $m \times n$-matrise, og la $T \colon \R^n \to \R^m$
være lineærtransformasjonen gitt ved $T(\x) = A\x$.  Da er
\[
\ker T = \Null A
\qquad\text{og}\qquad
\im T = \Col A.
\]
\end{thm}

\begin{ex}
La $\e_1 = \vvS{1}{0}$ og $\e_2 = \vvS{0}{1}$ være enhetsvektorene
i~$\R^2$, og la $T \colon \R^2 \to \R^3$ være en lineærtransformasjon
slik at
\[
T(\e_1) = \vvv{5}{-7}{2}
\quad\text{og}\quad
T(\e_2) = \vvv{-3}{8}{0}
\]
Basert på dette kan vi finne ut hva $T(\x)$ er for en vilkårlig vektor
$\x$ i~$\R^2$.  Vi kan nemlig skrive
\[
\x = \vv{x_1}{x_2} = x_1 \e_1 + x_2 \e_2,
\]
og da får vi:
\begin{align*}
T(\x)
&= T(x_1 \e_1 + x_2 \e_2)
 = x_1 \cdot T(\e_1) + x_2 \cdot T(\e_2) \\
&= x_1 \cdot \vvv{5}{-7}{2} + x_2 \cdot \vvv{-3}{8}{0} \\
&=
\begin{bmatrix}
 5 & -3 \\
-7 &  8 \\
 2 &  0
\end{bmatrix}
\x
\end{align*}
Her har vi endt opp med å skrive lineærtransformasjonen ved hjelp av
en matrise.  La $A$ være denne matrisen:
\[
A = \begin{bmatrix} \ T(\e_1) & T(\e_2)\;\ \end{bmatrix}
  =
\begin{bmatrix}
 5 & -3 \\
-7 &  8 \\
 2 &  0
\end{bmatrix}
\]
Da har vi altså at $T(\x) = A\x$ for alle vektorer~$\x$ i~$\R^2$.
\end{ex}

På samme måte som i dette eksempelet kan vi skrive enhver
lineærtransformasjon $T \colon \R^n \to \R^m$ på matriseform ved å
lage en matrise av vektorene som $T$ sender enhetsvektorene i~$\R^n$
til.

\begin{thm}
\label{thm:standardmatrise}
La $T \colon \R^n \to \R^m$ være en lineærtransformasjon.
Da finnes en $m \times n$-matrise $A$ slik at
\[
T(\x) = A\x
\quad\text{for alle $\x$ i $\R^n$.}
\]
Matrisen $A$ er entydig bestemt av~$T$, og er gitt ved
\[
\begin{bmatrix} \ T(\e_1) & T(\e_2) & \cdots & T(\e_n) \;\ \end{bmatrix},
\]
der $(\e_1, \e_2, \ldots, \e_n)$ er standardbasisen for~$\R^n$.
\end{thm}
% TODO bevis?

\begin{defn}
Matrisen $A$ i teorem~\ref{thm:standardmatrise} kalles
\defterm{standardmatrisen} til lineærtransformasjonen~$T$.
\end{defn}

% TODO eks?

% TODO tekst

\begin{thm}
La $V$ og~$W$ være endeligdimensjonale vektorrom, og la $\B$ og
$\mathscr{C}$ være basiser for henholdsvis $V$ og~$W$.  La
$T \colon V \to W$ være en lineærtransformasjon.  Da finnes en
matrise~$A$, entydig bestemt av~$T$, slik at
\[
\koord{T(\x)}{\mathscr{C}} = A \cdot \koord{\x}{\B}
\]
for alle vektorer~$\x$ i~$V$.
\end{thm}

% TODO forklaring, eks?


\section*{Isomorfi}

\begin{defn}
La $T \colon V \to W$ være en lineærtransformasjon.  En
\defterm{invers} til~$T$ er en lineærtransformasjon $S \colon W \to V$
som er slik at
\begin{align*}
S(T(\v)) &= \v &&\text{for alle $\v$ i~$V$, og} \\
T(S(\w)) &= \w &&\text{for alle $\w$ i~$W$.}
\end{align*}
\end{defn}

\begin{defn}
Hvis $T \colon V \to W$ er en lineærtransformasjon som har en invers,
så er $T$ en \defterm{isomorfi}.  Da sier vi dessuten at vektorrommene
\end{defn}

 %TODO


\kapittelslutt
