\ifx\inkludert\undefined
\documentclass[norsk,a4paper,twocolumn,oneside]{memoir}

\usepackage[utf8]{inputenc}
\usepackage{babel}
\usepackage{amsmath,amssymb,amsthm}
\usepackage[total={17cm,27cm}]{geometry}
\usepackage[table]{xcolor}
%\usepackage{tabularx}
\usepackage{systeme}
%\usepackage{hyperref}
%\usepackage{enumerate}

%\usepackage{sectsty}
\setsecheadstyle{\bfseries\large}
%\subsectionfont{\bf\normalsize}

\usepackage{tikz}
\usetikzlibrary{arrows.meta}

\newcommand{\defterm}[1]{\emph{#1}}

\newcommand{\N}{\mathbb{N}}
\newcommand{\Z}{\mathbb{Z}}
\newcommand{\Q}{\mathbb{Q}}
\newcommand{\R}{\mathbb{R}}

\newcommand{\abs}[1]{|#1|}

\newcommand{\roweq}{\sim}
\DeclareMathOperator{\Span}{Span}

\newcommand{\V}[1]{\mathbf{#1}}
\newcommand{\vv}[2]{\begin{bmatrix} #1 \\ #2 \end{bmatrix}}
\newcommand{\vvv}[3]{\begin{bmatrix} #1 \\ #2 \\ #3 \end{bmatrix}}
\newcommand{\vvvv}[4]{\begin{bmatrix} #1 \\ #2 \\ #3 \\ #4 \end{bmatrix}}
\newcommand{\vn}[2]{\vvvv{#1_1}{#1_2}{\vdots}{#1_#2}}

\newenvironment{amatrix}[1]{% "augmented matrix"
  \left[\begin{array}{*{#1}{c}|c}
}{%
  \end{array}\right]
}

% \newcounter{notatnr}
% \newcommand{\notatnr}[2]
% {\setcounter{notatnr}{#1}%
%  \setcounter{page}{#2}%
% }

\newtheorem{thm}{Teorem}[chapter]
\newtheorem*{thm-nn}{Teorem}
\newtheorem{cor}[thm]{Korollar}
\newtheorem{lem}[thm]{Lemma}
\newtheorem{prop}[thm]{Proposisjon}
\theoremstyle{definition}
\newtheorem{exx}[thm]{Eksempel}
\newtheorem*{defnx}{Definisjon}
\newtheorem*{oppg}{Oppgave}
\newtheorem*{merkx}{Merk}
\newtheorem*{spmx}{Spørsmål}

\newenvironment{defn}
  {\pushQED{\qed}\renewcommand{\qedsymbol}{$\triangle$}\defnx}
  {\popQED\enddefnx}
\newenvironment{ex}
  {\pushQED{\qed}\renewcommand{\qedsymbol}{$\triangle$}\exx}
  {\popQED\endexx}
\newenvironment{merk}
  {\pushQED{\qed}\renewcommand{\qedsymbol}{$\triangle$}\merkx}
  {\popQED\endmerkx}
\newenvironment{spm}
  {\pushQED{\qed}\renewcommand{\qedsymbol}{$\triangle$}\spmx}
  {\popQED\endspmx}

\setlength{\columnsep}{26pt}

\newcommand{\Tittel}[2]{%
\twocolumn[
\begin{center}
\Large
\begin{tabularx}{\textwidth}{cXr}
\cellcolor{black}\color{white}%
\bf {#1} &
#2
\hfill &
\footnotesize TMA4110 høsten 2018
\\ \hline
\end{tabularx}
\end{center}
]}

\newcommand{\tittel}[1]{\Tittel{\arabic{notatnr}}{#1}}

\newcommand{\linje}{%
\begin{center}
\rule{.8\linewidth}{0.4pt}
\end{center}
}


\newcommand{\chapternumber}{}

\makechapterstyle{tma4110}{%
 \renewcommand*{\chapterheadstart}{}
 \renewcommand*{\printchaptername}{}
 \renewcommand*{\chapternamenum}{}
 \renewcommand*{\printchapternum}{\renewcommand{\chapternumber}{\thechapter}}
 \renewcommand*{\afterchapternum}{}
 \renewcommand*{\printchapternonum}{\renewcommand{\chapternumber}{}}
 \renewcommand*{\printchaptertitle}[1]{
\LARGE
\begin{tabularx}{\textwidth}{cXr}
\cellcolor{black}\color{white}%
\textbf{\chapternumber} &
\textbf{##1}
\hfill &
%\footnotesize TMA4110 høsten 2018
\\ \hline
\end{tabularx}%
}
 \renewcommand*{\afterchaptertitle}{\par\nobreak\vskip \afterchapskip}
 % \newcommand{\chapnamefont}{\normalfont\huge\bfseries}
 % \newcommand{\chapnumfont}{\normalfont\huge\bfseries}
 % \newcommand{\chaptitlefont}{\normalfont\Huge\bfseries}
 \setlength{\beforechapskip}{0pt}
 \setlength{\midchapskip}{0pt}
 \setlength{\afterchapskip}{10pt}
}


\newcounter{oppgnr}[chapter]
\newcounter{punktnr}[oppgnr]
\newenvironment{oppgave}
 {\par\noindent\stepcounter{oppgnr}\textbf{{\arabic{oppgnr}}.}}
 {\par\bigskip}
\newenvironment{punkt}
 {\par\smallskip\noindent\stepcounter{punktnr}\textbf{\alph{punktnr})} }
 {\par}

\newcommand{\oppgaver}{\linje\section*{Oppgaver}}

\usepackage{xr}
\externaldocument{tma4110-2018h}
\newcommand{\kapittel}[2]{\setcounter{chapter}{#1}\addtocounter{chapter}{-1}\chapter{#2}}
\newcommand{\kapittelslutt}{\enddocument}
\begin{document}
\chapterstyle{tma4110}
\pagestyle{plain}
\fi


\kapittel{12}{Komplekse tall}
\label{ch:komplekse-tall}
\section*{Den imaginære enheten}
Generelt er det slik at en polynomlikning
\[
a_nx^n+a_{n-1}x^{n-1}+...+a_1x+a_0=0
\]
ikke nødvendigvis har $n$ løsninger. For eksempel har likningen 
\[
x^2+1=0
\]
ingen reell løsning. Men hva skjer om vi prøver å løse den med annengradsformelen? Vi får
\[
x=\frac{\sqrt{-4}}{2}=\frac{\sqrt{4\cdot (-1)}}{2}=\frac{2\sqrt{-1}}{2}=\sqrt{-1}.
\]
Vi gir det en spesiell plass i våre hjerter ved å donere en bokstav
\[
i=\sqrt{-1},
\]
og døpe det \emph{den imaginære enheten}. Nå kan vi skrive kvadratroten av alle negative tall på en pen måte:
\begin{equation*}
\sqrt{-4}=\sqrt{4\cdot (-1)}=\sqrt{4}\cdot \sqrt{(-1)}=2i
\end{equation*}


Hvis vi løser likningen
\[
x^2+x+1=0
\]
får vi 
\[
x=\frac{-1\pm \sqrt{-3}}{2}=\frac{-1\pm \sqrt{3}i}{2}=-\frac{1}{2}\pm\frac{\sqrt{3}}{2}i
\]
Dette inspirerer oss til å definere \emph{komplekse tall} som 
\[
z=a+bi.
\]
Her er $a$ og $b$ reelle tall. De kalles henholdsvis \emph{realdelen} og \emph{imaginærdelen} til $z$, og skrives Re $z$ og Im $z$. 

Hvis du plukker opp en tilfeldig bok i tallteori eller kompleks analyse, er det bevist følgende teorem et eller annet sted. Beviset er litt for hardt for dette kurset, men vi kommer til å trenge resultatet.
\begin{thm}
En polynomlikning
\[
a_nx^n+a_{n-1}x^{n-1}+...+a_1x+a_0=0
\]
har $n$ komplekse løsninger.
\end{thm}

\section*{Operasjoner på komplekse tall}
La $z=2+3i$ og $w=4+5i$. De kan adderes
\[
z+w=2+4+(3+5)i=6+8i,
\]
subtraheres
\[
z-w=2-4+(3-5)i=-2-2i,
\]
og ganges 
\begin{align*}
z\cdot w&=(2+3i)\cdot(4+5i)\\&=2\cdot 4+3\cdot 4i+2\cdot 5i+3\cdot 5 i^2\\&=8-15+(12+10)i=-7+22i.
\end{align*}
De kan også deles
\begin{align*}
\frac{z}{w}&=\frac{2+3i}{4+5i}=\frac{2+3i}{4+5i}\cdot\frac{4-5i}{4-5i}\\&=\frac{8+15+(12-10)i}{16+25}=\frac{22}{41}-\frac{2}{41}i.
\end{align*}
Hva skjedde her? La $z=a+bi$. Vi ganget oppe og nede med '$z$ konjugert'
\[
\overline z =a-bi.
\]
Merk at $z\overline z$ blir et reelt tall. 

\section*{Det komplekse planet}

Et komplekst tall har en viss ytre likhet med vektorer i $\R^2$. Hvis komponentene til $\V x$ er $x_1$ og $x_2$ og enhetsvektorer i koordinatretningene er $\V{e}_1$ og $\V{e}_2$, skriver vi gjerne
\[
\V{x}=x_1 \V{e}_1+x_2 \V{e}_2.
\]
På lignende vis kan vi tenke at realdelen $a$ og imaginærdelen $b$ er komponenter i en vektor, og avmerke $z$ i \emph{det komplekse planet}.
\begin{center}
\begin{tikzpicture}[scale=.42]
\draw[->] (-4,0) -- (7,0);
\draw[->] (0,-4) -- (0,6);
\node[anchor=east] at (9.8,0) {\footnotesize Re $z$};
\node[anchor=south] at (0,6.8) {\footnotesize Im $z$};
\foreach \x in {-4,-3,-2,-1,1,2,3,4,5,6}
\draw (\x,5pt) -- (\x,-5pt);
\foreach \y in {-4,-3,-2,-1,1,2,3,4,5}
\draw (5pt,\y) -- (-5pt,\y);
\filldraw (2,3) circle [radius=3pt] node[anchor=west] {$z=2+3i$};
\filldraw (2,-3) circle [radius=3pt] node[anchor=west] {$\overline z=2-3i$};
\filldraw (4,5) circle [radius=3pt] node[anchor=west] {$w=4+5i$};
%\filldraw (0,1) circle [radius=3pt] node[anchor=east] {$\V{e}_2$};
%\filldraw (-1,-2) circle [radius=3pt] node[anchor=east] {$\V{u}$};
%\filldraw (3,2) circle [radius=3pt] node[anchor=east] {$\V{v}$};
%\filldraw (1,4) circle [radius=3pt] node[anchor=south] {$A \V{e}_1$};
%\filldraw (3,-3) circle [radius=3pt] node[anchor=north] {$A \V{e}_2$};
%\filldraw (-7,2) circle [radius=3pt] node[anchor=east] {$A \V{u}$};
%\filldraw (9,6) circle [radius=3pt] node[anchor=north] {$A \V{v}$};
%\draw[->,shorten <=4pt,shorten >=4pt] (1,0) to[bend right=20] (1,4);
%\draw[->,shorten <=4pt,shorten >=4pt] (0,1) to[bend right=30] (3,-3);
%\draw[->,shorten <=4pt,shorten >=4pt] (-1,-2) to[bend right=20] (-7,2);
%\draw[->,shorten <=4pt,shorten >=4pt] (3,2) to[bend left=20] (9,6);
\end{tikzpicture}
\\
{\small \textit{Det komplekse planet}}
\end{center}

\section*{Polar form}
La $r$ være avstanden fra $z$ til origo i det komplekse planet, og la $\theta$ være vinkelen $z$ gjør med den reelle aksen. Noen enkle geometriske betraktninger gir oss at 
\begin{align*}
a=\text{Re}\, z = r\cos \theta \\
b=\text{Im}\, z = r\sin \theta \\
\end{align*}

\begin{center}
\begin{tikzpicture}[scale=.42]
\draw[->] (-5,0) -- (8,0);
\draw[->] (0,-2.5) -- (0,6);
\node[anchor=west] at (9,0) {\footnotesize Re $z$};
\node[anchor=south] at (0,7) {\footnotesize Im $z$};
\foreach \x in {-5,-4,-3,-2,-1,1,2,3,4,5,6,7}
\draw (\x,5pt) -- (\x,-5pt);
\foreach \y in {-2,-1,1,2,3,4,5}
\draw (5pt,\y) -- (-5pt,\y);
\filldraw (4,5) circle [radius=3pt] node[anchor=west] {$z=a+bi$};
%\filldraw (0,1) circle [radius=3pt] node[anchor=east] {$\V{e}_2$};
%\filldraw (-1,-2) circle [radius=3pt] node[anchor=east] {$\V{u}$};
%\filldraw (3,2) circle [radius=3pt] node[anchor=east] {$\V{v}$};
%\filldraw (1,4) circle [radius=3pt] node[anchor=south] {$A \V{e}_1$};
%\filldraw (3,-3) circle [radius=3pt] node[anchor=north] {$A \V{e}_2$};
%\filldraw (-7,2) circle [radius=3pt] node[anchor=east] {$A \V{u}$};
%\filldraw (9,6) circle [radius=3pt] node[anchor=north] {$A \V{v}$};
\draw[-] (0,0) to (4,5);
\node[anchor=south] at (2,3) {\footnotesize $r$};
\draw (3,0) arc (0:51:3);
\node[anchor=south] at (3.3,1.1) {\footnotesize $\theta$};
%\draw[->,shorten <=4pt,shorten >=4pt] (0,1) to[bend right=30] (3,-3);
%\draw[->,shorten <=4pt,shorten >=4pt] (-1,-2) to[bend right=20] (-7,2);
%\draw[->,shorten <=4pt,shorten >=4pt] (3,2) to[bend left=20] (9,6);
\end{tikzpicture}
\\
{\small \textit{Polare koordinater}}
\end{center}
Formlene over gir $a$ og $b$ som funksjon av $r$ og $\theta$. Litt mer enkel trigonometri gir den andre veien
\begin{align*}
r&=a^2+b^2=z\overline z \\
\theta&= \begin{cases} \tan \frac{b}{a} \quad &\text{for}\; a>0\\ \tan \frac{b}{a} + \pi \quad &\text{for}\;  a<0 \end{cases}
\end{align*}
Formelen for $\theta$ må deles opp i to tilfeller siden
\[
\frac{b}{a}=\frac{-b}{-a} \quad \text{og} \quad \frac{-b}{a}=\frac{b}{-a}.
\]
 Tangens ser ikke forskjell på disse, og skjønner ikke av seg selv om $z$ ligger til høyre eller venstre for den imaginære aksen.


\section*{Eulers formel}
Fra envariabel kalkulus husker du kanskje de tre taylorrekkene
\[
e^{x}=1+x+\frac{x^{2}}{2}+\frac{x^{3}}{3!}+\dots=\sum_{n=0}^{\infty}\frac{x^{n}}{n!}, 
\]
\[
\sin{x}=x-\frac{x^{3}}{3!}+\frac{x^{5}}{5!}-\dots=\sum_{n=0}^{\infty}(-1)^{n}\frac{x^{2n+1}}{(2n+1)!} 
\]
og
\[
\cos{x}=1-\frac{x^{2}}{2!}+\frac{x^{4}}{4!}-\dots=\sum_{n=0}^{\infty}(-1)^{n}\frac{x^{2n}}{(2n)!}
\]
Dersom vi skriver 
\[
\cos{x}=1+\frac{(ix)^{2}}{2!}+\frac{(ix)^{4}}{4!}-\dots=\sum_{n=0}^{\infty}\frac{(ix)^{2n}}{(2n)!}
\]
og 
\[
i\sin{x}=ix+\frac{(ix)^{3}}{3!}+\frac{(ix)^{5}}{5!}-\dots=\sum_{n=0}^{\infty}\frac{(ix)^{2n+1}}{(2n+1)!}
\]
og legger disse sammen, får vi 
\[
\cos x + i\sin x=\sum_{n=0}^{\infty}\frac{(ix)^{n}}{n!}=e^{ix}.
\]
Dette er kun en symbolsk manipulasjon, vi vet strengt tatt ikke hva som skjer med konvergensen til en taylorrekke når du ganger den med $i$, men det er allikevel en plausibel hypotese at det er går fint å definere
\[
e^{ix}=\cos x + i\sin x,
\]
som kalles \emph{Eulers formel}. Vanlige regneregler for eksponensialfunksjonen er lette å utlede herfra.
Tar vi Eulers formel for god fisk, kan vi skrive komplekse tall veldig kompakt på polar form:
\[
z=r(\cos \theta+i\sin \theta)=re^{i\theta}.
\]


\section*{Røtter av komplekse tall}
Et spesialtilfelle av teorem HÅHÅHÅ kan vi analysere med det vi kjenner til så langt, nemlig komplekse $n$-te røtter, altså løsninger av 
\[
z^n=a
\]
for et vilkårlig komplekst talll $a$. Dette er en polynomlikning i $z$, og vi skal se med egne øyne at den alltid har $n$ løsninger. Vi begynner med å skrive $a$ på polar form med valgfritt antall omdreininger rundt origo
\[
a = re^{i \theta}=re^{i \theta+2m\pi}.
\]
Dersom vi skriver 
\[
\sqrt[n]{a}=a^{1/n} = (re^{i \theta+2m\pi})^{1/n}=\sqrt[n]{r}e^{i \theta/n+2m\pi/n},
\]
ser vi at det nå finnes $n$ potensielle verdier for $\sqrt[n]{a}$, alle sammen gyldige løsninger av $z^n=a$. Hvis du velger $0\leq m \leq n-1$ får du ut alle sammen. 
\begin{ex}
Vi finner alle løsninger av ligningen
\[
z^5=-1.
\]
Siden 
\[
-1=e^{i\pi+2m\pi},
\]
får vi at 
\[
\sqrt[5]{-1}=e^{i\pi/5+2m\pi/5},
\]
og for $0\leq m\leq 4$ spyttes ut
\[
\sqrt[5]{-1}=e^{i\pi/5},\;e^{i3\pi/5},\;e^{i5\pi/5}=-1,\; e^{i7\pi/5}\; \text{og}\; e^{i9\pi/5}.
\]
\begin{center}
\begin{tikzpicture}[scale=.42]
\draw[->] (-5,0) -- (7,0);
\draw[->] (0,-3.1) -- (0,5);
\node[anchor=west] at (8,0) {\footnotesize Re $z$};
\node[anchor=south] at (0,6) {\footnotesize Im $z$};
%\foreach \x in {-5,-4,-3,-2,-1,1,2,3,4,5,6,7,8,9}
%\draw (\x,5pt) -- (\x,-5pt);
%\foreach \y in {-2,-1,1,2,3,4,5,6}
%\draw (5pt,\y) -- (-5pt,\y);
\filldraw (-3,0) circle [radius=3pt] node[anchor=south]{};

\filldraw (3*0.80901699437,3*0.587785252294) circle [radius=3pt] node[anchor=south] {};
\node[anchor=west] at (3.2*0.80901699437,3.6*0.587785252294) {$e^{i\pi/5}\; (m=0)$};
\filldraw (-3*.30901699437,3*0.95105651629) circle [radius=3pt] node[anchor=east] {};
\node[anchor=west] at (-8*.30901699437,4*0.95105651629) {$e^{i3\pi/5}\;(m=1)$};
\node[anchor=south] at (-4,.3) {$-1$};
\filldraw (-3*.30901699437,-3*0.95105651629) circle [radius=3pt] node[anchor=east] {};
\node[anchor=west] at (-8*.30901699437,-4*0.95105651629) {$e^{i7\pi/5}$};
\filldraw (3*0.80901699437,-3*0.587785252294) circle [radius=3pt] node[anchor=south] {};
\node[anchor=west] at (3.2*0.80901699437,-3.6*0.587785252294) {$e^{i9\pi/5}\; (m=4)$};

%\filldraw (3,-3) circle [radius=3pt] node[anchor=north] {$A \V{e}_2$};
%\filldraw (-7,2) circle [radius=3pt] node[anchor=east] {$A \V{u}$};
%\filldraw (9,6) circle [radius=3pt] node[anchor=north] {$A \V{v}$};
%\draw[-] (0,0) to (4,5);
%\node[anchor=south] at (2,3) {\footnotesize $r$};
\draw (3,0) arc (0:360:3);
%\node[anchor=south] at (3.3,1.1) {\footnotesize $\theta$};
%\draw[->,shorten <=4pt,shorten >=4pt] (0,1) to[bend right=30] (3,-3);
%\draw[->,shorten <=4pt,shorten >=4pt] (-1,-2) to[bend right=20] (-7,2);
%\draw[->,shorten <=4pt,shorten >=4pt] (3,2) to[bend left=20] (9,6);
\end{tikzpicture}
\\
{\small \textit{Femterøttene til -1}}
\end{center}
Merk hvordan røttene spres jevnt på en sirkel om origo. Merk også at om vi lar $m> 4$ eller $m<0$, får vi røtter som allerede er listet opp.
\end{ex}
\kapittelslutt
