% -*- TeX-master: "oving03"; -*-
\oppgaver{4}

\begin{oppgave}
La $A$ og~$B$ være matriser, og $\V{v}$ en vektor:
\[
A =
\begin{bmatrix}
 0 & 1 &  5 \\
 2 & 3 & -1 \\
-8 & 0 & 2
\end{bmatrix}
\quad
B =
\begin{bmatrix}
1 & 2 & 5 \\
0 & 0 & 3
\end{bmatrix}
\quad
\V{v} = \vvv{7}{2}{-4}
\]
Regn ut (eller forklar hvorfor uttrykkene ikke gir mening):
\begin{punkt}
$AB$
\end{punkt}
\begin{punkt}
$BA$
\end{punkt}
\begin{punkt}
$A^2$
\end{punkt}
\begin{punkt}
$B^2$
\end{punkt}
\begin{punkt}
$A+B$
\end{punkt}
\begin{punkt}
$(A + I_3) \V{v}$
\end{punkt}
\begin{punkt}
$B\V{v}$
\end{punkt}
\begin{punkt}
$BA\V{v}$
\end{punkt}
\end{oppgave}

\begin{losning}

\begin{punkt}
Gir ikke mening.
\end{punkt}

\begin{punkt}
$\begin{bmatrix}
-36 & 7 & -15\\
24 & 0 & 6
\end{bmatrix}$
\end{punkt}

\begin{punkt}
$\begin{bmatrix}
-38 & 3 & 9\\
14 & 11 & 5\\
-16 & -8 & -36
\end{bmatrix}$
\end{punkt}

\begin{punkt}
Gir ikke mening.
\end{punkt}

\begin{punkt}
Gir ikke mening.
\end{punkt}

\begin{punkt}
$\begin{bmatrix}
-11\\
26\\
-68
\end{bmatrix}$
\end{punkt}


\begin{punkt}
$\begin{bmatrix}
-9\\
-12
\end{bmatrix}$
\end{punkt}


\begin{punkt}
$\begin{bmatrix}
-299\\
-204
\end{bmatrix}$
\end{punkt}

\end{losning}

\begin{oppgave}
%(TODO: gitt Av_1 og Av_2, finn Aw, der w er lin.komb. av v_1 og v_2)
La $\V{v}_1$ og $\V{v}_2$ være vektorer i $\mathbb{R}^2$ slik at $A\V{v}_1=\vv{1}{-1}$ og $A\V{v}_2=\vv{2}{3}$ for en $2\times 2$-matrise $A$. Regn ut $A\V{w}$ for $\V{w}=2\V{v}_1-\V{v}_2$.
\end{oppgave}

\begin{losning}
$\vv{0}{-5}$
\end{losning}


\begin{oppgave}
Finn en kvadratisk matrise~$A$ slik at:
\begin{punkt}
$A \vv{1}{0} = \vv{3}{5}$
og
$A \vv{0}{1} = \vv{-1}{0}$.
\end{punkt}
\begin{punkt}
$A \vv{1}{0} = \vv{2}{8}$
og
$A \vv{1}{1} = \vv{3}{4}$.
\end{punkt}
\begin{punkt}
$A \vv{1}{0} = \vv{3}{5}$ 
og
$A \vv{0}{1} = \vv{-1}{0}$.
\end{punkt}
\begin{punkt}
$A \begin{bmatrix}
1\\
0\\
0
\end{bmatrix}
 =  \begin{bmatrix}
0\\
1\\
0
\end{bmatrix} 
 $, $A \begin{bmatrix}
0\\
1\\
0
\end{bmatrix}
 =  \begin{bmatrix}
99\\
-1\\
145
\end{bmatrix} 
$ \newline 
og $A \begin{bmatrix}
0\\
0\\
1
\end{bmatrix}
 =  \begin{bmatrix}
-3\\
0\\
2
\end{bmatrix}.
 $
\end{punkt}

\begin{punkt}
$A \begin{bmatrix}
1\\
1\\
1
\end{bmatrix}
 =  \begin{bmatrix}
1\\
0\\
2
\end{bmatrix} 
 $, $A \begin{bmatrix}
0\\
1\\
1
\end{bmatrix}
 =  \begin{bmatrix}
3\\
4\\
5
\end{bmatrix} 
$ \newline 
og $A \begin{bmatrix}
0\\
0\\
-1
\end{bmatrix}
 =  \begin{bmatrix}
0\\
4\\
-1
\end{bmatrix}.
 $
\end{punkt}
\end{oppgave}




\begin{losning}
La $\V{e}_i$ være vektoren med 1 i komponent $i$ og null ellers. Husk at for en $n\times n$-matrise er $A\V{e}_i$ kolonnevektor nummer $i$. Vi kan derfor løse \textbf{a)}, \textbf{c)} og \textbf{d)} direkte; det er jo kolonnevektorene til $A$ som er oppgitt. I del \textbf{b)} og \textbf{e)} må vi skrive ut ligningene vi får, og deretter løse dem direkte. Eksempelvis, hvis vi skriver $A=\begin{bmatrix}
a_1 & a_2\\
a_3 & a_4
\end{bmatrix}
$ og setter inn kravene fra del \textbf{b)} får vi $$\vv{a_1}{a_3}=\vv{2}{8} \text{ og } \vv{a_1+a_2}{a_3+a_4}=\vv{3}{4}.$$ Dette er fire ligninger med fire ukjente. Tilsvarende fremgangsmåte fungerer i \textbf{e)}, men nå er $A$ en $3\times 3$-matrise. Du kan sjekke om det endelige svaret ditt er korrekt; tilfredstiller $A$ kravene i oppgaven?
\end{losning}




\begin{oppgave}
Løs ligningene
\[
A\V{x}=\V{b}_1\text{ og }A\V{x}=\V{b}_2
\]
samtidig for \ldots


\begin{punkt}
$A=\begin{bmatrix}
1 & 1\\
1 & 1
\end{bmatrix}
$, $\V{b}_1=\vv{1}{1}$ og $\V{b}_2=\vv{2}{2}$
\end{punkt}


\begin{punkt}
$A=\begin{bmatrix}
1 & 1\\
1 & 1
\end{bmatrix}
$, $\V{b}_1=\vv{1}{1}$ og $\V{b}_2=\vv{1}{-1}$
\end{punkt}


\begin{punkt}
$A=\begin{bmatrix}
1 & 1\\
1 & -1
\end{bmatrix}
$, $\V{b}_1=\vv{1}{2}$ og $\V{b}_2=\vv{3}{-1}$
\end{punkt}


\begin{punkt}
$A=\begin{bmatrix}
	1 & 2 & 3\\
	2 & 3 & 4\\
	3 & 4 & 5\\
	4 & 5 & 6
	\end{bmatrix}
	$,
$\V{b}_1=
	\begin{bmatrix}
	-3  \\
	-7 \\
	-3 \\
	0
	\end{bmatrix}
$ og 
$\V{b}_2=
	\begin{bmatrix}
	1  \\
	1 \\
	1 \\
	1
	\end{bmatrix}
$
\end{punkt}



\end{oppgave}


\begin{losning}

\begin{punkt}
Begge ligningene har uendelig mange løsninger.
\end{punkt}


\begin{punkt}
Den første har uendelig mange løsninger; den andre har ingen løsning.
\end{punkt}

\begin{punkt}
Den første har entydig løsning $\begin{bmatrix} 
\frac{3}{2}\\
-\frac{1}{2}
\end{bmatrix}$; den andre har entydig løsning $\begin{bmatrix} 
\frac{1}{2}\\
\frac{1}{2}
\end{bmatrix}.$

\end{punkt}


\begin{punkt}
Den første har ingen løsninger; den andre har uendelig mange løsninger.
\end{punkt}

\end{losning}






\begin{oppgave}
La $A$ og $B$ være to gitte $2\times 2$-matriser. Betrakt ligningen $$AX=B,$$ hvor $X$ er en ukjent $2\times 2$-matrise.

\begin{punkt}
Forklar hvorfor den oppgitte ligningen er ekvaivalent med å løse to $2\times 2$-ligningssystem samtidig. Hvordan generaliseres denne påstanden for $n\times n$-matriser?
\end{punkt}

\begin{punkt}
Løs den oppgitte ligningen for $A=\begin{bmatrix}
1 & 1\\
1 & 1
\end{bmatrix}
$ og $B=\begin{bmatrix}
1 & 2\\
1 & 2
\end{bmatrix}.$
\end{punkt}

\begin{punkt}
Løs den oppgitte ligningen for $A=\begin{bmatrix}
1 & 1\\
1 & -1
\end{bmatrix}
$ og $B=\begin{bmatrix}
1 & 3\\
2 & -1
\end{bmatrix}.$
\end{punkt}

\end{oppgave}


\begin{losning}
\begin{punkt}
Uttrykk $X$ og $B$ ved kolonnevektorer: $X=\begin{bmatrix}
\V{x}_1 & \V{x}_2
\end{bmatrix}$ og $B=\begin{bmatrix}
\V{b}_1 & \V{b}_2
\end{bmatrix}.$ Nå kan vi reformulere $AX=B$ som $\begin{bmatrix}
A\V{x}_1 & A\V{x}_2
\end{bmatrix}=\begin{bmatrix}
\V{b}_1 & \V{b}_2
\end{bmatrix}.$ Vi skal altså løse ligningene $A\V{x}_1=\V{b}_1$ og $A\V{x}_2=\V{b}_2$. Dette kan selvfølgelig gjøres samtidig.
\end{punkt}

\begin{punkt}
Se \textbf{4.4. a)}.
\end{punkt}


\begin{punkt}
Se \textbf{4.4. c)}.
\end{punkt}

\end{losning}

\begin{oppgave}
La $A$, $B$ og $C$ være gitte $2\times 2$-matriser. Betrakt ligningen $$AX+XB=C,$$ hvor $X$ er en ukjent $2\times 2$-matrise.

\begin{punkt}
Hvorfor kan man \emph{ikke} løse denne ligningen som to $2\times 2$-ligningssystem samtidig?
\end{punkt}

\begin{punkt}
Skriv om ligningen til fire ligninger med fire ukjente. Hva er totalmatrisen? 
\end{punkt}


\begin{punkt}
Løs den oppgitte ligningen for 
$A=\begin{bmatrix}
-1 & 1\\
1 & -1
\end{bmatrix}$, $B=\begin{bmatrix}
2 & 1\\
1 & 2
\end{bmatrix}$ og $C=\begin{bmatrix}
1 & 1\\
1 & 1
\end{bmatrix}.$
\end{punkt}

\end{oppgave}


\begin{losning}

\begin{punkt}
La $X=\begin{bmatrix}
x_1 & x_2\\
x_3 & x_4
\end{bmatrix}.$ I forrige oppgave ble ligningene for kolonnevektorene $\V{x}_1=\begin{bmatrix}
x_1\\
x_3
\end{bmatrix}$ og $\V{x}_2=\begin{bmatrix}
x_2\\
x_4
\end{bmatrix}$ uavhengige. Vi kunne altså finne $x_1$ og $x_3$ uten at dette påvirket $x_2$ og $x_4$, og vice versa. Dette er ikke sant i denne oppgaven.
\end{punkt}

\begin{punkt}
Notasjon: $A=\begin{bmatrix}
a_1 & a_2\\
a_3 & a_4
\end{bmatrix}$, $B=\begin{bmatrix}
b_1 & b_2\\
b_3 & b_4
\end{bmatrix}$ og $C=\begin{bmatrix}
c_1 & c_2\\
c_3 & c_4
\end{bmatrix}$. Totalmatrisen blir da $$\begin{bmatrix}
a_1+b_1 & b_3 & a_2 & 0 & c_1\\
b_2 & a_1+b_4 & 0 & a_2 & c_2\\
a_3 & 0 & a_4+b_1 & b_3 & c_3\\
0 & a_3 & b_2 & a_4+b_4 & c_4
\end{bmatrix}.$$ Merk: avhengig av hvordan du nummererte ligningene kan du ha en totalmatrise hvor radene er byttet om på.
\end{punkt}


\begin{punkt}
Med oppgitte matriser blir totalmatrisen fra \textbf{b)}$$\begin{bmatrix}
1 & 1 & 1 & 0 & 1\\
1 & 1 & 0 & 1 & 1\\
1 & 0 & 1 & 1 & 1\\
0 & 1 & 1 & 1 & 1
\end{bmatrix},$$ som har løsning $x_1=\frac{1}{3}$, $x_2=\frac{1}{3}$, $x_3=\frac{1}{3}$ og $x_4=\frac{1}{3}$. Løsningen er altså   
$$X=\begin{bmatrix}
\frac{1}{3} & \frac{1}{3}\\
\frac{1}{3} & \frac{1}{3}
\end{bmatrix}.$$
\end{punkt}

\end{losning}


\begin{oppgave}
Er følgende matriser invertible? I så fall, finn den inverse og sjekk at svaret ditt er riktig. 

\begin{punkt}
$\begin{bmatrix}
1 & 1\\
1 & 1
\end{bmatrix}$
\end{punkt}

\begin{punkt}
$\begin{bmatrix}
1 & 1\\
1 & -1
\end{bmatrix}$
\end{punkt}

\begin{punkt}
$\begin{bmatrix}
1 & 2\\
2 & 3\\
3 & 4
\end{bmatrix}$
\end{punkt}

\begin{punkt}
$\begin{bmatrix}
1 & 2 & 3\\
2 & 3 & 4\\
3 & 4 & 5
\end{bmatrix}$
\end{punkt}

\begin{punkt}
$\begin{bmatrix}
1 & 0 & 0\\
1 & 1 & 0\\
1 & 1 & 1
\end{bmatrix}$
\end{punkt}


\end{oppgave}


\begin{losning}
Hint: Du kan sjekke om en $n\times n$-matrise er inverterbar hvis du prøver å finne den inverse ved regning. Dette koker altså ned til å radredusere $\begin{bmatrix}A & I \end{bmatrix}$.

\begin{punkt}
Ikke invertibel.
\end{punkt}

\begin{punkt}
$\begin{bmatrix}
\frac{1}{2} & \frac{1}{2}\\
\frac{1}{2} & -\frac{1}{2}
\end{bmatrix}.$
\end{punkt}

\begin{punkt}
Ikke invertibel (den er ikke kvadratisk).
\end{punkt}


\begin{punkt}
Ikke invertibel.
\end{punkt}


\begin{punkt}
$\begin{bmatrix}
1 & 0 & 0 \\
-1 & 1 & 0 \\
0 & -1 & 1
\end{bmatrix}.$
\end{punkt}


\end{losning}


\begin{oppgave}
% (TODO: A 3x3-matrise, finn c og v\ ne 0 slik at Av = cv

La $A=\begin{bmatrix}
1 & 0\\
0 & 0
\end{bmatrix}$. Kan du finne en ikke-null vektor $\vv{a}{b}$ slik at $A\vv{a}{b}=c\vv{a}{b}$ for en konstant $c$? I så fall, for hvilke valg av $c$ eksisterer en slik ikke-null vektor? Kan du gi en gometrisk forklaring på hva som skjer når du multipliserer $A$ med vektorerene i de ulike tilfellene for $c$?
\end{oppgave}



\begin{losning}

Konstanten $c$ kan være 0 eller 1. For $c=1$ må $b=0$ men vi kan velge $a$ fritt. Dette svarer geometrisk til at vektorer som kun har komponent langs $y$-aksen blir null-vektoren etter multiplikasjon med $A$. For $c=0$ må $a=0$ men vi kan velge $b$ fritt. Dette svarer geometrisk til at vektorer som kun har komponent langs $x$-aksen ikke påvirkes av multiplikasjon med $A$.


Hint: Vi ønsker at $A\vv{a}{b}=c\vv{a}{b}$, som gir ligningene $a=c\cdot a$ og $0=b\cdot c$. Det er nå to muligheter: 1) $c=1$ og $b=0$, eller 2) $c=0$ og $a=0$.



\end{losning}




\begin{oppgave}
% TODO: oppgave med å se grafisk hvordan en matrise (helst 2x2) transformerer vektorer

La $A=\begin{bmatrix}
0 & -1\\
1 & 0
\end{bmatrix}.$

\begin{punkt}
Skisser $A\vv{1}{0}$ og $A\vv{0}{1}$ i planet. Hva har skjedd med $\vv{1}{0}$ og $\vv{0}{1}$ geometrisk?
\end{punkt}

\begin{punkt}
Hva skjer -  geometrisk - med en vilkårlig vektor $\vv{a}{b}$ når vi multipliserer med A?

Hint: $\vv{a}{b}=a\vv{1}{0}+b\vv{0}{1}$ og $A\vv{a}{b}=aA\vv{1}{0}+bA\vv{0}{1}$.
\end{punkt}

\begin{punkt}
Kan du gi en geometrisk forklaring på hvorfor $A$ burde være invertibel? Har du et forslag til hvordan den inverse til $A$ burde påvirke en vilkårlig vektor ved multiplikasjon?

Hint: Hva er den inverse geometriske operasjonen til det du fant i $\textbf{b)}$?
\end{punkt}

\begin{punkt}
Finn den inverse matrisen til $A$, og verifiser svaret ditt i del $\textbf{c)}$.
\end{punkt}

\end{oppgave}

\begin{losning}

\begin{punkt}

Vektorene blir rotert 90 grader mot klokken.

\begin{center}
	\begin{tikzpicture}
	\draw[->] (-2,0) -- (2,0) node[right] {$x$};
	\draw[->] (0,-1) -- (0,2) node[above] {$y$};
	\draw[->] (0,0) -- (0,1) node[right] {$A\vv{1}{0}$};
	\draw[->] (0,0) -- (-1,0) node[above] {$A\vv{0}{1}$};
	%\draw[->] (0,0) -- (7/2,-1) node[below] {$\frac{1}{2}\V{u}-2\V{v}$};
	%\draw[->] (0,0) -- (2,3) node[above] {$\V{u}+\V{v}$};
	\end{tikzpicture}
\end{center}


\end{punkt}

\begin{punkt}
Hint: Dersom vi deler opp en vilkårlig vektor $\vv{a}{b}$ i vektoren $\vv{a}{0}$ langs $x$-aksen og $\vv{0}{b}$ langs $y$-aksen, ser vi fra hintet i oppgaveteksten og $\textbf{a)}$ at hver av disse roteres 90 grader mot klokken. Altså roteres hele vektoren 90 grader mot klokken.
\end{punkt}


\begin{punkt}
Den omvendte geometriske operasjonen er å rotere 90 grader med klokken. Derfor skulle man tro at det finnes en matrise som gjør dette og er inversen til $A$.
\end{punkt}


\begin{punkt}
Vi finner at $A^{-1}=\begin{bmatrix}
0 & 1\\
-1 & 0
\end{bmatrix}.$ Du kan nå gjenta \textbf{a)}-\textbf{b)} med $A^{-1}$ i stedet for $A$, og dermed sjekke at $A^{-1}$ roterer en vektor 90 grader med klokken.
\end{punkt}

\end{losning}



\begin{oppgave}
La $A=\begin{bmatrix}
\V{a}_1 & \V{a}_2 & \cdots & \V{a}_n
\end{bmatrix}
$ være en $n\times n$-matrise slik at $\V{a}_i\cdot \V{a}_i=1$ og $\V{a}_i\cdot \V{a}_j=1$ for $i\neq j$.

\begin{punkt}
Vis at $A^{-1}=A^T$.

Hint: Hva er radene til $A^T$?
\end{punkt}

\begin{punkt}
Sjekk at $A=\begin{bmatrix}
\frac{1}{\sqrt{2}} & \frac{1}{\sqrt{2}}\\
-\frac{1}{\sqrt{2}} & \frac{1}{\sqrt{2}}
\end{bmatrix}$
tilfredstiller de oppgitte kravene. Bruk \textbf{a)} for å finne $A^{-1}$. Sjekk at svaret ditt er riktig.
\end{punkt}

\end{oppgave}


\begin{losning}

\begin{punkt}
Hint: $A^T=\begin{bmatrix}
\V{a}_1^T\\
\V{a}_2^T\\
\vdots\\
\V{a}_n^T
\end{bmatrix}$. Derfor blir
$$A^TA=\begin{bmatrix}
\V{a}_1^T \V{a}_1 & \V{a}_1^T \V{a}_2 & \cdots & \V{a}_1^T\V{a}_n\\
\V{a}_2^T \V{a}_1 & \V{a}_2^T \V{a}_2 & \cdots & \V{a}_2^T\V{a}_n\\
\vdots & \vdots & \ddots & \vdots\\
\V{a}_n^T \V{a}_1 & \V{a}_n^T \V{a}_2 & \cdots & \V{a}_n^T\V{a}_n
\end{bmatrix}.
$$ Hva er antagelsen i oppgaven? Kan du fullføre opgaven nå?

\end{punkt}


\begin{punkt}
Den oppgitte matrisen tilfredstiller kravene ettersom $$\vv{\frac{1}{\sqrt{2}}}{-\frac{1}{\sqrt{2}}}\cdot \vv{\frac{1}{\sqrt{2}}}{-\frac{1}{\sqrt{2}}}=1,$$ $$\vv{\frac{1}{\sqrt{2}}}{\frac{1}{\sqrt{2}}}\cdot \vv{\frac{1}{\sqrt{2}}}{\frac{1}{\sqrt{2}}}=1,$$ $$\vv{\frac{1}{\sqrt{2}}}{\frac{1}{\sqrt{2}}}\cdot \vv{\frac{1}{\sqrt{2}}}{-\frac{1}{\sqrt{2}}}=0 \text{ og}$$ $$\vv{\frac{1}{\sqrt{2}}}{-\frac{1}{\sqrt{2}}}\cdot \vv{\frac{1}{\sqrt{2}}}{\frac{1}{\sqrt{2}}}=0.$$ Inversen er derfor $A^{-1}=A^T=\begin{bmatrix}
\frac{1}{\sqrt{2}} & -\frac{1}{\sqrt{2}}\\
\frac{1}{\sqrt{2}} & \frac{1}{\sqrt{2}}
\end{bmatrix}.$ Du kan sjekke dette ved å bekrefte at både $AA^T$ og $A^TA$ er lik $I$.
\end{punkt}


\end{losning}



