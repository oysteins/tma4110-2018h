\ifx\inkludert\undefined
\documentclass[norsk,a4paper,twocolumn,oneside]{memoir}

\usepackage[utf8]{inputenc}
\usepackage{babel}
\usepackage{amsmath,amssymb,amsthm}
\usepackage[total={17cm,27cm}]{geometry}
\usepackage[table]{xcolor}
%\usepackage{tabularx}
\usepackage{systeme}
%\usepackage{hyperref}
%\usepackage{enumerate}

%\usepackage{sectsty}
\setsecheadstyle{\bfseries\large}
%\subsectionfont{\bf\normalsize}

\usepackage{tikz}
\usetikzlibrary{arrows.meta}

\newcommand{\defterm}[1]{\emph{#1}}

\newcommand{\N}{\mathbb{N}}
\newcommand{\Z}{\mathbb{Z}}
\newcommand{\Q}{\mathbb{Q}}
\newcommand{\R}{\mathbb{R}}

\newcommand{\abs}[1]{|#1|}

\newcommand{\roweq}{\sim}
\DeclareMathOperator{\Span}{Span}

\newcommand{\V}[1]{\mathbf{#1}}
\newcommand{\vv}[2]{\begin{bmatrix} #1 \\ #2 \end{bmatrix}}
\newcommand{\vvv}[3]{\begin{bmatrix} #1 \\ #2 \\ #3 \end{bmatrix}}
\newcommand{\vvvv}[4]{\begin{bmatrix} #1 \\ #2 \\ #3 \\ #4 \end{bmatrix}}
\newcommand{\vn}[2]{\vvvv{#1_1}{#1_2}{\vdots}{#1_#2}}

\newenvironment{amatrix}[1]{% "augmented matrix"
  \left[\begin{array}{*{#1}{c}|c}
}{%
  \end{array}\right]
}

% \newcounter{notatnr}
% \newcommand{\notatnr}[2]
% {\setcounter{notatnr}{#1}%
%  \setcounter{page}{#2}%
% }

\newtheorem{thm}{Teorem}[chapter]
\newtheorem*{thm-nn}{Teorem}
\newtheorem{cor}[thm]{Korollar}
\newtheorem{lem}[thm]{Lemma}
\newtheorem{prop}[thm]{Proposisjon}
\theoremstyle{definition}
\newtheorem{exx}[thm]{Eksempel}
\newtheorem*{defnx}{Definisjon}
\newtheorem*{oppg}{Oppgave}
\newtheorem*{merkx}{Merk}
\newtheorem*{spmx}{Spørsmål}

\newenvironment{defn}
  {\pushQED{\qed}\renewcommand{\qedsymbol}{$\triangle$}\defnx}
  {\popQED\enddefnx}
\newenvironment{ex}
  {\pushQED{\qed}\renewcommand{\qedsymbol}{$\triangle$}\exx}
  {\popQED\endexx}
\newenvironment{merk}
  {\pushQED{\qed}\renewcommand{\qedsymbol}{$\triangle$}\merkx}
  {\popQED\endmerkx}
\newenvironment{spm}
  {\pushQED{\qed}\renewcommand{\qedsymbol}{$\triangle$}\spmx}
  {\popQED\endspmx}

\setlength{\columnsep}{26pt}

\newcommand{\Tittel}[2]{%
\twocolumn[
\begin{center}
\Large
\begin{tabularx}{\textwidth}{cXr}
\cellcolor{black}\color{white}%
\bf {#1} &
#2
\hfill &
\footnotesize TMA4110 høsten 2018
\\ \hline
\end{tabularx}
\end{center}
]}

\newcommand{\tittel}[1]{\Tittel{\arabic{notatnr}}{#1}}

\newcommand{\linje}{%
\begin{center}
\rule{.8\linewidth}{0.4pt}
\end{center}
}


\newcommand{\chapternumber}{}

\makechapterstyle{tma4110}{%
 \renewcommand*{\chapterheadstart}{}
 \renewcommand*{\printchaptername}{}
 \renewcommand*{\chapternamenum}{}
 \renewcommand*{\printchapternum}{\renewcommand{\chapternumber}{\thechapter}}
 \renewcommand*{\afterchapternum}{}
 \renewcommand*{\printchapternonum}{\renewcommand{\chapternumber}{}}
 \renewcommand*{\printchaptertitle}[1]{
\LARGE
\begin{tabularx}{\textwidth}{cXr}
\cellcolor{black}\color{white}%
\textbf{\chapternumber} &
\textbf{##1}
\hfill &
%\footnotesize TMA4110 høsten 2018
\\ \hline
\end{tabularx}%
}
 \renewcommand*{\afterchaptertitle}{\par\nobreak\vskip \afterchapskip}
 % \newcommand{\chapnamefont}{\normalfont\huge\bfseries}
 % \newcommand{\chapnumfont}{\normalfont\huge\bfseries}
 % \newcommand{\chaptitlefont}{\normalfont\Huge\bfseries}
 \setlength{\beforechapskip}{0pt}
 \setlength{\midchapskip}{0pt}
 \setlength{\afterchapskip}{10pt}
}


\newcounter{oppgnr}[chapter]
\newcounter{punktnr}[oppgnr]
\newenvironment{oppgave}
 {\par\noindent\stepcounter{oppgnr}\textbf{{\arabic{oppgnr}}.}}
 {\par\bigskip}
\newenvironment{punkt}
 {\par\smallskip\noindent\stepcounter{punktnr}\textbf{\alph{punktnr})} }
 {\par}

\newcommand{\oppgaver}{\linje\section*{Oppgaver}}

\usepackage{xr}
\externaldocument{tma4110-2018h}
\newcommand{\kapittel}[2]{\setcounter{chapter}{#1}\addtocounter{chapter}{-1}\chapter{#2}}
\newcommand{\kapittelslutt}{\enddocument}
\begin{document}
\chapterstyle{tma4110}
\pagestyle{plain}
\fi


\kapittel{6}{Matriser}
\label{ch:matriser}


\section*{Definisjoner og notasjon}

En \defterm{$m \times n$-matrise} er en rektangulær tabell med tall
som har $m$~tall i høyden og $n$~tall i bredden:
\[
\begin{bmatrix}
a_{11} & a_{12} & \cdots & a_{1n} \\
a_{21} & a_{22} & \cdots & a_{2n} \\
\vdots & \vdots & \ddots & \vdots \\
a_{m1} & a_{m2} & \cdots & a_{mn}
\end{bmatrix}
\]
\defterm{Kolonnene} i matrisen er følgende kolonnevektorer:
\[
\vvvv{a_{11}}{a_{21}}{\vdots}{a_{m1}}\qquad
\vvvv{a_{12}}{a_{22}}{\vdots}{a_{m2}}\qquad
\cdots\qquad
\vvvv{a_{1n}}{a_{2n}}{\vdots}{a_{mn}}\qquad
\]
\defterm{Radene} i matrisen er følgende radvektorer:
\begin{gather*}
\begin{bmatrix}
a_{11} & a_{12} & \cdots & a_{1n}
\end{bmatrix}
\\
\begin{bmatrix}
a_{21} & a_{22} & \cdots & a_{2n}
\end{bmatrix}
\\
\vdots
\\
\begin{bmatrix}
a_{m1} & a_{m2} & \cdots & a_{mn}
\end{bmatrix}
\end{gather*}

\begin{ex}
Her er et eksempel på en $2 \times 3$-matrise:
\[
\begin{bmatrix}
5 & 0 & -2 \\
3 & 1 &  4
\end{bmatrix}
\]
Kolonnene i denne matrisen er:
\[
\vv{5}{3},\quad
\vv{0}{1}\quad\text{og}\quad
\vv{-2}{4}
\]
Radene er:
\[
\begin{bmatrix}
5 & 0 & -2
\end{bmatrix}
\quad\text{og}\quad
\begin{bmatrix}
3 & 1 &  4
\end{bmatrix}
\qedhere
\]
\end{ex}

Noen ganger har vi en liste med vektorer, si
\[
\V{v}_1,\ \V{v}_2,\ \ldots,\ \V{v}_n,
\]
og vil lage en matrise som har disse vektorene som kolonner.
Den matrisen kan vi skrive slik:
\[
\begin{bmatrix}
\V{v}_1 & \V{v}_2 & \cdots & \V{v}_n
\end{bmatrix}
\]
Hvis vektorene ligger i~$\R^m$, blir dette en $m \times n$-matrise.

\begin{ex}
La
\[
\V{v}_1 = \vvv{2}{0}{4}
\qquad\text{og}\qquad
\V{v}_2 = \vvv{1}{1}{2}
\]
være to vektorer i~$\R^3$.  Matrisen
$\begin{bmatrix} \V{v}_1 & \V{v}_2 \end{bmatrix}$ med disse vektorene
som kolonner blir da følgende $3 \times 2$-matrise:
\[
\begin{bmatrix} \V{v}_1 & \V{v}_2 \end{bmatrix}
=
\begin{bmatrix}
2 & 1 \\
1 & 0 \\
2 & 4
\end{bmatrix}
\qedhere
\]
\end{ex}


\section*{Produkt av matrise og vektor}

La
\[
A =
\begin{bmatrix}
\V{a}_1 & \V{a}_2 & \cdots & \V{a}_n
\end{bmatrix}
\]
være en $m \times n$-matrise med vektorene $\V{a}_1$, $\V{a}_2$,
\ldots, $\V{a}_n$ som kolonner, og la
\[
\V{v} = \vn{v}{n}
\]
være en vektor i~$\R^n$.  Vi definerer produktet $A\V{v}$ av $A$
og~$V$ som lineærkombinasjonen av kolonnene i~$A$ med tallene
i~$\V{v}$ som vekter:
\[
A \V{v} = \V{a}_1 v_1 + \V{a}_2 v_2 + \cdots + \V{a}_n v_n
\]
Merk at produktet~$A \V{v}$ bare er definert når bredden av matrisen
$A$ er lik høyden av vektoren~$\V{v}$.

\begin{ex}
Vi regner ut produktet av en $2 \times 3$-matrise og en vektor
i~$\R^3$:
\[
\begin{bmatrix}
5 & 0 & -2 \\
3 & 1 &  4
\end{bmatrix}
\vvv{2}{-1}{3}
=
\vv{5}{3} \cdot 2 +
\vv{0}{1} \cdot (-1) +
\vv{-2}{4} \cdot 3
=
\vv{4}{17}
\]
Merk at resultatet blir en vektor i~$\R^2$.
\end{ex}

\begin{thm}
Hvis $A$ er en $m \times n$-matrise, $\V{v}$ og~$\V{w}$ er vektorer
i~$\R^n$ og $c$ er et tall, så har vi følgende likheter:
\[
A (\V{v} + \V{w}) = A \V{v} + A \V{w}
\qquad\text{og}\qquad
A (c \V{v}) = c (A \V{v})
\]
\end{thm}


\section*{Sum av matriser}% og multiplikasjon med skalar}

La $A$ og~$B$ være to $m \times n$-matriser:
\[
A =
\begin{bmatrix}
a_{11} & \cdots & a_{1n} \\
a_{21} & \cdots & a_{2n} \\
\vdots & \ddots & \vdots \\
a_{m1} & \cdots & a_{mn}
\end{bmatrix}
\qquad
B =
\begin{bmatrix}
b_{11} & \cdots & b_{1n} \\
b_{21} & \cdots & b_{2n} \\
\vdots & \ddots & \vdots \\
b_{m1} & \cdots & b_{mn}
\end{bmatrix}
\]
% \[
% A =
% \begin{bmatrix}
% a_{11} & a_{12} & \cdots & a_{1n} \\
% a_{21} & a_{22} & \cdots & a_{2n} \\
% \vdots & \vdots & \ddots & \vdots \\
% a_{m1} & a_{m2} & \cdots & a_{mn}
% \end{bmatrix}
% \qquad
% B =
% \begin{bmatrix}
% b_{11} & b_{12} & \cdots & b_{1n} \\
% b_{21} & b_{22} & \cdots & b_{2n} \\
% \vdots & \vdots & \ddots & \vdots \\
% b_{m1} & b_{m2} & \cdots & b_{mn}
% \end{bmatrix}
% \]
Vi definerer summen $A+B$ på den mest åpenbare måten -- vi legger
sammen tallene fra de to matrisene i hver posisjon, og får en ny
$m \times n$-matrise:
\[
A + B =
\begin{bmatrix}
a_{11} + b_{11} & a_{12} + b_{12} & \cdots & a_{1n} + b_{1n} \\
a_{21} + b_{21} & a_{22} + b_{22} & \cdots & a_{2n} + b_{2n} \\
\vdots          & \vdots          & \ddots & \vdots          \\
a_{m1} + b_{m1} & a_{m2} + b_{m2} & \cdots & a_{mn} + b_{mn} \\
\end{bmatrix}
\]

Produktet av en matrise og et tall defineres også på den åpenbare
måten:
\[
c A =
\begin{bmatrix}
c \cdot a_{11} & c \cdot a_{12} & \cdots & c \cdot a_{1n} \\
c \cdot a_{21} & c \cdot a_{22} & \cdots & c \cdot a_{2n} \\
\vdots         & \vdots         & \ddots & \vdots         \\
c \cdot a_{m1} & c \cdot a_{m2} & \cdots & c \cdot a_{mn}
\end{bmatrix}
\]

% \begin{ex}
% TODO: nødvendig med eksempel?
% \end{ex}

\begin{thm}
Hvis $A$ og~$B$ er $m \times n$-matriser, $\V{v}$ er en vektor
i~$\R^n$ og $c$ er et tall, så har vi følgende likheter:
\[
(A + B) \V{v} = A \V{v} + B \V{v}
\qquad\text{og}\qquad
(c A) \V{v} = c (A \V{v})
\]
\end{thm}


\section*{Matrisemultiplikasjon}

Nå kommer vi til den viktigste av de aritmetiske operasjonene vi kan
gjøre med matriser, nemlig \emph{matrisemultiplikasjon}.

Det er litt mer komplisert å beskrive hvordan vi multipliserer
matriser enn hvordan vi summerer dem.  Det er imidlertid en god grunn
til at det er slik.  Vi kunne valgt å definere multiplikasjon av
matriser på tilsvarende måte som sum, men det ville ikke blitt
spesielt nyttig.  Vi vil nemlig at matrisemultiplikasjon skal oppføre
seg pent sammen med multiplikasjon av matriser med vektorer.  Spesielt
vil vi at følgende likhet skal holde:
\[
(AB) \V{v} = A (B \V{v})
\]
Vi vil altså at vi fritt skal kunne flytte parentesene, akkurat slik
vi kan gjøre med et produkt av tre tall.

Hvordan kan vi definere produkt av matriser slik at dette fungerer?
La oss først se på et eksempel.

\begin{ex}
% TODO eksempel med 3x3 e.l.?
\end{ex}

\begin{ex}
\[
A =
\begin{bmatrix}
a_{11} & a_{12} \\
a_{21} & a_{22}
\end{bmatrix}
\qquad
B =
\begin{bmatrix}
b_{11} & b_{12} \\
b_{21} & b_{22}
\end{bmatrix}
\]
\[
B \vv{1}{0} = \vv{b_{11}}{b_{21}}
\]


\[
\begin{bmatrix}
a_{11} b_{11} + a_{12} b_{21} & a_{11} b_{12} + a_{12} b_{22} \\
a_{21} b_{11} + a_{22} b_{21} & a_{21} b_{12} + a_{21} b_{22}
\end{bmatrix}
\]
\end{ex}


\[
AB =
\begin{bmatrix}
A \V{b}_1 & A \V{b}_2 & \cdots & A \V{b}_p
\end{bmatrix}
\]

% \[
% AB =
% \begin{bmatrix}
% a_{11} b_{11} + a_{12} b_{21} + \cdots + a_{1n} b_{n1}
% \end{bmatrix}
% \]


\begin{thm}
\begin{align*}
(AB) \V{v} &= A (B \V{v}) \\
(c A) B &= c (AB) = A (cB) \\
A(BC) &= (AB)C
\end{align*}
\end{thm}


\section*{Identitetsmatriser}


\section*{Vektorer som matriser}


\kapittelslutt
