\chapter{Gausseliminasjon}

I dette notatet skal vi formalisere ideene fra forrige notat.  Vi
skal se hvordan vi kan løse et hvilket som helst lineært
likningssystem ved å skrive om totalmatrisen til systemet etter
bestemte regler.

Reglene for hvordan totalmatrisen kan skrives om kalles
\emph{radoperasjoner}, og målet er å få en matrise som er på
\emph{trappeform}.  Denne prosessen kalles \emph{gausseliminasjon}.


\section*{Radoperasjoner}

Følgende tre måter å endre en matrise på kalles
\defterm{radoperasjoner}:
\begin{enumerate}
\item Gange alle tallene i en rad med det samme tallet (ikke~$0$).
\item Legge til (et multiplum av) en rad i en annen.
\item Bytte rekkefølge på radene.
\end{enumerate}

Vi sier at to matriser er \defterm{radekvivalente} hvis vi kan komme
fra den ene til den andre ved å utføre en eller flere radoperasjoner.
Vi bruker notasjonen $M \roweq N$ for å si at to matriser $M$ og~$N$
er radekvivalente.

\begin{ex}
\label{ex:radekvivalent}
Disse matrisene er radekvivalente, siden vi får den andre matrisen fra
den første ved å gange øverste rad med~$4$:
\begin{align*}
\begin{bmatrix}
 2 & 5 \\
 1 & 7
\end{bmatrix}
&\roweq
\begin{bmatrix}
 8 & 20 \\
 1 &  7
\end{bmatrix}
\end{align*}
Merk at vi også kan gå motsatt vei: Ved å gange øverste rad i den
andre matrisen med $1/4$ får vi tilbake den første matrisen.

Disse to matrisene er også radekvivalente:
\begin{align*}
\begin{bmatrix}
 3 & 1 \\
 9 & 5
\end{bmatrix}
&\roweq
\begin{bmatrix}
 3 & 1 \\
 0 & 2
\end{bmatrix}
\end{align*}
Her har vi brukt den andre typen radoperasjon: Vi la til $-3$ ganger
øverste rad i nederste rad for å komme fra den første matrisen til den
andre.  Merk igjen at vi også kan gå motsatt vei: Ved å legge til $3$
ganger øverste rad i nederste rad, kommer vi fra den andre matrisen
til den første.
\end{ex}

Hele poenget med radoperasjoner er at det å utføre en radoperasjon på
en totalmatrise tilsvarer å skrive om likningssystemet til et nytt
system som er ekvivalent med det opprinnelige.  Vi formulerer dette
som et teorem:

\begin{thm}
\label{thm:radekvivalens}
Hvis to likningssystemer har radekvivalente totalmatriser, så er de to
likningssystemene ekvivalente.
\end{thm}
\begin{proof}
For å bevise dette, er det nok å vise at det å gjøre en radoperasjon
på totalmatrisen til et likningssystem tilsvarer å gjøre en gyldig
omskrivning av systemet selv.

Den første typen radoperasjon -- å gange alle tallene i en rad med
samme tall -- tilsvarer å gange med det samme tallet på begge sider av
en ligning.  Litt mer detaljert: La oss si at
\[
a_{i1}\ a_{i2}\ \cdots\ a_{in}\ |\ b_i
\]
er en av radene i totalmatrisen, og at vi ganger opp denne med
tallet~$c$ slik at vi får:
\[
(c a_{i1})\ (c a_{i2})\ \cdots\ (c a_{in})\ |\ (c b_i)
\]
Dette tilsvarer at vi bytter ut likningen
\[
a_{i1} x_1 + a_{i2} x_2 + \cdots + a_{in} x_n = b_i
\]
med den nye likningen
\[
(c a_{i1}) x_1 + (c a_{i2}) x_2 + \cdots + (c a_{in}) x_n = c b_i.
\]
Men det er klart at hvis den opprinnelige likningen var sann, så må
også den nye være det.  Og siden det ikke tillates at tallet~$c$ som
vi ganger med er~$0$, så har vi også det motsatte: Hvis den nye
likningen er sann, så må også den opprinnelige være det.  Altså gjør
vi ingen endring i løsningene av likningssystemet ved å utføre denne
typen radoperasjon.

For den andre typen radoperasjon -- legge til et multiplum av en rad i
en annen -- kan vi på tilsvarende måte se at den nye raden vi lager
tilsvarer en likning som må være sann hvis de gamle likningene var
sanne.  Sett at vi legger til $c$ ganger rad~$i$ i rad~$j$.  Dette
tilsvarer at vi ganger opp den $i$-te likningen med~$c$, og legger til
resultatet i den $j$-te likningen.  Alle løsninger av de gamle
likningene må da også være løsninger av denne nye likningen.  Dessuten
kan vi komme tilbake til det gamle systemet (ved å legge til $-c$
ganger rad~$i$ i rad~$j$), og dermed må alle løsninger av det nye
systemet også være løsninger av det gamle.

Den tredje og siste typen radoperasjon -- bytte rekkefølge på radene
-- gjør åpenbart ingen endringer i løsningene av likningssystemet,
siden dette bare tilsvarer å skrive likningene i en annen rekkefølge.
\end{proof}

\begin{ex}
\label{ex:gausseliminasjon1}
Vi gjentar regningen i eksempel~2.5 fra forrige notat, denne gangen
ved å utføre radoperasjoner på totalmatrisen til likningssystemet:
\begin{align*}
\begin{amatrix}{3}
1 & 2 & -2 & -5 \\
1 & 5 &  9 & 33 \\
2 & 5 & -1 &  0
\end{amatrix}
&\roweq
\begin{amatrix}{3}
1 & 2 & -2 & -5 \\
0 & 3 & 11 & 38 \\
2 & 5 & -1 &  0
\end{amatrix}
\\
&\roweq
\begin{amatrix}{3}
1 & 2 & -2 & -5 \\
0 & 3 & 11 & 38 \\
0 & 1 &  3 & 10
\end{amatrix}
\\
&\roweq
\begin{amatrix}{3}
1 & 2 & -2 & -5 \\
0 & 1 &  3 & 10 \\
0 & 3 & 11 & 38
\end{amatrix}
\\
&\roweq
\begin{amatrix}{3}
1 & 2 & -2 & -5 \\
0 & 1 &  3 & 10 \\
0 & 0 &  2 &  8
\end{amatrix}
\end{align*}
Her gjorde vi følgende radoperasjoner: Legge til $-1$ ganger første
rad i andre rad, legge til $-2$ ganger første rad i tredje rad, bytte
andre og tredje rad, og legge til $-3$ ganger andre rad i tredje rad.

Den siste matrisen her er på det som kalles trappeform, og da er det
(som vi så i eksempel~2.5) lett å finne løsningen.  Hvis vi vil
gjøre det enda lettere, kan vi fortsette med radoperasjoner til vi
oppnår det som kalles \emph{redusert trappeform}:
\begin{align*}
\begin{amatrix}{3}
1 & 2 & -2 & -5 \\
0 & 1 &  3 & 10 \\
0 & 0 &  2 &  8
\end{amatrix}
&\roweq
\begin{amatrix}{3}
1 & 2 & -2 & -5 \\
0 & 1 &  3 & 10 \\
0 & 0 &  1 &  4
\end{amatrix}
\\
&\roweq
\begin{amatrix}{3}
1 & 2 &  0 &  3 \\
0 & 1 &  0 & -2 \\
0 & 0 &  1 &  4
\end{amatrix}
\\
&\roweq
\begin{amatrix}{3}
1 & 0 &  0 &  7 \\
0 & 1 &  0 & -2 \\
0 & 0 &  1 &  4
\end{amatrix}
\end{align*}
Den siste totalmatrisen her svarer til følgende likningssystem:
\[
\systeme*{
x = 7,
y = -2,
z = 4
}
\]
Her har vi altså kommet helt frem til løsningen.
\end{ex}


\section*{Trappeform}

Vi vil nå gi en presis definisjon av begrepene «trappeform» og
«redusert trappeform».  Da trenger vi også et annet begrep, nemlig
«pivotelement».

% TODO: endre til «lederelement»?  ta med «pivotposisjon», «pivotkolonne»?
\begin{defn}
Det første tallet i en rad i en matrise som ikke er~$0$ kalles
\defterm{pivotelementet} for den raden.  (En rad med bare nuller har
ikke noe pivotelement.)
\end{defn}

\begin{ex}
\label{ex:pivotelement}
Se på følgende matrise:
\[
\begin{bmatrix}
3 & -2 & 0 & 2 \\
0 &  0  & 5 & 12 \\
1 &  8 & 3 & 7 \\
0 &  0  & 0 & 0
\end{bmatrix}
\]
Pivotelementene her er tallet~$3$ i den øverste raden, tallet~$5$ i
den andre raden og tallet~$1$ i den tredje raden.  Den siste raden
består av bare nuller, og har derfor ikke noe pivotelement.
\end{ex}

\begin{defn}
En matrise er på \defterm{trappeform} dersom hvert pivotelement er til
høyre for alle pivotelementer i tidligere rader, og eventuelle
nullrader er helt nederst.
\end{defn}

\begin{ex}
\label{ex:trappeform}
Denne matrisen er på trappeform:
\[
\begin{bmatrix}
3 & 7 & 6 \\
0 & 1 & -2 \\
0 & 0 & 2
\end{bmatrix}
\]
Pivotelementene er $3$, $1$ og~$2$, og hvert av dem er til høyre for
alle de tidligere pivotelementene.

Denne matrisen er også på trappeform:
\[
\begin{bmatrix}
-2 & 7 & 1 & 5 \\
 0 & 0 & 4 & 9 \\
 0 & 0 & 0 & 0
\end{bmatrix}
\]

Denne matrisen er ikke på trappeform fordi nullradene ikke er samlet
nederst:
\[
\begin{bmatrix}
3 & 8 & 2 & 0 \\
0 & 0 & 0 & 0 \\
0 & 2 & 1 & 4 \\
0 & 0 & 0 & 0
\end{bmatrix}
\]

Denne matrisen ser «trappete» ut, men er likevel ikke på trappeform:
\[
\begin{bmatrix}
5 & 8 & 7 & 2 \\
0 & 4 & 1 & 6 \\
0 & 9 & 2 & 0 \\
0 & 0 & 3 & 1
\end{bmatrix}
\]
Grunnen til at den ikke er på trappeform er at pivotelementet $9$ i
tredje rad ikke er til høyre for pivotelementet i andre rad, men rett
under det isteden.
\end{ex}

\begin{defn}
En matrise er på \defterm{redusert trappeform} hvis den er på
trappeform og dessuten oppfyller:
\begin{itemize}
\item Alle pivotelementene er~$1$.
\item Alle tall som står over pivotelementer er~$0$.\qedhere
\end{itemize}
\end{defn}

Den siste matrisen i eksempel~\ref{ex:gausseliminasjon1} er på
redusert trappeform, og der så vi også hva som gjør redusert
trappeform nyttig: Løsningen av systemet kan leses av direkte.

\medskip
Det å  skrive om en matrise til trappeform
ved hjelp av radoperasjoner
kalles \defterm{gausseliminasjon}, oppkalt etter den tyske
matematikeren Carl Friedrich Gauss (1777--1855).  Noen velger også å
ha et eget navn på det å komme frem til \emph{redusert} trappeform, og
kaller den prosessen for \defterm{Gauss--Jordan-eliminasjon}, oppkalt
etter Wilhelm Jordan (1842--1899).  Vi tar det ikke så nøye med den
forskjellen, og sier «gauss\-eliminasjon» uansett.

(Disse begrepene er uansett historisk sett fullstendig misvisende.
Metoden som vi kaller gauss\-eli\-minasjon var kjent i Kina for flere
tusen år siden, og Gauss -- som riktignok var et universalgeni og fant
opp mengder av flotte ting -- har ikke egentlig så mye med den å
gjøre.)


\section*{Eksistens og entydighet av løsninger}

Når vi vil løse et likningssystem, er det noen åpenbare spørsmål vi
kan stille:
\begin{itemize}
\item Har systemet noen løsning? (\emph{Eksistens})
\item Hvis systemet har løsning: Har det også flere løsninger, eller
bare én?  (\emph{Entydighet})
\end{itemize}

I eksempel~\ref{ex:gausseliminasjon1} hadde vi et system med entydig
løsning.  Vi tar noen flere eksempler for å vise andre ting som kan
skje.

\begin{ex}
\label{ex:gausseliminasjon2}
La oss løse følgende system:
\[
\systeme{
   x -  2 y = 1,
-5 x + 10 y = -1
}
\]
Vi setter opp totalmatrisen og gausseliminerer:
\[
\begin{amatrix}{2}
 1 & -2 &  1 \\
-5 & 10 & -1
\end{amatrix}
\roweq
\begin{amatrix}{2}
1 & -2 & 1 \\
0 &  0 & 4
\end{amatrix}
\]
Den siste matrisen svarer til følgende system:
\[
\systeme{
  x + 2 y = 1,
0 x + 0 y = 4
}
\]
Likningen $0x + 0y = 4$ kan også skrives som $0 = 4$, og den kan ikke
stemme uansett hva vi setter $x$ og~$y$ til å være.  Dette systemet
har altså ingen løsning.
\end{ex}

Generelt er det slik at hvis vi får en rad i totalmatrisen vår på
formen
\[
0\ 0\ \cdots\ 0\ |\ b,
\]
der $b$ er et tall som ikke er~$0$, så har systemet ingen løsning.
Denne raden svarer jo til likningen $0 = b$, som ikke kan være sann.
Hvis vi har en matrise på trappeform der ingen av radene er på denne
formen, så har systemet minst én løsning.

Men et lineært likningssystem kan også ha mer enn én løsning, som vi
skal se i det neste eksempelet.

\begin{ex}
\label{ex:gausseliminasjon3}
La oss løse følgende system:
\[
\systeme{
  x_1 + 3 x_2 + 2 x_3 + 3 x_4 = 16,
  x_1 + 3 x_2 + 3 x_3 + 1 x_4 = 21,
2 x_1 + 6 x_2 + 4 x_3 + 6 x_4 = 32
}
\]
Vi setter opp totalmatrisen og gausseliminerer:
\begin{align*}
\begin{amatrix}{4}
1 & 3 & 2 &  3 & 16 \\
1 & 3 & 3 &  1 & 21 \\
2 & 6 & 4 &  6 & 32
\end{amatrix}
&\roweq
\begin{amatrix}{4}
1 & 3 & 2 &  3 & 16 \\
0 & 0 & 1 & -2 &  5 \\
0 & 0 & 0 &  0 &  0
\end{amatrix}
\\
&\roweq
\begin{amatrix}{4}
1 & 3 & 0 &  7 & 6 \\
0 & 0 & 1 & -2 & 5 \\
0 & 0 & 0 &  0 & 0
\end{amatrix}
\end{align*}
Den siste matrisen svarer til følgende system:
\[
\systeme*{
x_1 + 3 x_2 + 7 x_4 = 6,
x_3 - 2 x_4 = 5
}
\]
(Her har vi ikke tatt med noen likning for nullraden i matrisen.  Det
er fordi nullraden står for likningen $0x_1 + 0x_2 + 0x_3 + 0x_4 = 0$,
eller med andre ord $0 = 0$.  Denne likningen er åpenbart oppfylt
uansett hva $x_1$, $x_2$, $x_3$ og~$x_4$ er, så vi trenger ikke ta den
med.)

Hvis vi flytter alt unntatt $x_1$ og~$x_3$ til høyresiden, ser
systemet slik ut:
\[
\left\{
\begin{aligned}
x_1 &= - 3 x_2 - 7 x_4 + 6 \\
x_3 &= 2 x_4 + 5
\end{aligned}
\right.
\]
Vi kan altså finne løsninger av systemet ved å sette $x_2$ og~$x_4$
til å være hva vi vil, og deretter bruke disse to likhetene til å
bestemme $x_1$ og~$x_3$.

Hvis vi for eksempel velger å sette $x_2 = 0$ og $x_4 = 1$, så får vi
følgende løsning:
\[
\left\{
\begin{aligned}
x_1 &= -3 \cdot 0 - 7 \cdot 1 + 6 = -1 \\
x_2 &= 0 \\
x_3 &= 2 \cdot 1 + 5 = 7 \\
x_4 &= 1
\end{aligned}
\right.
\]

For å beskrive alle løsningene av systemet på en ryddig måte, kan vi
sette $x_2 = s$ og $x_4 = t$, der $s$ og~$t$ står for to vilkårlige
tall.  Da er alle løsningene gitt ved:
\[
\raisebox{23pt}{$
\left\{
\begin{aligned}
x_1 &= -3 s - 7 t + 6 \\
x_2 &= s \\
x_3 &= 2 t + 5 \\
x_4 &= t
\end{aligned}
\right.
$}
\qedhere
\]
\end{ex}

Variabler som vi kan sette til hva vi vil, slik som $x_2$ og~$x_4$ i
eksempelet over, kalles \defterm{frie variabler}.

Når vi løser et lineært likningssystem, og har funnet ut at det har
minst én løsning, så er det to muligheter.  Den ene muligheten er at
vi ikke får noen frie variabler (slik som i
eksempel~\ref{ex:gausseliminasjon1}).  Da har systemet entydig
løsning.  Den andre muligheten er at det er en eller flere frie
variabler.  Da har systemet uendelig mange løsninger, siden hver av de
frie variablene kan settes til å være et hvilket som helst tall.

Dette betyr at det ikke er mulig at vi får for eksempel to løsninger,
eller tre løsninger, og så videre.  Om det først er mer enn én
løsning, må det være uendelig mange.

\medskip

La oss oppsummere det vi har funnet ut om eksistens og entydighet av
løsninger.  For ethvert lineært likningssystem må én av følgende være
sant:
\begin{itemize}
\item Systemet har ingen løsning.
\item Systemet har entydig løsning.
\item Systemet har uendelig mange løsninger.
\end{itemize}


\section*{Valgfrihet}

Når vi gausseliminerer har vi en viss grad av valgfrihet.
(TODO)
