\documentclass[norsk,a4paper,twocolumn,oneside]{memoir}

\usepackage[utf8]{inputenc}
\usepackage{babel}
\usepackage{amsmath,amssymb,amsthm}
\usepackage[total={17cm,27cm}]{geometry}
\usepackage[table]{xcolor}
%\usepackage{tabularx}
\usepackage{systeme}
%\usepackage{hyperref}
%\usepackage{enumerate}

%\usepackage{sectsty}
\setsecheadstyle{\bfseries\large}
%\subsectionfont{\bf\normalsize}

\usepackage{tikz}
\usetikzlibrary{arrows.meta}

\newcommand{\defterm}[1]{\emph{#1}}

\newcommand{\N}{\mathbb{N}}
\newcommand{\Z}{\mathbb{Z}}
\newcommand{\Q}{\mathbb{Q}}
\newcommand{\R}{\mathbb{R}}

\newcommand{\abs}[1]{|#1|}

\newcommand{\roweq}{\sim}
\DeclareMathOperator{\Span}{Span}

\newcommand{\V}[1]{\mathbf{#1}}
\newcommand{\vv}[2]{\begin{bmatrix} #1 \\ #2 \end{bmatrix}}
\newcommand{\vvv}[3]{\begin{bmatrix} #1 \\ #2 \\ #3 \end{bmatrix}}
\newcommand{\vvvv}[4]{\begin{bmatrix} #1 \\ #2 \\ #3 \\ #4 \end{bmatrix}}
\newcommand{\vn}[2]{\vvvv{#1_1}{#1_2}{\vdots}{#1_#2}}

\newenvironment{amatrix}[1]{% "augmented matrix"
  \left[\begin{array}{*{#1}{c}|c}
}{%
  \end{array}\right]
}

% \newcounter{notatnr}
% \newcommand{\notatnr}[2]
% {\setcounter{notatnr}{#1}%
%  \setcounter{page}{#2}%
% }

\newtheorem{thm}{Teorem}[chapter]
\newtheorem*{thm-nn}{Teorem}
\newtheorem{cor}[thm]{Korollar}
\newtheorem{lem}[thm]{Lemma}
\newtheorem{prop}[thm]{Proposisjon}
\theoremstyle{definition}
\newtheorem{exx}[thm]{Eksempel}
\newtheorem*{defnx}{Definisjon}
\newtheorem*{oppg}{Oppgave}
\newtheorem*{merkx}{Merk}
\newtheorem*{spmx}{Spørsmål}

\newenvironment{defn}
  {\pushQED{\qed}\renewcommand{\qedsymbol}{$\triangle$}\defnx}
  {\popQED\enddefnx}
\newenvironment{ex}
  {\pushQED{\qed}\renewcommand{\qedsymbol}{$\triangle$}\exx}
  {\popQED\endexx}
\newenvironment{merk}
  {\pushQED{\qed}\renewcommand{\qedsymbol}{$\triangle$}\merkx}
  {\popQED\endmerkx}
\newenvironment{spm}
  {\pushQED{\qed}\renewcommand{\qedsymbol}{$\triangle$}\spmx}
  {\popQED\endspmx}

\setlength{\columnsep}{26pt}

\newcommand{\Tittel}[2]{%
\twocolumn[
\begin{center}
\Large
\begin{tabularx}{\textwidth}{cXr}
\cellcolor{black}\color{white}%
\bf {#1} &
#2
\hfill &
\footnotesize TMA4110 høsten 2018
\\ \hline
\end{tabularx}
\end{center}
]}

\newcommand{\tittel}[1]{\Tittel{\arabic{notatnr}}{#1}}

\newcommand{\linje}{%
\begin{center}
\rule{.8\linewidth}{0.4pt}
\end{center}
}


\newcommand{\chapternumber}{}

\makechapterstyle{tma4110}{%
 \renewcommand*{\chapterheadstart}{}
 \renewcommand*{\printchaptername}{}
 \renewcommand*{\chapternamenum}{}
 \renewcommand*{\printchapternum}{\renewcommand{\chapternumber}{\thechapter}}
 \renewcommand*{\afterchapternum}{}
 \renewcommand*{\printchapternonum}{\renewcommand{\chapternumber}{}}
 \renewcommand*{\printchaptertitle}[1]{
\LARGE
\begin{tabularx}{\textwidth}{cXr}
\cellcolor{black}\color{white}%
\textbf{\chapternumber} &
\textbf{##1}
\hfill &
%\footnotesize TMA4110 høsten 2018
\\ \hline
\end{tabularx}%
}
 \renewcommand*{\afterchaptertitle}{\par\nobreak\vskip \afterchapskip}
 % \newcommand{\chapnamefont}{\normalfont\huge\bfseries}
 % \newcommand{\chapnumfont}{\normalfont\huge\bfseries}
 % \newcommand{\chaptitlefont}{\normalfont\Huge\bfseries}
 \setlength{\beforechapskip}{0pt}
 \setlength{\midchapskip}{0pt}
 \setlength{\afterchapskip}{10pt}
}


\newcounter{oppgnr}[chapter]
\newcounter{punktnr}[oppgnr]
\newenvironment{oppgave}
 {\par\noindent\stepcounter{oppgnr}\textbf{{\arabic{oppgnr}}.}}
 {\par\bigskip}
\newenvironment{punkt}
 {\par\smallskip\noindent\stepcounter{punktnr}\textbf{\alph{punktnr})} }
 {\par}

\newcommand{\oppgaver}{\linje\section*{Oppgaver}}


\begin{document}

\chapterstyle{tma4110}
\pagestyle{plain}

%\book{TMA4110 Matematikk 3 høsten 2018}

\setlength{\cftpartnumwidth}{3em}

\frontmatter
\tableofcontents*
\documentclass[norsk,a4paper,twocolumn,oneside]{memoir}

\usepackage[utf8]{inputenc}
\usepackage{babel}
\usepackage{amsmath,amssymb,amsthm}
\usepackage[total={17cm,27cm}]{geometry}
\usepackage[table]{xcolor}
%\usepackage{tabularx}
\usepackage{systeme}
%\usepackage{hyperref}
%\usepackage{enumerate}

%\usepackage{sectsty}
\setsecheadstyle{\bfseries\large}
%\subsectionfont{\bf\normalsize}

\usepackage{tikz}
\usetikzlibrary{arrows.meta}

\newcommand{\defterm}[1]{\emph{#1}}

\newcommand{\N}{\mathbb{N}}
\newcommand{\Z}{\mathbb{Z}}
\newcommand{\Q}{\mathbb{Q}}
\newcommand{\R}{\mathbb{R}}

\newcommand{\abs}[1]{|#1|}

\newcommand{\roweq}{\sim}
\DeclareMathOperator{\Span}{Span}

\newcommand{\V}[1]{\mathbf{#1}}
\newcommand{\vv}[2]{\begin{bmatrix} #1 \\ #2 \end{bmatrix}}
\newcommand{\vvv}[3]{\begin{bmatrix} #1 \\ #2 \\ #3 \end{bmatrix}}
\newcommand{\vvvv}[4]{\begin{bmatrix} #1 \\ #2 \\ #3 \\ #4 \end{bmatrix}}
\newcommand{\vn}[2]{\vvvv{#1_1}{#1_2}{\vdots}{#1_#2}}

\newenvironment{amatrix}[1]{% "augmented matrix"
  \left[\begin{array}{*{#1}{c}|c}
}{%
  \end{array}\right]
}

% \newcounter{notatnr}
% \newcommand{\notatnr}[2]
% {\setcounter{notatnr}{#1}%
%  \setcounter{page}{#2}%
% }

\newtheorem{thm}{Teorem}[chapter]
\newtheorem*{thm-nn}{Teorem}
\newtheorem{cor}[thm]{Korollar}
\newtheorem{lem}[thm]{Lemma}
\newtheorem{prop}[thm]{Proposisjon}
\theoremstyle{definition}
\newtheorem{exx}[thm]{Eksempel}
\newtheorem*{defnx}{Definisjon}
\newtheorem*{oppg}{Oppgave}
\newtheorem*{merkx}{Merk}
\newtheorem*{spmx}{Spørsmål}

\newenvironment{defn}
  {\pushQED{\qed}\renewcommand{\qedsymbol}{$\triangle$}\defnx}
  {\popQED\enddefnx}
\newenvironment{ex}
  {\pushQED{\qed}\renewcommand{\qedsymbol}{$\triangle$}\exx}
  {\popQED\endexx}
\newenvironment{merk}
  {\pushQED{\qed}\renewcommand{\qedsymbol}{$\triangle$}\merkx}
  {\popQED\endmerkx}
\newenvironment{spm}
  {\pushQED{\qed}\renewcommand{\qedsymbol}{$\triangle$}\spmx}
  {\popQED\endspmx}

\setlength{\columnsep}{26pt}

\newcommand{\Tittel}[2]{%
\twocolumn[
\begin{center}
\Large
\begin{tabularx}{\textwidth}{cXr}
\cellcolor{black}\color{white}%
\bf {#1} &
#2
\hfill &
\footnotesize TMA4110 høsten 2018
\\ \hline
\end{tabularx}
\end{center}
]}

\newcommand{\tittel}[1]{\Tittel{\arabic{notatnr}}{#1}}

\newcommand{\linje}{%
\begin{center}
\rule{.8\linewidth}{0.4pt}
\end{center}
}


\newcommand{\chapternumber}{}

\makechapterstyle{tma4110}{%
 \renewcommand*{\chapterheadstart}{}
 \renewcommand*{\printchaptername}{}
 \renewcommand*{\chapternamenum}{}
 \renewcommand*{\printchapternum}{\renewcommand{\chapternumber}{\thechapter}}
 \renewcommand*{\afterchapternum}{}
 \renewcommand*{\printchapternonum}{\renewcommand{\chapternumber}{}}
 \renewcommand*{\printchaptertitle}[1]{
\LARGE
\begin{tabularx}{\textwidth}{cXr}
\cellcolor{black}\color{white}%
\textbf{\chapternumber} &
\textbf{##1}
\hfill &
%\footnotesize TMA4110 høsten 2018
\\ \hline
\end{tabularx}%
}
 \renewcommand*{\afterchaptertitle}{\par\nobreak\vskip \afterchapskip}
 % \newcommand{\chapnamefont}{\normalfont\huge\bfseries}
 % \newcommand{\chapnumfont}{\normalfont\huge\bfseries}
 % \newcommand{\chaptitlefont}{\normalfont\Huge\bfseries}
 \setlength{\beforechapskip}{0pt}
 \setlength{\midchapskip}{0pt}
 \setlength{\afterchapskip}{10pt}
}


\newcounter{oppgnr}[chapter]
\newcounter{punktnr}[oppgnr]
\newenvironment{oppgave}
 {\par\noindent\stepcounter{oppgnr}\textbf{{\arabic{oppgnr}}.}}
 {\par\bigskip}
\newenvironment{punkt}
 {\par\smallskip\noindent\stepcounter{punktnr}\textbf{\alph{punktnr})} }
 {\par}

\newcommand{\oppgaver}{\linje\section*{Oppgaver}}

\usepackage{tikz}
\usetikzlibrary{arrows.meta}

\begin{document}

\notatnr{1}{1}
\tittel{Introduksjon}

\noindent%
Velkommen til emnet TMA4110 Matematikk 3.

Det du leser nå er det første i en serie med notater som vil følge deg
gjennom emnet.  Notatene tilsvarer det som gjennomgås i
forelesningene, og det er notatene som angir hva som er pensum.

I dette første notatet skal vi ta et overblikk over temaene vi skal
innom i løpet av semesteret.  Det gjør ikke noe om du ikke forstår alt
i dette notatet; vi skal gjennomgå det i detalj siden.  Fra notat~2
starter den virkelige gjennomgangen av pensum.

\smallskip

Emnet er delt inn i tre separate deler som egentlig hører til helt
forskjellige områder innenfor matematikk:
\begin{enumerate}
\item Lineær algebra
\item Komplekse tall
\item Lineære differensiallikninger
\end{enumerate}
Delen om lineær algebra utgjør hoveddelen av emnet og er den vi vil
bruke mest tid på.

\smallskip

Grunnen til at disse tilsynelatende urelaterte temaene er gruppert
sammen i ett emne er at det er noen viktige tilknytningspunkter mellom
dem som gjør at det er fornuftig å lære om dem sammen.


\section*{Lineær algebra}

Algebra er et område innenfor matematikk som i utgangspunktet handler
om å løse likninger.  For å få en idé om hva algebra er for noe, kan
det være nyttig å vite hva det \emph{ikke} er.  Her er en liten guide
til hvordan noen av de tingene du antagelig har gjort i ditt
matematiske liv så langt passer inn i ulike områder innen matematikk:

Hvis du deriverer eller integrerer, så driver du med \emph{analyse}.
Hvis du tegner en figur, driver du antagelig med \emph{geometri}.
Hvis du teller antall måter du kan plukke opp forskjellige fargede
kuler fra en pose på, så driver du med \emph{kombinatorikk}; men hvis
du deretter regner ut sannsynligheten for at du får to røde kuler, så
har du flyttet deg over til \emph{sannsynlighetsregning}.  Hvis du
prøver å forstå reglene for hvordan et matematisk bevis utføres, så
driver du med \emph{logikk}.  Og hvis du løser en likning, da driver
du med algebra.

\smallskip

Da du for eksempel (en gang for mange år siden) lærte at en
andregradslikning
\[
ax^2 + bx + c = 0
\]
kan løses ved hjelp av formelen
\[
x = \frac{-b \pm \sqrt{b^2 - 4ac}}{2a},
\]
så var det en del av den klassiske, grunnleggende algebraen du lærte.

\smallskip

Ordet «algebra» kommer av det arabiske \emph{al-jabr}, som var en del
av tittelen på en bok skrevet omkring år 820 av den persiske
matematikeren al-Khwarizmi.  Denne boken inneholdt metoder for å løse
likninger (blant annet andregradslikningen nevnt over), og la
grunnlaget for algebra som fagområde.

Men så er det slik med matematikk at den stadig endrer seg.
Matematisk forskning er en evig runddans av at matematikere finner opp
nye teknikker og konsepter for å finne svar på spørsmål de synes er
interessante, og så finner de ut at man kan stille nye interessante
spørsmål om de nye tingene de har funnet opp, og så har man det
gående.

I løpet av 1800-tallet og begynnelsen på 1900-tallet førte den
matematiske utviklingen til at algebraen som fagfelt skiftet fokus.
Man utviklet visse former for matematiske strukturer som ble
brukt til å forstå ulike typer likninger, og over tid fant man ut at
disse strukturene kunne generaliseres og være nyttige også til andre
ting enn løsing av likninger.  Dermed endte man opp med forskjellige
typer \emph{abstrakte algebraiske strukturer} (noen av disse kalles
\emph{grupper}, \emph{ringer} og \emph{moduler}), og algebra i dag
handler primært om å forstå disse strukturene.  Denne nye formen for
algebra kalles \emph{abstrakt algebra} (eller \emph{moderne algebra}),
for å skille den fra den mer tradisjonelle algebraen som handler om
løsing av likninger.

\medskip
Greit, det var veldig mye prat om algebra.  Men hva er \emph{lineær
  algebra} for noe?

Algebra handler opprinnelig om likninger, og lineær algebra er den
delen av algebraen som handler om lineære likninger.

En lineær likning med én ukjent ser generelt slik ut:
\[
ax = b
\]
Dersom $a \ne 0$, kan vi løse denne likningen ved å dele på $a$:
\[
x = b/a
\]
Og det er egentlig omtrent alt som er å si om lineære likninger med én
ukjent.

Men hvis vi ser på et system av flere lineære likninger, med flere
ukjente, blir det straks mer interessant.  Her er et eksempel på et
system av lineære likninger:
\[
\systeme{
  3x + 5y - 2z = 14,
   x - 9y + 7z = 0,
 -5x + 4y + 8z = -3
}
\]
Studiet av slike systemer er utgangspunktet for lineær algebra.

\smallskip
Det første vi skal lære er en metode (som kalles
\emph{gausseliminasjon}) for å løse lineære likningssystemer på en
effektiv måte.

Deretter skal vi se at ved å innføre noen nye konsepter -- nemlig
\emph{vektorer} og \emph{matriser} -- kan vi få en bedre forståelse av
lineære likningssystemer.

Du har antagelig hørt om vektorer før, og antagelig har du lært å
tenke på en vektor som en pil -- noe som har en lengde og en retning.
\begin{center}
\begin{tikzpicture}[scale=.9]
\draw[->] (0,1) -- (3,0);
\end{tikzpicture}
\\
{\small\it En pil}
\end{center}
Dette er imidlertid bare ett aspekt av vektorer.  Vi vil velge å se på
et \emph{punkt} i planet, og \emph{pilen} fra origo til dette punktet, og
\emph{koordinatene} til punktet, som tre forskjellige representasjoner
av den samme vektoren.
\begin{center}
\begin{tikzpicture}[scale=.9]
\draw[->] (-3.5,0) -- (3.5,0);
\draw[->] (0,-1.2) -- (0,2.4);
\foreach \x in {-3,-2,-1,1,2,3}
\draw (\x cm,1pt) -- (\x cm,-1pt) node[anchor=north] {$\x$};
\foreach \y in {-1,1,2}
\draw (1pt,\y cm) -- (-1pt,\y cm) node[anchor=east] {$\y$};
\draw[-{>[length=7,width=5]},line width=1pt] (0,0) -- (2,1);
\filldraw (2,1) circle [radius=2pt];
\node[anchor=west] at (2,1) {$(2,1)$};
\end{tikzpicture}
\\
{\small\it Vektorenes treenighet:\\punkt -- pil -- koordinater}
\end{center}
Koordinatene som angir en vektor, for eksempel $(2,1)$, vil vi
vanligvis skrive på denne måten:
\[
\vv{2}{1}
\]
Dette kaller vi en \emph{kolonnevektor}.

Det er klart at vi kan se på vektorer i to dimensjoner eller i tre
dimensjoner -- altså i et plan eller i et rom.  Men hvis vi tenker på
vektorer bare som kolonnevektorer, og glemmer det med punkter og
piler, så er det ikke noe i veien for å snakke om vektorer i fire
dimensjoner, eller fem, eller så mange dimensjoner vi vil.  Og det
skal vi gjøre.

Men hva har dette med lineære likningssystemer å gjøre?  Vi skal
definere aritmetiske operasjoner for vektorer (på ganske naturlige
måter), slik at likningssystemet vi så for en stund siden kan skrives
på denne måten:
\[
\vvv{3}{1}{-5} \cdot x + \vvv{5}{-9}{4} \cdot y + \vvv{-2}{7}{8} \cdot z = \vvv{14}{0}{-3}
\]
Istedenfor et system med flere likninger har vi altså én likning, der
koeffisientene og høyresiden er vektorer.  Istedenfor å tenke på
problemet som det å finne tall som er løsninger av flere forskjellige
likninger, tenker vi på det som å finne ut om visse vektorer kan
kombineres slik at vi får en viss annen vektor.

Så innfører vi matriser, som er rektangulære tabeller med tall, og vi
kan skrive om systemet vårt til:
\[
\begin{bmatrix}
 3 &  5 & -2 \\
 1 & -9 &  7 \\
-5 &  4 &  8
\end{bmatrix}
\vvv{x}{y}{z}
= \vvv{14}{0}{-3}
\]
Her har vi en matrise ganger en vektor på venstresiden, og en vektor
på høyresiden.  Når vi har lært om matriser, kan vi skrive et generelt
lineært likningssystem på den konsise formen
\[
A \V{x} = \V{b},
\]
der $A$ er en matrise, $\V{b}$ er en konstant vektor, og $\V{x}$ er en
ukjent vektor.

Men det stopper ikke der!  Vi kan snu likheten $A \V{x} = \V{b}$ litt
på hodet.  Istedenfor å si at $\V{x}$ er en ukjent som vi vil finne,
så kan vi si at når matrisen $A$ ganges med en vektor~$\V{x}$, så får
vi ut en ny vektor~$\V{b}$.  Med andre ord: En matrise gir opphav til
en funksjon som tar inn vektorer og gir ut vektorer.  Hvis matrisen
har $m$~rader og $n$~kolonner, så kan vi bruke den til å lage en
funksjon~$T$ som tar inn $n$-dimensjonale vektorer og gir ut
$m$-dimensjonale vektorer.  En slik funksjon kalles en
\emph{lineærtransformasjon}, og vi skriver:
\[
T \colon \R^n \to \R^m
\]
Her står $\R^n$ for mengden av alle $n$-dimensjonale kolonnevektorer,
og en slik mengde kalles et \emph{vektorrom}.

Nå viser det seg at vektorrom, lineærtransformasjoner og matriser er
interessante ting å studere i seg selv, og at vi med utgangspunkt i
disse tingene kan bygge opp en stor og flott matematisk teori med
anvendelsesområder som går langt ut over det å løse likningssystemer.
Den teorien er lineær algebra.


\section*{Komplekse tall}

Hva er et tall?  Da du som barn lærte å telle, lærte du tallene
\[
1,\ 2,\ 3,\ 4,\ \ldots
\]
Så lærte du å legge sammen tall, og å trekke tall fra hverandre.  Da
viste det seg at det går an å stille spørsmål som ikke har svar: Hva
er $3 - 5$?

Men så lærte du at slike spørsmål likevel kan besvares når man bare
har innført noen nye tall:
\[
0,\ -1,\ -2,\ -3,\ \ldots
\]

Videre utvidet du tallforståelsen din med \emph{rasjonale tall}
(brøker av heltall, for eksempel $7/5$), og så lærte du at det også
finnes tall som ikke er rasjonale, for eksempel $\sqrt{2}$ og $\pi$.
Når vi tar med alle slike tall, har vi mengden av \emph{reelle tall}.

Fremdeles er det spørsmål som ikke kan besvares, for eksempel: Hva er
$\sqrt{-1}$?  Det finnes ikke noe reelt tall som vi kan opphøye i
andre og få $-1$.

Men denne situasjonen er egentlig helt tilsvarende som da vi bare
kjente til de positive tallene og lurte på hva $3 - 5$ er.  Akkurat
som vi da kunne utvide tallsystemet ved å legge til negative tall, kan
vi nå legge til nye tall som gjør at uttrykket $\sqrt{-1}$ gir mening.

Vi lager et nytt tall, som vi kaller~$i$, og som er definert til å
være slik at
\[
i^2 = -1
\]
For å få et tallsystem som oppfører seg slik det skal, må vi kunne
gange og legge sammen $i$ med et hvilket som helst annet tall.  Det
gjør at vi må ha med flere nye tall, og til sammen får vi at alle tall
som kan skrives som
\[
a + bi
\qquad
\text{(der $a$ og~$b$ er reelle tall)}
\]
må være med i det nye tallsystemet.  Disse tallene kalles
\emph{komplekse tall}.

Mengden av reelle tall visualiserer vi som en uendelig lang tallinje.
Mengden av komplekse tall visualiserer vi som et todimensjonalt plan.
\begin{center}
\begin{tikzpicture}[scale=.9]
\draw[->] (-3.3,0) -- (3.5,0);
\draw[->] (0,-1.3) -- (0,2.4);
\foreach \x in {-3,-2,-1,1,2,3}
\draw (\x cm,1pt) -- (\x cm,-1pt) node[anchor=north] {$\x$};
\draw (1pt, 2 cm) -- (-1pt, 2 cm) node[anchor=east] {$2i$};
\draw (1pt, 1 cm) -- (-1pt, 1 cm) node[anchor=east] {$i$};
\draw (1pt,-1 cm) -- (-1pt,-1 cm) node[anchor=east] {$-i$};
\filldraw (1,2) circle [radius=2pt];
\node[anchor=west] at (1,2) {$(1 + 2i)$};
\end{tikzpicture}
\\
{\small\it Det komplekse planet}
\end{center}

Du må ikke la deg lure av navnet til å tro at komplekse tall er
kompliserte å ha med å gjøre.  På mange måter er det enklere å jobbe
med komplekse tall enn med reelle tall.

Innenfor de komplekse tallene kan vi for eksempel alltid ta
kvadratrøtter av hvilke som helst tall, uten å måtte tenke på om de er
negative.  Med andre ord: alle likninger på formen $x^2 = a$ har
løsninger.  Og ikke bare det, men alle andregradslikninger
\[
a x^2 + bx + c = 0
\]
har løsninger.  Og ikke bare det, men alle polynomlikninger av hvilken
som helst grad, altså alle likninger på formen
\[
a_n x^n + a_{n-1} x^{n-1} + \cdots + a_1 x + a_0 = 0,
\]
har løsninger.

Det som imidlertid kan gjøre komplekse tall litt vanskelige er at de
ikke helt passer inn i vår intuitive forståelse av hva et tall skal
være.  Med reelle tall kan vi lett se for oss hvordan et tall kan
representere noe målbart i den virkelige verden: en avstand, et areal,
en hastighet eller liknende.  Med komplekse tall er det vanskeligere å
se for seg hva tallene kan representere.

Når vi får bruk for komplekse tall er det ofte slik at vi starter med
noe som bare handler om reelle tall, og får et sluttsvar med bare
reelle tall, men må innom de komplekse tallene i mellomregningen.  Vi
kommer til å se eksempler på nettopp dette når vi i emnets siste del
skal løse differensiallikninger.


\section*{Lineære differensiallikninger}

En \emph{differensiallikning} er en likning der den ukjente er en
funksjon, og der den deriverte av denne ukjente funksjonen også er med
i likningen.

Vi skal se på to forskjellige typer differensiallikninger.  Den ene
typen er lineære andreordens differensiallikninger, som vil si
likninger på formen
\[
y'' + p y' + q y = g,
\]
der $y$ er en ukjent funksjon av en variabel~$t$, og $g$ er en kjent
funksjon av~$t$, og $p$ og~$q$ er konstanter.

Når vi skal løse en slik likning får vi bruk for å lage en
«hjelpelikning», nemlig andregradslikningen
\[
\lambda^2 + p \lambda + q = 0
\]
der $\lambda$ er den ukjente, og $p$ og~$q$ er de samme konstantene
som vi hadde i differensiallikningen.  Det viser seg nemlig at
løsningene av denne likningen gir oss viktig informasjon om hvordan
løsningene av differensiallikningen ser ut.  Men avhengig av hva $p$
og~$q$ er, er det ikke sikkert at denne andregradslikningen har noen
løsning i reelle tall.  Her blir vi reddet av at vi har lært om
komplekse tall!  Vi kan alltid finne komplekse tall som er løsninger
av hjelpelikningen vår, og disse kan vi igjen bruke til å finne
løsningene av differensiallikningen vi startet med.

\smallskip
Den andre typen differensiallikninger vi skal se på er systemer av
førsteordens lineære differensiallikninger, som vil si
likningssystemer på formen
\[
\left\{
\begin{aligned}
x_1' &= a_{11} x_1 + a_{12} x_2 + \cdots + a_{1n} x_n \\
x_2' &= a_{21} x_1 + a_{22} x_2 + \cdots + a_{2n} x_n \\
     &\ \ \vdots \\
x_n' &= a_{n1} x_1 + a_{n2} x_2 + \cdots + a_{nn} x_n
\end{aligned}
\right.
\]
der koeffisientene $a_{ij}$ er konstanter og hver $x_i$ er en ukjent
funksjon av~$t$.

Når vi har lært om lineær algebra og matriser, ser vi at et slikt
system også kan skrives på den mer kompakte formen
\[
\V{x}' = A \V{x},
\]
der $\V{x}$ er en vektorfunksjon og $A$ er en matrise.  For å løse
systemet vil vi bruke avansert lineær-algebraisk magi som litt
forenklet sagt går ut på å vri det $n$-dimensjonale rommet om til et
nytt $n$-dimensjonalt rom der systemet blir enkelt å løse, så løse det
der, og til slutt vri rommet tilbake og ta løsningene med oss.


\end{document}


\mainmatter
\part{Lineær algebra}
\ifx\inkludert\undefined
\documentclass[norsk,a4paper,twocolumn,oneside]{memoir}

\usepackage[utf8]{inputenc}
\usepackage{babel}
\usepackage{amsmath,amssymb,amsthm}
\usepackage[total={17cm,27cm}]{geometry}
\usepackage[table]{xcolor}
%\usepackage{tabularx}
\usepackage{systeme}
%\usepackage{hyperref}
%\usepackage{enumerate}

%\usepackage{sectsty}
\setsecheadstyle{\bfseries\large}
%\subsectionfont{\bf\normalsize}

\usepackage{tikz}
\usetikzlibrary{arrows.meta}

\newcommand{\defterm}[1]{\emph{#1}}

\newcommand{\N}{\mathbb{N}}
\newcommand{\Z}{\mathbb{Z}}
\newcommand{\Q}{\mathbb{Q}}
\newcommand{\R}{\mathbb{R}}

\newcommand{\abs}[1]{|#1|}

\newcommand{\roweq}{\sim}
\DeclareMathOperator{\Span}{Span}

\newcommand{\V}[1]{\mathbf{#1}}
\newcommand{\vv}[2]{\begin{bmatrix} #1 \\ #2 \end{bmatrix}}
\newcommand{\vvv}[3]{\begin{bmatrix} #1 \\ #2 \\ #3 \end{bmatrix}}
\newcommand{\vvvv}[4]{\begin{bmatrix} #1 \\ #2 \\ #3 \\ #4 \end{bmatrix}}
\newcommand{\vn}[2]{\vvvv{#1_1}{#1_2}{\vdots}{#1_#2}}

\newenvironment{amatrix}[1]{% "augmented matrix"
  \left[\begin{array}{*{#1}{c}|c}
}{%
  \end{array}\right]
}

% \newcounter{notatnr}
% \newcommand{\notatnr}[2]
% {\setcounter{notatnr}{#1}%
%  \setcounter{page}{#2}%
% }

\newtheorem{thm}{Teorem}[chapter]
\newtheorem*{thm-nn}{Teorem}
\newtheorem{cor}[thm]{Korollar}
\newtheorem{lem}[thm]{Lemma}
\newtheorem{prop}[thm]{Proposisjon}
\theoremstyle{definition}
\newtheorem{exx}[thm]{Eksempel}
\newtheorem*{defnx}{Definisjon}
\newtheorem*{oppg}{Oppgave}
\newtheorem*{merkx}{Merk}
\newtheorem*{spmx}{Spørsmål}

\newenvironment{defn}
  {\pushQED{\qed}\renewcommand{\qedsymbol}{$\triangle$}\defnx}
  {\popQED\enddefnx}
\newenvironment{ex}
  {\pushQED{\qed}\renewcommand{\qedsymbol}{$\triangle$}\exx}
  {\popQED\endexx}
\newenvironment{merk}
  {\pushQED{\qed}\renewcommand{\qedsymbol}{$\triangle$}\merkx}
  {\popQED\endmerkx}
\newenvironment{spm}
  {\pushQED{\qed}\renewcommand{\qedsymbol}{$\triangle$}\spmx}
  {\popQED\endspmx}

\setlength{\columnsep}{26pt}

\newcommand{\Tittel}[2]{%
\twocolumn[
\begin{center}
\Large
\begin{tabularx}{\textwidth}{cXr}
\cellcolor{black}\color{white}%
\bf {#1} &
#2
\hfill &
\footnotesize TMA4110 høsten 2018
\\ \hline
\end{tabularx}
\end{center}
]}

\newcommand{\tittel}[1]{\Tittel{\arabic{notatnr}}{#1}}

\newcommand{\linje}{%
\begin{center}
\rule{.8\linewidth}{0.4pt}
\end{center}
}


\newcommand{\chapternumber}{}

\makechapterstyle{tma4110}{%
 \renewcommand*{\chapterheadstart}{}
 \renewcommand*{\printchaptername}{}
 \renewcommand*{\chapternamenum}{}
 \renewcommand*{\printchapternum}{\renewcommand{\chapternumber}{\thechapter}}
 \renewcommand*{\afterchapternum}{}
 \renewcommand*{\printchapternonum}{\renewcommand{\chapternumber}{}}
 \renewcommand*{\printchaptertitle}[1]{
\LARGE
\begin{tabularx}{\textwidth}{cXr}
\cellcolor{black}\color{white}%
\textbf{\chapternumber} &
\textbf{##1}
\hfill &
%\footnotesize TMA4110 høsten 2018
\\ \hline
\end{tabularx}%
}
 \renewcommand*{\afterchaptertitle}{\par\nobreak\vskip \afterchapskip}
 % \newcommand{\chapnamefont}{\normalfont\huge\bfseries}
 % \newcommand{\chapnumfont}{\normalfont\huge\bfseries}
 % \newcommand{\chaptitlefont}{\normalfont\Huge\bfseries}
 \setlength{\beforechapskip}{0pt}
 \setlength{\midchapskip}{0pt}
 \setlength{\afterchapskip}{10pt}
}


\newcounter{oppgnr}[chapter]
\newcounter{punktnr}[oppgnr]
\newenvironment{oppgave}
 {\par\noindent\stepcounter{oppgnr}\textbf{{\arabic{oppgnr}}.}}
 {\par\bigskip}
\newenvironment{punkt}
 {\par\smallskip\noindent\stepcounter{punktnr}\textbf{\alph{punktnr})} }
 {\par}

\newcommand{\oppgaver}{\linje\section*{Oppgaver}}

\usepackage{xr}
\externaldocument{tma4110-2018h}
\newcommand{\kapittel}[2]{\setcounter{chapter}{#1}\addtocounter{chapter}{-1}\chapter{#2}}
\newcommand{\kapittelslutt}{\enddocument}
\begin{document}
\chapterstyle{tma4110}
\pagestyle{plain}
\fi


\kapittel{1}{Lineære likningssystemer}

Grunnlaget for lineær algebra er \emph{lineære liknings\-systemer}.  I
dette notatet starter vi vår reise inn i den lineære algebraen ved å
først se på noen forskjellige måter å løse likningssystemer på, og så
forsikre oss om at vi er helt enige om nøyaktig hva det vil si at et
likningssystem er lineært.

\section*{Forskjellige fremgangsmåter}

Her er et eksempel på et lineært likningssystem:
\[
(\ast)
\systeme{
2x + 3y = 9,
-x + 6y = 3
}
\]
I dette systemet har vi to likninger og to ukjente.
En løsning av systemet består av to tall som vi kan sette inn
for $x$ og~$y$ slik at begge likningene er oppfylt samtidig.

Vi kjenner fra før til flere måter å løse et slikt system på.  La oss
løse systemet over med noen forskjellige metoder.

\begin{ex}
Den kanskje mest åpenbare metoden for å løse et likningssystem er å
først løse én likning med hensyn på én av de ukjente, og så sette inn
i den andre likningen (eller i \emph{de} andre likningene, hvis vi har
et system med mer enn to likninger).

For å løse systemet~$(\ast)$ med denne metoden kan vi først løse den
andre likningen med hensyn på~$x$; da får vi:
\[
x = 6y - 3
\]
Så setter vi dette inn i den første likningen og forenkler:
\begin{align*}
2 \cdot (6y - 3) + 3y &= 9 \\
12y - 6 + 3y &= 9 \\
15y &= 15 \\
y &= 1
\end{align*}
Til slutt setter vi denne $y$-verdien inn i uttrykket vi fant for~$x$,
og får:
\[
x = 6y - 3 = 6 - 3 = 3
\]

Vi har altså funnet ut at for at begge likningene skal være oppfylt,
må vi ha at $x = 3$ og $y = 1$.  Vi sjekker at dette virkelig er en
løsning av~$(\ast)$ ved å sette inn disse verdiene i begge likningene
i systemet:
\begin{align*}
2x + 3y &= 2 \cdot 3 + 3 \cdot 1 = 6 + 3 = 9 \\
-x + 6y &= -3 + 6 \cdot 1 = 3
\end{align*}
Vi har nå funnet ut at systemet~$(\ast)$ har nøyaktig én løsning,
nemlig $x = 3$ og $y = 1$.
\end{ex}

% Metoden i eksempelet over er enkel og grei, men kan bli temmelig
% tungvint å bruke hvis vi har mer enn to ukjente.  Vi 

Vi ser på en annen løsningsmetode for det samme systemet:

\begin{ex}
Vi ganger opp den andre likningen med~$2$, og deretter legger vi
sammen de to likningene:
\begin{align*}
2x + 3y &= 9   &&\text{(første likning)} \\
-2x + 12y &= 6 &&\text{(andre likning ganget med~$2$)} \\[4pt] \hline
\rule{0pt}{11pt}
15y &= 15      &&\text{(sum av likningene over)}
\end{align*}
På denne måten får vi $x$ til å forsvinne, og vi står igjen med en
likning med bare~$y$.

Den nye likningen $15y = 15$ kan vi forenkle til $y=1$.  Nå kan vi
gange opp denne med $-3$ og legge sammen med den første likningen fra
systemet for å få en likning der $y$ forsvinner:
\begin{align*}
2x + 3y &= 9 &&\text{(første likning fra $(\ast)$)}\\
-3y &= -3    &&\text{(ny likning ganget med $-3$)} \\[4pt] \hline
\rule{0pt}{11pt}
2x &= 6      &&\text{(sum av likningene over)}
\end{align*}
Nå har vi fått en likning med bare~$x$, og vi forenkler den til
$x = 3$.
Vi har dermed igjen funnet løsningen $x = 3$ og $y = 6$.
\end{ex}

% Når vi (i neste notat) skal beskrive en  ...

\begin{ex}
Vi kan også løse systemet~$(\ast)$ grafisk.  Vi lager et
koordinatsystem med en $x$-akse og en $y$-akse.  Hver av de to
likningene $2x+3y=9$ og $-x+6y=3$ beskriver en rett linje:
\begin{center}
\begin{tikzpicture}[scale=.9]
\draw[->] (-1.5,0) -- (6.5,0);
\draw[->] (0,-1.5) -- (0,5.5);
\foreach \x in {-1,1,2,3,4,5,6}
\draw (\x cm,1pt) -- (\x cm,-1pt) node[anchor=north] {$\x$};
\foreach \y in {-1,1,2,3,4,5}
\draw (1pt,\y cm) -- (-1pt,\y cm) node[anchor=east] {$\y$};
\draw (-1.5,4) -- (6,-1);
\node[anchor=west] at (.4,3) {$2x + 3y = 9$};
\draw (-1.5,0.25) -- (6,1.5);
\node at (6,2) {$-x + 6y = 3$};
\filldraw (3,1) circle [radius=2pt];
\end{tikzpicture}
\end{center}
Løsningene av $2x + 3y = 9$ er  alle punkter som ligger på den
ene linjen, mens løsningene av $-x + 6y = 3$ er alle punkter som
ligger på den andre linjen.  Den felles løsningen av begge likningene
er punktet der de to linjene møtes, nemlig $(3,1)$.  Løsningen er
altså $x = 3$ og $y = 1$.

Denne metoden er fin for å visualisere problemet og se løsningen på en
intuitiv måte, men ikke nødvendigvis den beste for å finne svaret
eksakt.  Dessuten blir det vanskelig å tegne hvis vi har mer enn to
ukjente (men det kan likevel være nyttig å prøve å se for seg
løsningene av for eksempel en likning med tre ukjente på en grafisk
måte).
\end{ex}


\section*{Hva er et lineært likningssystem?}

Et lineært likningssystem er et system av lineære likninger.  Men
nøyaktig hva mener vi med at en likning er lineær?

Ordet «lineær» kommer fra det latinske «linea», som betyr «linje».
Hvis vi har en likning med to ukjente, så kan vi tegne grafen til
denne likningen.  Vi sier at likningen er lineær hvis grafen dens er
en rett linje.  I eksempelet over så vi at grafene til likningene
$2x + 3y = 9$ og $-x + 6y = 3$ er rette linjer.  Men det er også lett
å finne likninger som ikke har rettlinjede grafer, for eksempel
$y = x^2$ eller $x^2 + y^2 = 4$.

Generelt er det slik at grafen til en likning er en rett linje hvis og
bare hvis likningen kan skrives på formen
\[
ax + by = c,
\]
der $a$, $b$ og~$c$ er konstanter.

Når vi vil se på likninger med mer enn to ukjente, gir det ikke lenger
mening å snakke om at grafen blir en rett linje.  Men det at likningen
kan skrives på formen $ax + by = c$ kan vi lett utvide til å ta med
flere ukjente, så det er dette vi bruker i den generelle definisjonen
av lineære likninger.

\begin{defn}
En likning med $n$ ukjente $x_1$, $x_2$, \ldots, $x_n$ kalles
\defterm{lineær} dersom den kan skrives på formen
\[
a_1 x_1 + a_2 x_2 + \cdots a_n x_n = b,
\]
der $a_1$, $a_2$, \ldots, $a_n$ og~$b$ er konstanter.  Et
\defterm{lineært likningssystem} er en samling av én eller flere
lineære likninger med de samme ukjente.
\end{defn}

% \begin{merk}
% Hvis vi har bare to eller tre ukjente i likningene våre, kaller vi dem
% gjerne $x$, $y$ og~$z$.  Hvis vi har mange ukjente, blir det tungvint
% å skulle finne en ny bokstav for hver av dem, så da kan vi for
% eksempel kalle dem $x_1$, $x_2$, $x_3$ og så videre, som i
% definisjonen over.  Men det spiller egentlig ingen rolle hva de ukjente heter.
% \end{merk}

\begin{ex}
Her er noen eksempler på lineære likninger:
\begin{gather*}
5 x_1 + x_2 - 7 x_3 + x_4 = 27 \\
17 x - 5y = \pi
\end{gather*}
Og her er noen eksempler på likninger som ikke er lineære:
\begin{gather*}
2 x^2 + y = 3 \\
x_1 + 2 x_1 x_2 + x_3 = 1
\qedhere
\end{gather*}
\end{ex}


\section*{Ekvivalente systemer}

Vi sier at to likningssystemer er \defterm{ekvivalente} dersom de har
samme løsninger.  Vi kan løse et likningssystem ved å erstatte det med
stadig enklere ekvivalente systemer.  Vi tar et eksempel for å se
hvordan dette kan gjøres.

\begin{ex}
\label{ex:gausseliminasjon}
La oss løse det følgende lineære likningssystemet:
\[
\systeme{
   x + 2y - 2z = -5,
   x + 5y + 9z = 33,
  2x + 5y -  z = 0
}
\]
Vi vil begynne med å eliminere $x$-en fra de to siste likningene.
Hvis vi trekker den første likningen fra den andre, får vi den nye
likningen
\[
3y + 11z = 38.
\]
Vi bytter ut den andre likningen i systemet med denne nye likningen:
\[
\systeme{
   x + 2y -  2z = -5,
       3y + 11z = 38,
  2x + 5y -   z = 0
}
\]
På samme måte kan vi eliminere $x$ fra den siste likningen ved å
trekke fra første likning ganget med~$2$:
\[
\systeme{
   x + 2y -  2z = -5,
       3y + 11z = 38,
        y +  3z = 10
}
\]
Nå vil vi eliminere $y$ fra den siste likningen.  Men det er lettere å
eliminere $y$ fra den midterste likningen (ved å trekke fra $3$ ganger den
siste).  Likningenes rekkefølge spiller imidlertid ingen rolle, så vi
kan bytte om på de to nederste likningene først:
\[
\systeme{
   x + 2y -  2z = -5,
        y +  3z = 10,
       3y + 11z = 38
}
\]
Så eliminerer vi $y$ fra den siste likningen ved å trekke fra andre
likning ganget med~$3$:
\[
\systeme{
   x + 2y - 2z = -5,
        y + 3z = 10,
            2z =  8
}
\]
Nå ser vi fra den siste likningen at vi må ha $z = 4$.  Ved å sette
inn det i den midterste likningen får vi $y = 10 - 3 \cdot 4 = -2$.
Til slutt får vi ved å sette inn i den øverste likningen at
$x = -5 - 2 \cdot (-2) + 2 \cdot 4 = 7$.
\end{ex}

Alle likningssystemene vi skrev opp i dette eksempelet er ekvivalente
med hverandre, men det siste er mye enklere å håndtere enn det vi
startet med.  Prosessen vi utførte for å forenkle systemet kalles
\emph{gauss\-eliminasjon}, og er nærmere beskrevet i neste notat.


\section*{Totalmatrisen til et system}

Generelt kan et lineært likningssystem (med $m$ likninger og $n$
ukjente) se slik ut:
\[
\left\{
\begin{aligned}
  a_{11} x_1 + a_{12} x_2 + \cdots + a_{1n} x_n &= b_1 \\
  a_{21} x_1 + a_{22} x_2 + \cdots + a_{2n} x_n &= b_2 \\
                                                &\ \ \vdots \\
  a_{m1} x_1 + a_{m2} x_2 + \cdots + a_{mn} x_n &= b_m
\end{aligned}
\right.
\]
Når vi skal løse et slikt system, vil vi skrive opp en rekke nye
systemer som er ekvivalent med dette, men som er stadig enklere.  Da
er det unødvendig tungvint å skrive alle de ukjente og alle $+$-ene og
$=$-tegnene hver eneste gang.  Den eneste informasjonen vi trenger å
ha med oss gjennom utregningen er koeffisientene ($a$-ene) og tallene
på høyresiden ($b$-ene).

\defterm{Totalmatrisen} til et likningssystem er en tabell som
inneholder akkurat disse tallene:
\[
\left[
\begin{array}{cccc|c}
  a_{11} & a_{12} & \cdots & a_{1n} & b_1 \\
  a_{21} & a_{22} & \cdots & a_{2n} & b_2 \\
  \vdots & \vdots & \ddots & \vdots & \vdots \\
  a_{m1} & a_{m2} & \cdots & a_{mn} & b_m
\end{array}
\right]
\]

\begin{ex}
Likningssystemet vi startet med i eksempel~\ref{ex:gausseliminasjon}
har følgende totalmatrise:
\[
\raisebox{15pt}{$
\begin{amatrix}{3}
1 & 2 & -2 & -5 \\
1 & 5 &  9 & 33 \\
2 & 5 & -1 & 0
\end{amatrix}
$}\qedhere
\]
\end{ex}

Den loddrette streken inni matrisen er egentlig ikke nødvendig å ha
med, men den er praktisk for å hjelpe oss med å huske at det som står
til høyre for streken hører til høyre side av likningene.

\kapittelslutt

\oppgaver{1}

\begin{oppgave}
Hvilke av disse likningene er lineære?
\begin{punkt}
$14x + 3y = 2x + 1 - 5z$
\end{punkt}
\begin{punkt}
$x + 2xy + y = 1$
\end{punkt}
\begin{punkt}
$\frac{x + y}{2} = z$
\end{punkt}
\end{oppgave}

\begin{losning}
Likning (a) og (c) er lineære; (b) er ikke.
\end{losning}


\begin{oppgave}
Lag et lineært likningssystem med to likninger og to ukjente som
\begin{punkt}
\ldots\ har entydig løsning.
\end{punkt}
\begin{punkt}
\ldots\ ikke har noen løsning.
\end{punkt}
\begin{punkt}
\ldots\ har uendelig mange løsninger.
\end{punkt}
\smallskip\noindent
I hver deloppgave: Tegn grafene til de to likningene i systemet ditt.
\end{oppgave}

\begin{losning}
Det finnes mange eksempler som alle tilfredstiller at
\begin{punkt}
\ldots\ linjene skjærer hverandre i ett punkt:
\end{punkt}
\begin{center}
	\begin{tikzpicture}
	\draw[->] (-2,0) -- (2,0) node[right] {$x$};
	\draw[->] (0,-2) -- (0,2) node[above] {$y$};
	\draw[scale=0.7,domain=-2:2,smooth,variable=\x] plot ({\x},{\x});
	\draw[scale=0.7,domain=-2:2,smooth,variable=\y]  plot ({\y},{(1)});
	\end{tikzpicture}
\end{center}
\begin{punkt}
\ldots\ linjene er parallelle:
\end{punkt}
\begin{center}
	\begin{tikzpicture}
	\draw[->] (-2,0) -- (2,0) node[right] {$x$};
	\draw[->] (0,-2) -- (0,2) node[above] {$y$};
	\draw[scale=0.7,domain=-2:2,smooth,variable=\x] plot ({\x},{\x});
	\draw[scale=0.7,domain=-2:2,smooth,variable=\y]  plot ({\y},{(\y+1)});
	\end{tikzpicture}
\end{center}
\begin{punkt}
\ldots\ linjene er helt like:
\end{punkt}
\begin{center}
	\begin{tikzpicture}
	\draw[->] (-2,0) -- (2,0) node[right] {$x$};
	\draw[->] (0,-2) -- (0,2) node[above] {$y$};
	\draw[scale=0.7,domain=-2:2,smooth,variable=\x] plot ({\x},{\x});
	\draw[scale=0.7,domain=-2:2,smooth,variable=\y]  plot ({\y},{(\y)});
	\end{tikzpicture}
\end{center}
\end{losning}



\begin{oppgave}
En lineær likning med to ukjente kan tegnes som en rett linje i $x$--$y$-planet.
\begin{punkt}
Hvordan kan vi på tilsvarende måte se for oss en lineær likning med tre ukjente?
\end{punkt}
\begin{punkt}
Se på følgende likningssystem:
\[
\systeme{
  x + y + z = 5,
  z = 3
}
\]
Tegn en figur som illustrerer løsningene av hver av disse likningene
og løsningene av systemet.
\end{punkt}
\end{oppgave}

\chapter{Gausseliminasjon}

I dette notatet skal vi formalisere ideene fra forrige notat.  Vi
skal se hvordan vi kan løse et hvilket som helst lineært
likningssystem ved å skrive om totalmatrisen til systemet etter
bestemte regler.

Reglene for hvordan totalmatrisen kan skrives om kalles
\emph{radoperasjoner}, og målet er å få en matrise som er på
\emph{trappeform}.  Denne prosessen kalles \emph{gausseliminasjon}.


\section*{Radoperasjoner}

Følgende tre måter å endre en matrise på kalles
\defterm{radoperasjoner}:
\begin{enumerate}
\item Gange alle tallene i en rad med det samme tallet (ikke~$0$).
\item Legge til (et multiplum av) en rad i en annen.
\item Bytte rekkefølge på radene.
\end{enumerate}

Vi sier at to matriser er \defterm{radekvivalente} hvis vi kan komme
fra den ene til den andre ved å utføre en eller flere radoperasjoner.
Vi bruker notasjonen $M \roweq N$ for å si at to matriser $M$ og~$N$
er radekvivalente.

\begin{ex}
\label{ex:radekvivalent}
Disse matrisene er radekvivalente, siden vi får den andre matrisen fra
den første ved å gange øverste rad med~$4$:
\begin{align*}
\begin{bmatrix}
 2 & 5 \\
 1 & 7
\end{bmatrix}
&\roweq
\begin{bmatrix}
 8 & 20 \\
 1 &  7
\end{bmatrix}
\end{align*}
Merk at vi også kan gå motsatt vei: Ved å gange øverste rad i den
andre matrisen med $1/4$ får vi tilbake den første matrisen.

Disse to matrisene er også radekvivalente:
\begin{align*}
\begin{bmatrix}
 3 & 1 \\
 9 & 5
\end{bmatrix}
&\roweq
\begin{bmatrix}
 3 & 1 \\
 0 & 2
\end{bmatrix}
\end{align*}
Her har vi brukt den andre typen radoperasjon: Vi la til $-3$ ganger
øverste rad i nederste rad for å komme fra den første matrisen til den
andre.  Merk igjen at vi også kan gå motsatt vei: Ved å legge til $3$
ganger øverste rad i nederste rad, kommer vi fra den andre matrisen
til den første.
\end{ex}

Hele poenget med radoperasjoner er at det å utføre en radoperasjon på
en totalmatrise tilsvarer å skrive om likningssystemet til et nytt
system som er ekvivalent med det opprinnelige.  Vi formulerer dette
som et teorem:

\begin{thm}
\label{thm:radekvivalens}
Hvis to likningssystemer har radekvivalente totalmatriser, så er de to
likningssystemene ekvivalente.
\end{thm}
\begin{proof}
For å bevise dette, er det nok å vise at det å gjøre en radoperasjon
på totalmatrisen til et likningssystem tilsvarer å gjøre en gyldig
omskrivning av systemet selv.

Den første typen radoperasjon -- å gange alle tallene i en rad med
samme tall -- tilsvarer å gange med det samme tallet på begge sider av
en ligning.  Litt mer detaljert: La oss si at
\[
a_{i1}\ a_{i2}\ \cdots\ a_{in}\ |\ b_i
\]
er en av radene i totalmatrisen, og at vi ganger opp denne med
tallet~$c$ slik at vi får:
\[
(c a_{i1})\ (c a_{i2})\ \cdots\ (c a_{in})\ |\ (c b_i)
\]
Dette tilsvarer at vi bytter ut likningen
\[
a_{i1} x_1 + a_{i2} x_2 + \cdots + a_{in} x_n = b_i
\]
med den nye likningen
\[
(c a_{i1}) x_1 + (c a_{i2}) x_2 + \cdots + (c a_{in}) x_n = c b_i.
\]
Men det er klart at hvis den opprinnelige likningen var sann, så må
også den nye være det.  Og siden det ikke tillates at tallet~$c$ som
vi ganger med er~$0$, så har vi også det motsatte: Hvis den nye
likningen er sann, så må også den opprinnelige være det.  Altså gjør
vi ingen endring i løsningene av likningssystemet ved å utføre denne
typen radoperasjon.

For den andre typen radoperasjon -- legge til et multiplum av en rad i
en annen -- kan vi på tilsvarende måte se at den nye raden vi lager
tilsvarer en likning som må være sann hvis de gamle likningene var
sanne.  Sett at vi legger til $c$ ganger rad~$i$ i rad~$j$.  Dette
tilsvarer at vi ganger opp den $i$-te likningen med~$c$, og legger til
resultatet i den $j$-te likningen.  Alle løsninger av de gamle
likningene må da også være løsninger av denne nye likningen.  Dessuten
kan vi komme tilbake til det gamle systemet (ved å legge til $-c$
ganger rad~$i$ i rad~$j$), og dermed må alle løsninger av det nye
systemet også være løsninger av det gamle.

Den tredje og siste typen radoperasjon -- bytte rekkefølge på radene
-- gjør åpenbart ingen endringer i løsningene av likningssystemet,
siden dette bare tilsvarer å skrive likningene i en annen rekkefølge.
\end{proof}

\begin{ex}
\label{ex:gausseliminasjon1}
Vi gjentar regningen i eksempel~2.5 fra forrige notat, denne gangen
ved å utføre radoperasjoner på totalmatrisen til likningssystemet:
\begin{align*}
\begin{amatrix}{3}
1 & 2 & -2 & -5 \\
1 & 5 &  9 & 33 \\
2 & 5 & -1 &  0
\end{amatrix}
&\roweq
\begin{amatrix}{3}
1 & 2 & -2 & -5 \\
0 & 3 & 11 & 38 \\
2 & 5 & -1 &  0
\end{amatrix}
\\
&\roweq
\begin{amatrix}{3}
1 & 2 & -2 & -5 \\
0 & 3 & 11 & 38 \\
0 & 1 &  3 & 10
\end{amatrix}
\\
&\roweq
\begin{amatrix}{3}
1 & 2 & -2 & -5 \\
0 & 1 &  3 & 10 \\
0 & 3 & 11 & 38
\end{amatrix}
\\
&\roweq
\begin{amatrix}{3}
1 & 2 & -2 & -5 \\
0 & 1 &  3 & 10 \\
0 & 0 &  2 &  8
\end{amatrix}
\end{align*}
Her gjorde vi følgende radoperasjoner: Legge til $-1$ ganger første
rad i andre rad, legge til $-2$ ganger første rad i tredje rad, bytte
andre og tredje rad, og legge til $-3$ ganger andre rad i tredje rad.

Den siste matrisen her er på det som kalles trappeform, og da er det
(som vi så i eksempel~2.5) lett å finne løsningen.  Hvis vi vil
gjøre det enda lettere, kan vi fortsette med radoperasjoner til vi
oppnår det som kalles \emph{redusert trappeform}:
\begin{align*}
\begin{amatrix}{3}
1 & 2 & -2 & -5 \\
0 & 1 &  3 & 10 \\
0 & 0 &  2 &  8
\end{amatrix}
&\roweq
\begin{amatrix}{3}
1 & 2 & -2 & -5 \\
0 & 1 &  3 & 10 \\
0 & 0 &  1 &  4
\end{amatrix}
\\
&\roweq
\begin{amatrix}{3}
1 & 2 &  0 &  3 \\
0 & 1 &  0 & -2 \\
0 & 0 &  1 &  4
\end{amatrix}
\\
&\roweq
\begin{amatrix}{3}
1 & 0 &  0 &  7 \\
0 & 1 &  0 & -2 \\
0 & 0 &  1 &  4
\end{amatrix}
\end{align*}
Den siste totalmatrisen her svarer til følgende likningssystem:
\[
\systeme*{
x = 7,
y = -2,
z = 4
}
\]
Her har vi altså kommet helt frem til løsningen.
\end{ex}


\section*{Trappeform}

Vi vil nå gi en presis definisjon av begrepene «trappeform» og
«redusert trappeform».  Da trenger vi også et annet begrep, nemlig
«pivotelement».

% TODO: endre til «lederelement»?  ta med «pivotposisjon», «pivotkolonne»?
\begin{defn}
Det første tallet i en rad i en matrise som ikke er~$0$ kalles
\defterm{pivotelementet} for den raden.  (En rad med bare nuller har
ikke noe pivotelement.)
\end{defn}

\begin{ex}
\label{ex:pivotelement}
Se på følgende matrise:
\[
\begin{bmatrix}
3 & -2 & 0 & 2 \\
0 &  0  & 5 & 12 \\
1 &  8 & 3 & 7 \\
0 &  0  & 0 & 0
\end{bmatrix}
\]
Pivotelementene her er tallet~$3$ i den øverste raden, tallet~$5$ i
den andre raden og tallet~$1$ i den tredje raden.  Den siste raden
består av bare nuller, og har derfor ikke noe pivotelement.
\end{ex}

\begin{defn}
En matrise er på \defterm{trappeform} dersom hvert pivotelement er til
høyre for alle pivotelementer i tidligere rader, og eventuelle
nullrader er helt nederst.
\end{defn}

\begin{ex}
\label{ex:trappeform}
Denne matrisen er på trappeform:
\[
\begin{bmatrix}
3 & 7 & 6 \\
0 & 1 & -2 \\
0 & 0 & 2
\end{bmatrix}
\]
Pivotelementene er $3$, $1$ og~$2$, og hvert av dem er til høyre for
alle de tidligere pivotelementene.

Denne matrisen er også på trappeform:
\[
\begin{bmatrix}
-2 & 7 & 1 & 5 \\
 0 & 0 & 4 & 9 \\
 0 & 0 & 0 & 0
\end{bmatrix}
\]

Denne matrisen er ikke på trappeform fordi nullradene ikke er samlet
nederst:
\[
\begin{bmatrix}
3 & 8 & 2 & 0 \\
0 & 0 & 0 & 0 \\
0 & 2 & 1 & 4 \\
0 & 0 & 0 & 0
\end{bmatrix}
\]

Denne matrisen ser «trappete» ut, men er likevel ikke på trappeform:
\[
\begin{bmatrix}
5 & 8 & 7 & 2 \\
0 & 4 & 1 & 6 \\
0 & 9 & 2 & 0 \\
0 & 0 & 3 & 1
\end{bmatrix}
\]
Grunnen til at den ikke er på trappeform er at pivotelementet $9$ i
tredje rad ikke er til høyre for pivotelementet i andre rad, men rett
under det isteden.
\end{ex}

\begin{defn}
En matrise er på \defterm{redusert trappeform} hvis den er på
trappeform og dessuten oppfyller:
\begin{itemize}
\item Alle pivotelementene er~$1$.
\item Alle tall som står over pivotelementer er~$0$.\qedhere
\end{itemize}
\end{defn}

Den siste matrisen i eksempel~\ref{ex:gausseliminasjon1} er på
redusert trappeform, og der så vi også hva som gjør redusert
trappeform nyttig: Løsningen av systemet kan leses av direkte.

\medskip
Det å  skrive om en matrise til trappeform
ved hjelp av radoperasjoner
kalles \defterm{gausseliminasjon}, oppkalt etter den tyske
matematikeren Carl Friedrich Gauss (1777--1855).  Noen velger også å
ha et eget navn på det å komme frem til \emph{redusert} trappeform, og
kaller den prosessen for \defterm{Gauss--Jordan-eliminasjon}, oppkalt
etter Wilhelm Jordan (1842--1899).  Vi tar det ikke så nøye med den
forskjellen, og sier «gauss\-eliminasjon» uansett.

(Disse begrepene er uansett historisk sett fullstendig misvisende.
Metoden som vi kaller gauss\-eli\-minasjon var kjent i Kina for flere
tusen år siden, og Gauss -- som riktignok var et universalgeni og fant
opp mengder av flotte ting -- har ikke egentlig så mye med den å
gjøre.)


\section*{Eksistens og entydighet av løsninger}

Når vi vil løse et likningssystem, er det noen åpenbare spørsmål vi
kan stille:
\begin{itemize}
\item Har systemet noen løsning? (\emph{Eksistens})
\item Hvis systemet har løsning: Har det også flere løsninger, eller
bare én?  (\emph{Entydighet})
\end{itemize}

I eksempel~\ref{ex:gausseliminasjon1} hadde vi et system med entydig
løsning.  Vi tar noen flere eksempler for å vise andre ting som kan
skje.

\begin{ex}
\label{ex:gausseliminasjon2}
La oss løse følgende system:
\[
\systeme{
   x -  2 y = 1,
-5 x + 10 y = -1
}
\]
Vi setter opp totalmatrisen og gausseliminerer:
\[
\begin{amatrix}{2}
 1 & -2 &  1 \\
-5 & 10 & -1
\end{amatrix}
\roweq
\begin{amatrix}{2}
1 & -2 & 1 \\
0 &  0 & 4
\end{amatrix}
\]
Den siste matrisen svarer til følgende system:
\[
\systeme{
  x + 2 y = 1,
0 x + 0 y = 4
}
\]
Likningen $0x + 0y = 4$ kan også skrives som $0 = 4$, og den kan ikke
stemme uansett hva vi setter $x$ og~$y$ til å være.  Dette systemet
har altså ingen løsning.
\end{ex}

Generelt er det slik at hvis vi får en rad i totalmatrisen vår på
formen
\[
0\ 0\ \cdots\ 0\ |\ b,
\]
der $b$ er et tall som ikke er~$0$, så har systemet ingen løsning.
Denne raden svarer jo til likningen $0 = b$, som ikke kan være sann.
Hvis vi har en matrise på trappeform der ingen av radene er på denne
formen, så har systemet minst én løsning.

Men et lineært likningssystem kan også ha mer enn én løsning, som vi
skal se i det neste eksempelet.

\begin{ex}
\label{ex:gausseliminasjon3}
La oss løse følgende system:
\[
\systeme{
  x_1 + 3 x_2 + 2 x_3 + 3 x_4 = 16,
  x_1 + 3 x_2 + 3 x_3 + 1 x_4 = 21,
2 x_1 + 6 x_2 + 4 x_3 + 6 x_4 = 32
}
\]
Vi setter opp totalmatrisen og gausseliminerer:
\begin{align*}
\begin{amatrix}{4}
1 & 3 & 2 &  3 & 16 \\
1 & 3 & 3 &  1 & 21 \\
2 & 6 & 4 &  6 & 32
\end{amatrix}
&\roweq
\begin{amatrix}{4}
1 & 3 & 2 &  3 & 16 \\
0 & 0 & 1 & -2 &  5 \\
0 & 0 & 0 &  0 &  0
\end{amatrix}
\\
&\roweq
\begin{amatrix}{4}
1 & 3 & 0 &  7 & 6 \\
0 & 0 & 1 & -2 & 5 \\
0 & 0 & 0 &  0 & 0
\end{amatrix}
\end{align*}
Den siste matrisen svarer til følgende system:
\[
\systeme*{
x_1 + 3 x_2 + 7 x_4 = 6,
x_3 - 2 x_4 = 5
}
\]
(Her har vi ikke tatt med noen likning for nullraden i matrisen.  Det
er fordi nullraden står for likningen $0x_1 + 0x_2 + 0x_3 + 0x_4 = 0$,
eller med andre ord $0 = 0$.  Denne likningen er åpenbart oppfylt
uansett hva $x_1$, $x_2$, $x_3$ og~$x_4$ er, så vi trenger ikke ta den
med.)

Hvis vi flytter alt unntatt $x_1$ og~$x_3$ til høyresiden, ser
systemet slik ut:
\[
\left\{
\begin{aligned}
x_1 &= - 3 x_2 - 7 x_4 + 6 \\
x_3 &= 2 x_4 + 5
\end{aligned}
\right.
\]
Vi kan altså finne løsninger av systemet ved å sette $x_2$ og~$x_4$
til å være hva vi vil, og deretter bruke disse to likhetene til å
bestemme $x_1$ og~$x_3$.

Hvis vi for eksempel velger å sette $x_2 = 0$ og $x_4 = 1$, så får vi
følgende løsning:
\[
\left\{
\begin{aligned}
x_1 &= -3 \cdot 0 - 7 \cdot 1 + 6 = -1 \\
x_2 &= 0 \\
x_3 &= 2 \cdot 1 + 5 = 7 \\
x_4 &= 1
\end{aligned}
\right.
\]

For å beskrive alle løsningene av systemet på en ryddig måte, kan vi
sette $x_2 = s$ og $x_4 = t$, der $s$ og~$t$ står for to vilkårlige
tall.  Da er alle løsningene gitt ved:
\[
\raisebox{23pt}{$
\left\{
\begin{aligned}
x_1 &= -3 s - 7 t + 6 \\
x_2 &= s \\
x_3 &= 2 t + 5 \\
x_4 &= t
\end{aligned}
\right.
$}
\qedhere
\]
\end{ex}

Variabler som vi kan sette til hva vi vil, slik som $x_2$ og~$x_4$ i
eksempelet over, kalles \defterm{frie variabler}.

Når vi løser et lineært likningssystem, og har funnet ut at det har
minst én løsning, så er det to muligheter.  Den ene muligheten er at
vi ikke får noen frie variabler (slik som i
eksempel~\ref{ex:gausseliminasjon1}).  Da har systemet entydig
løsning.  Den andre muligheten er at det er en eller flere frie
variabler.  Da har systemet uendelig mange løsninger, siden hver av de
frie variablene kan settes til å være et hvilket som helst tall.

Dette betyr at det ikke er mulig at vi får for eksempel to løsninger,
eller tre løsninger, og så videre.  Om det først er mer enn én
løsning, må det være uendelig mange.

\medskip

La oss oppsummere det vi har funnet ut om eksistens og entydighet av
løsninger.  For ethvert lineært likningssystem må én av følgende være
sant:
\begin{itemize}
\item Systemet har ingen løsning.
\item Systemet har entydig løsning.
\item Systemet har uendelig mange løsninger.
\end{itemize}


\section*{Valgfrihet}

Når vi gausseliminerer har vi en viss grad av valgfrihet.
(TODO)

% -*- TeX-master: "oving01"; -*-
\oppgaver{2}

\begin{oppgave}
Hvilke av disse matrisene er på trappeform?  Hvilke av dem er på
redusert trappeform?
\begin{punkt}
$
\begin{bmatrix}
1 & 5 & 0 & 0 \\
0 & 0 & 0 & 1
\end{bmatrix}
$
\end{punkt}
\begin{punkt}
$
\begin{bmatrix}
1 & 0 \\
0 & 1 \\
0 & -1
\end{bmatrix}
$
\end{punkt}
\begin{punkt}
$
\begin{bmatrix}
0 & 2 & 1 \\
0 & 0 & 4 \\
0 & 0 & 0
\end{bmatrix}
$
\end{punkt}
\begin{punkt}
$
\begin{bmatrix}
0 & 0 & 0 \\
0 & 0 & 0 \\
0 & 0 & 0
\end{bmatrix}
$
\end{punkt}
\end{oppgave}


\begin{oppgave}
Løs likningssystemene.
\begin{punkt}
$
\systeme{
  2x - 4y + 9z = -38,
  4x - 3y + 8z = -26,
 -2x + 4y - 2z =  17
}
$
\end{punkt}
\begin{punkt}
(TODO: system med uendelig mange løsninger)
\end{punkt}
\begin{punkt}
(TODO: system med ingen løsninger)
\end{punkt}
\end{oppgave}


\begin{oppgave}
\begin{punkt}
Er disse to likningssystemene ekvivalente?
(TODO: to likningssystemer som er ekvivalente)
\end{punkt}
\begin{punkt}
Er disse to matrisene radekvivalente?
(TODO: to totalmatriser som ikke er radekvivalente)
\end{punkt}
\end{oppgave}


\begin{oppgave}
Anta at vi har et likningssystem med $m$~likninger og~$n$ ukjente.
Hvilke av de ni forskjellige tilfellene i følgende tabell er mulige?
\[
\begin{array}{r|c|c|c|}
                                & m < n & m = n & m > n \\ \hline
\text{ingen løsninger}          &       &       &       \\ \hline
\text{én løsning}               &       &       &       \\ \hline
\text{uendelig mange løsninger} &       &       &       \\ \hline
\end{array}
\]
\end{oppgave}


\begin{oppgave}
Se på likningssystemet
\[
\systeme[xy]{
  ax + by = m,
  cx + dy = n
}
\]
der $a$, $b$, $c$, $d$, $m$ og~$n$ er konstanter, og vi antar at $ad \ne bc$.

Hvor mange løsninger har systemet?  Finn løsningen(e) uttrykt ved $a$,
$b$, $c$, $d$, $m$ og~$n$.
\end{oppgave}


\begin{oppgave}
(TODO: velg tre passende punkter og skriv ferdig oppgaveteksten)

tre punkter i planet, vil finne andregradspolynom $ax^2 + bx + c$ slik
at grafen går gjennom de tre punktene
\begin{punkt}
Sett opp et lineært likningssystem for $a$, $b$ og~$c$.
\end{punkt}
\begin{punkt}
Løs systemet, og finn andregradspolynomet som går gjennom alle punktene.
\end{punkt}
\end{oppgave}


\begin{oppgave}
% 1  4  1
% 0 -2  3
% 0  0 17
% ---
% 1  4  1
% 0 -2  3
% 0  8  5
% ---
% 1  4  1
% 0 -2  3
%-3 -4  2
% ---
% 1  4  1
% 5 18  8
%-3  4  2
La $p$ og~$q$ være følgende polynomer:
\begin{align*}
p(x) &= x^2 + 5x - 3 \\
q(x) &= 4x^2 + 18x + 4
\end{align*}
\begin{punkt}
La $s$ være polynomet $s(x) = x^2 + 8x + 2$.  Finnes det konstanter
$a$ og~$b$ slik at
\[
s(x) = a \cdot p(x) + b \cdot q(x)
\]
for alle~$x$?
\end{punkt}
\begin{punkt}
Finn et andregradspolynom $t$ som oppfyller følgende: For hvert
andregradspolynom $r$ skal det være mulig å finne konstanter $a$, $b$
og~$c$ slik at
\[
r(x) = a \cdot p(x) + b \cdot q(x) + c \cdot t(x)
\]
\end{punkt}
\end{oppgave}


\begin{oppgave}
Vis at følgende påstander er sanne for alle matriser $M$, $N$ og~$L$:
\begin{punkt}
$M \roweq M$.
\end{punkt}
\begin{punkt}
Hvis $M \roweq N$, så: $N \roweq M$.
\end{punkt}
\begin{punkt}
Hvis $M \roweq L$ og $L \roweq N$, så: $M \roweq N$.
\end{punkt}
\end{oppgave}

\ifx\inkludert\undefined
\documentclass[norsk,a4paper,twocolumn,oneside]{memoir}

\usepackage[utf8]{inputenc}
\usepackage{babel}
\usepackage{amsmath,amssymb,amsthm}
\usepackage[total={17cm,27cm}]{geometry}
\usepackage[table]{xcolor}
%\usepackage{tabularx}
\usepackage{systeme}
%\usepackage{hyperref}
%\usepackage{enumerate}

%\usepackage{sectsty}
\setsecheadstyle{\bfseries\large}
%\subsectionfont{\bf\normalsize}

\usepackage{tikz}
\usetikzlibrary{arrows.meta}

\newcommand{\defterm}[1]{\emph{#1}}

\newcommand{\N}{\mathbb{N}}
\newcommand{\Z}{\mathbb{Z}}
\newcommand{\Q}{\mathbb{Q}}
\newcommand{\R}{\mathbb{R}}

\newcommand{\abs}[1]{|#1|}

\newcommand{\roweq}{\sim}
\DeclareMathOperator{\Span}{Span}

\newcommand{\V}[1]{\mathbf{#1}}
\newcommand{\vv}[2]{\begin{bmatrix} #1 \\ #2 \end{bmatrix}}
\newcommand{\vvv}[3]{\begin{bmatrix} #1 \\ #2 \\ #3 \end{bmatrix}}
\newcommand{\vvvv}[4]{\begin{bmatrix} #1 \\ #2 \\ #3 \\ #4 \end{bmatrix}}
\newcommand{\vn}[2]{\vvvv{#1_1}{#1_2}{\vdots}{#1_#2}}

\newenvironment{amatrix}[1]{% "augmented matrix"
  \left[\begin{array}{*{#1}{c}|c}
}{%
  \end{array}\right]
}

% \newcounter{notatnr}
% \newcommand{\notatnr}[2]
% {\setcounter{notatnr}{#1}%
%  \setcounter{page}{#2}%
% }

\newtheorem{thm}{Teorem}[chapter]
\newtheorem*{thm-nn}{Teorem}
\newtheorem{cor}[thm]{Korollar}
\newtheorem{lem}[thm]{Lemma}
\newtheorem{prop}[thm]{Proposisjon}
\theoremstyle{definition}
\newtheorem{exx}[thm]{Eksempel}
\newtheorem*{defnx}{Definisjon}
\newtheorem*{oppg}{Oppgave}
\newtheorem*{merkx}{Merk}
\newtheorem*{spmx}{Spørsmål}

\newenvironment{defn}
  {\pushQED{\qed}\renewcommand{\qedsymbol}{$\triangle$}\defnx}
  {\popQED\enddefnx}
\newenvironment{ex}
  {\pushQED{\qed}\renewcommand{\qedsymbol}{$\triangle$}\exx}
  {\popQED\endexx}
\newenvironment{merk}
  {\pushQED{\qed}\renewcommand{\qedsymbol}{$\triangle$}\merkx}
  {\popQED\endmerkx}
\newenvironment{spm}
  {\pushQED{\qed}\renewcommand{\qedsymbol}{$\triangle$}\spmx}
  {\popQED\endspmx}

\setlength{\columnsep}{26pt}

\newcommand{\Tittel}[2]{%
\twocolumn[
\begin{center}
\Large
\begin{tabularx}{\textwidth}{cXr}
\cellcolor{black}\color{white}%
\bf {#1} &
#2
\hfill &
\footnotesize TMA4110 høsten 2018
\\ \hline
\end{tabularx}
\end{center}
]}

\newcommand{\tittel}[1]{\Tittel{\arabic{notatnr}}{#1}}

\newcommand{\linje}{%
\begin{center}
\rule{.8\linewidth}{0.4pt}
\end{center}
}


\newcommand{\chapternumber}{}

\makechapterstyle{tma4110}{%
 \renewcommand*{\chapterheadstart}{}
 \renewcommand*{\printchaptername}{}
 \renewcommand*{\chapternamenum}{}
 \renewcommand*{\printchapternum}{\renewcommand{\chapternumber}{\thechapter}}
 \renewcommand*{\afterchapternum}{}
 \renewcommand*{\printchapternonum}{\renewcommand{\chapternumber}{}}
 \renewcommand*{\printchaptertitle}[1]{
\LARGE
\begin{tabularx}{\textwidth}{cXr}
\cellcolor{black}\color{white}%
\textbf{\chapternumber} &
\textbf{##1}
\hfill &
%\footnotesize TMA4110 høsten 2018
\\ \hline
\end{tabularx}%
}
 \renewcommand*{\afterchaptertitle}{\par\nobreak\vskip \afterchapskip}
 % \newcommand{\chapnamefont}{\normalfont\huge\bfseries}
 % \newcommand{\chapnumfont}{\normalfont\huge\bfseries}
 % \newcommand{\chaptitlefont}{\normalfont\Huge\bfseries}
 \setlength{\beforechapskip}{0pt}
 \setlength{\midchapskip}{0pt}
 \setlength{\afterchapskip}{10pt}
}


\newcounter{oppgnr}[chapter]
\newcounter{punktnr}[oppgnr]
\newenvironment{oppgave}
 {\par\noindent\stepcounter{oppgnr}\textbf{{\arabic{oppgnr}}.}}
 {\par\bigskip}
\newenvironment{punkt}
 {\par\smallskip\noindent\stepcounter{punktnr}\textbf{\alph{punktnr})} }
 {\par}

\newcommand{\oppgaver}{\linje\section*{Oppgaver}}

\usepackage{xr}
\externaldocument{tma4110-2018h}
\newcommand{\kapittel}[2]{\setcounter{chapter}{#1}\addtocounter{chapter}{-1}\chapter{#2}}
\newcommand{\kapittelslutt}{\enddocument}
\begin{document}
\chapterstyle{tma4110}
\pagestyle{plain}
\fi


\kapittel{3}{Vektor- og matriselikninger}
\label{ch:vektor-og-matriselikninger}






\noindent%

\noindent
I denne uken skal vi bruke enkel vektorregning til å analysere lineære ligningssystemer. Vi skal ha et spesielt fokus på $\mathbb{R}^3$, for det går an å visualisere; klarer man det, går det lettere å abstrahere til $\mathbb{R}^n$. Senere i kurset skal vi se hvordan noen av konseptene under kan generaliseres, slik at vi kan konstruere teori som kan behandle matematiske emner som tilsynelatende ser veldig forskjellige ut, men følger akkurat de samme lovene. 

\section*{Vektorregning}
Inntil videre skal vi skrive vektorer på høykant
\begin{align*}
\mathbf{x}=
\begin{bmatrix}
x_1  \\
x_2 \\
\vdots \\
x_n
\end{bmatrix},
\end{align*}
kalt søylevektor. Du kan også tenke på dette som et punkt i $\mathbb{R}^n$. De to viktigste regnereglene for vektorer er skalarmultiplikasjon
\begin{align*}
a\mathbf{x}=
\begin{bmatrix}
ax_1  \\
ax_2 \\
\vdots \\
ax_n
\end{bmatrix}
\end{align*}
og vektoraddisjon
\begin{align*}
\mathbf{x}+\mathbf{y}=
\begin{bmatrix}
x_1  \\
x_2 \\
\vdots \\
x_n
\end{bmatrix}
+
\begin{bmatrix}
y_1  \\
y_2 \\
\vdots \\
y_n
\end{bmatrix}
=
\begin{bmatrix}
x_1 + y_1  \\
x_2 + y_2\\
\vdots \\
x_n + y_n
\end{bmatrix}.
\end{align*}
En sammensetning av disse to operasjonene
\begin{align*}
a\mathbf{x}+b\mathbf{y}=
a
\begin{bmatrix}
x_1  \\
x_2 \\
\vdots \\
x_n
\end{bmatrix}
+
b
\begin{bmatrix}
y_1  \\
y_2 \\
\vdots \\
y_n
\end{bmatrix}
=
\begin{bmatrix}
ax_1 + by_1  \\
ax_2 + by_2\\
\vdots \\
ax_n + by_n
\end{bmatrix}
\end{align*}
kalles en \emph{lineærkombinasjon}. Skalarene $a$ og $b$ kalles vekter. Hvis vi har $m$ vektorer $\mathbf{x}_k$, definerer vi \emph{det lineære spennet}, eller
\begin{equation*}
\text{Sp}\{\mathbf{x}_1,\mathbf{x}_2,...,\mathbf{x}_m\}
\end{equation*}
som alle lineærkombinasjoner av vektorene, altså alle vektorer på formen
\begin{equation*}
a_1\mathbf{x}_1+a_2\mathbf{x}_2+...+a_m\mathbf{x}_m.
\end{equation*}
\begin{ex}
	\begin{equation*}
	3\begin{bmatrix}1 \\  2 \\ 3 \end{bmatrix}+ 2\begin{bmatrix}4 \\  5 \\ 6 \end{bmatrix}=\begin{bmatrix}11 \\  16 \\ 21 \end{bmatrix}.
	\end{equation*}
\end{ex}
%\begin{center}
%\begin{tikzpicture}[scale=.9]
%\draw[->] (-1.5,0) -- (6.5,0);
%\draw[->] (0,-1.5) -- (0,5.5);
%\draw[->] (0,0) -- (1,2);
%\draw[->] (0,0) -- (2,1);
%\draw[->] (0,0) -- (3,3);
%\draw[->] (0,0) -- (5,4);
%\foreach \x in {-1,1,2,3,4,5,6}
%\draw (\x cm,1pt) -- (\x cm,-1pt) node[anchor=north] {$\x$};
%\foreach \y in {-1,1,2,3,4,5}
%\draw (1pt,\y cm) -- (-1pt,\y cm) node[anchor=east] {$\y$};
%\node[anchor=west] at (.2,3.5) {$\begin{bmatrix} 1 \\ 3\end{bmatrix}$};
%\node[anchor=west] at (1,1.7) {$\begin{bmatrix} 2 \\ 4\end{bmatrix}$};
%\node[anchor=west] at (3,1) {$3\begin{bmatrix} 1 \\ 3\end{bmatrix}-2\begin{bmatrix} 2 \\ 4\end{bmatrix}=\begin{bmatrix} -1 \\ -2\end{bmatrix}$};
%\end{tikzpicture}
%\end{center}
\begin{ex}
	\noindent Spennet til vektorene i eksemplet over, er alle vektorer på formen
	\begin{equation*}
	a\begin{bmatrix}1 \\  2 \\ 3 \end{bmatrix}+ b\begin{bmatrix}4 \\  5 \\ 6 \end{bmatrix}.
	\end{equation*}
\end{ex}


\section*{Vektorligninger}
Ved å ta i bruk lineærkombinasjon, kan vi skrive ligningssystemet fra forrige uke
\[
\systeme{
	x + 2y - 2z = -5,
	x + 5y + 9z = 33,
	2x + 5y -  z = 0
}
\]
som vektorligningen
\begin{equation*}
x
\begin{bmatrix}
1     \\
1   \\
2   
\end{bmatrix}
+
y
\begin{bmatrix}
2   \\
5   \\
5    
\end{bmatrix}
+
z
\begin{bmatrix}
-2   \\
9 \\
-1 
\end{bmatrix}
=
\begin{bmatrix}
-5   \\
33 \\
0 
\end{bmatrix}.
\end{equation*}
Dette gir oss en ny måte å se ligningssystemer på: oppgaven er å finne vektene $x$, $y$  og $z$ slik at s{\o}ylene i matrisen line{\ae}rkombineres til {\aa} bli lik h{\o}yresiden. 
\begin{ex}
	Løsningen til systemet over er
	\begin{equation*}
	\begin{bmatrix}
	x  \\
	y \\
	z
	\end{bmatrix}
	=
	\begin{bmatrix}
	7  \\
	-2 \\
	4
	\end{bmatrix}
	\end{equation*}
	og du kan verifisere at 
	\begin{equation*}
	7
	\begin{bmatrix}
	1     \\
	1   \\
	2   
	\end{bmatrix}
	-2
	\begin{bmatrix}
	2   \\
	5   \\
	5    
	\end{bmatrix}
	+
	4
	\begin{bmatrix}
	-2   \\
	9 \\
	-1 
	\end{bmatrix}
	=
	\begin{bmatrix}
	-5   \\
	33 \\
	0 
	\end{bmatrix}.
	\end{equation*}
\end{ex}

\section*{Matriseligninger}
Produktet av en $n\times n$-matrise og en søylevektor i $\mathbb{R}^n$ defineres som følgende lineærkombinasjon av matrisens søyler
\begin{equation*}
\begin{bmatrix}
a_{11}    &  \cdots & a_{1n}   \\
\vdots  & \ddots & \vdots\\
a_{n1}  & \cdots &  a_{nn}  
\end{bmatrix}
\begin{bmatrix}
x_1   \\
x_2 \\
\vdots \\
x_n 
\end{bmatrix}=
x_1
\begin{bmatrix}
a_{11}     \\
a_{21}   \\
\vdots \\
a_{n1}   
\end{bmatrix}
+
x_2
\begin{bmatrix}
a_{12}    \\
a_{22}    \\
\vdots \\
a_{n2}    
\end{bmatrix}
+
\cdots
+
x_n
\begin{bmatrix}
a_{1n}   \\
a_{2n}  \\
\vdots \\
a_{3n}  
\end{bmatrix}.
\end{equation*}
\begin{ex}
	\begin{equation*}
	\begin{bmatrix}
	1  &  2  &  -2   \\
	1  & 5  &  9  \\
	2  & 5  &  -1  
	\end{bmatrix}
	\begin{bmatrix}
	7   \\
	-2 \\
	4 
	\end{bmatrix}=
	7
	\begin{bmatrix}
	1     \\
	1   \\
	2   
	\end{bmatrix}
	-
	2
	\begin{bmatrix}
	2   \\
	5   \\
	5    
	\end{bmatrix}
	+
	4
	\begin{bmatrix}
	-2   \\
	9 \\
	-1 
	\end{bmatrix}.
	\end{equation*}
\end{ex}
\noindent Nå kan vi skrive ligningssystemet fra forrige uke som
\begin{equation*}
\begin{bmatrix}
1  &  2  &  -2   \\
1  & 5  &  9  \\
2  & 5  &  -1  
\end{bmatrix}
\begin{bmatrix}
x_1   \\
x_2 \\
x_3 
\end{bmatrix}=
\begin{bmatrix}
-5   \\
33 \\
0 
\end{bmatrix}.
\end{equation*}
Dersom vi skriver 
\begin{equation*}
A=
\begin{bmatrix}
1  &  2  &  -2   \\
1  & 5  &  9  \\
2  & 5  &  -1  
\end{bmatrix},
\end{equation*}
\begin{equation*}
\mathbf{x}=
\begin{bmatrix}
x_1  \\
x_2 \\
x_3
\end{bmatrix}
\end{equation*}
og
\begin{equation*}
\mathbf{b}=
\begin{bmatrix}
-5   \\
33 \\
0 
\end{bmatrix},
\end{equation*} 
kan vi innføre den kompakte notasjonen
\begin{equation*}
A\mathbf{x}=\mathbf{b}.
\end{equation*}

\section*{Eksistens og entydighet av løsninger II}
Ligningssystemer deler seg naturlig i tre kategorier; de som har en unik løsning, de som har ingen løsning, og de som har uendelig mange løsninger. Vi skal nå gi en geometrisk illustrasjon av hva som skjer i de forskjellige  tilfellene. 

\begin{ex}
	Hvis vi utf{\o}rer Gauss-eliminasjon p{\aa} systemet
	\begin{equation*}
	\begin{bmatrix}2 & 3& 4 \\  3& 4 & 5  \\ 4 &5 & 6 \end{bmatrix}\begin{bmatrix}x_{1} \\  x_{2}\\ x_{3} \end{bmatrix}=
	\begin{bmatrix}4 \\ 5\\ 3 \end{bmatrix},
	\end{equation*}
	f{\aa}r vi 
	\begin{equation*}
	\begin{bmatrix}2 & 3& 4 \\  0& 1 & 2  \\ 0 & 1 & 2 \end{bmatrix}\begin{bmatrix}x_{1} \\  x_{2}\\ x_{3} \end{bmatrix}=
	\begin{bmatrix}4 \\ 2\\ 5 \end{bmatrix}.
	\end{equation*}
	De to nederste linjene sier at $x_{2}+2x_{3}$ skal v{\ae}re b{\aa}de 2 og 5. Dette er åpenbart umulig, og systemet har ingen løsning. Grunnen er at søylene ligger i samme i plan i $\mathbb{R}^{3}$, og siden høyresiden ikke ligger i dette planet, er det umulig {\aa} skrive den som en line{\ae}rkombinasjon av disse vektorene.
\end{ex}

\begin{ex}
	Hvis vi derimot utf{\o}rer Gauss-eliminasjon p{\aa} systemet
	\begin{equation*}
	\begin{bmatrix}2 & 3& 4 \\  3& 4 & 5  \\ 4 &5 & 6 \end{bmatrix}\begin{bmatrix}x_{1} \\  x_{2}\\ x_{3} \end{bmatrix}=
	\begin{bmatrix}9 \\ 12\\ 15 \end{bmatrix},
	\end{equation*}
	f{\aa}r vi 
	\begin{equation*}
	\begin{bmatrix}2 & 3& 4 \\  0& 1 & 2  \\ 0 & 1 & 2 \end{bmatrix}\begin{bmatrix}x_{1} \\  x_{2}\\ x_{3} \end{bmatrix}=
	\begin{bmatrix}9 \\ 3\\ 3 \end{bmatrix},
	\end{equation*}
	Nå er to nederste linjene identiske. Søylene i matrisen ligger som kjent i samme plan i $\mathbb{R}^{3}$, men nå ligger tilfeldigvis høyresiden også i dette planet, og systemet kan derfor løses. Hvis du ønsker å skrive en vektor i et plan som en lineærkombinasjon av tre andre vektorer i samme plan, har du uendelig mange måter å gjøre det på, og derfor har ligningssystemet uendelig mange løsninger.
\end{ex}

\begin{ex}
	Ligningssystemet 
	\begin{equation*}
	\begin{bmatrix}
	1  &  2  &  -2   \\
	1  & 5  &  9  \\
	2  & 5  &  -1  
	\end{bmatrix}
	\begin{bmatrix}
	x_1   \\
	x_2 \\
	x_3 
	\end{bmatrix}=
	\begin{bmatrix}
	-5   \\
	33 \\
	0
	\end{bmatrix}.
	\end{equation*}
	har som kjent en unik løsning. Vi sier at søylene \emph{spenner ut} $\mathbb{R}^3$, siden alle punkter i $\mathbb{R}^3$ kan skrives som en unik lineærkombinasjon av dem. Merk at søylene i matrisen danner et parallellepiped med volum 2.
\end{ex}

\noindent Du kan avgjøre hvorvidt tre vektorer i $\mathbb{R}^3$ ligger i samme plan ved å beregne volumet til parallellepipedet spent ut av de tre vektorene. Dersom volumet blir 0, ligger de i samme plan. Dersom søylene i matrisen $A$ kalles $\mathbf{a}_1$, $\mathbf{a}_2$ og $\mathbf{a}_3$, kalles dette volumet \emph{determinanten} til $A$, og er gitt ved
\begin{equation*}
\det A= \mathbf{a}_1\cdot \mathbf{a}_2 \times \mathbf{a}_3.
\end{equation*}
\begin{ex}	 
	\begin{equation*}
	\det
	\begin{bmatrix}
	2  &  3  &  4   \\
	3  & 4  &  5 \\
	4  & 5  &  6 
	\end{bmatrix}
	=0.
	\end{equation*}
\end{ex}


\noindent Presise kriterier for når et ligningssystem har én, ingen, eller mange løsninger, får vi ikke uten litt mer matematisk maskineri. Men et mentalt bilde av $3\times 3$-systemer kan vi lage oss. 
\begin{itemize}
	\item Hvis matrisens søyler danner et parallellepiped har systemet en unik løsning uansett høyreside.
	\item Hvis matrisens søyler ligger i samme plan, og høyresiden ikke ligger i dette planet, har systemet ingen løsning.
	\item Hvis matrisens søyler ligger i samme plan, og høyresiden ligger i dette planet, har systemet uendelig mange løsninger.
\end{itemize}

\section*{En forsmak på lineær uavhengighet}
Hvis man har en samling vektorer, sier vi at de er lineært avhengige dersom en av vektorene i samlingen kan skrives som en lineærkombinasjon av de andre, for eksempel dersom tre vektorer i $\mathbb{R}^3$ ligger i samme plan.
\begin{ex}	
	Hvis du utfører gausseliminasjon på systemet
	\begin{equation*}
	\begin{bmatrix}
	2  &  3  &  4 & 0   \\
	3  & 4  &  5  & 0\\
	4  & 5  &  6 & 0 
	\end{bmatrix}
	\end{equation*}
	får du
	\begin{equation*}
	\begin{bmatrix}
	2  &  3  &  4 & 0   \\
	0  & 1  &  2  & 0\\
	0  & 0 &  0 & 0 
	\end{bmatrix}
	\end{equation*}
	altså at 
	\[
	\systeme{
		2x + 3y + 4z = 0,
		y + 2z = 0}.
	\]
	setter vi $z=s$, får vi $y=-2s$ av den siste ligningen, og $x=s$ av den første, slik at 
	\begin{equation*}
	\begin{bmatrix}
	x_1  \\
	x_2 \\
	x_3
	\end{bmatrix}
	=
	s
	\begin{bmatrix}
	1  \\
	-2 \\
	1
	\end{bmatrix}
	\end{equation*} 
	er en løsning av systemet for vilkårlige $s$. Dette betyr at søylene i den opprinnelige matrisen er lineært avhengige. Vi dobbeltsjekker:
	\begin{equation*}
	\begin{bmatrix}
	2  \\
	3 \\
	4
	\end{bmatrix}
	-2
	\begin{bmatrix}
	3  \\
	4 \\
	5
	\end{bmatrix}
	+
	\begin{bmatrix}
	4  \\
	5 \\
	6
	\end{bmatrix}
	=
	\begin{bmatrix}
	0  \\
	0 \\
	0
	\end{bmatrix}
	\end{equation*}
\end{ex}




\kapittelslutt

% -*- TeX-master: "oving02"; -*-
\oppgaver{3}


\begin{oppgave}
La $\V{u} = \vv{3}{2}$ og~$\V{v} = \vv{-1}{1}$ være to vektorer
i~$\R^2$.

\begin{punkt}
Regn ut $\V{u} + \V{v}$ og $\frac{1}{2} \V{u} - 2 \V{v}$.
\end{punkt}

\begin{punkt}
Tegn en figur som viser vektorene $\V{u}$, $\V{v}$, $\V{u} + \V{v}$ og
$\frac{1}{2} \V{u} - 2 \V{v}$ i planet.
\end{punkt}

\end{oppgave}

\begin{losning}
\begin{punkt}
	$\V{u} + \V{v}=
	\begin{bmatrix}
	2\\
	3
	\end{bmatrix}
	$ og $\frac{1}{2} \V{u} - 2 \V{v}=
	\begin{bmatrix}
	\frac{7}{2}\\
	-1
	\end{bmatrix}
	$.
\end{punkt}

\begin{punkt}
\end{punkt}
\begin{center}
	\begin{tikzpicture}
	\draw[->] (-1,0) -- (4,0) node[right] {$x$};
	\draw[->] (0,-1) -- (0,4) node[above] {$y$};
	\draw[->] (0,0) -- (3,2) node[above] {$\V{u}$};
	\draw[->] (0,0) -- (-1,1) node[above] {$\V{v}$};
	\draw[->] (0,0) -- (7/2,-1) node[below] {$\frac{1}{2}\V{u}-2\V{v}$};
	\draw[->] (0,0) -- (2,3) node[above] {$\V{u}+\V{v}$};
	\end{tikzpicture}
\end{center}

\end{losning}


\begin{oppgave}
Skriv alle ligningssystemene fra oppgave~\textbf{2.2} i øving~1 som

\begin{punkt}
\ldots vektorlikninger.
\end{punkt}

\begin{punkt}
\ldots matriselikninger.
\end{punkt}

\end{oppgave}

\begin{losning}
Se diskusjonen om matrise- og vektorligninger i kapittel \textbf{3}.
\end{losning}



\begin{oppgave}
Løs ligningen $A\V{x}=\V{b}$ for 
% \begin{align*}
% \text{\textbf{a)}\hspace{20pt}}
% A &=
% \begin{bmatrix}
% \;0 & 1 & 1 & 0 & 0 & 0 & 1\;\\
% \;0 & 0 & 0 & 0 & 1 & 0 & 1\;\\
% \;0 & 0 & 0 & 0 & 0 & 1 & 1\;\\
% \;0 & 0 & 0 & 0 & 0 & 0 & 0\;
% \end{bmatrix},
% &
% \V{b} &=
% \begin{bmatrix}
% 0  \\
% 0 \\
% 0 \\
% 0
% \end{bmatrix}
% \\[10pt]
% \text{\textbf{b)}\hspace{20pt}}
% A &=
% \begin{bmatrix}
% 1 & 2 & 3\\
% 2 & 3 & 4\\
% 3 & 4 & 5\\
% 4 & 5 & 6
% \end{bmatrix},
% &
% \V{b} &=
% \begin{bmatrix}
% -3  \\
% -7 \\
% -3 \\
% 0
% \end{bmatrix}
% \\[10pt]
% \text{\textbf{c)}\hspace{20pt}}
%  A &=
% 	\begin{bmatrix}
% 	8  & -7 & 0 \\
% 	-8 & -7 & 3 \\
% 	-4 & 5  & -8\\
% 	-6 & 6  & -4
% 	\end{bmatrix},
% &
%         \V{b} &=
% 	\begin{bmatrix}
% 	-3  \\
% 	-7 \\
% 	-3 \\
% 	0
% 	\end{bmatrix}
% \end{align*}
\begin{punkt}
\vspace{-10pt}
$$A=
\begin{bmatrix}
\;0 & 1 & 1 & 0 & 0 & 0 & 1\;\\
\;0 & 0 & 0 & 0 & 1 & 0 & 1\;\\
\;0 & 0 & 0 & 0 & 0 & 1 & 1\;\\
\;0 & 0 & 0 & 0 & 0 & 0 & 0\;
\end{bmatrix},
\qquad
\V{b}=
\begin{bmatrix}
0  \\
0 \\
0 \\
0
\end{bmatrix}
$$
\end{punkt}

\begin{punkt}
\vspace{-10pt}
$$A=
\begin{bmatrix}
1 & 2 & 3\\
2 & 3 & 4\\
3 & 4 & 5\\
4 & 5 & 6
\end{bmatrix},
\qquad
\V{b}=
\begin{bmatrix}
-3  \\
-7 \\
-3 \\
0
\end{bmatrix}
$$
\end{punkt}


\begin{punkt}
\vspace{-10pt}
	$$A=
	\begin{bmatrix}
	8  & -7 & 0 \\
	-8 & -7 & 3 \\
	-4 & 5  & -8\\
	-6 & 6  & -4
	\end{bmatrix},
\qquad
        \V{b}=
	\begin{bmatrix}
	-3  \\
	-7 \\
	-3 \\
	0
	\end{bmatrix}
	$$
	
\end{punkt}

\end{oppgave}

\begin{losning}

\begin{punkt}
Det er fire frie variabler. Husk at du kan sjekke om det endelige svaret ditt er riktig ved å multiplisere med $A$ og se om resultatet faktisk blir null-vektoren.
\end{punkt}

\begin{punkt}
Det finnes ingen løsning.
\end{punkt}

\begin{punkt}
$x=\frac{83}{215}$, $y=\frac{187}{215}$ og $z=\frac{156}{215}$.
\end{punkt}

\end{losning}



\begin{oppgave}
%(TODO: noen gitte vektorer, finn ut om en av dem er lineærkombinasjon av de andre)
Finn ut om en vektor er en lineærkombinasjon av de andre:
\begin{punkt}
$$
\begin{bmatrix}
1\\
2
\end{bmatrix},
\begin{bmatrix}
2\\
3
\end{bmatrix},
\begin{bmatrix}
3\\
4
\end{bmatrix}
$$
\end{punkt}
\begin{punkt}
	$$
	\begin{bmatrix}
	1\\
	2\\
	3\\
	4
	\end{bmatrix},
	\begin{bmatrix}
	2\\
	3\\
	4\\
	5
	\end{bmatrix},
	\begin{bmatrix}
	3\\
	4\\
	5\\
	6
	\end{bmatrix}
	$$
\end{punkt}

\end{oppgave}


\begin{losning}
	
\begin{punkt}
Et vilkårlig valg av en vektor vil kunne skrives som en lineærkombinasjon av de to andre. Det er derfor tre mulige fremgangsmåter som alle er riktig.
Hint: La $\V{v}_1$ være en av vektorene. Du ønsker å sjekke om $\V{v}_1$ er en lineærkombinasjon av de to resterende vektorene. Dette kan formuleres med likningen $$\V{v}_1=a\V{v}_2+b\V{v}_3$$ hvor $\V{v}_2$ og~$\V{v}_3$ er de to resterende vektorene, og $a$ og~$b$ er ukjente koeffisienter. Spørsmålet er altså om denne likningen har en løsning; som kan sjekkes ved regning. Prøv å skissere løsningen din i $x$--$y$-planet.
\end{punkt}

\begin{punkt}
Et vilkårlig valg av en vektor vil kunne skrives som en lineærkombinasjon av de to andre. Fremgangsmåte som i $\textbf{a)}$, men merk at du ikke kan skissere løsningen din ettersom dette krever fire dimensjoner. 
\end{punkt}


\end{losning}



\begin{oppgave}
%(TODO: gitt vektorer u, v, w (to eller tre), finn vektor som ikke er lineærkombinasjon av disse)
Finn en vektor som ikke er en lineærkombinasjon av:
\begin{punkt}
	$$
	\begin{bmatrix}
	1\\
	5\\
	-3
	\end{bmatrix},
	\begin{bmatrix}
	4\\
	18\\
	4
	\end{bmatrix}
	$$
\end{punkt}

\begin{punkt}
$$
\begin{bmatrix}
1\\
2\\
3\\
4
\end{bmatrix},
\begin{bmatrix}
2\\
3\\
4\\
5
\end{bmatrix},
\begin{bmatrix}
3\\
4\\
5\\
6
\end{bmatrix}
$$
\end{punkt}

\begin{punkt}
	$$
	\begin{bmatrix}
	8  \\
	-8 \\
	-4 \\
	-6
	\end{bmatrix},
	\begin{bmatrix}
	-7  \\
	-7 \\
	5 \\
	6
	\end{bmatrix},
	\begin{bmatrix}
	0  \\
	3 \\
	-8 \\
	-4
	\end{bmatrix}
	$$
\end{punkt}

\end{oppgave}

\begin{losning}
	
\begin{punkt}
Hint: Vi ønsker å finne en vektor $\begin{bmatrix}
a\\
b\\
c
\end{bmatrix}$ som ikke er en lineærkombinasjon av 
$
\begin{bmatrix}
	1\\
	5\\
	-3
\end{bmatrix}$ og~$
\begin{bmatrix}
	4\\
	18\\
	4
\end{bmatrix}$. Vektoren skal altså \emph{ikke} tilfredstille likningen 
$$x
\begin{bmatrix}
1\\
5\\
-3
\end{bmatrix}+y
\begin{bmatrix}
4\\
18\\
4
\end{bmatrix}=
\begin{bmatrix}
a\\
b\\
c
\end{bmatrix},$$
som har totalmatrise
$$\begin{bmatrix}
1  & 4  & a\\
5  & 18 & b\\
-3 & 4  & c
\end{bmatrix}.$$ Radreduser og velg $a$, $b$ og~$c$ slik at systemet ikke har løsning. Eksempelvis fungerer $a=0$, $b=1$ og~$c=0$.
\end{punkt}

\begin{punkt}
Hint: Som i $\textbf{a)}$, men nå er totalmatrisen
$$\begin{bmatrix}
1 & 2 & 3 & a\\
2 & 3 & 4 & b\\
3 & 4 & 5 & c\\
4 & 5 & 6 & d
\end{bmatrix}.$$
\end{punkt}


\begin{punkt}
Hint: Som i $\textbf{a)}$, men nå er totalmatrisen
$$\begin{bmatrix}
8  & -7 & 0  & a\\
-8 & -7 & 3  & b\\
-4 & 5  & -8 & c\\
-6 & 6  & -4 & d
\end{bmatrix}.$$
\end{punkt}
\end{losning}


\begin{oppgave}
	Er 
	$
	\begin{bmatrix}
	-3  \\
	-7 \\
	-3 \\
	0
	\end{bmatrix}
	$
	en lineærkombinasjon av vektorene i 
	
\begin{punkt}
\ldots oppgave \textbf{5. a)}?
\end{punkt} 

\begin{punkt}
	\ldots oppgave \textbf{5. b)}?
\end{punkt} 

\begin{punkt}
	\ldots oppgave \textbf{5. c)}?
\end{punkt} 

\end{oppgave}


\begin{losning}

\begin{punkt}
Spørsmålet gir ikke mening siden vektorene i \textbf{5.}~\textbf{a)} er vektorer i $\mathbb{R}^3$.
\end{punkt}

\begin{punkt}
Hint: Ligningen i \textbf{3. b)} har ingen løsning. 
\end{punkt}


\begin{punkt}
Hint: Ligningen i \textbf{3. c)} har én løsning.
\end{punkt}
	
\end{losning}



\begin{oppgave}
	Finn en tredje vektor i samme plan som disse to vektorene:
\[
	\begin{bmatrix}
	-3  \\
	-7 \\
	-3 \\
	\end{bmatrix}
	\quad\text{og}\quad
	\begin{bmatrix}
	8  \\
	-8 \\
	-4 \\
	\end{bmatrix}
\]
\end{oppgave}

\begin{losning}
Planet som inneholder 
$
\begin{bmatrix}
-3  \\
-7 \\
-3 \\
\end{bmatrix}$ og $\begin{bmatrix}
8  \\
-8 \\
-4 \\
\end{bmatrix}
$ er akkurat det lineære spennet deres. Altså, alle vektorer på formen 
\[
a\begin{bmatrix}
-3  \\
-7 \\
-3 \\
\end{bmatrix} + b\begin{bmatrix}
8  \\
-8 \\
-4 \\
\end{bmatrix}.
\] Alle valg av $a$ og $b$ er riktig.
\end{losning}





\begin{oppgave}
La $\V{v}$ og~$\V{w}$ være disse vektorene i~$\R^3$:
\[
	\V{v}=
	\begin{bmatrix}
	-3  \\
	-7 \\
	-3 \\
	\end{bmatrix}
	\quad\text{og}\quad
	\V{w}=
	\begin{bmatrix}
	8  \\
	-8 \\
	-4 \\
	\end{bmatrix}
\]
	Finn en vektor 
\[
	\V{u}=\begin{bmatrix}
	u_1    \\
	u_2 \\
	u_3  
	\end{bmatrix}
\]
slik at $\V{u}$, $\V{v}$ og~$\V{w}$
	spenner ut $\R^3$, og løs likningen $x\V{u}+y\V{v}+z\V{w}=0$.
\end{oppgave}

\begin{losning}
Tre vektorer i $\mathbb{R}^3$ spenner ut et parallellepiped. Disse ligger i et plan hvis og bare hvis volumet til dette parallellepipedet - determinanten - er lik null. Derfor må vi finne en tredje vektor $\V{u}=\begin{bmatrix}
a\\
b\\
c
\end{bmatrix}$ slik at determinanten til $\V{u}$, $\V{v}$ og $\V{w}$ ikke er lik null. Matematisk formulert: $$\begin{bmatrix}
a\\
b\\
c
\end{bmatrix} \times \begin{bmatrix}
-3  \\
-7 \\
-3 \\
\end{bmatrix}\cdot \begin{bmatrix}
8  \\
-8 \\
-4 \\
\end{bmatrix}\neq 0.$$ Det finnes mange valg av $a$, $b$ og $c$ som fungerer. Eksempelvis fungerer $\begin{bmatrix}
1\\
-9\\
20
\end{bmatrix}$. Likningen $x\V{u}+y\V{v}+z\V{w}=0$ har kun triviell løsning; $x=0$, $y=0$ og $z=0$.

\end{losning}



\begin{oppgave}
% 1  4  1
% 0 -2  3
% 0  0 17
% ---
% 1  4  1
% 0 -2  3
% 0  8  5
% ---
% 1  4  1
% 0 -2  3
%-3 -4  2
% ---
% 1  4  1
% 5 18  8
%-3  4  2
La $p$ og~$q$ være følgende polynomer:
\begin{align*}
p(x) &= x^2 + 5x - 3 \\
q(x) &= 4x^2 + 18x + 4
\end{align*}

\begin{punkt}
La $s$ være polynomet $s(x) = x^2 + 8x + 2$.  Finnes det konstanter
$a$ og~$b$ slik at
\[
s(x) = a \cdot p(x) + b \cdot q(x)
\]
for alle~$x$?
\end{punkt}

\begin{punkt}
Finn et andregradspolynom $t$ som oppfyller følgende: For hvert
andregradspolynom $r$ skal det være mulig å finne konstanter $a$, $b$
og~$c$ slik at
\[
r(x) = a \cdot p(x) + b \cdot q(x) + c \cdot t(x)
\]
\end{punkt}

\end{oppgave}

\begin{losning}
Introduksjon til løsning: Vi ønsker å beskrive problemet med lineær algebra. Et polynom er entydig bestemt av koeffisientene sine. Derfor kan all informasjon om et andregradspolynom $p(x)=ax^2+bx+c$ lagres i vektoren $\V{p}=\begin{bmatrix}
a\\
b\\
c
\end{bmatrix}.$ Summen av to andregradspolynom $$ax^2+bx+c$$ og $$dx^2+ex+f$$ kan skrives $$(a+d)x^2+(b+e)x+(c+f).$$ Vi summerer altså koeffisientene foran tilhørende potens av $x$. Dette svarer akkurat til addisjon av tilhørende vektorer: $$\begin{bmatrix}
a\\
b\\
c
\end{bmatrix}+\begin{bmatrix}
d\\
e\\
f
\end{bmatrix}=\begin{bmatrix}
a+d\\
b+e\\
c+f
\end{bmatrix}.$$ Tilsvarende blir en konstant multiplisert med et andregradspolynom multiplisert i hver koeffisient: $$k\cdot(ax^2+bx+c)=(k\cdot a)x^2+(k\cdot b)x+(k\cdot c),$$ som svarer til skalarmultiplikasjon av tilhørende vektor: $$k\cdot \begin{bmatrix}
a\\
b\\
c
\end{bmatrix}=\begin{bmatrix}
k\cdot a\\
k\cdot b\\
k\cdot c
\end{bmatrix}.$$

\begin{punkt}
I dette lineær algebra-språket blir spørsmålet om $\V{s}=\begin{bmatrix}
1\\
8\\
2
\end{bmatrix}$ er en lineærkombinasjon av $\V{p}=\begin{bmatrix}
1\\
5\\
-3
\end{bmatrix}$ og $\V{q}=\begin{bmatrix}
4\\
18\\
4
\end{bmatrix}.$ Spørsmålet er altså om likningen $$x\begin{bmatrix}
1\\
5\\
-3
\end{bmatrix}+y\begin{bmatrix}
4\\
18\\
4
\end{bmatrix}=\begin{bmatrix}
1\\
8\\
2
\end{bmatrix},$$ 

som svarer til totalmatrisen $$\begin{bmatrix}
1  & 4  & 1\\
5  & 18 & 8\\
-3 & 4  & 2
\end{bmatrix},$$ har en løsning. Svaret er nei.
\end{punkt}

\begin{punkt}
Hint: I lineær algebra-språk skal du finne en vektor $\V{t}$ slik at alle vektorer $\V{r}$ kan skrives som en lineærkombinasjon av $\V{p}$, $\V{q}$ og~$\V{t}$. La $\V{t}$ være løsningen du fant i oppgave \textbf{3.5} del \textbf{a)}. Du kan nå sjekke at likningen $$x\V{p}+y\V{q}+z\V{t}=\V{r}$$ har en entydig løsning for alle valg av vektorer $\V{r}=\begin{bmatrix}
a\\
b\\
c
\end{bmatrix}.$ Det tilhørende polynomet $t$ - til $\V{t}$ - er altså en løsning på oppgaven.

Merk: Grunnen til at $\V{t}$ fungerer er at den er lineært uavhengig av $\V{p}$ og~$\V{q}$. Derfor får vi tre lineært uavhengige vektorer som til sammen spenner ut $\mathbb{R}^3.$
\end{punkt}

\end{losning}







\begin{oppgave}
	La $m<n$. Kan $m$ vektorer spenne ut $\mathbb{R}^n$? 
\end{oppgave}

\begin{losning}
Eksempel; $m=2$, $n=2$. En vektor i $\mathbb{R}^2$ spenner ut en linje, altså ikke hele $\mathbb{R}^2$. Dette generaliseres til $\mathbb{R}^n$, svaret er altså nei.

\noindent
Hint: Dersom $m$ vektorer, $\V{v}_1,\dots,\V{v}_m$, spenner ut $\mathbb{R}^n$ betyr dette at vi alltid kan løse likningen $x_1\V{v}_1+\dots+x_n\V{v}_m=\V{b}$ hvor $\V{b}$ er en vilkårlig vektor i $\mathbb{R}^n$. Dette svarer til $m$ likninger med $n$ ukjente hvor $m<n$. Kan et slikt system alltid ha løsning?

\noindent
Merk: Senere skal vi se at $m$ vektorer spenner ut et underrom av dimensjon $\leq m$, altså kan de ikke spenne ut hele $\mathbb{R}^n$ dersom $m<n$.
\end{losning}






\part{Komplekse tall}

\chapter{Komplekse tall}

TODO

\part{Lineære differensiallikninger}

\chapter{Andreordens differensiallikninger}

TODO


\addtocontents{toc}{\par\vspace{2em}}
\backmatter

\chapter{Løsninger på oppgaver}

\end{document}
