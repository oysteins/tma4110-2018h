\documentclass[titlepage,a4paper,12pt,norsk]{IMFeksamen}
%\geometry{left=3.5cm,right=3.5cm,bottom=2cm}
\usepackage[utf8]{inputenc}
\trykkinfo[tosidig,sorthvit]
\emnekode{TMA4110}
\emnenavn{Matematikk 3 -- EKSEMPEL~2}
\eksamensdato{Når du vil}
\eksamenstid{Fire timer}
\fagligkontaktinfo{Spør på mattelab hvis du lurer på noe}{Nei}
\hjelpemiddel{Ingen trykte eller håndskrevne hjelpemidler tillatt.
 Bestemt, enkel kalkulator tillatt.
 (Casio fx-82ES PLUS, Casio fx-82EX,
  Citizen SR-270X, Citizen SR-270X College,
  Hewlett Packard HP30S)}
\anneninfo{Eksamenen består av ti oppgaver
Hver av disse teller like mye.
Alle svar må begrunnes.

\textbf{Merk.}
Dette er ikke en virkelig eksamen, men et eksempel for å vise
hvordan en eksamen kan se ut.
Hvis du ikke har nok godkjente øvinger, kan du levere svar på disse
oppgavene som en ekstra øving.
}
\runninghead{TMA4110 -- Eksamen høsten 2018 -- EKSEMPEL~2}
\usepackage[T1]{fontenc}
\usepackage{lmodern,amsmath,amssymb,amsfonts}
\usepackage{mathrsfs}
\usepackage{systeme}


\newcommand{\N}{\mathbb{N}}
\newcommand{\Z}{\mathbb{Z}}
\newcommand{\Q}{\mathbb{Q}}
\newcommand{\R}{\mathbb{R}}
\newcommand{\C}{\mathbb{C}}

\newcommand{\M}{\mathcal{M}} % vektorrom av matriser
\newcommand{\Cf}{\mathcal{C}} % vektorrom av kontinuerlige funksjoner
\renewcommand{\P}{\mathcal{P}} % vektorrom av polynomer
\newcommand{\B}{\mathscr{B}} % basis

\renewcommand{\Im}{\operatorname{Im}}
\renewcommand{\Re}{\operatorname{Re}}

\newcommand{\abs}[1]{|#1|}
\newcommand{\intersect}{\cap}
\newcommand{\union}{\cup}
\newcommand{\fcomp}{\circ}
\newcommand{\iso}{\cong}

\newcommand{\roweq}{\sim}
\DeclareMathOperator{\Sp}{Sp}
\DeclareMathOperator{\Null}{Null}
\DeclareMathOperator{\Col}{Col}
\DeclareMathOperator{\Row}{Row}
\DeclareMathOperator{\rank}{rank}
\DeclareMathOperator{\im}{im}
\DeclareMathOperator{\id}{id}
\DeclareMathOperator{\Hom}{Hom}
\newcommand{\tr}{^\top}
\newcommand{\koord}[2]{[\,{#1}\,]_{#2}} % koordinater mhp basis

\newcommand{\V}[1]{\mathbf{#1}}
\newcommand{\vv}[2]{\begin{bmatrix} #1 \\ #2 \end{bmatrix}}
\newcommand{\vvS}[2]{\left[ \begin{smallmatrix} #1 \\ #2 \end{smallmatrix} \right]}
\newcommand{\vvv}[3]{\begin{bmatrix} #1 \\ #2 \\ #3 \end{bmatrix}}
\newcommand{\vvvv}[4]{\begin{bmatrix} #1 \\ #2 \\ #3 \\ #4 \end{bmatrix}}
\newcommand{\vvvvv}[5]{\begin{bmatrix} #1 \\ #2 \\ #3 \\ #4 \\ #5 \end{bmatrix}}
\newcommand{\vn}[2]{\vvvv{#1_1}{#1_2}{\vdots}{#1_#2}}

\newcommand{\e}{\V{e}}
\renewcommand{\u}{\V{u}}
\renewcommand{\v}{\V{v}}
\newcommand{\w}{\V{w}}
\renewcommand{\b}{\V{b}}
\newcommand{\x}{\V{x}}
\newcommand{\0}{\V{0}}

\newenvironment{amatrix}[1]{% "augmented matrix"
  \left[\begin{array}{*{#1}{c}|c}
}{%
  \end{array}\right]
}



\begin{document}


\begin{oppgave}
Likningssystemet 
\[
\systeme{
x + 2y = 1,
3x + 4y = 2,
4x + 6y = 3
}
\]
beskriver tre rette linjer i $\mathbb R^2$. Tegn dem, og avgjør om systemet har en løsning.
\end{oppgave}


\begin{oppgave}
Regn ut inversen til matrisen
\[
A =
\begin{bmatrix}
1 & -2 & 3 \\
2 & -4 & 7 \\
1 & -1 & 3
\end{bmatrix}
\]
og bruk den til å løse likningen
\[
A \x = \vvv{1}{2}{3}.
\]
\end{oppgave}


\begin{oppgave}
Betrakt differensiallikningen
\[
y''-3y'-2=e^t
\]
\begin{punkt}
Skriv den homogene likningen om til et første ordens system, og finn løsnigen. 
\end{punkt}
\begin{punkt}
Finn den generelle løsningen av likningen.
\end{punkt}
\end{oppgave}


\begin{oppgave}
Anta at $A$ er en $3 \times 3$-matrise med egenverdiene $2$, $5$ og $-5$.
Finn alle egenverdiene til hver av matrisene $3A$ og $A^2$.
(Pass på at du forklarer hvorfor egenverdiene du finner er \emph{alle} egenverdiene
til disse matrisene.)
\end{oppgave}


\begin{oppgave}
Bruk minste kvadrats metode til å finne en approksimasjon til likningssystemet
\[
\systeme{
x + 2y = 1,
3x + 4y = 2,
5x + 6y = 3
}
\]
\end{oppgave}


\begin{oppgave}
Vis at rotasjonsmatrisen 
\[
\begin{bmatrix}
\cos \theta  & -\sin \theta  \\  \sin \theta & \cos \theta 
\end{bmatrix}
\]
er en ortogonal matrise.  Finn en matrise som har kolonnene i rotasjonsmatrisen som egenvektorer, og egenverdier 1 og 2. \\
\end{oppgave}


\begin{oppgave}
La $A$ være følgende matrise:
\[
A =
\begin{bmatrix}
-9 &  20 & -10 \\
 0 &   1 &   0 \\
 5 & -10 &   6
\end{bmatrix}
\]
Finn alle vektorer $\x$ slik at $A^2 \x = A \x$.
% \Sp \{ \vvv{2}{1}{0}, \vvv{-1}{0}{1} \}
\end{oppgave}


\begin{oppgave}
Husk at vi skriver $\P_2$ for vektorrommet som består av alle
polynomfunksjoner av grad~$2$ eller lavere.  La $p_1$, $p_2$, $p_3$
og~$q$ være følgende polynomer i~$\P_2$:
\begin{align*}
p_1(x) &= x^2 - 2    &   q(x) &= 4x^2 + x - 7  \\
p_2(x) &= -x^2 + x + 2 \\
p_3(x) &= 3x^2 + x - 5
\end{align*}
Vis at $\B = (p_1, p_2, p_3)$ er en basis for $\P_2$,
og finn koordinatene $[ q ]_\B$ til $q$ med hensyn på denne basisen.
\end{oppgave}


\begin{oppgave}
Vis at en determinanten til en diagonaliserbar matrise er lik produktet av egenverdiene.
\end{oppgave}


\begin{oppgave}
La $T \colon \R^n \to \R^m$ være en lineærtransformasjon.
Vis at $T$ er surjektiv hvis og bare hvis det finnes en
lineærtransformasjon $S \colon \R^m \to \R^n$ slik at
$T(S(\v)) = \v$ for alle vektorer $\v$ i $\R^m$.
\end{oppgave}


\end{document}
