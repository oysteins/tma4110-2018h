% -*- TeX-master: "oving11"; -*-
\oppgaver{14}


\begin{oppgave}
Skissér faseplottet til systemet $A\V y=\V{y}'$ når
\begin{punkt}
$A= \begin{bmatrix}
2 & 3\\
-1 & -2
\end{bmatrix}
$	
\end{punkt}

\begin{punkt}
	$A= \begin{bmatrix}
	7 & -1\\
	3 & 3
	\end{bmatrix}
	$	
\end{punkt}

\begin{punkt}
	$A= \begin{bmatrix}
	-3 & 2\\
	-1 & -1
	\end{bmatrix}
	$	
\end{punkt}

\begin{punkt}
	$A= \begin{bmatrix}
	-3 & -9\\
	2 & 3
	\end{bmatrix}
	$	
\end{punkt}

\end{oppgave}


\begin{losning}
Vi ber kun om en skisse. Det holder å finne egenverdiene for å se hvordan systemet oppfører seg.
	\begin{punkt}
		Egenverdiene er $-1$ og~$1$, derfor får vi en sadel om origo.
	\end{punkt}
	
	\begin{punkt}
		Egenverdiene er $4$ og~$6$, derfor får vi en ustabilt likevektsløsning.
	\end{punkt}
	
	\begin{punkt}
		Egenverdiene er $2+i$ og~$2-i$, derfor får vi spiraler som er sirkulære og utgående fra origo.
	\end{punkt}
	
	\begin{punkt}
		Egenverdiene er $3i$ og~$-3i$, derfor får vi sirkulære baner om origo.
	\end{punkt}
	
\end{losning}


\begin{oppgave}
Finn generell løsning av $A\V y=\V y'$ når
\begin{punkt}
$
A=
\begin{bmatrix}
2 & 3 & 5\\
0 & 3 & 5\\
0 & 0 & 5
\end{bmatrix}
$ 
\end{punkt}
%
\begin{punkt}
$
A=
\begin{bmatrix}
3 & -1 & 2\\
3 & -1 & 6\\
-2 & 2 & -2
\end{bmatrix}$
\end{punkt}

\begin{punkt}
$
A=
\begin{bmatrix}
0 & -1 & 0\\
1 & 0 & 0\\
0 & 0 & 3
\end{bmatrix}$
\end{punkt}



\begin{punkt}
$
A=
\begin{bmatrix}
0 & -1 & 1& 5\\
1 & 0 & 2& 6\\
0 & 0 & 3 & 7\\
0 & 0 & 4 & 0
\end{bmatrix}$
\end{punkt}


\end{oppgave}

\begin{losning}
Du fant egenverdier og egenvektorer til alle matrisene i kapittel 13. Du trenger derfor bare å sette inn i formelen for generell løsning.


\begin{punkt}
	$$\V{y}=c_1\vvv{1}{0}{0}e^{2t}+c_2\vvv{3}{1}{0}e^{3t}+c_3\vvv{25}{15}{6}e^{5t}$$
\end{punkt}

\begin{punkt}
	$$\V{y}=c_1\vvv{1}{1}{0}e^{2t}+c_2\vvv{-2}{0}{1}e^{2t}+c_3\vvv{1}{3}{-2}e^{-4t}$$
\end{punkt}

\begin{punkt}
$$	\begin{aligned}
	\V{y}= {} & c_1\vvv{0}{0}{1}e^{3t}+c_2\left(\cos (t) \vvv{0}{1}{0}-\sin (t) \vvv{1}{0}{0}\right)\\
	& +c_3\left(\cos (t) \vvv{1}{0}{0}+\sin (t) \vvv{0}{1}{0}\right)
	\end{aligned}$$
	
\end{punkt}

\begin{punkt} 
$$	\begin{aligned}
	\V{y}= {} & c_1\vvvv{20}{12}{17}{-17}e^{-4t} +c_2\vvvv{151}{293}{350}{200}e^{7t}\\
	& +c_3\left(\cos (t) \vvvv{0}{1}{0}{0}-\sin (t) \vvvv{1}{0}{0}{0}\right)\\
	& +c_4\left(\cos (t) \vvvv{1}{0}{0}{0}+\sin (t) \vvvv{0}{1}{0}{0}\right)
	\end{aligned}$$

\end{punkt}


	
\end{losning}

\begin{oppgave}
Løs initialverdiproblemene $A\V y= \V y '$, $\V{y}(0)=\V{y}_0$ når 

\begin{punkt}
	$
	A=
	\begin{bmatrix}
	2 & 3 & 5\\
	0 & 3 & 5\\
	0 & 0 & 5
	\end{bmatrix}
	$, $
	\V{y}_0=\vvv{1}{0}{0}
	$
\end{punkt}
%
\begin{punkt}
	$
	A=
	\begin{bmatrix}
	3 & -1 & 2\\
	3 & -1 & 6\\
	-2 & 2 & -2
	\end{bmatrix}$, $
	\V{y}_0=\vvv{1}{-1}{1}
	$
\end{punkt}

\begin{punkt}
	$
	A=
	\begin{bmatrix}
	0 & -1 & 0\\
	1 & 0 & 0\\
	0 & 0 & 3
	\end{bmatrix}$, $
	\V{y}_0=\vvv{1}{2}{3}
	$
\end{punkt}


\end{oppgave}

\begin{losning}


\begin{punkt}
	$$\V{y}=\vvv{1}{0}{0}e^{2t}$$
\end{punkt}

\begin{punkt}
	$$\V{y}=2\vvv{1}{1}{0}e^{2t}-\vvv{1}{3}{-2}e^{-4t}$$
\end{punkt}

\begin{punkt}
	$$	\begin{aligned}
	\V{y}= {} & 3\vvv{0}{0}{1}e^{3t}+2(\cos (t) \vvv{0}{1}{0}-\sin (t) \vvv{1}{0}{0})\\
	& +(\cos (t) \vvv{1}{0}{0}+\sin (t) \vvv{0}{1}{0})
	\end{aligned}$$
	
\end{punkt}


	

\end{losning}

\begin{oppgave}
Løs initialverdiproblemet $$
\begin{bmatrix}
0 & -1 & 0\\
1 & 0 & 0\\
0 & 0 & 3
\end{bmatrix}
\V{y}=\V{y}',\quad \V{y}\left(\frac{\pi}{2}\right)=\vvv{1}{1}{1}.
$$
\end{oppgave}

\begin{losning}
	$$	\begin{aligned}
	\V{y}= {} & \vvv{0}{0}{1}e^{3(t-\frac{\pi}{2})}-\left(\cos (t) \vvv{0}{1}{0}-\sin (t) \vvv{1}{0}{0}\right)\\
	& +\left(\cos (t) \vvv{1}{0}{0}+\sin (t) \vvv{0}{1}{0}\right)
	\end{aligned}$$
\end{losning}


\begin{oppgave}
Vis at systemet
$$
\begin{bmatrix}
3 & -1 & 2\\
3 & -1 & 6\\
-2 & 2 & -2
\end{bmatrix} \V y= \V{y} '$$
med en gitt initialverdi $\V{y}(0)=\V{y}_0$ har en entydig løsning.\\

\noindent
\emph{Hint}: Du kan anta at den generelle løsningen inneholder alle løsninger.
\end{oppgave}

\begin{losning}
Vi har sett at den generelle løsningen er 
$$\V{y}=c_1\vvv{1}{1}{0}e^{2t}+c_2\vvv{-2}{0}{1}e^{2t}+c_3\vvv{1}{3}{-2}e^{-4t}.$$ Det gjenstår kun å bestemme koeffisientene fra initialverdiene:
$$c_1\vvv{1}{1}{0}+c_2\vvv{-2}{0}{1}+c_3\vvv{1}{3}{-2}=\V{y}_0.$$ Dette kan skrives som en matriselikning
$$\begin{bmatrix}
1 & -2 & 1\\
1 & 0 & 3\\
0 & 1 & -1
\end{bmatrix} \V c= \V{y}_0$$ hvor $\V c=\vvv{c_1}{c_2}{c_3}$. Du kan sjekke at kolonnene til denne matrisen er inverterbar. Dermed er det et entydig valg av koeffisienter $c_1$, $c_2$ og~$c_3$; løsningen er dermed entydig ettersom den generelle løsningen inneholder alle løsninger.
\end{losning}

%\newpage
\begin{oppgave}\textbf{[Utfordring]}
	
\noindent
Husk at glatte funksjoner $f :\mathbb{R}\rightarrow \mathbb{R}$ er et vektorrom. På samme måte danner glatte vektorfunksjoner $\mathbf f:\mathbb{R}\rightarrow \mathbb{R}^n$ et vektorrom $V$ med addisjon
$$(\mathbf f + \mathbf g)(t)=\mathbf{f}(t)+\mathbf{g}(t),$$ og skalarmultiplikasjon
$$(c \mathbf f )(t)=c\mathbf{f}(t).$$

\begin{punkt}
Forklar hvorfor alle løsningene til et system $A\V y=\V{y}'$ danner et underrom av $V$.
\end{punkt}

\begin{punkt}
Hva er dimensjonen til rommet av alle løsninger til $$
\begin{bmatrix}
3 & -1 & 2\\
3 & -1 & 6\\
-2 & 2 & -2
\end{bmatrix} \V y= \V{y} ' ?$$

\noindent
\emph{Hint}: Bruk forrige oppgave.
\end{punkt}


\end{oppgave}





\begin{losning}
	\begin{punkt}
		Dette er superposisjonsprinsippet.
	\end{punkt}
	
	\begin{punkt}
	Vi viste i forrige oppgave at en løsning er entydig bestemt av en gitt initialverdi. Initialverdiene danner et tredimensjonalt rom. Derfor er dimensjonen lik tre.
	\end{punkt}
	
	
\end{losning}
