% -*- TeX-master: "oving11"; -*-
\oppgaver{14}


\begin{oppgave}
Finn likningssystemet $A\V y=\V y'$ sin generelle løsning og tegn fasediagram når
\begin{punkt}
$
A=
\begin{bmatrix}
2 & 2 & 5\\
0 & 3 & 5\\
0 & 0 & 5
\end{bmatrix}
$ 
\end{punkt}
%
\begin{punkt}
$
A=
\begin{bmatrix}
3 & -1 & 2\\
3 & -1 & 6\\
-2 & 2 & -2
\end{bmatrix}$
\end{punkt}

\begin{punkt}
$
A=
\begin{bmatrix}
0 & -1 & 0\\
1 & 0 & 0\\
0 & 0 & 3
\end{bmatrix}$
\end{punkt}



\begin{punkt}
$
A=
\begin{bmatrix}
0 & -1 & 1& 5\\
1 & 0 & 2& 6\\
0 & 0 & 3 & 7\\
0 & 0 & 4 & 0
\end{bmatrix}$
\end{punkt}


\end{oppgave}


\begin{losning}
\begin{punkt}
Matrisen har egenverdien~$0$, med tilhørende egenrom $\Sp \{ \vvS{1}{0} \}$.
Siden den ikke har to lineært uavhengige egenvektorer er den
ikke diagonaliserbar.
\end{punkt}

\begin{punkt}
Matrisen har egenverdiene $2$ og~$5$, med tilhørende egenrom
henholdsvis
\[
\Sp \left\{ \vvv{1}{0}{0} \right\}
\qquad\text{og}\qquad
\Sp \left\{ \vvv{25}{15}{9} \right\}.
\]
Siden den ikke har tre lineært uavhengige egenvektorer er den
ikke diagonaliserbar.
\end{punkt}

\begin{punkt}
Det karakteristiske polynomet er
\[
\begin{vmatrix}
3-\lambda & -1         & 2          \\
3         & -1-\lambda & 6          \\
-2        & 2          & -2-\lambda
\end{vmatrix}
=
-\lambda^3 + 12\lambda - 16.
\]
Vi må altså løse likningen
\[
-\lambda^3 + 12\lambda - 16 = 0.
\]
Ved å prøve oss frem finner vi ganske raskt at $\lambda=2$ er en
løsning.  Det vil si at vi kan dele ut faktoren $(\lambda-2)$ fra det
karakteristiske polynomet ved polynomdivisjon.  Da får vi at
\[
-\lambda^3 + 12\lambda - 16 = (\lambda - 2) (-\lambda^2 - 2\lambda + 8).
\]
Vi finner dermed de andre løsningene av likningen ved å løse
\[
-\lambda^2 - 2\lambda + 8 = 0,
\]
som vi gjør ved å bruke den vanlige formelen for andregradslikninger:
\[
\lambda
 = \frac{-(-2) \pm \sqrt{(-2)^2 - 4 \cdot (-1) \cdot 8}}{2 (-1)}
 = -1 \pm 3
\]
Egenverdiene til matrisen er altså $2$ og~$-4$.

Vi finner egenrommene på vanlig måte.  Egenrommet til egenverdien~$2$ er
\[
\Sp \left\{ \vvv{1}{1}{0}, \vvv{-2}{0}{1} \right\},
\]
og egenrommet til egenverdien~$-4$ er
\[
\Sp \left\{ \vvv{1}{3}{-1} \right\}.
\]
Matrisen har altså tre lineært uavhengige egenvektorer, så den er
diagonaliserbar.
\end{punkt}

\begin{punkt}
Egenverdiene er $3$, $-i$ og~$i$, med tilhørende egenrom henholdsvis
\[
\Sp \left\{ \vvv{0}{0}{1} \right\},\qquad
\Sp \left\{ \vvv{-i}{1}{0} \right\},\qquad
\Sp \left\{ \vvv{i}{1}{0} \right\}.
\]
Matrisen har altså tre lineært uavhengige egenvektorer, så den er
diagonaliserbar.
\end{punkt}

\begin{punkt}
Egenverdiene er $0$, $3$, $-i$ og~$i$, med tilhørende egenrom
henholdsvis
\[
\Sp \left\{ \vvvv{-6}{5}{0}{1} \right\},\quad
\Sp \left\{ \vvvv{39}{113}{30}{40} \right\},\quad
\Sp \left\{ \vvvv{-i}{1}{0}{0} \right\},\quad
\Sp \left\{ \vvvv{i}{1}{0}{0} \right\}.
\]
Matrisen har altså fire lineært uavhengige egenvektorer, så den er
diagonaliserbar.
\end{punkt}

\begin{punkt}
Egenverdiene er $-4$, $7$, $-i$ og~$i$, med tilhørende egenrom
henholdsvis
\[
\Sp \left\{ \vvvv{20}{12}{17}{-17} \right\},\quad
\Sp \left\{ \vvvv{151}{293}{350}{200} \right\},\quad
\Sp \left\{ \vvvv{-i}{1}{0}{0} \right\},\quad
\Sp \left\{ \vvvv{i}{1}{0}{0} \right\}.
\]
Matrisen har altså fire lineært uavhengige egenvektorer, så den er
diagonaliserbar.
\end{punkt}
\end{losning}



\begin{oppgave}
Løs initialverdiproblemet
\begin{punkt}
$
A=
\begin{bmatrix}
1 & 2 & 2\\
2 & 6 & 2\\
2 & 2 & 6
\end{bmatrix}
$ 
\end{punkt}
\begin{punkt}
$\begin{bmatrix}
\frac{\sqrt{3}}{2} & -\frac{1}{2} \\ \frac{1}{2} & \frac{\sqrt{3}}{2}
\end{bmatrix}
$
\end{punkt}
\begin{punkt}
% T(1) = 1
% T(x) = (x+1) + x = 2x + 1
% T(x^2) = (x+1) 2x + x^2 = 3x^2 + 2x
$
A =
\begin{bmatrix}
1 & 1 & 0 \\
0 & 2 & 2 \\
0 & 0 & 3
\end{bmatrix}
$
\end{punkt}


\end{oppgave}




\begin{losning}
\begin{punkt}
Rotasjonsvinkelen er $\pi/6$.
\end{punkt}
\begin{punkt}
Det karakteristiske polynomet er:
\[
\begin{vmatrix}
\frac{\sqrt{3}}{2} - \lambda & -\frac{1}{2}                 \\
\frac{1}{2}                  & \frac{\sqrt{3}}{2} - \lambda
\end{vmatrix}
=
\left( \frac{\sqrt{3}}{2} - \lambda \right)^2 + \frac{1}{4}
\]
% \sqrt{3}/2 - \lambda = \pm \sqrt{-1/4} = \pm i/2
% \lambda = \sqrt{3}/2 \pm i/2
Egenverdiene er
\[
\frac{\sqrt{3}}{2} + \frac{i}{2}
\qquad\text{og}\qquad
\frac{\sqrt{3}}{2} - \frac{i}{2},
\]
med tilhørende egenrom henholdsvis
% -i/2 -1/2
%  1/2 -i/2
% --
% i  1
% 1 -i
% --
% 1 -i
% 0  0
% -----------------
% i/2 -1/2
% 1/2  i/2
% --
% i -1
% 1  i
% --
% 1 i
% 0 0
\[
\Sp \left\{ \vv{i}{1} \right\}
\qquad\text{og}\qquad
\Sp \left\{ \vv{i}{-1} \right\}.
\]
\end{punkt}
\begin{punkt}
La
\[
\v_1 = \vv{i}{1}
\qquad\text{og}\qquad
\v_2 = \vv{i}{-1}
\]
være de to egenvektorene i fant i del~\textbf{b)}.  Vi skal finne
matrisen til~$T$ med hensyn på basisen $\B = (\v_1, \v_2)$.  Vi vet at
\[
T(\v_1) = \left( \frac{\sqrt{3}}{2} + \frac{i}{2} \right) \v_1
\quad\text{og}\quad
T(\v_2) = \left( \frac{\sqrt{3}}{2} - \frac{i}{2} \right) \v_2.
\]
Dermed får vi følgende matrise:
\[
\begin{bmatrix}
\frac{\sqrt{3}}{2} + \frac{i}{2} & 0 \\
0 & \frac{\sqrt{3}}{2} - \frac{i}{2}
\end{bmatrix}
\]
\end{punkt}
\end{losning}







\begin{losning}
\begin{punkt}
Det karakteristiske polynomet er
\begin{align*}
\begin{vmatrix}
2 - \lambda & 3 - i       \\
3 + i       & 4 - \lambda
\end{vmatrix}
&= (2 - \lambda)(4 - \lambda) - (3 - i)(3 + i) \\
&= \lambda^2 - 6\lambda - 2,
\end{align*}
og egenverdiene er $3 + \sqrt{11}$ og $3 - \sqrt{11}$.

Vi finner lett ut at
\[
\v_1 = \vv{3-i}{1+\sqrt{11}}
\]
er en egenvektor for egenverdien $3 + \sqrt{11}$, og at
\[
\v_2 = \vv{3-i}{1-\sqrt{11}}
\]
er en egenvektor for egenverdien $3 - \sqrt{11}$.  For å lage en
ortogonal diagonalisering må vi ha egenvektorer med lengde~$1$, så vi
deler hver av disse på lengden sin og får normaliserte egenvektorer
$\hat\v_1$ og~$\hat\v_2$:
\begin{align*}
\hat\v_1
&= \frac{1}{\| \v_1 \|} \v_1
 = \frac{1}{\sqrt{2 (11+\sqrt{11})}} \v_1 \\
&= \vv{(3-i) \Big/ \sqrt{2 (11+\sqrt{11})}}
      {(1+\sqrt{11}) \Big/ \sqrt{2 (11+\sqrt{11})}}
\\
\hat\v_2
&= \frac{1}{\| \v_2 \|} \v_2
 = \frac{1}{\sqrt{2 (11-\sqrt{11})}} \v_2 \\
&= \vv{(3-i) \Big/ \sqrt{2 (11-\sqrt{11})}}
      {(1-\sqrt{11}) \Big/ \sqrt{2 (11-\sqrt{11})}}
\end{align*}
Da kan vi sette
\[
V = \begin{bmatrix} \hat\v_1 & \hat\v_2 \end{bmatrix}
\quad\text{og}\quad
D =
\begin{bmatrix}
3+\sqrt{11} & 0           \\
0           & 3-\sqrt{11}
\end{bmatrix}
\]
\end{punkt}
\begin{punkt}
Det karakteristiske polynomet er
\[
\begin{vmatrix}
1-\lambda & 1-i\\
1+i & -1-\lambda
\end{vmatrix}
= \lambda^2 - 3.
\]
Egenverdiene er $\sqrt{3}$ og~$-\sqrt{3}$.  Vi finner tilhørende
egenvektorer
\[
\v_1 = \vv{-1+i}{1-\sqrt{3}}
\qquad\text{og}\qquad
\v_2 = \vv{-1+i}{1+\sqrt{3}}.
\]
Vi normaliserer disse:
\begin{align*}
\hat\v_1
&= \vv{(-1+i)/\sqrt{(6-2\sqrt{3})}}
      {(1-\sqrt{3})/\sqrt{(6-2\sqrt{3})}}
\\
\hat\v_1
&= \vv{(-1+i)/\sqrt{(6+2\sqrt{3})}}
      {(1+\sqrt{3})/\sqrt{(6+2\sqrt{3})}}
\end{align*}
Da kan vi sette
\[
V = \begin{bmatrix} \hat\v_1 & \hat\v_2 \end{bmatrix}
\quad\text{og}\quad
D =
\begin{bmatrix}
\sqrt{3} & 0         \\
0        & -\sqrt{3}
\end{bmatrix}
\]
\end{punkt}
\begin{punkt}
Det karakteristiske polynomet er
\[
\begin{vmatrix}
1 & 2 & 2\\
2 & 6 & 2\\
2 & 2 & 6
\end{vmatrix}
= -\lambda^3 + 13\lambda^2 - 36\lambda
\]
Egenverdiene er $0$, $4$ og~$9$.  Vi finner tilhørende egenvektorer
\[
\v_1 = \vvv{-4}{1}{1},\qquad
\v_2 = \vvv{0}{1}{-1}\qquad\text{og}\qquad
\v_3 = \vvv{1}{2}{2}.
\]
Vi normaliserer disse:
\begin{align*}
\hat\v_1 &= \vvv{-2/\sqrt{3}}{1/(2\sqrt{3})}{1/(2\sqrt{3})} \\
\hat\v_2 &= \vvv{0}{1/\sqrt{2}}{-1/\sqrt{2}} \\
\hat\v_3 &= \vvv{1/3}{2/3}{2/3}
\end{align*}
Da kan vi sette
\[
V = \begin{bmatrix} \hat\v_1 & \hat\v_2 & \hat\v_3 \end{bmatrix}
\quad\text{og}\quad
D =
\begin{bmatrix}
0 & 0 & 0 \\
0 & 4 & 0 \\
0 & 0 & 9
\end{bmatrix}
\]
\end{punkt}
\end{losning}


\oppgaver{15}

\begin{oppgave}
Skriv om til system.
\end{oppgave}

\begin{oppgave}
Løs likningen.
\end{oppgave}

%\end{losning}