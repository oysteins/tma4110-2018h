% -*- TeX-master: "oving02"; -*-
\oppgaver{3}

\begin{oppgave}
La $\V{u} = \vv{3}{2}$ og~$\V{v} = \vv{-1}{1}$ være to vektorer
i~$\R^2$.

\begin{punkt}
Regn ut $\V{u} + \V{v}$ og $\frac{1}{2} \V{u} - 2 \V{v}$.
\end{punkt}

\begin{punkt}
Tegn en figur som viser vektorene $\V{u}$, $\V{v}$, $\V{u} + \V{v}$ og
$\frac{1}{2} \V{u} - 2 \V{v}$ i planet.
\end{punkt}
\end{oppgave}

\begin{losning}
\begin{punkt}
	$\V{u} + \V{v}=
	\begin{bmatrix}
	2\\
	3
	\end{bmatrix}
	$ og $\frac{1}{2} \V{u} - 2 \V{v}=
	\begin{bmatrix}
	\frac{7}{2}\\
	-1
	\end{bmatrix}
	$.
\end{punkt}

\begin{punkt}
\end{punkt}
\begin{center}
	\begin{tikzpicture}
	\draw[->] (-1,0) -- (4,0) node[right] {$x$};
	\draw[->] (0,-1) -- (0,4) node[above] {$y$};
	\draw[->] (0,0) -- (3,2) node[above] {$\V{u}$};
	\draw[->] (0,0) -- (-1,1) node[above] {$\V{v}$};
	\draw[->] (0,0) -- (7/2,-1) node[below] {$\frac{1}{2}\V{u}-2\V{v}$};
	\draw[->] (0,0) -- (2,3) node[above] {$\V{u}+\V{v}$};
	\end{tikzpicture}
\end{center}
\end{losning}

\begin{oppgave}
Skriv alle ligningssystemene fra oppgave~\textbf{2.2} i øving~1 som

\begin{punkt}
\ldots vektorlikninger.
\end{punkt}

\begin{punkt}
\ldots matriselikninger.
\end{punkt}

\end{oppgave}

\begin{oppgave}
Løs ligningen $A\V{x}=\V{b}$ for 
% \begin{align*}
% \text{\textbf{a)}\hspace{20pt}}
% A &=
% \begin{bmatrix}
% \;0 & 1 & 1 & 0 & 0 & 0 & 1\;\\
% \;0 & 0 & 0 & 0 & 1 & 0 & 1\;\\
% \;0 & 0 & 0 & 0 & 0 & 1 & 1\;\\
% \;0 & 0 & 0 & 0 & 0 & 0 & 0\;
% \end{bmatrix},
% &
% \V{b} &=
% \begin{bmatrix}
% 0  \\
% 0 \\
% 0 \\
% 0
% \end{bmatrix}
% \\[10pt]
% \text{\textbf{b)}\hspace{20pt}}
% A &=
% \begin{bmatrix}
% 1 & 2 & 3\\
% 2 & 3 & 4\\
% 3 & 4 & 5\\
% 4 & 5 & 6
% \end{bmatrix},
% &
% \V{b} &=
% \begin{bmatrix}
% -3  \\
% -7 \\
% -3 \\
% 0
% \end{bmatrix}
% \\[10pt]
% \text{\textbf{c)}\hspace{20pt}}
%  A &=
% 	\begin{bmatrix}
% 	8  & -7 & 0 \\
% 	-8 & -7 & 3 \\
% 	-4 & 5  & -8\\
% 	-6 & 6  & -4
% 	\end{bmatrix},
% &
%         \V{b} &=
% 	\begin{bmatrix}
% 	-3  \\
% 	-7 \\
% 	-3 \\
% 	0
% 	\end{bmatrix}
% \end{align*}
\begin{punkt}
\vspace{-10pt}
$$A=
\begin{bmatrix}
\;0 & 1 & 1 & 0 & 0 & 0 & 1\;\\
\;0 & 0 & 0 & 0 & 1 & 0 & 1\;\\
\;0 & 0 & 0 & 0 & 0 & 1 & 1\;\\
\;0 & 0 & 0 & 0 & 0 & 0 & 0\;
\end{bmatrix},
\qquad
\V{b}=
\begin{bmatrix}
0  \\
0 \\
0 \\
0
\end{bmatrix}
$$
\end{punkt}

\begin{punkt}
\vspace{-10pt}
$$A=
\begin{bmatrix}
1 & 2 & 3\\
2 & 3 & 4\\
3 & 4 & 5\\
4 & 5 & 6
\end{bmatrix},
\qquad
\V{b}=
\begin{bmatrix}
-3  \\
-7 \\
-3 \\
0
\end{bmatrix}
$$
\end{punkt}


\begin{punkt}
\vspace{-10pt}
	$$A=
	\begin{bmatrix}
	8  & -7 & 0 \\
	-8 & -7 & 3 \\
	-4 & 5  & -8\\
	-6 & 6  & -4
	\end{bmatrix},
\qquad
        \V{b}=
	\begin{bmatrix}
	-3  \\
	-7 \\
	-3 \\
	0
	\end{bmatrix}
	$$
	
\end{punkt}

\end{oppgave}


\begin{oppgave}
%(TODO: noen gitte vektorer, finn ut om en av dem er lineærkombinasjon av de andre)
Finn ut om en vektor er en lineærkombinasjon av de andre:
\begin{punkt}
$$
\begin{bmatrix}
1\\
2
\end{bmatrix},
\begin{bmatrix}
2\\
3
\end{bmatrix},
\begin{bmatrix}
3\\
4
\end{bmatrix}
$$
\end{punkt}
\begin{punkt}
	$$
	\begin{bmatrix}
	1\\
	2\\
	3\\
	4
	\end{bmatrix},
	\begin{bmatrix}
	2\\
	3\\
	4\\
	5
	\end{bmatrix},
	\begin{bmatrix}
	3\\
	4\\
	5\\
	6
	\end{bmatrix}
	$$
\end{punkt}

\end{oppgave}
\begin{losning}
\begin{punkt}
Et vilkårlig valg av en vektor vil kunne skrives som en lineærkombinasjon av de to andre. Det er derfor tre mulige fremgangsmåter som alle er riktig.
Hint: La $\V{v_1}$ være en av vektorene. Du ønsker å sjekke om $\V{v_1}$ er en lineærkombinasjon av de to resterende vektorene. Dette kan formuleres med likningen $$\V{v_1}=a\V{v_2}+b\V{v_3}$$ hvor $\V{v_2}$ og~$\V{v_3}$ er de to resterende vektorene, og $a$ og~$b$ er ukjente koeffisienter. Prøv å skissere løsningen din i $x$--$y$-planet.
\end{punkt}
\begin{punkt}
Se $\textbf{a)}$, men merk at du ikke kan skissere løsningen din ettersom dette krever fire dimensjoner. 
\end{punkt}
\end{losning}



\begin{oppgave}
%(TODO: gitt vektorer u, v, w (to eller tre), finn vektor som ikke er lineærkombinasjon av disse)
Finn en vektor som ikke er en lineærkombinasjon av:
\begin{punkt}
	$$
	\begin{bmatrix}
	1\\
	5\\
	-3
	\end{bmatrix},
	\begin{bmatrix}
	4\\
	18\\
	4
	\end{bmatrix}
	$$
\end{punkt}
\begin{punkt}
$$
\begin{bmatrix}
1\\
2\\
3\\
4
\end{bmatrix},
\begin{bmatrix}
2\\
3\\
4\\
5
\end{bmatrix},
\begin{bmatrix}
3\\
4\\
5\\
6
\end{bmatrix}
$$
\end{punkt}

\begin{punkt}
	$$
	\begin{bmatrix}
	8  \\
	-8 \\
	-4 \\
	-6
	\end{bmatrix},
	\begin{bmatrix}
	-7  \\
	-7 \\
	5 \\
	6
	\end{bmatrix},
	\begin{bmatrix}
	0  \\
	3 \\
	-8 \\
	-4
	\end{bmatrix}
	$$
\end{punkt}
\end{oppgave}

\begin{losning}
\begin{punkt}
Hint: Vi ønsker å finne en vektor $\begin{bmatrix}
a\\
b\\
c
\end{bmatrix}$ som ikke er en lineærkombinasjon av 
$
\begin{bmatrix}
	1\\
	5\\
	-3
\end{bmatrix}$ og~$
\begin{bmatrix}
	4\\
	18\\
	4
\end{bmatrix}$. Vektoren skal altså \emph{ikke} tilfredstille likningen 
$$x
\begin{bmatrix}
1\\
5\\
-3
\end{bmatrix}+y
\begin{bmatrix}
4\\
18\\
4
\end{bmatrix}=
\begin{bmatrix}
a\\
b\\
c
\end{bmatrix},$$
som har totalmatrise
$$\begin{bmatrix}
1  & 4  & a\\
5  & 18 & b\\
-3 & 4  & c
\end{bmatrix}.$$ Radreduser og velg $a$, $b$ og~$c$ slik at systemet ikke har løsning. Eksempelvis fungerer $a=0$, $b=1$ og~$c=0$.
\end{punkt}
\begin{punkt}
Hint: Som i $\textbf{a)}$, men nå er totalmatrisen
$$\begin{bmatrix}
1 & 2 & 3 & a\\
2 & 3 & 4 & b\\
3 & 4 & 5 & c\\
4 & 5 & 6 & d
\end{bmatrix}.$$
\end{punkt}


\begin{punkt}
Hint: Som i $\textbf{a)}$, men nå er totalmatrisen
$$\begin{bmatrix}
8  & -7 & 0  & a\\
-8 & -7 & 3  & b\\
-4 & 5  & -8 & c\\
-6 & 6  & -4 & d
\end{bmatrix}.$$
\end{punkt}
\end{losning}


\begin{oppgave}
	Er 
	$
	\begin{bmatrix}
	-3  \\
	-7 \\
	-3 \\
	0
	\end{bmatrix}
	$
	en lineærkombinasjon av vektorene i 
	
\begin{punkt}
\ldots oppgave \textbf{5. a)}?
\end{punkt} 
\begin{punkt}
	\ldots oppgave \textbf{5. b)}?
\end{punkt} 
\begin{punkt}
	\ldots oppgave \textbf{5. c)}?
\end{punkt} 
\end{oppgave}



\begin{oppgave}
	Finn en tredje vektor i samme plan som disse to vektorene:
\[
	\begin{bmatrix}
	-3  \\
	-7 \\
	-3 \\
	\end{bmatrix}
	\quad\text{og}\quad
	\begin{bmatrix}
	8  \\
	-8 \\
	-4 \\
	\end{bmatrix}
\]
\end{oppgave}




\begin{oppgave}
La $\V{v}$ og~$\V{w}$ være disse vektorene i~$\R^3$:
\[
	\V{v}=
	\begin{bmatrix}
	-3  \\
	-7 \\
	-3 \\
	\end{bmatrix}
	\quad\text{og}\quad
	\V{w}=
	\begin{bmatrix}
	8  \\
	-8 \\
	-4 \\
	\end{bmatrix}
\]
	Finn en vektor 
\[
	\V{u}=\begin{bmatrix}
	u_1    \\
	u_2 \\
	u_3  
	\end{bmatrix}
\]
slik at $\V{u}$, $\V{v}$ og~$\V{w}$
	spenner ut $\R^3$, og løs likningen $x\V{u}+y\V{v}+z\V{w}=0$.
\end{oppgave}





\begin{oppgave}
% 1  4  1
% 0 -2  3
% 0  0 17
% ---
% 1  4  1
% 0 -2  3
% 0  8  5
% ---
% 1  4  1
% 0 -2  3
%-3 -4  2
% ---
% 1  4  1
% 5 18  8
%-3  4  2
La $p$ og~$q$ være følgende polynomer:
\begin{align*}
p(x) &= x^2 + 5x - 3 \\
q(x) &= 4x^2 + 18x + 4
\end{align*}
\begin{punkt}
La $s$ være polynomet $s(x) = x^2 + 8x + 2$.  Finnes det konstanter
$a$ og~$b$ slik at
\[
s(x) = a \cdot p(x) + b \cdot q(x)
\]
for alle~$x$?
\end{punkt}
\begin{punkt}
Finn et andregradspolynom $t$ som oppfyller følgende: For hvert
andregradspolynom $r$ skal det være mulig å finne konstanter $a$, $b$
og~$c$ slik at
\[
r(x) = a \cdot p(x) + b \cdot q(x) + c \cdot t(x)
\]
\end{punkt}
\end{oppgave}
\begin{losning}
Introduksjon til løsning: Vi ønsker å beskrive problemet med lineær algebra. Et polynom er entydig bestemt av koeffisientene sine. Derfor kan all informasjonen om et andregradspolynom $p(x)=ax^2+bx+c$ lagres i vektoren $\V{p}=\begin{bmatrix}
a\\
b\\
c
\end{bmatrix}.$ Summen av to andregradspolynom $$ax^2+bx+c$$ og $$dx^2+ex+f$$ kan skrives $$(a+d)x^2+(b+e)x+(c+f).$$ Vi summerer altså koeffisientene foran tilhørende potens av $x$. Dette svarer akkurat til addisjon av tilhørende vektorer: $$\begin{bmatrix}
a\\
b\\
c
\end{bmatrix}+\begin{bmatrix}
d\\
e\\
f
\end{bmatrix}=\begin{bmatrix}
a+d\\
b+e\\
c+f
\end{bmatrix}.$$ Tilsvarende blir en konstant multiplisert med et andregradspolynom multiplisert i hver koeffisient: $$k\cdot(ax^2+bx+c)=(k\cdot a)x^2+(k\cdot b)x+(k\cdot c),$$ som svarer til skalarmultiplikasjon av tilhørende vektor: $$k\cdot \begin{bmatrix}
a\\
b\\
c
\end{bmatrix}=\begin{bmatrix}
k\cdot a\\
k\cdot b\\
k\cdot c
\end{bmatrix}.$$
\begin{punkt}
I dette lineær algebra-språket blir spørsmålet om $\V{s}=\begin{bmatrix}
1\\
8\\
2
\end{bmatrix}$ er en lineærkombinasjon av $\V{p}=\begin{bmatrix}
1\\
5\\
-3
\end{bmatrix}$ og $\V{q}=\begin{bmatrix}
4\\
18\\
4
\end{bmatrix}.$ Spørsmålet er altså om likningen $$x\begin{bmatrix}
1\\
5\\
-3
\end{bmatrix}+y\begin{bmatrix}
4\\
18\\
4
\end{bmatrix}=\begin{bmatrix}
1\\
8\\
2
\end{bmatrix},$$ 

som svarer til totalmatrisen $$\begin{bmatrix}
1  & 4  & 1\\
5  & 18 & 8\\
-3 & 4  & 2
\end{bmatrix},$$ har en løsning. Svaret er nei.
\end{punkt}
\begin{punkt}
Hint: I lineær algebra-språk skal du finne en vektor $\V{t}$ slik at alle vektorer $\V{r}$ kan skrives som en lineærkombinasjon av $\V{p}$, $\V{q}$ og~$\V{t}$. La $\V{t}$ være løsningen du fant i oppgave \textbf{3.3} del \textbf{a)}. Du kan nå sjekke at likningen $$x\V{p}+y\V{q}+z\V{t}=\V{r}$$ har en entydig løsning for alle valg av vektorer $\V{r}=\begin{bmatrix}
a\\
b\\
c
\end{bmatrix}.$ Det tilhørende polynomet $t$ - til $\V{t}$ - er altså en løsning på oppgaven.

Merk: Grunnen til at $\V{t}$ fungerer er at den er lineært uavhengig av $\V{p}$ og~$\V{q}$. Derfor får vi tre lineært uavhengige vektorer som til sammen utspenner $\mathbb{R}^3.$
\end{punkt}
\end{losning}







\begin{oppgave}
	La $m<n$. Kan $m$ vektorer spenne ut $\mathbb{R}^n$? 
\end{oppgave}





