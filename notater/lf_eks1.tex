\documentclass[titlepage,a4paper,12pt,norsk]{IMFeksamen}
\geometry{left=3.5cm,right=3.5cm,bottom=2cm}
\usepackage[utf8]{inputenc}
\trykkinfo[ensidig,sorthvit]
\emnekode{TMA4110}
\emnenavn{Matematikk 3 -- EKSEMPEL~1}
\eksamensdato{23. november 2018}
\eksamenstid{09:00–13:00}
\fagligkontaktinfo{Øystein Skartsæterhagen}{}
\hjelpemiddel{Ingen trykte eller håndskrevne hjelpemidler tillatt.
 Bestemt, enkel kalkulator tillatt.
 (Casio fx-82ES PLUS, Casio fx-82EX,
  Citizen SR-270X, Citizen SR-270X College,
  Hewlett Packard HP30S)}
\anneninfo{Eksamenen består av ti oppgaver
Hver av disse teller like mye.
Alle svar må begrunnes.

\textbf{Merk.}
Dette er ikke en virkelig eksamen, men et eksempel for å vise
hvordan en eksamen kan se ut.
Hvis du ikke har nok godkjente øvinger, kan du levere svar på disse
oppgavene som en ekstra øving.
}
\runninghead{TMA4110 -- Eksamen høsten 2018 -- EKSEMPEL~1}
\usepackage[T1]{fontenc}
\usepackage{lmodern,amsmath,amssymb,amsfonts}
\usepackage{mathrsfs}
\usepackage{systeme}


\newcommand{\N}{\mathbb{N}}
\newcommand{\Z}{\mathbb{Z}}
\newcommand{\Q}{\mathbb{Q}}
\newcommand{\R}{\mathbb{R}}
\newcommand{\C}{\mathbb{C}}

\newcommand{\M}{\mathcal{M}} % vektorrom av matriser
\newcommand{\Cf}{\mathcal{C}} % vektorrom av kontinuerlige funksjoner
\renewcommand{\P}{\mathcal{P}} % vektorrom av polynomer
\newcommand{\B}{\mathscr{B}} % basis

\renewcommand{\Im}{\operatorname{Im}}
\renewcommand{\Re}{\operatorname{Re}}

\newcommand{\abs}[1]{|#1|}
\newcommand{\intersect}{\cap}
\newcommand{\union}{\cup}
\newcommand{\fcomp}{\circ}
\newcommand{\iso}{\cong}

\newcommand{\roweq}{\sim}
\DeclareMathOperator{\Sp}{Sp}
\DeclareMathOperator{\Null}{Null}
\DeclareMathOperator{\Col}{Col}
\DeclareMathOperator{\Row}{Row}
\DeclareMathOperator{\rank}{rank}
\DeclareMathOperator{\im}{im}
\DeclareMathOperator{\id}{id}
\DeclareMathOperator{\Hom}{Hom}
\newcommand{\tr}{^\top}
\newcommand{\koord}[2]{[\,{#1}\,]_{#2}} % koordinater mhp basis

\newcommand{\V}[1]{\mathbf{#1}}
\newcommand{\vv}[2]{\begin{bmatrix} #1 \\ #2 \end{bmatrix}}
\newcommand{\vvS}[2]{\left[ \begin{smallmatrix} #1 \\ #2 \end{smallmatrix} \right]}
\newcommand{\vvv}[3]{\begin{bmatrix} #1 \\ #2 \\ #3 \end{bmatrix}}
\newcommand{\vvvv}[4]{\begin{bmatrix} #1 \\ #2 \\ #3 \\ #4 \end{bmatrix}}
\newcommand{\vvvvv}[5]{\begin{bmatrix} #1 \\ #2 \\ #3 \\ #4 \\ #5 \end{bmatrix}}
\newcommand{\vn}[2]{\vvvv{#1_1}{#1_2}{\vdots}{#1_#2}}

\newcommand{\e}{\V{e}}
\renewcommand{\u}{\V{u}}
\renewcommand{\v}{\V{v}}
\newcommand{\w}{\V{w}}
\renewcommand{\b}{\V{b}}
\newcommand{\x}{\V{x}}
\newcommand{\0}{\V{0}}

\newenvironment{amatrix}[1]{% "augmented matrix"
  \left[\begin{array}{*{#1}{c}|c}
}{%
  \end{array}\right]
}



\begin{document}


\begin{oppgave}
Totalmatrise:
$
\begin{bmatrix}
0 & 2 & 4 &   2 & 6 \\
2 & 1 & 2 & -13 & 3 \\
1 & 3 & 6 &  -4 & 9
\end{bmatrix}
$.
Dersom du radreduserer finner du relasjonene
$$x_1+7x_2+14x_3=21$$
og $$x_2+2x_3+x_4=3.$$
Vi har altså to frie variabler; det finnes mange valg av parametrisering. Du kan for eksemepl velge $x_2=t$ og $x_3=s$ som parametere. Dette gir løsninger
$$\{\vvvv{-7t-14s+21}{t}{s}{-t-2s+3}|\;\; t,s \in \mathbb{R}\}.$$
\end{oppgave}


\begin{oppgave}
Lineært avhengige. Du kan se dette ved å enten beregne determinanten (og se at den er lik null), eller radredusere (det er ikke tre pivotelement).
\end{oppgave}


\begin{oppgave}
Egenverdiene til
\[
\begin{bmatrix}
-42 & 8 & 27 \\
  0 & 2 & -2 \\
-30 & 6 & 19
\end{bmatrix}
\]
er $-22,1,0$ med egenrom utspent av henholdsvis 
$$\vvv{83}{5}{60} \quad \vvv{1}{2}{1} \quad \vvv{5}{6}{6}.$$

Sett inn i formelen for generell løsning:
$$\V{y} = c_1\vvv{83}{5}{60}e^{-22t}+c_2\vvv{1}{2}{1}e^{t}+c_3 \vvv{5}{6}{6}.$$
\end{oppgave}


\begin{oppgave}
Radreduser matrisen og bruk regneregler for determinanten. Determinanten er $-3+6i$.
\end{oppgave}


\begin{oppgave}
En matrise som tilfredstiller oppgaveteksten kan skrives på formen $A=PDP^{-1}$ hvor
\[
P=\begin{bmatrix}
1  & 1  & 2  & -1 \\
2  & 3  & 4  & -2 \\
0  & 0  & 1  & 3  \\
-3 & -3 & -6 & 4
\end{bmatrix},
\]
og
\[
D=\begin{bmatrix}
1 & 0 & 0  & 0 \\
0 & 1 & 0  & 0 \\
0 & 0 &-1  & 0  \\
0 & 0 & 0  & 3
\end{bmatrix}.
\]

Regn ut
\[
P^{-1}=\begin{bmatrix}
24 & -1 & -2 & 7  \\
-2 & 1  & 0  & 0  \\
-9 & 0  & 1  & -3 \\
3  & 0  & 0  & 1
\end{bmatrix},
\]
og sett inn i $A=PDP^{-1}$:
\[
A=\begin{bmatrix}
34  & 0 & -4 & 11 \\
66  & 1 & -8 & 22 \\
27  & 0 & -1 & 9  \\
-96 & 0 & 12 & -31
\end{bmatrix}.
\]


\end{oppgave}


\begin{oppgave}
Radredusering viser at vi har pivotelement i første og andre kolonne. En basis er dermed gitt av 

$$\vvv{1}{2}{-1}, \quad \vvv{5}{8}{3}.$$
Men merk at 
$$\vvv{1}{2}{-1} \boldsymbol{\cdot} \vvv{5}{8}{3} =18. $$
Derfor må vi bruke Gram-Schmidt for å ortogonalisere basisen:
Ta $\V{v}_1=\vvv{1}{2}{-1}$ som første basiselement, og trekk fra den ortogonale projeksjonen ned på denne:
$$\V{v}_2=\vvv{5}{8}{3}-\frac{\V{v}_1 \boldsymbol{\cdot} \vvv{5}{8}{3}}{\V{v}_1 \boldsymbol{\cdot} \V{v}_1} \V{v}_1=\vvv{2}{2}{6}.$$

% \vvv{1}{2}{-1}, \vvv{2}{2}{6}
\end{oppgave}


\begin{oppgave}
La % $A$ være følgende matrise:
$
A =
\begin{bmatrix}
3 & 0 & 4 \\
2 & 1 & 1
\end{bmatrix}
$.
Finn en matrise $B$ slik at $AB = I_2$.
% f.eks.
%  1/3  0
% -2/3  1
%   0   0
\end{oppgave}


\begin{oppgave}
La $T \colon \R^n \to \R^m$ være en lineærtransformasjon, og la $\u$,
$\v$ og~$\w$ være vektorer i~$\R^n$.  Vis at hvis $T(\u)$, $T(\v)$
og~$T(\w)$ er lineært uavhengige, så er $\u$, $\v$ og~$\w$ også
lineært uavhengige.
\end{oppgave}


\begin{oppgave}
Husk at vi skriver $\M_2$ for vektorrommet som består av alle $2 \times 2$-matriser.
La $U$ være mengden av alle symmetriske $2 \times 2$-matriser.
Vis at $U$ er et underrom av~$\M_2$.
Finn en basis for~$U$.  Hva er dimensjonen til~$\M_2$, og hva er dimensjonen til~$U$?
\end{oppgave}


\begin{oppgave}
La $A$ være en $m \times n$-matrise med rang~$r$.
Vis at det finnes en $m \times r$-matrise~$B$ og
en $r \times n$-matrise~$C$
slik at $A = BC$.
\end{oppgave}


\end{document}
