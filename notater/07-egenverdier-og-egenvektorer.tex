\ifx\inkludert\undefined
\documentclass[norsk,a4paper,twocolumn,oneside]{memoir}

\usepackage[utf8]{inputenc}
\usepackage{babel}
\usepackage{amsmath,amssymb,amsthm}
\usepackage[total={17cm,27cm}]{geometry}
\usepackage[table]{xcolor}
%\usepackage{tabularx}
\usepackage{systeme}
%\usepackage{hyperref}
%\usepackage{enumerate}

%\usepackage{sectsty}
\setsecheadstyle{\bfseries\large}
%\subsectionfont{\bf\normalsize}

\usepackage{tikz}
\usetikzlibrary{arrows.meta}

\newcommand{\defterm}[1]{\emph{#1}}

\newcommand{\N}{\mathbb{N}}
\newcommand{\Z}{\mathbb{Z}}
\newcommand{\Q}{\mathbb{Q}}
\newcommand{\R}{\mathbb{R}}

\newcommand{\abs}[1]{|#1|}

\newcommand{\roweq}{\sim}
\DeclareMathOperator{\Span}{Span}

\newcommand{\V}[1]{\mathbf{#1}}
\newcommand{\vv}[2]{\begin{bmatrix} #1 \\ #2 \end{bmatrix}}
\newcommand{\vvv}[3]{\begin{bmatrix} #1 \\ #2 \\ #3 \end{bmatrix}}
\newcommand{\vvvv}[4]{\begin{bmatrix} #1 \\ #2 \\ #3 \\ #4 \end{bmatrix}}
\newcommand{\vn}[2]{\vvvv{#1_1}{#1_2}{\vdots}{#1_#2}}

\newenvironment{amatrix}[1]{% "augmented matrix"
  \left[\begin{array}{*{#1}{c}|c}
}{%
  \end{array}\right]
}

% \newcounter{notatnr}
% \newcommand{\notatnr}[2]
% {\setcounter{notatnr}{#1}%
%  \setcounter{page}{#2}%
% }

\newtheorem{thm}{Teorem}[chapter]
\newtheorem*{thm-nn}{Teorem}
\newtheorem{cor}[thm]{Korollar}
\newtheorem{lem}[thm]{Lemma}
\newtheorem{prop}[thm]{Proposisjon}
\theoremstyle{definition}
\newtheorem{exx}[thm]{Eksempel}
\newtheorem*{defnx}{Definisjon}
\newtheorem*{oppg}{Oppgave}
\newtheorem*{merkx}{Merk}
\newtheorem*{spmx}{Spørsmål}

\newenvironment{defn}
  {\pushQED{\qed}\renewcommand{\qedsymbol}{$\triangle$}\defnx}
  {\popQED\enddefnx}
\newenvironment{ex}
  {\pushQED{\qed}\renewcommand{\qedsymbol}{$\triangle$}\exx}
  {\popQED\endexx}
\newenvironment{merk}
  {\pushQED{\qed}\renewcommand{\qedsymbol}{$\triangle$}\merkx}
  {\popQED\endmerkx}
\newenvironment{spm}
  {\pushQED{\qed}\renewcommand{\qedsymbol}{$\triangle$}\spmx}
  {\popQED\endspmx}

\setlength{\columnsep}{26pt}

\newcommand{\Tittel}[2]{%
\twocolumn[
\begin{center}
\Large
\begin{tabularx}{\textwidth}{cXr}
\cellcolor{black}\color{white}%
\bf {#1} &
#2
\hfill &
\footnotesize TMA4110 høsten 2018
\\ \hline
\end{tabularx}
\end{center}
]}

\newcommand{\tittel}[1]{\Tittel{\arabic{notatnr}}{#1}}

\newcommand{\linje}{%
\begin{center}
\rule{.8\linewidth}{0.4pt}
\end{center}
}


\newcommand{\chapternumber}{}

\makechapterstyle{tma4110}{%
 \renewcommand*{\chapterheadstart}{}
 \renewcommand*{\printchaptername}{}
 \renewcommand*{\chapternamenum}{}
 \renewcommand*{\printchapternum}{\renewcommand{\chapternumber}{\thechapter}}
 \renewcommand*{\afterchapternum}{}
 \renewcommand*{\printchapternonum}{\renewcommand{\chapternumber}{}}
 \renewcommand*{\printchaptertitle}[1]{
\LARGE
\begin{tabularx}{\textwidth}{cXr}
\cellcolor{black}\color{white}%
\textbf{\chapternumber} &
\textbf{##1}
\hfill &
%\footnotesize TMA4110 høsten 2018
\\ \hline
\end{tabularx}%
}
 \renewcommand*{\afterchaptertitle}{\par\nobreak\vskip \afterchapskip}
 % \newcommand{\chapnamefont}{\normalfont\huge\bfseries}
 % \newcommand{\chapnumfont}{\normalfont\huge\bfseries}
 % \newcommand{\chaptitlefont}{\normalfont\Huge\bfseries}
 \setlength{\beforechapskip}{0pt}
 \setlength{\midchapskip}{0pt}
 \setlength{\afterchapskip}{10pt}
}


\newcounter{oppgnr}[chapter]
\newcounter{punktnr}[oppgnr]
\newenvironment{oppgave}
 {\par\noindent\stepcounter{oppgnr}\textbf{{\arabic{oppgnr}}.}}
 {\par\bigskip}
\newenvironment{punkt}
 {\par\smallskip\noindent\stepcounter{punktnr}\textbf{\alph{punktnr})} }
 {\par}

\newcommand{\oppgaver}{\linje\section*{Oppgaver}}

\usepackage{xr}
\externaldocument{tma4110-2018h}
\newcommand{\kapittel}[2]{\setcounter{chapter}{#1}\addtocounter{chapter}{-1}\chapter{#2}}
\newcommand{\kapittelslutt}{\enddocument}
\begin{document}
\chapterstyle{tma4110}
\pagestyle{plain}
\fi


\kapittel{7}{Egenverdier og egenvektorer}
\label{ch:egenverdier-og-egenvektorer}

% TODO intro

\section*{Definisjon av egenverdier og egenvektorer}

\begin{ex}
\label{ex:egenverdier-intro}
La $A$ være følgende $2 \times 2$-matrise:
\[
A = \begin{bmatrix}
1 &  3 \\
4 & -3
\end{bmatrix}
\]
Vi vil se på hva som skjer med punkter i planet når vi ganger dem
med~$A$, altså når vi sender en vektor~$\V{x}$ til vektoren $A \V{x}$.

Vi velger følgende fire vektorer og ser hva $A$ gjør med dem:
\[
\V{e}_1 = \vv{1}{0},\quad
\V{e}_2 = \vv{0}{1},\quad
\V{u} = \vv{-1}{-2},\quad
\V{v} = \vv{3}{2}
\]
Vi får:
\[
A \V{e}_1 = \vv{1}{4},\ 
A \V{e}_2 = \vv{3}{-3},\ 
A \V{u} = \vv{-7}{2},\ 
A \V{v} = \vv{9}{6}
\]
La oss tegne opp de fire vektorene, samt vektorene $A$ sender dem til,
som punkter i planet.
\begin{center}
\begin{tikzpicture}[scale=.42]
\draw[->] (-7.5,0) -- (9.8,0);
\draw[->] (0,-3.5) -- (0,6.8);
\foreach \x in {-7,-6,-5,-4,-3,-2,-1,1,2,3,4,5,6,7,8,9}
\draw (\x,5pt) -- (\x,-5pt);
\foreach \y in {-3,-2,-1,1,2,3,4,5,6}
\draw (5pt,\y) -- (-5pt,\y);
\filldraw (1,0) circle [radius=3pt] node[anchor=north] {$\V{e}_1$};
\filldraw (0,1) circle [radius=3pt] node[anchor=east] {$\V{e}_2$};
\filldraw (-1,-2) circle [radius=3pt] node[anchor=east] {$\V{u}$};
\filldraw (3,2) circle [radius=3pt] node[anchor=east] {$\V{v}$};
\filldraw (1,4) circle [radius=3pt] node[anchor=south] {$A \V{e}_1$};
\filldraw (3,-3) circle [radius=3pt] node[anchor=north] {$A \V{e}_2$};
\filldraw (-7,2) circle [radius=3pt] node[anchor=east] {$A \V{u}$};
\filldraw (9,6) circle [radius=3pt] node[anchor=north] {$A \V{v}$};
\draw[->,shorten <=4pt,shorten >=4pt] (1,0) to[bend right=20] (1,4);
\draw[->,shorten <=4pt,shorten >=4pt] (0,1) to[bend right=30] (3,-3);
\draw[->,shorten <=4pt,shorten >=4pt] (-1,-2) to[bend right=20] (-7,2);
\draw[->,shorten <=4pt,shorten >=4pt] (3,2) to[bend left=20] (9,6);
\end{tikzpicture}
\\
{\small \textit{Matrisen~$A$ kaster vektorene rundt i planet}}
\end{center}
Vi ser at matrisen sender de fire eksempelvektorene våre i
forskjellige retninger.  Men akkurat vektoren
\[
\V{v} = \vv{3}{2}
\]
er interessant.  Det som skjer med den er at den blir sendt til
\[
\vv{9}{6},
\]
men det er jo det samme som $3 \cdot \V{v}$.  Virkningen av
matrisen~$A$ på akkurat denne vektoren er altså bare å skalere den opp
med~$3$:
\[
A \V{v} = 3 \V{v}\qedhere
\]
\end{ex}

Teorien om egenverdier og egenvektorer handler om å identifisere slike
situasjoner som den vi så i eksempelet, der virkningen av en matrise
på en vektor blir det samme som å bare gange opp vektoren med et tall.

\begin{defn}
La $A$ være en $n \times n$-matrise, $\lambda$ et tall og $\V{v}$ en
vektor i~$\R^n$ som ikke er nullvektoren.  Hvis
\[
A \V{v} = \lambda \V{v},
\]
så sier vi at tallet~$\lambda$ er en \defterm{egenverdi} for
matrisen~$A$, og at vektoren~$\V{v}$ er en \defterm{egenvektor}
for~$A$ som hører til egenverdien~$\lambda$.
\end{defn}

Et par merknader -- én matematisk og én språklig -- er på sin plass
etter denne definisjonen.

\begin{merk}
Hvorfor sier vi at $\V{v}$ ikke skal være nullvektoren?  Jo, for hvis
vi setter inn nullvektoren for~$\V{v}$, så får vi likningen
\[
A \cdot \V{0} = \lambda \cdot \V{0},
\]
som er oppfylt for alle matriser~$A$ og alle tall~$\lambda$.  Hvis vi
hadde tillatt nullvektoren som en egenvektor, så ville vi altså fått
at alle tall er egenverdier for alle matriser.  Da blir
egenverdibegrepet ganske meningsløst.
\end{merk}

\begin{merk}
Det er vanlig å bruke den greske bokstaven~$\lambda$ som variabelnavn
for egenverdier.  Vi kunne i og for seg brukt en hvilken som helst
annen bokstav, men siden det er $\lambda$ man vanligvis bruker, så
gjør vi det.  Navnet på bokstaven uttales «lambda», og den tilsvarer
bokstaven \emph{l} i det latinske alfabetet.
% Hvis vi for eksempel
% skriver navnet på filosofen Platon på hans eget språk, så er $\lambda$
% den andre bokstaven: $\Pi\lambda\alpha\tau\omicron\nu$.
% Πλάτων.
\end{merk}

\begin{ex}
Vi ser igjen på matrisen~$A$ fra eksempel~\ref{ex:egenverdier-intro}.
Vi så at vektoren
\[
\V{v} = \vv{3}{2}
\]
oppfylte likheten
\[
A \V{v} = 3 \V{v}.
\]
Det betyr at $3$ er en egenverdi for matrisen~$A$, og at $\V{v}$ er en
egenvektor som hører til egenverdien~$3$.

Finnes det flere egenvektorer?  Hvis vi ser på en vektor som er
parallell med~$\V{v}$, altså som er på formen $\V{w} = c \cdot \V{v}$
der $c$ er et tall, så får vi:
\[
A \V{w}
= A \cdot (c \V{v})
= c \cdot (A \V{v})
= c \cdot (3 \V{v})
%= 3 \cdot (c \V{v})
= 3 \cdot \V{w}
\]
Enhver slik vektor er altså en egenvektor som hører til
egenverdien~$3$, forutsatt at den ikke er nullvektoren.

Vi har altså funnet ut at~$A$ i hvert fall har én egenverdi,
nemlig~$3$, og uendelig mange egenvektorer som hører til denne
egenverdien, nemlig alle vektorene på denne linjen (unntatt
nullvektoren):
\begin{center}
\begin{tikzpicture}[scale=.42]
\draw[->] (-6.5,0) -- (6.8,0);
\draw[->] (0,-4.5) -- (0,4.8);
\foreach \x in {-6,-5,-4,-3,-2,-1,1,2,3,4,5,6}
\draw (\x,5pt) -- (\x,-5pt);
\foreach \y in {-4,-3,-2,-1,1,2,3,4}
\draw (5pt,\y) -- (-5pt,\y);
\draw (-6,-4) -- (6,4);
\end{tikzpicture}
\end{center}

Det vi foreløpig ikke vet, er om det kan finnes enda flere
egenvektorer, og om $A$ har flere egenverdier enn~$3$.  Vi skal vende
tilbake til dette eksempelet om en stund og finne ut av dette, etter
at vi har kommet frem til en generell metode for å finne alle
egenverdiene og egenvektorene til en matrise.
\end{ex}

% TODO tekst

\begin{thm}
\label{thm:egenvektor-sp}
Anta at $\lambda$ er en egenverdi for en $n \times n$-matrise~$A$, og
at $\V{v}$ er en tilhørende egenvektor.  Da er alle multipler
$c \V{v}$ av vektoren~$\V{v}$, der $c$ er et tall som ikke er~$0$,
også egenvektorer som hører til egenverdien~$\lambda$.  Med andre ord
er alle vektorer i mengden $\Sp \{\V{v}\}$, unntatt nullvektoren,
egenvektorer som hører til egenverdien~$\lambda$.
\end{thm}
% TODO bevis?

% \section*{Noen todimensjonale eksempler} ikke egen seksjon?

% TODO refleksjon, rotasjon, projeksjon

% TODO thm:egenverdi-0


\section*{Hvordan finne dem}

Gitt en $n \times n$-matrise~$A$, hvordan kan vi finne alle
egenverdiene og egenvektorene dens?

Ut fra definisjonen er vi på jakt etter tall~$\lambda$ og
vektorer~$\V{v} \ne \V{0}$ som oppfyller likheten
\[
A \V{v} = \lambda \V{v}.
\]
Vi har altså en likning med både $\lambda$ og~$\V{v}$ som ukjente, og
den ser ved første øyekast ganske uhåndterlig ut.  Men vi kan trikse
litt med den.  Vi kan først flytte alt over til venstre side:
\[
A \V{v} - \lambda \V{v} = \V{0}
\]
Nå fremstår det som veldig fristende å sette $\V{v}$-en utenfor
parentes, altså å skrive $(A - \lambda) \V{v}$.  Men det går ikke an,
for uttrykket $A - \lambda$, altså en matrise minus et tall, gir ikke
mening.

Nå kan vi bruke et lurt triks: Vi ganger inn identitetsmatrisen $I_n$.
Vi vet at $I_n \V{v}$ bare blir $\V{v}$ uansett hva vektoren~$\V{v}$
er, så vi kan skrive om likningen til:
\[
A \V{v} - \lambda I_n \V{v} = \V{0}
\]
Det vi har oppnådd nå er at vi har en $n \times n$-matrise ganget
med~$\V{v}$ i hvert ledd, og da kan vi sette $\V{v}$ utenfor parentes:
\[
(A - \lambda I_n) \cdot \V{v} = \V{0}
\]
Nå ser vi at $\lambda$ er en egenverdi for~$A$ hvis og bare hvis
likningen
\[
(A - \lambda I_n) \cdot \V{x} = \V{0}
\]
har en ikketriviell løsning.  Men dette er bare et vanlig lineært
likningssystem med
\[
A - \lambda I_n
\]
som koeffisientmatrise, og vi vet fra før %TODO ref?
at et slikt system har ikketrivielle løsninger hvis og bare hvis
\[
\det (A - \lambda I_n) = 0.
\]
Her har vi endt opp med en likning med bare~$\lambda$ som ukjent.  Vi
kan altså løse denne for å finne egenverdiene, uten at vi samtidig må
tenke på hva de tilhørende egenvektorene skal være.

Vi oppsummerer det vi har funnet ut i et teorem.

\begin{thm}
\label{thm:finne-egenverdier}
La $A$ være en $n \times n$-matrise.
\begin{enumerate}
\item[(a)] Egenverdiene til~$A$ er alle løsninger~$\lambda$ av
likningen
\[
\det (A - \lambda I_n) = 0.
\]
\item[(b)] Hvis $\lambda$ er en egenverdi for~$A$, så er de tilhørende
egenvektorene gitt ved alle ikketrivielle løsninger av likningen
\[
(A - \lambda I_n) \cdot \V{x} = \V{0}.
\]
\end{enumerate}
\end{thm}

Uttrykket
\[
\det (A - \lambda I_n),
\]
som står på venstresiden av likningen vi løser for å finne
egenverdiene, blir et $n$-tegradspolynom i~$\lambda$.  Vi kaller det
for det \defterm{karakteristiske polynomet} til~$A$.

\begin{ex}
\label{ex:finne-egenverdier}
Nå kan vi bruke teorem~\ref{thm:finne-egenverdier} til å finne alle
egenverdiene og egenvektorene til matrisen
\[
A = \begin{bmatrix}
1 &  3 \\
4 & -3
\end{bmatrix}
\]
fra eksempel~\ref{ex:egenverdier-intro}.

Vi finner egenverdiene ved å løse likningen
\[
\det (A - \lambda I_2) = 0,
\]
der venstresiden er det karakteristiske polynomet til~$A$.  La oss
først se hvordan matrisen $A - \lambda I_2$ ser ut:
\[
A - \lambda I_2
=
\begin{bmatrix}
1 &  3 \\
4 & -3
\end{bmatrix}
-
\begin{bmatrix}
\lambda &       0 \\
      0 & \lambda
\end{bmatrix}
=
\begin{bmatrix}
1 - \lambda &  3           \\
4           & -3 - \lambda
\end{bmatrix}
\]
Det karakteristiske polynomet blir:
\begin{align*}
\det (A - \lambda I_2)
&=
\begin{vmatrix}
1 - \lambda &  3           \\
4           & -3 - \lambda
\end{vmatrix}
\\
&= (1 - \lambda)(-3 - \lambda) - 3 \cdot 4 \\
&= \lambda^2 + 2\lambda - 15
\end{align*}
Det betyr at vi kan løse andregradslikningen
\[
\lambda^2 + 2\lambda - 15 = 0
\]
for å finne egenverdiene.  Vi løser den på vanlig måte og får:
\[
\lambda
 = \frac{-2 \pm \sqrt{2^2 - 4 \cdot (-15)}}{2}
 = -1 \pm 4
\]
Vi får altså to egenverdier: $3$ og~$-5$.

Vi finner alle egenvektorer som hører til egenverdien~$3$ ved å løse
likningen $(A - 3I_2) \V{x} = \V{0}$.  Vi kan løse denne likningen ved
å gausseliminere matrisen $(A - 3I_2)$:
\[
\begin{bmatrix}
-2 &  3 \\
 4 & -6
\end{bmatrix}
\roweq
\begin{bmatrix}
-2 & 3 \\
 0 & 0
\end{bmatrix}
\]
Vi får én fri variabel, og løsningene blir
\[
\V{x} = \vv{3}{2} \cdot t
\]
for alle tall~$t$.  Egenvektorene som hører til egenverdien~$3$ er
altså alle vektorer i
\[
\Sp \left\{ \vv{3}{2} \right\},
\]
unntatt nullvektoren.

Vi finner alle egenvektorer som hører til egenverdien~$-5$ ved å løse
likningen $(A + 5I_2) \V{x} = \V{0}$.  Vi kan løse denne likningen ved
å gausseliminere matrisen $(A + 5I_2)$:
\[
\begin{bmatrix}
6 & 3 \\
4 & 2
\end{bmatrix}
\roweq
\begin{bmatrix}
2 & 1 \\
0 & 0
\end{bmatrix}
\]
Vi får én fri variabel, og løsningene blir
\[
\V{x} = \vv{1}{-2} \cdot t
\]
for alle tall~$t$.  Egenvektorene som hører til egenverdien~$-5$ er
altså alle vektorer i
\[
\Sp \left\{ \vv{1}{-2} \right\},
\]
unntatt nullvektoren.
\end{ex}


\section*{Egenrom}

Vi har sett at egenverdiene til en matrise er noen enkeltverdier, mens
egenvektorene er uendelig mange (dersom matrisen har egenverdier og
egenvektorer).  I eksempel~\ref{ex:finne-egenverdier} beskrev vi
egenvektorene tilhørende en gitt egenvektor ved å si «alle vektorer i
(\ldots) unntatt nullvektoren».  Vi innfører nå et nytt begrep som
gjør det litt enklere å beskrive alle egenvektorene til en egenverdi.

\begin{defn}
La $A$ være en $n \times n$-matrise, og anta at $\lambda$ er en
egenverdi for~$A$.  Da er \defterm{egenrommet} til~$\lambda$ mengden
av alle egenvektorer som hører til~$\lambda$, samt nullvektoren; altså
mengden
\[
\{\, \V{v} \in \R^n \mid A \V{v} = \lambda \V{v} \,\}.\qedhere
\]
\end{defn}

% \begin{merk}
% TODO merknad om notasjonen \{ \mid \}
% \end{merk}

\begin{ex}
I eksempel~\ref{ex:finne-egenverdier} kunne vi sagt at egenrommet til
egenvektoren~$3$ er
\[
\Sp \left\{ \vv{3}{2} \right\},
\]
og at egenrommet til egenvektoren~$-5$ er
\[
\Sp \left\{ \vv{1}{-2} \right\}.
\]
Hvert av disse to egenrommene er en linje i planet:
\begin{center}
\begin{tikzpicture}[scale=.42]
\draw[->] (-6.5,0) -- (6.8,0);
\draw[->] (0,-4.5) -- (0,4.8);
\foreach \x in {-6,-5,-4,-3,-2,-1,1,2,3,4,5,6}
\draw (\x,5pt) -- (\x,-5pt);
\foreach \y in {-4,-3,-2,-1,1,2,3,4}
\draw (5pt,\y) -- (-5pt,\y);
\draw (-6,-4) -- (6,4);
\draw (-2,4) -- (2,-4);
\node[anchor=west] at (4,2.5) {\footnotesize egenrommet til~$3$};
\node[anchor=west] at (1.5,-3) {\footnotesize egenrommet til~$-5$};
\end{tikzpicture}
\end{center}
\vspace{-20pt}
\end{ex}

La oss nå ta et litt større eksempel.

\begin{ex}
Vi finner egenverdiene til matrisen
\[
A =
\begin{bmatrix}
 -8 & 0 &  6 \\
 12 & 4 & -6 \\
-20 & 0 & 14
\end{bmatrix},
\]
og de tilhørende egenrommene.

Det karakteristiske polynomet til~$A$ er:
\begin{align*}
\det (A - \lambda I_3)
&=
\begin{vmatrix}
 -8 - \lambda & 0           &  6           \\
 12           & 4 - \lambda & -6           \\
-20           & 0           & 14 - \lambda
\end{vmatrix}
\\
&= (4 - \lambda) \cdot
\begin{vmatrix}
 -8 - \lambda &  6           \\
-20           & 14 - \lambda
\end{vmatrix}
\\
&= (4 - \lambda) \big((-8 - \lambda) (14 - \lambda) + 6 \cdot 20\big)
\\
&= (4 - \lambda) (\lambda^2 - 6\lambda + 8)
\end{align*}
Vi finner altså egenverdiene til~$A$ ved å løse tredjegradslikningen
\[
(4 - \lambda) (\lambda^2 - 6\lambda + 8) = 0.
\]
Denne likningen er ekvivalent med at
\[
4 - \lambda = 0
\qquad\text{eller}\qquad
\lambda^2 - 6\lambda + 8 = 0.
\]
Andregradslikningen $\lambda^2 - 6\lambda + 8 = 0$ har løsninger
\[
\lambda = \frac{6 \pm \sqrt{6^2 - 4 \cdot 8}}{2} = 3 \pm 1,
\]
så vi får to egenverdier: $2$ og~$4$.

Vi finner egenrommene ved å løse likningene
\[
(A - 2 I_3) \V{x} = \V{0}
\qquad\text{og}\qquad
(A - 4 I_3) \V{x} = \V{0}.
\]
Vi tar ikke med all utregningen her, men du bør gjøre den selv.
Resultatet blir at egenrommet til egenverdien~$2$ er
\[
\Sp \left\{ \vvv{3}{-3}{5} \right\},
\]
og egenrommet til egenverdien~$4$ er
\[
\Sp \left\{ \vvv{0}{1}{0}, \vvv{1}{0}{2} \right\}.
\]
Egenrommet til~$2$ er altså en linje i~$\R^3$, mens egenrommet til~$4$
er et plan.
\end{ex}


\section*{Diagonalmatriser}

La oss se på noen eksempler på matriser der det er veldig lett å se
hva egenverdiene er.

\begin{ex}
Har identitetsmatrisen~$I_n$ noen egenverdier?  Vi vet at
\[
I_n \cdot \V{v} = 1 \cdot \V{v}
\]
for alle vektorer~$\V{v}$ i~$\R^n$.  Dermed ser vi at $I_n$ har
tallet~$1$ som egenverdi, med hele~$\R^n$ som det tilhørende
egenrommet.
\end{ex}

\begin{ex}
La $A$ være følgende $2 \times 2$-matrise:
\[
A =
\begin{bmatrix}
9 & 0 \\
0 & 9
\end{bmatrix}
\]
Da har vi at
\[
A \V{v} = 9 \V{v}
\]
for alle $\V{v}$ i~$\R^n$.  Det betyr at $9$ er en egenverdi for~$A$,
og det tilhørende egenrommet er hele~$\R^n$.
\end{ex}

\begin{ex}
La $A$ være følgende $3 \times 3$-matrise:
\[
A =
\begin{bmatrix}
7 &  0 & 0 \\
0 & -3 & 0 \\
0 &  0 & 1
\end{bmatrix}
\]
Da har vi at
\[
A \cdot \vvv{v_1}{v_2}{v_3} = \vvv{7v_1}{-3v_2}{1v_3}
\]
for en vektor $(v_1,v_2,v_3)$ i~$\R^3$.  Da ser vi lett at de tre
enhetsvektorene
\[
\vvv{1}{0}{0},\quad
\vvv{0}{1}{0}\quad\text{og}\quad
\vvv{0}{0}{1}
\]
er egenvektorer, med $7$, $-3$ og~$1$ som tilhørende egenverdier.  Det
er også lett å se at hvis vi har en vektor $(v_1,v_2,v_3)$ der minst
to av komponentene $v_1$, $v_2$ og~$v_3$ ikke er~$0$, så kan den ikke
være en egenvektor, siden komponentene ganges opp med ulike tall når
vi ganger vektoren med~$A$.

Vi ser altså at matrisen har egenverdiene $7$, $-3$ og~$1$, med
\[
\Sp \left\{ \vvv{1}{0}{0} \right\},\quad
\Sp \left\{ \vvv{0}{1}{0} \right\}\quad\text{og}\quad
\Sp \left\{ \vvv{0}{0}{1} \right\}
\]
som tilhørende egenrom.
\end{ex}

I alle disse tre eksemplene hadde vi matriser der de eneste tallene
som ikke er~$0$ er på diagonalen.  Vi gir et navn til slike matriser.

\begin{defn}
En \defterm{diagonalmatrise} er en kvadratisk matrise der alle tall
utenfor diagonalen er~$0$, altså en matrise på følgende form:
\[
\begin{bmatrix}
a_{11} & 0      & 0      & \cdots & 0      \\
0      & a_{22} & 0      & \cdots & 0      \\
0      & 0      & a_{33} & \cdots & 0      \\
\vdots & \vdots & \vdots & \ddots & \vdots \\
0      & 0      & 0      & \vdots & a_{nn}
\end{bmatrix}
\vspace{-17pt}
\]
\end{defn}

På samme måte som i eksemplene over kan vi lett finne egenverdiene til
enhver diagonalmatrise.

\begin{thm}
Egenverdiene til en diagonalmatrise er tallene på diagonalen.
\end{thm}


\section*{Lineær uavhengighet av egenvektorer}

Vi skal nå ta en ganske omfattende diskusjon om hvorvidt egenvektorer
er lineært uavhengige av hverandre, og hva vi kan si om vektorer som
ligger i mengden utspent av noen gitte egenvektorer.

Hele diskusjonen kommer til å bli konkludert med et fint og flott
teorem (teorem~\ref{thm:egenvektorer-lin-uavh}), men la oss ikke se på
konklusjonen helt med en gang.  For å prøve å forstå hvordan man kunne
kommet frem til det teoremet på egen hånd, skal vi bygge oss opp mot
det i små steg.  Vi begynner med å se på én egenvektor, og deretter to
egenvektorer som hører til forskjellige egenverdier.  Vi ser hva vi
kan si om disse, og etter hvert kommer vi til å nærme oss et generelt
resultat.

Gjennom alt det vi skal gjøre nå lar vi $A$ være en
$n \times n$-matrise.

Vi vet fra før (teorem~\ref{thm:egenvektor-sp}) at hvis vi har en
egenvektor~$\V{v}$ som hører til en egenverdi~$\lambda$, så er enhver
vektor i $\Sp \{ \V{v} \}$, utenom nullvektoren, også en egenvektor
som hører til~$\lambda$.  Vi har altså:

\smallskip
\noindent\emph{Hvis $\V{w}$ er en skalar ganger en egenvektor~$\V{v}$,
 og ikke er lik nullvektoren,
 så er $\V{w}$ en egenvektor tilhørende samme egenverdi som~$\V{v}$.}
\smallskip

Dette enkle og greie resultatet skal vi utnytte når vi nå går videre
til å se på flere egenvektorer som hører til forskjellige egenverdier.

\smallskip
La oss se på hva vi kan si hvis vi har to egenvektorer
\[
\V{v}_1\ \text{og}\ \V{v}_2
\]
som hører til to forskjellige egenverdier
\[
\lambda_1\ \text{og}\ \lambda_2.
\]
Vi vet at alle vektorer i $\Sp \{ \V{v}_1 \}$ er egenvektorer som
hører til egenverdien~$\lambda_1$.  Dermed er det klart at $\V{v}_2$
ikke kan være i $\Sp \{ \V{v}_1 \}$, siden $\V{v}_2$ hører til
egenverdien~$\lambda_2$.  På samme måte ser vi at $\V{v}_1$ ikke kan
være i $\Sp \{ \V{v}_2 \}$.  Dermed ser vi (ved å bruke
teorem~\ref{thm:lin-uavh-2}, eller ved å tenke selv) at vektorene
$\V{v}_1$ og~$\V{v}_2$ er lineært uavhengige.  Vi har altså vist:

\smallskip
\noindent\emph{To egenvektorer som hører til forskjellige egenverdier
  må være lineært uavhengige.}
\smallskip

Dette betyr at vi kan se for oss de to vektorene $\V{v}_1$
og~$\V{v}_2$ som to piler som peker i forskjellige retninger:
\begin{center}
\begin{tikzpicture}[scale=.35]
\draw[->] (0,0) -- (3,1) node[anchor=north,inner sep=7pt,pos=.9] {$\V{v}_1$};
\draw[->] (0,0) -- (1,4) node[anchor=east,inner sep=6pt] {$\V{v}_2$};
\end{tikzpicture}
\\
{\small \textit{De to egenvektorene våre}}
\end{center}

La oss nå se hva vi kan si om en vektor $\V{w}$ i
$\Sp \{ \V{v}_1, \V{v}_2 \}$, altså i planet utspent av $\V{v}_1$
og~$\V{v}_2$.  En slik vektor $\V{w}$ er en lineærkombinasjon av
$\V{v}_1$ og~$\V{v}_2$, så det finnes tall $c_1$ og~$c_2$ slik at
\[
\V{w} = c_1 \V{v}_1 + c_2 \V{v}_2.
\]
For det første: Hvis $c_2 = 0$, så er $\V{w}$ bare et tall
ganger~$\V{v}_1$, og da vet vi at den er en egenvektor som hører til
egenverdien~$\lambda_1$.  På samme måte får vi at hvis $c_1 = 0$, så
er $\V{w}$ en egenvektor som hører til egenverdien~$\lambda_2$.  Men
hva kan vi si dersom både $c_1$ og~$c_2$ er forskjellige fra~$0$,
altså dersom $\V{w}$ ikke ligger i $\Sp \{\V{v}_1\}$ eller i
$\Sp \{\V{v}_2\}$?

Vi tegner inn $\V{w}$ på tegningen vår, og får med hvordan den er en
lineærkombinasjon av $\V{v}_1$ og~$\V{v}_2$:
\begin{center}
\begin{tikzpicture}[scale=.35]
\draw[->] (0,0) -- (3,1) node[anchor=north,inner sep=7pt,pos=.9] {$\V{v}_1$};
\draw[->] (0,0) -- (1,4) node[anchor=east,inner sep=6pt] {$\V{v}_2$};
\draw[->] (0,0) -- (4.5,1.5) node[anchor=north,inner sep=7pt,pos=1.1] {$c_1 \V{v}_1$};
\draw[->] (0,0) -- (2,8) node[anchor=east,inner sep=6pt] {$c_2 \V{v}_2$};
\draw[->] (0,0) -- (6.5,9.5) node[anchor=west] {$\V{w}$};
\draw[dashed] (2,8) -- (6.5,9.5);
\draw[dashed] (4.5,1.5) -- (6.5,9.5);
\end{tikzpicture}
\\
{\small \textit{Vektoren~$\V{w}$ er en lineærkombinasjon av $\V{v}_1$ og~$\V{v}_2$}}
\end{center}
Nå vil vi finne ut om $\V{w}$ er en egenvektor eller ikke, altså om
det finnes et tall~$\lambda$ slik at $A \V{w} = \lambda \V{w}$.  Vi
ser på uttrykket $A \V{w}$.  Siden vi vet at $\V{v}_1$ og~$\V{v}_2$ er
egenvektorer tilhørende egenverdiene $\lambda_1$ og~$\lambda_2$, får
vi:
\[
A \V{w}
 = A c_1 \V{v}_1 + A c_2 \V{v}_2
 = \lambda_1 c_1 \V{v}_1 + \lambda_2 c_2 \V{v}_2
\]
Vi tegner inn $A \V{w}$ også på tegningen vår:
\begin{center}
\begin{tikzpicture}[scale=.35]
\draw[->] (0,0) -- (3,1) node[anchor=north,inner sep=7pt,pos=.9] {$\V{v}_1$};
\draw[->] (0,0) -- (1,4) node[anchor=east,inner sep=6pt] {$\V{v}_2$};
\draw[->] (0,0) -- (4.5,1.5) node[anchor=north,inner sep=7pt,pos=1.1] {$c_1 \V{v}_1$};
\draw[->] (0,0) -- (2,8) node[anchor=east,inner sep=6pt] {$c_2 \V{v}_2$};
\draw[->] (0,0) -- (6.5,9.5) node[anchor=west] {$\V{w}$};
\draw[dashed] (2,8) -- (6.5,9.5);
\draw[dashed] (4.5,1.5) -- (6.5,9.5);
\draw[->] (0,0) -- (12,4) node[anchor=north,inner sep=7pt,pos=1.05] {$\lambda_1 c_1 \V{v}_1$};
\draw[->] (0,0) -- (2.5,10) node[anchor=east,inner sep=6pt,pos=1.05] {$\lambda_2 c_2 \V{v}_2$};
\draw[->] (0,0) -- (14.5,14) node[anchor=west,inner sep=6pt,pos=.95] {$A \V{w}$};
\draw[dashed] (12,4) -- (14.5,14);
\draw[dashed] (2.5,10) -- (14.5,14);
\end{tikzpicture}
\\
{\small \textit{Vektoren $A \V{w}$ må peke i en annen\\
 retning enn $\V{w}$ fordi $\lambda_1 \ne \lambda_2$}}
\end{center}
Denne siste figuren viser at $\V{w}$ ikke kan være en egenvektor.  Den
eneste muligheten for å få $A \V{w}$ til å peke i samme retning
som~$\V{w}$ er å anta at $\lambda_1 = \lambda_2$, og vi har jo antatt
akkurat det motsatte, nemlig at $\lambda_1 \ne \lambda_2$.

Denne figuren er det man bør se for seg for å forstå hva som skjer,
men vi kan vise det samme på en mer presis og rent algebraisk måte som
ikke avhenger av figuren.

Hvis vi ganger $\V{w}$ med et tall~$\lambda$, så får vi:
\[
\lambda \V{w} = \lambda c_1 \V{v}_1 + \lambda c_2 \V{v}_2
\]
Og vi så at hvis vi ganger $A$ med~$\V{w}$, så får vi:
\[
A \V{w} = \lambda_1 c_1 \V{v}_1 + \lambda_2 c_2 \V{v}_2
\]
Hvis $\V{w}$ er en egenverdi, så finnes det en $\lambda$ slik at
$A \V{w} = \lambda \V{w}$, og dermed:
\[
\lambda_1 c_1 \V{v}_1 + \lambda_2 c_2 \V{v}_2 = \lambda c_1 \V{v}_1 + \lambda c_2 \V{v}_2
\]
Siden $\V{v}_1$ og~$\V{v}_2$ er lineært uavhengige, betyr dette at
\[
\lambda_1 c_1 = \lambda c_1
\qquad\text{og}\qquad
\lambda_2 c_2 = \lambda c_2.
\]
Men vi har antatt at både $c_1$ og~$c_2$ er forskjellige fra~$0$ og at
$\lambda_1 \ne \lambda_2$, og da er dette umulig.  Det vil si at
$\V{w}$ ikke er en egenvektor.

Nå har vi vist:

\smallskip
\noindent\emph{Hvis $\V{w}$ er en lineærkombinasjon av to egenvektorer som
  hører til forskjellige egenverdier, så er det to muligheter: Enten
  er $\V{w}$ lik en skalar ganger én av de to egenvektorene, eller så
  er $\V{w}$ ikke en egenvektor.}
\smallskip

Nå går vi videre til å se på tre egenvektorer
\[
\V{v}_1,\ \V{v}_2\ \text{og}\ \V{v}_3
\]
som hører til forskjellige egenverdier
\[
\lambda_1,\ \lambda_2\ \text{og}\ \lambda_3.
\]
Da kan vi bruke det vi nettopp viste til å konkludere med at ingen av
disse egenvektorene kan være en lineærkombinasjon av de andre to.  For
hvis $\V{v}_3$ skulle vært i $\Sp \{ \V{v}_1, \V{v}_2 \}$, så ville vi
fått at den enten er nullvektoren, eller en skalar ganger $\V{v}_1$
eller $\V{v}_2$ (men da ville den hatt $\lambda_1$ eller $\lambda_2$
som tilhørende egenverdi istedenfor~$\lambda_3$), eller så ville den
ikke vært en egenvektor i det hele tatt.  Alle disse alternativene er
utelukket, siden vi har antatt at $\V{v}_3$ er en egenvektor
tilhørende egenverdien~$\lambda_3$.

På samme måte får vi at $\V{v}_2$ ikke kan være i
$\Sp \{ \V{v}_1, \V{v}_3 \}$ og at $\V{v}_1$ ikke kan være i
$\Sp \{ \V{v}_2, \V{v}_3 \}$.

Siden vi har vist at ingen av de tre vektorene er en lineærkombinasjon
av de andre to, får vi (fra teorem~\ref{thm:lin-uavh-lin-komb}) at de
er lineært uavhengige.  Vi har altså vist:

\smallskip
\noindent\emph{Tre egenvektorer som hører til forskjellige egenverdier
  må være lineært uavhengige.}
\smallskip

Men vi trenger ikke å gi oss der.  Ved et helt tilsvarende argument
som det vi hadde i tilfellet med to vektorer kan
vi vise:

\smallskip
\noindent\emph{Hvis $\V{w}$ er en lineærkombinasjon av tre egenvektorer som
  hører til forskjellige egenverdier, så er det to muligheter: Enten
  er $\V{w}$ lik en skalar ganger én av de tre egenvektorene, eller så
  er $\V{w}$ ikke en egenvektor.}
\smallskip

Hvis du fremdeles henger med, så ser du antagelig hvordan vi nå kan gå
videre til:

\smallskip
\noindent\emph{Fire egenvektorer som hører til forskjellige egenverdier
  må være lineært uavhengige.}
\smallskip

Herfra kan vi fortsette på akkurat samme måte, og vi får de samme
resultatene for enhver liste av vilkårlig mange egenvektorer som hører
til forskjellige egenverdier.  Da har vi kommet til slutten på vår
lange diskusjon om lineær uavhengighet for egenvektorer, og vi
oppsummerer det vi har funnet ut i et teorem.

\begin{thm}
\label{thm:egenvektorer-lin-uavh}
La $\lambda_1$, $\lambda_2$, \ldots, $\lambda_t$ være
forskjellige egenverdier for en matrise~$A$, og la $\V{v}_1$,
$\V{v}_2$, \ldots, $\V{v}_t$ være egenvektorer som hører til hver av
disse egenverdiene.  Da har vi:
\begin{enumerate}
\item[(a)] Vektorene $\V{v}_1$, $\V{v}_2$, \ldots, $\V{v}_t$ er lineært uavhengige.
\item[(b)] I mengden
\[
\Sp \{ \V{v}_1,\ \V{v}_2,\ \cdots,\ \V{v}_t \}
\]
utspent av egenvektorene vi ser på finnes det ingen andre egenvektorer
enn de som er et multippel $c \cdot \V{v}_i$ av én av vektorene i
listen.
\end{enumerate}
\end{thm}

\section*{Markovkjeder}

% TODO markovkjeder!


\kapittelslutt
