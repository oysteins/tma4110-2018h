\ifx\inkludert\undefined
\documentclass[norsk,a4paper,twocolumn,oneside]{memoir}

\usepackage[utf8]{inputenc}
\usepackage{babel}
\usepackage{amsmath,amssymb,amsthm}
\usepackage[total={17cm,27cm}]{geometry}
\usepackage[table]{xcolor}
%\usepackage{tabularx}
\usepackage{systeme}
%\usepackage{hyperref}
%\usepackage{enumerate}

%\usepackage{sectsty}
\setsecheadstyle{\bfseries\large}
%\subsectionfont{\bf\normalsize}

\usepackage{tikz}
\usetikzlibrary{arrows.meta}

\newcommand{\defterm}[1]{\emph{#1}}

\newcommand{\N}{\mathbb{N}}
\newcommand{\Z}{\mathbb{Z}}
\newcommand{\Q}{\mathbb{Q}}
\newcommand{\R}{\mathbb{R}}

\newcommand{\abs}[1]{|#1|}

\newcommand{\roweq}{\sim}
\DeclareMathOperator{\Span}{Span}

\newcommand{\V}[1]{\mathbf{#1}}
\newcommand{\vv}[2]{\begin{bmatrix} #1 \\ #2 \end{bmatrix}}
\newcommand{\vvv}[3]{\begin{bmatrix} #1 \\ #2 \\ #3 \end{bmatrix}}
\newcommand{\vvvv}[4]{\begin{bmatrix} #1 \\ #2 \\ #3 \\ #4 \end{bmatrix}}
\newcommand{\vn}[2]{\vvvv{#1_1}{#1_2}{\vdots}{#1_#2}}

\newenvironment{amatrix}[1]{% "augmented matrix"
  \left[\begin{array}{*{#1}{c}|c}
}{%
  \end{array}\right]
}

% \newcounter{notatnr}
% \newcommand{\notatnr}[2]
% {\setcounter{notatnr}{#1}%
%  \setcounter{page}{#2}%
% }

\newtheorem{thm}{Teorem}[chapter]
\newtheorem*{thm-nn}{Teorem}
\newtheorem{cor}[thm]{Korollar}
\newtheorem{lem}[thm]{Lemma}
\newtheorem{prop}[thm]{Proposisjon}
\theoremstyle{definition}
\newtheorem{exx}[thm]{Eksempel}
\newtheorem*{defnx}{Definisjon}
\newtheorem*{oppg}{Oppgave}
\newtheorem*{merkx}{Merk}
\newtheorem*{spmx}{Spørsmål}

\newenvironment{defn}
  {\pushQED{\qed}\renewcommand{\qedsymbol}{$\triangle$}\defnx}
  {\popQED\enddefnx}
\newenvironment{ex}
  {\pushQED{\qed}\renewcommand{\qedsymbol}{$\triangle$}\exx}
  {\popQED\endexx}
\newenvironment{merk}
  {\pushQED{\qed}\renewcommand{\qedsymbol}{$\triangle$}\merkx}
  {\popQED\endmerkx}
\newenvironment{spm}
  {\pushQED{\qed}\renewcommand{\qedsymbol}{$\triangle$}\spmx}
  {\popQED\endspmx}

\setlength{\columnsep}{26pt}

\newcommand{\Tittel}[2]{%
\twocolumn[
\begin{center}
\Large
\begin{tabularx}{\textwidth}{cXr}
\cellcolor{black}\color{white}%
\bf {#1} &
#2
\hfill &
\footnotesize TMA4110 høsten 2018
\\ \hline
\end{tabularx}
\end{center}
]}

\newcommand{\tittel}[1]{\Tittel{\arabic{notatnr}}{#1}}

\newcommand{\linje}{%
\begin{center}
\rule{.8\linewidth}{0.4pt}
\end{center}
}


\newcommand{\chapternumber}{}

\makechapterstyle{tma4110}{%
 \renewcommand*{\chapterheadstart}{}
 \renewcommand*{\printchaptername}{}
 \renewcommand*{\chapternamenum}{}
 \renewcommand*{\printchapternum}{\renewcommand{\chapternumber}{\thechapter}}
 \renewcommand*{\afterchapternum}{}
 \renewcommand*{\printchapternonum}{\renewcommand{\chapternumber}{}}
 \renewcommand*{\printchaptertitle}[1]{
\LARGE
\begin{tabularx}{\textwidth}{cXr}
\cellcolor{black}\color{white}%
\textbf{\chapternumber} &
\textbf{##1}
\hfill &
%\footnotesize TMA4110 høsten 2018
\\ \hline
\end{tabularx}%
}
 \renewcommand*{\afterchaptertitle}{\par\nobreak\vskip \afterchapskip}
 % \newcommand{\chapnamefont}{\normalfont\huge\bfseries}
 % \newcommand{\chapnumfont}{\normalfont\huge\bfseries}
 % \newcommand{\chaptitlefont}{\normalfont\Huge\bfseries}
 \setlength{\beforechapskip}{0pt}
 \setlength{\midchapskip}{0pt}
 \setlength{\afterchapskip}{10pt}
}


\newcounter{oppgnr}[chapter]
\newcounter{punktnr}[oppgnr]
\newenvironment{oppgave}
 {\par\noindent\stepcounter{oppgnr}\textbf{{\arabic{oppgnr}}.}}
 {\par\bigskip}
\newenvironment{punkt}
 {\par\smallskip\noindent\stepcounter{punktnr}\textbf{\alph{punktnr})} }
 {\par}

\newcommand{\oppgaver}{\linje\section*{Oppgaver}}

\usepackage{xr}
\externaldocument{tma4110-2018h}
\newcommand{\kapittel}[2]{\setcounter{chapter}{#1}\addtocounter{chapter}{-1}\chapter{#2}}
\newcommand{\kapittelslutt}{\enddocument}
\begin{document}
\chapterstyle{tma4110}
\pagestyle{plain}
\fi


\kapittel{13}{Diagonalisering}
\label{ch:diagonalisering}
\section*{Distinkte egenverdier}
La $A$ være en $n \times n$-matrise med egenverdier $\lambda_1$, $\lambda_2$,$\cdots$, $\lambda_n$, og tilhørende egenvektorer $\V{v}_1$, $\V{v}_2\dots \V{v}_n$. For hvert par av egenverdier og egenvektorer gjelder
\[
A\V{v}_k=\lambda \V{v}_k.
\]
Disse $n$ ligningene kan like gjerne organiseres i en matriseligning
\[
AP=PD, 
\]
der 
\[
D=
\begin{bmatrix}
\lambda_1      & 0      & 0      & \cdots & 0 \\
0      & \lambda_2      & 0      & \cdots & 0 \\
0      & 0      & \lambda_3      & \cdots & 0 \\
\vdots & \vdots & \vdots & \ddots & \vdots \\
0      & 0      & 0      & \cdots & \lambda_n
\end{bmatrix}
\]
og 
\[
P=
\begin{bmatrix}
\V{v}_1 & \V{v}_2 & \cdots & \V{v}_n
\end{bmatrix}.
\]
Fra teorem HEIHEIHEI vet vi at $\V{v}_1$, $\V{v}_2\dots \V{v}_n$ er lineært uavhengige, og dermed at matrisen $P$ har en invers. Vi ganger med $P^{-1}$ fra venstre og får
\[
P^{-1}AP=D.
\]
Denne operasjonen kalles å \emph{diagonalisere} $A$. 

\begin{ex}	
Vi diagonaliserer matrisen 
\[
A=
\begin{bmatrix}
2     & 1 \\
1      & 2
\end{bmatrix}.
\]
\end{ex}

\begin{thm}
En $n \times n$-matrise $A$ kan diagonaliseres dersom den har $n$ lineært uavhengige egenvektorer.
\end{thm}

\section*{Ikke distinkte egenverdier}
Teorem HIHIHI har en slags triviell aura over seg. Så klart man kan diagonlisere en matrise dersom den har $n$ lineært uavhengige egenvektorer - det er jo bare å stille opp likningene for egenvektorene i matriseform. Dersom alle egenverdiene er forskjellige, finnes det automatisk $n$ lineært uavhengige egenvektor. Det er først når vi har doble egenverdier det blir interessant.

\begin{ex}
La oss finne den doble egenverdien til matrisen
\end{ex}


\section*{Fullstendig opptelling av alle $2 \times 2$-projeksjonsmatriser}

\section*{Symmetriske matriser}
Dersom matrisen er symmetrisk, går det an å vise at ingen 
\begin{ex}
La oss prøve å finne egenverdiene til denne matrisen hvis karakteristiske polynom ikke har noen røtter
\end{ex}


\kapittelslutt
