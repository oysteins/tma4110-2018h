\ifx\inkludert\undefined
\documentclass[norsk,a4paper,twocolumn,oneside]{memoir}

\usepackage[utf8]{inputenc}
\usepackage{babel}
\usepackage{amsmath,amssymb,amsthm}
\usepackage{mathrsfs}
\usepackage[total={17cm,27cm}]{geometry}
\usepackage[table]{xcolor}
%\usepackage{tabularx}
\usepackage{systeme}
%\usepackage{hyperref}
%\usepackage{enumerate}
\usepackage{ifthen}
\usepackage{textgreek}
\usepackage{multirow}
\usepackage{placeins}

%\usepackage{sectsty}
\setsecheadstyle{\bfseries\large}
%\subsectionfont{\bf\normalsize}

\usepackage{tikz,pgfplots}
\usetikzlibrary{calc}
\usetikzlibrary{arrows.meta}
\def\centerarc[#1](#2)(#3:#4:#5)% Syntax: [draw options] (center) (initial angle:final angle:radius)
    { \draw[#1] ($(#2)+({#5*cos(#3)},{#5*sin(#3)})$) arc (#3:#4:#5); }
\usepackage{pgfornament}

\newcommand{\defterm}[1]{\emph{#1}}

\newcommand{\N}{\mathbb{N}}
\newcommand{\Z}{\mathbb{Z}}
\newcommand{\Q}{\mathbb{Q}}
\newcommand{\R}{\mathbb{R}}

\newcommand{\M}{\mathcal{M}} % vektorrom av matriser
\newcommand{\C}{\mathcal{C}} % vektorrom av kontinuerlige funksjoner
\renewcommand{\P}{\mathcal{P}} % vektorrom av polynomer
\newcommand{\B}{\mathscr{B}} % basis

\renewcommand{\Im}{\operatorname{Im}}
\renewcommand{\Re}{\operatorname{Re}}

\newcommand{\abs}[1]{|#1|}
\newcommand{\intersect}{\cap}
\newcommand{\union}{\cup}
\newcommand{\fcomp}{\circ}
\newcommand{\iso}{\cong}

\newcommand{\roweq}{\sim}
\DeclareMathOperator{\Sp}{Sp}
\DeclareMathOperator{\Null}{Null}
\DeclareMathOperator{\Col}{Col}
\DeclareMathOperator{\Row}{Row}
\DeclareMathOperator{\rank}{rank}
\DeclareMathOperator{\im}{im}
\DeclareMathOperator{\id}{id}
\DeclareMathOperator{\Hom}{Hom}
\newcommand{\tr}{^\top}
\newcommand{\koord}[2]{[\,{#1}\,]_{#2}} % koordinater mhp basis

\newcommand{\V}[1]{\mathbf{#1}}
\newcommand{\vv}[2]{\begin{bmatrix} #1 \\ #2 \end{bmatrix}}
\newcommand{\vvS}[2]{\left[ \begin{smallmatrix} #1 \\ #2 \end{smallmatrix} \right]}
\newcommand{\vvv}[3]{\begin{bmatrix} #1 \\ #2 \\ #3 \end{bmatrix}}
\newcommand{\vvvv}[4]{\begin{bmatrix} #1 \\ #2 \\ #3 \\ #4 \end{bmatrix}}
\newcommand{\vvvvv}[5]{\begin{bmatrix} #1 \\ #2 \\ #3 \\ #4 \\ #5 \end{bmatrix}}
\newcommand{\vn}[2]{\vvvv{#1_1}{#1_2}{\vdots}{#1_#2}}

\newcommand{\e}{\V{e}}
\renewcommand{\u}{\V{u}}
\renewcommand{\v}{\V{v}}
\newcommand{\w}{\V{w}}
\renewcommand{\b}{\V{b}}
\newcommand{\x}{\V{x}}
\newcommand{\0}{\V{0}}

\newenvironment{amatrix}[1]{% "augmented matrix"
  \left[\begin{array}{*{#1}{c}|c}
}{%
  \end{array}\right]
}

\newcommand{\boks}[1]{\framebox{\strut $#1$}}

% \newcounter{notatnr}
% \newcommand{\notatnr}[2]
% {\setcounter{notatnr}{#1}%
%  \setcounter{page}{#2}%
% }

\newtheorem{thm}{Teorem}[chapter]
\newtheorem*{thm-nn}{Teorem}
\newtheorem{cor}[thm]{Korollar}
\newtheorem{lem}[thm]{Lemma}
\newtheorem{prop}[thm]{Proposisjon}
\theoremstyle{definition}
\newtheorem{exx}[thm]{Eksempel}
\newtheorem*{defnx}{Definisjon}
\newtheorem*{oppg}{Oppgave}
\newtheorem*{merkx}{Merk}
\newtheorem*{spmx}{Spørsmål}

\newenvironment{defn}
  {\pushQED{\qed}\renewcommand{\qedsymbol}{$\triangle$}\defnx}
  {\popQED\enddefnx}
\newenvironment{ex}
  {\pushQED{\qed}\renewcommand{\qedsymbol}{$\triangle$}\exx}
  {\popQED\endexx}
\newenvironment{merk}
  {\pushQED{\qed}\renewcommand{\qedsymbol}{$\triangle$}\merkx}
  {\popQED\endmerkx}
\newenvironment{spm}
  {\pushQED{\qed}\renewcommand{\qedsymbol}{$\triangle$}\spmx}
  {\popQED\endspmx}

\setlength{\columnsep}{26pt}

\newcommand{\Tittel}[2]{%
\twocolumn[
\begin{center}
\Large
\begin{tabularx}{\textwidth}{cXr}
\cellcolor{black}\color{white}%
\bf {#1} &
#2
\hfill &
\footnotesize TMA4110 høsten 2018
\\ \hline
\end{tabularx}
\end{center}
]}

\newcommand{\tittel}[1]{\Tittel{\arabic{notatnr}}{#1}}

\newcommand{\linje}{%
\begin{center}
\rule{.8\linewidth}{0.4pt}
\end{center}
}


\newcommand{\kapittelemnenavn}{TMA4110 høsten 2018}
\newcommand{\chapternumber}{}

\makechapterstyle{tma4110}{%
 \renewcommand*{\chapterheadstart}{}
 \renewcommand*{\printchaptername}{}
 \renewcommand*{\chapternamenum}{}
 \renewcommand*{\printchapternum}{\renewcommand{\chapternumber}{\thechapter}}
 \renewcommand*{\afterchapternum}{}
 \renewcommand*{\printchapternonum}{\renewcommand{\chapternumber}{}}
 \renewcommand*{\printchaptertitle}[1]{
\LARGE
\begin{tabularx}{\textwidth}{cXr}
\cellcolor{black}\color{white}%
\textbf{\chapternumber} &
\textbf{##1}
\hfill &
\footnotesize\kapittelemnenavn
\\ \hline
\end{tabularx}%
}
 \renewcommand*{\afterchaptertitle}{\par\nobreak\vskip \afterchapskip}
 % \newcommand{\chapnamefont}{\normalfont\huge\bfseries}
 % \newcommand{\chapnumfont}{\normalfont\huge\bfseries}
 % \newcommand{\chaptitlefont}{\normalfont\Huge\bfseries}
 \setlength{\beforechapskip}{0pt}
 \setlength{\midchapskip}{0pt}
 \setlength{\afterchapskip}{10pt}
}
\chapterstyle{tma4110}
\pagestyle{plain}


\newboolean{vis-oppgaver}
\newboolean{vis-losninger}
\setboolean{vis-oppgaver}{true}
\setboolean{vis-losninger}{false}

\newcounter{oppg-kap} % kapittelnummerering for oppgaver
\newcounter{oppgnr}[oppg-kap]
\newcounter{punktnr}[oppgnr]

\newenvironment{oppgave}%
 {\ifthenelse{\boolean{vis-oppgaver}}%
             {\par\noindent\stepcounter{oppgnr}\textbf{\arabic{oppgnr}.}}%
             {\expandafter\comment}}%
 {\ifthenelse{\boolean{vis-oppgaver}}%
             {\par\bigskip}%
             {\expandafter\endcomment}}

\newenvironment{losning}%
 {\ifthenelse{\boolean{vis-losninger}}%
             {\par\noindent\stepcounter{oppgnr}\textbf{\arabic{oppg-kap}.\arabic{oppgnr}.}}%
             {\expandafter\comment}}%
 {\ifthenelse{\boolean{vis-losninger}}%
             {\par\bigskip}%
             {\expandafter\endcomment}}

\newenvironment{punkt}
 {\par\smallskip\noindent\stepcounter{punktnr}\textbf{\alph{punktnr})} }
 {\par}

\newcommand{\kap}[1]{\setcounter{oppg-kap}{#1}\addtocounter{oppg-kap}{-1}\stepcounter{oppg-kap}}

\newcommand{\oppgaver}[1]{%
  \kap{#1}%
  \ifthenelse{\boolean{vis-oppgaver}}%
             {\linje\section*{Oppgaver}}%
             {}}

\usepackage{xr}
\externaldocument{tma4110-2018h}
\newcommand{\kapittel}[2]{\setcounter{chapter}{#1}\addtocounter{chapter}{-1}\chapter{#2}}
\newcommand{\kapittelslutt}{\enddocument}
\begin{document}
\chapterstyle{tma4110}
\pagestyle{plain}
\fi


\kapittel{6}{Determinanter}
\label{ch:determinanter}

En matrise inneholder mange tall og dermed mye informasjon -- så mye
at det kan være litt overveldende.  Vi kan kondensere ned all
informasjonen i en kvadratisk matrise til ett enkelt tall som kalles
\emph{determinanten} til matrisen.  Dette ene tallet sier oss en hel
del om hvordan matrisen oppfører seg.

Vi har to forskjellige notasjoner for determinanter.
Hvis
\[
A =
\begin{bmatrix}
a_{11} & a_{12} & \cdots & a_{1n} \\
a_{21} & a_{22} & \cdots & a_{2n} \\
\vdots & \vdots & \ddots & \vdots \\
a_{n1} & a_{n2} & \cdots & a_{nn}
\end{bmatrix}
\]
er en $n \times n$-matrise, så skriver vi enten
\[
\det A
\]
eller
\[
\begin{vmatrix}
a_{11} & a_{12} & \cdots & a_{1n} \\
a_{21} & a_{22} & \cdots & a_{2n} \\
\vdots & \vdots & \ddots & \vdots \\
a_{n1} & a_{n2} & \cdots & a_{nn}
\end{vmatrix}
\]
for determinanten til~$A$.%  De to notasjonene betyr akkurat det samme.

%TODO mer intro


\section*{Determinanter for $2 \times 2$-matriser}

For en $2 \times 2$-matrise
\[
A =
\begin{bmatrix}
a & b \\
c & d
\end{bmatrix}
\]
er determinanten definert ved:
\[
\det A =
\begin{vmatrix}
a & b \\
c & d
\end{vmatrix}
= ad - bc
\]
Determinanten har en fin geometrisk tolkning.  Vi ser på de to
kolonnene i~$A$ som vektorer i~$\R^2$, tegner dem som piler i planet,
og lager et parallellogram med disse som to av sidene.  Dette
paralellogrammet kan vi kalle parallellogrammet utspent av de to
vektorene.
\begin{center}
\begin{tikzpicture}[scale=.7]
\draw[->] (-1,0) -- (10,0);
\draw[->] (0,-1) -- (0,6);
\draw (2,2pt) -- (2,-6pt) node[anchor=north] {$b$};
\draw (7,0pt) -- (7,-6pt) node[anchor=north] {$a$};
%\draw (9,0pt) -- (9,-6pt) node[anchor=north] {$a + b$};
\draw (2pt,1.5) -- (-6pt,1.5) node[anchor=east] {$c$};
\draw (0,4) -- (-6pt,4) node[anchor=east] {$d$};
%\draw (0,5.5) -- (-6pt,5.5) node[anchor=east] {$c + d$};
\draw[->] (0,0) -- (7,1.5);
\draw[->] (0,0) -- (2,4);
\draw[dashed] (7,1.5) -- (9,5.5);
\draw[dashed] (2,4) -- (9,5.5);
\node[anchor=west] at (7.2,1.3) {$\vv{a}{c}$};
\node[anchor=south] at (1.8,4) {$\vv{b}{d}$};
\end{tikzpicture}
\\
{\small \textit{Parallellogrammet utspent av kolonnene i~$A$}}
\end{center}

La oss beregne arealet av dette parallellogrammet.  Vi tegner på noen
hjelpelinjer:
\begin{center}
\begin{tikzpicture}[scale=.7]
\draw[->] (-1,0) -- (10,0);
\draw[->] (0,-1) -- (0,6);
\draw (2,2pt) -- (2,-6pt) node[anchor=north] {$b$};
\draw (7,0pt) -- (7,-6pt) node[anchor=north] {$a$};
\draw (9,0pt) -- (9,-6pt) node[anchor=north] {$a + b$};
\draw (2pt,1.5) -- (-6pt,1.5) node[anchor=east] {$c$};
\draw (0,4) -- (-6pt,4) node[anchor=east] {$d$};
\draw (0,5.5) -- (-6pt,5.5) node[anchor=east] {$c + d$};
\draw[->] (0,0) -- (7,1.5);
\draw[->] (0,0) -- (2,4);
\draw (7,1.5) -- (9,5.5);
\draw (2,4) -- (9,5.5);
\draw (0,5.5) -- (9,5.5);
\draw (9,0) -- (9,5.5);
\draw (0,4) -- (2,4);
\draw (2,5.5) -- (2,4);
\draw (7,0) -- (7,1.5);
\draw (7,1.5) -- (9,1.5);
\end{tikzpicture}
\\
{\small \textit{Vi beregner arealet av parallellogrammet}}
\end{center}
Vi kan finne arealet av parallellogrammet ved å starte med arealet av
det store rektangelet, som er
\[
(a + b) (c + d),
\]
og trekke fra arealene av de to små rektanglene og de fire trekantene
som omgir parallellogrammet.  Vi ser at hvert av de små rektanglene
har areal~$bc$, at de to trekantene øverst og nederst til sammen har
areal~$ac$, og at trekantene til venstre og høyre til sammen har
areal~$bd$.  Det betyr at arealet av parallellogrammet er:
\begin{align*}
& \!\!\!\!\!\!\!\!(a + b) (c + d) - 2bc - ac - bd \\
&= ac + ad + bc + bd - 2bc - ac - bd \\
&= ad - bc \\
&= \det A
\end{align*}
Vi kommer altså frem til at arealet av parallellogrammet utspent av
kolonnene i~$A$ er lik determinanten til~$A$.  Denne utregningen var
imidlertid litt avhengig av hvordan disse to kolonnevektorene er
plassert i forhold til hverandre i planet.  Hvis vi hadde byttet plass
på kolonnene, så ville vi isteden fått $bc - ad$ som areal.  Da ville
altså determinanten vært negativ.

Det som holder i alle tilfeller, er at arealet av parallellogrammet
utspent av kolonnene i~$A$ er lik absoluttverdien til determinanten:
\[
\abs{\det A}
\]

\medskip

Så determinanten gir oss arealet til et parallellogram.  Men hva
forteller det oss om matrisen~$A$?

Vi kan tenke på~$A$ som en transformasjon av planet der hver
vektor~$\V{x}$ i~$\R^2$ sendes til vektoren~$A \V{x}$.  Determinanten
sier noe om hvordan planet endres under denne transformasjonen.

La
\[
\V{e}_1 = \vv{1}{0}
\qquad\text{og}\qquad
\V{e}_2 = \vv{0}{1}
\]
være de to enhetsvektorene i~$\R^2$, og se på kvadratet utspent av
disse:
\begin{center}
\begin{tikzpicture}[scale=2]
\draw[->] (-.6,0) -- (1.6,0);
\draw[->] (0,-.2) -- (0,1.2);
\draw[->,thick] (0,0) -- (1,0);
\draw[->,thick] (0,0) -- (0,1);
\node[anchor=north] at (1,0) {$\V{e}_1$};
\node[anchor=east] at (0,1) {$\V{e}_2$};
\draw (1,0) -- (1,1) -- (0,1);
\end{tikzpicture}
\\
{\small \textit{Enhetskvadratet}}
\end{center}


\section*{Determinanter for $3 \times 3$-matriser}



\section*{Den generelle definisjonen}



\section*{Kofaktorekspansjon}



\section*{Determinanter og radoperasjoner}



\section*{Triangulære matriser}



\section*{Flere regneregler}



\section*{Karakterisering av inverterbarhet}


\kapittelslutt

