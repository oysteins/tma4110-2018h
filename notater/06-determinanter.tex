\ifx\inkludert\undefined
\documentclass[norsk,a4paper,twocolumn,oneside]{memoir}

\usepackage[utf8]{inputenc}
\usepackage{babel}
\usepackage{amsmath,amssymb,amsthm}
\usepackage[total={17cm,27cm}]{geometry}
\usepackage[table]{xcolor}
%\usepackage{tabularx}
\usepackage{systeme}
%\usepackage{hyperref}
%\usepackage{enumerate}

%\usepackage{sectsty}
\setsecheadstyle{\bfseries\large}
%\subsectionfont{\bf\normalsize}

\usepackage{tikz}
\usetikzlibrary{arrows.meta}

\newcommand{\defterm}[1]{\emph{#1}}

\newcommand{\N}{\mathbb{N}}
\newcommand{\Z}{\mathbb{Z}}
\newcommand{\Q}{\mathbb{Q}}
\newcommand{\R}{\mathbb{R}}

\newcommand{\abs}[1]{|#1|}

\newcommand{\roweq}{\sim}
\DeclareMathOperator{\Span}{Span}

\newcommand{\V}[1]{\mathbf{#1}}
\newcommand{\vv}[2]{\begin{bmatrix} #1 \\ #2 \end{bmatrix}}
\newcommand{\vvv}[3]{\begin{bmatrix} #1 \\ #2 \\ #3 \end{bmatrix}}
\newcommand{\vvvv}[4]{\begin{bmatrix} #1 \\ #2 \\ #3 \\ #4 \end{bmatrix}}
\newcommand{\vn}[2]{\vvvv{#1_1}{#1_2}{\vdots}{#1_#2}}

\newenvironment{amatrix}[1]{% "augmented matrix"
  \left[\begin{array}{*{#1}{c}|c}
}{%
  \end{array}\right]
}

% \newcounter{notatnr}
% \newcommand{\notatnr}[2]
% {\setcounter{notatnr}{#1}%
%  \setcounter{page}{#2}%
% }

\newtheorem{thm}{Teorem}[chapter]
\newtheorem*{thm-nn}{Teorem}
\newtheorem{cor}[thm]{Korollar}
\newtheorem{lem}[thm]{Lemma}
\newtheorem{prop}[thm]{Proposisjon}
\theoremstyle{definition}
\newtheorem{exx}[thm]{Eksempel}
\newtheorem*{defnx}{Definisjon}
\newtheorem*{oppg}{Oppgave}
\newtheorem*{merkx}{Merk}
\newtheorem*{spmx}{Spørsmål}

\newenvironment{defn}
  {\pushQED{\qed}\renewcommand{\qedsymbol}{$\triangle$}\defnx}
  {\popQED\enddefnx}
\newenvironment{ex}
  {\pushQED{\qed}\renewcommand{\qedsymbol}{$\triangle$}\exx}
  {\popQED\endexx}
\newenvironment{merk}
  {\pushQED{\qed}\renewcommand{\qedsymbol}{$\triangle$}\merkx}
  {\popQED\endmerkx}
\newenvironment{spm}
  {\pushQED{\qed}\renewcommand{\qedsymbol}{$\triangle$}\spmx}
  {\popQED\endspmx}

\setlength{\columnsep}{26pt}

\newcommand{\Tittel}[2]{%
\twocolumn[
\begin{center}
\Large
\begin{tabularx}{\textwidth}{cXr}
\cellcolor{black}\color{white}%
\bf {#1} &
#2
\hfill &
\footnotesize TMA4110 høsten 2018
\\ \hline
\end{tabularx}
\end{center}
]}

\newcommand{\tittel}[1]{\Tittel{\arabic{notatnr}}{#1}}

\newcommand{\linje}{%
\begin{center}
\rule{.8\linewidth}{0.4pt}
\end{center}
}


\newcommand{\chapternumber}{}

\makechapterstyle{tma4110}{%
 \renewcommand*{\chapterheadstart}{}
 \renewcommand*{\printchaptername}{}
 \renewcommand*{\chapternamenum}{}
 \renewcommand*{\printchapternum}{\renewcommand{\chapternumber}{\thechapter}}
 \renewcommand*{\afterchapternum}{}
 \renewcommand*{\printchapternonum}{\renewcommand{\chapternumber}{}}
 \renewcommand*{\printchaptertitle}[1]{
\LARGE
\begin{tabularx}{\textwidth}{cXr}
\cellcolor{black}\color{white}%
\textbf{\chapternumber} &
\textbf{##1}
\hfill &
%\footnotesize TMA4110 høsten 2018
\\ \hline
\end{tabularx}%
}
 \renewcommand*{\afterchaptertitle}{\par\nobreak\vskip \afterchapskip}
 % \newcommand{\chapnamefont}{\normalfont\huge\bfseries}
 % \newcommand{\chapnumfont}{\normalfont\huge\bfseries}
 % \newcommand{\chaptitlefont}{\normalfont\Huge\bfseries}
 \setlength{\beforechapskip}{0pt}
 \setlength{\midchapskip}{0pt}
 \setlength{\afterchapskip}{10pt}
}


\newcounter{oppgnr}[chapter]
\newcounter{punktnr}[oppgnr]
\newenvironment{oppgave}
 {\par\noindent\stepcounter{oppgnr}\textbf{{\arabic{oppgnr}}.}}
 {\par\bigskip}
\newenvironment{punkt}
 {\par\smallskip\noindent\stepcounter{punktnr}\textbf{\alph{punktnr})} }
 {\par}

\newcommand{\oppgaver}{\linje\section*{Oppgaver}}

\usepackage{xr}
\externaldocument{tma4110-2018h}
\newcommand{\kapittel}[2]{\setcounter{chapter}{#1}\addtocounter{chapter}{-1}\chapter{#2}}
\newcommand{\kapittelslutt}{\enddocument}
\begin{document}
\chapterstyle{tma4110}
\pagestyle{plain}
\fi


\kapittel{6}{Determinanter}
\label{ch:determinanter}

En matrise inneholder mange tall og dermed mye informasjon -- så mye
at det kan være litt overveldende.  Vi kan kondensere ned all
informasjonen i en kvadratisk matrise til ett enkelt tall som kalles
\emph{determinanten} til matrisen.  Dette ene tallet sier oss en hel
del om hvordan matrisen oppfører seg.

Vi har to forskjellige notasjoner for determinanter.
Hvis
\[
A =
\begin{bmatrix}
a_{11} & a_{12} & \cdots & a_{1n} \\
a_{21} & a_{22} & \cdots & a_{2n} \\
\vdots & \vdots & \ddots & \vdots \\
a_{n1} & a_{n2} & \cdots & a_{nn}
\end{bmatrix}
\]
er en $n \times n$-matrise, så skriver vi enten
\[
\det A
\qquad\text{eller}\qquad
\begin{vmatrix}
a_{11} & a_{12} & \cdots & a_{1n} \\
a_{21} & a_{22} & \cdots & a_{2n} \\
\vdots & \vdots & \ddots & \vdots \\
a_{n1} & a_{n2} & \cdots & a_{nn}
\end{vmatrix}
\]
for determinanten til~$A$.  De to notasjonene betyr akkurat det samme.
Den første er praktisk å bruke når vi har en variabel som står for
matrisen vi vil ta determinanten av; den andre når vi vil liste opp
alle tallene i matrisen.

%TODO mer intro


\section*{Determinanter for $2 \times 2$-matriser}

For en $2 \times 2$-matrise
\[
A =
\begin{bmatrix}
a & b \\
c & d
\end{bmatrix}
\]
er determinanten definert ved:
\[
\det A =
\begin{vmatrix}
a & b \\
c & d
\end{vmatrix}
= ad - bc
\]
Determinanten har en fin geometrisk tolkning.  Vi ser på de to
kolonnene i~$A$ som vektorer i~$\R^2$, tegner dem som piler i planet,
og lager et parallellogram med disse som to av sidene.  Dette
paralellogrammet kan vi kalle parallellogrammet utspent av de to
vektorene.
\begin{center}
\begin{tikzpicture}[scale=.7]
\draw[->] (-1,0) -- (10,0);
\draw[->] (0,-1) -- (0,6);
\draw (2,2pt) -- (2,-6pt) node[anchor=north] {$b$};
\draw (7,0pt) -- (7,-6pt) node[anchor=north] {$a$};
%\draw (9,0pt) -- (9,-6pt) node[anchor=north] {$a + b$};
\draw (2pt,1.5) -- (-6pt,1.5) node[anchor=east] {$c$};
\draw (0,4) -- (-6pt,4) node[anchor=east] {$d$};
%\draw (0,5.5) -- (-6pt,5.5) node[anchor=east] {$c + d$};
\draw[->] (0,0) -- (7,1.5);
\draw[->] (0,0) -- (2,4);
\draw[dashed] (7,1.5) -- (9,5.5);
\draw[dashed] (2,4) -- (9,5.5);
\node[anchor=west] at (7.2,1.3) {$\vv{a}{c}$};
\node[anchor=south] at (1.8,4) {$\vv{b}{d}$};
\end{tikzpicture}
\\
{\small \textit{Parallellogrammet utspent av kolonnene i~$A$}}
\end{center}

La oss beregne arealet av dette parallellogrammet.  Vi tegner på noen
hjelpelinjer:
\begin{center}
\begin{tikzpicture}[scale=.6]
\draw[->] (-1,0) -- (10,0);
\draw[->] (0,-1) -- (0,6);
\draw (2,2pt) -- (2,-6pt) node[anchor=north] {$b$};
\draw (7,0pt) -- (7,-6pt) node[anchor=north] {$a$};
\draw (9,0pt) -- (9,-6pt) node[anchor=north] {$a + b$};
\draw (2pt,1.5) -- (-6pt,1.5) node[anchor=east] {$c$};
\draw (0,4) -- (-6pt,4) node[anchor=east] {$d$};
\draw (0,5.5) -- (-6pt,5.5) node[anchor=east] {$c + d$};
\draw[->] (0,0) -- (7,1.5);
\draw[->] (0,0) -- (2,4);
\draw (7,1.5) -- (9,5.5);
\draw (2,4) -- (9,5.5);
\draw (0,5.5) -- (9,5.5);
\draw (9,0) -- (9,5.5);
\draw (0,4) -- (2,4);
\draw (2,5.5) -- (2,4);
\draw (7,0) -- (7,1.5);
\draw (7,1.5) -- (9,1.5);
\end{tikzpicture}
\\
{\small \textit{Parallellogramarealberegningshjelpefigur}}
\end{center}
Vi kan finne arealet av parallellogrammet ved å starte med arealet av
det store rektangelet, som er
\[
(a + b) (c + d),
\]
og trekke fra arealene av de to små rektanglene og de fire trekantene
som omgir parallellogrammet.  Vi ser at hvert av de små rektanglene
har areal~$bc$, at de to trekantene øverst og nederst til sammen har
areal~$ac$, og at trekantene til venstre og høyre til sammen har
areal~$bd$.  Det betyr at arealet av parallellogrammet er:
\begin{align*}
& \!\!\!\!\!\!\!\!(a + b) (c + d) - 2bc - ac - bd \\
&= ac + ad + bc + bd - 2bc - ac - bd \\
&= ad - bc \\
&= \det A
\end{align*}
Vi kommer altså frem til at arealet av parallellogrammet utspent av
kolonnene i~$A$ er lik determinanten til~$A$.  Denne utregningen var
imidlertid litt avhengig av hvordan disse to kolonnevektorene er
plassert i forhold til hverandre i planet.  Hvis vi hadde byttet plass
på kolonnene, så ville vi isteden fått $bc - ad$ som areal.  Da ville
altså determinanten vært negativ.

Det som holder i alle tilfeller, er at arealet av parallellogrammet
utspent av kolonnene i~$A$ er lik absoluttverdien til determinanten:
\[
\abs{\det A}
\]

\medskip

Så determinanten gir oss arealet til et parallellogram.  Men hva
forteller det oss om matrisen~$A$?

Vi kan tenke på~$A$ som en transformasjon av planet der hver
vektor~$\V{v}$ i~$\R^2$ sendes til vektoren~$A \V{v}$.  Determinanten
sier noe om hvordan planet endres under denne transformasjonen.

La
\[
\V{e}_1 = \vv{1}{0}
\qquad\text{og}\qquad
\V{e}_2 = \vv{0}{1}
\]
være de to enhetsvektorene i~$\R^2$, og se på kvadratet utspent av
disse:
\begin{center}
\begin{tikzpicture}[scale=2]
\draw[fill=gray!35] (0,0) -- (0,1) -- (1,1) -- (1,0) -- cycle;
\draw[->] (-.6,0) -- (1.6,0);
\draw[->] (0,-.2) -- (0,1.2);
\draw[->,thick] (0,0) -- (1,0);
\draw[->,thick] (0,0) -- (0,1);
\node[anchor=north] at (1,0) {$\V{e}_1$};
\node[anchor=east] at (0,1) {$\V{e}_2$};
\draw (1,0) -- (1,1) -- (0,1);
\node at (.5,.5) {$1$};
\end{tikzpicture}
\\
{\small \textit{Enhetskvadratet}}
\end{center}
Nå vil vi se på hva som skjer med dette kvadratet dersom vi sender
hvert punkt $\V{v}$ i~$\R^2$ til $A \V{v}$, der
\[
A = \begin{bmatrix} \V{a}_1 & \V{a}_2 \end{bmatrix}
\]
er en $2 \times 2$-matrise med $\V{a}_1$ og~$\V{a}_2$ som kolonner.
Da sendes vektoren $\V{e}_1$ til $\V{a}_1$, og vektoren $\V{e}_2$
sendes til~$\V{a}_2$.  Alle vektorene som ligger inni enhetskvadratet
i forrige figur sendes til vektorer som ligger inni parallellogrammet
utspent av $\V{a}_1$ og~$\V{a}_2$.

Vi skisserer noen forskjellige muligheter, for forskjellige valg av
matrisen~$A$:
\begin{center}
\begin{tabular}{cl}
$\det A > 0$ &
\begin{tikzpicture}[scale=.5,baseline=30pt]
\draw[->] (-1,0) -- (10,0);
\draw[->] (0,-.2) -- (0,5);
\draw[fill=gray!35] (0,0) -- (6,1) -- (8,4) -- (2,3) -- cycle;
\draw[->] (0,0) -- (6,1);
\draw[->] (0,0) -- (2,3);
\draw (2,3) -- (8,4) -- (6,1);
\node at (4,2) {$\det A$};
\node[anchor=west] at (6,1) {$\V{a}_1$};
\node[anchor=south] at (2,3) {$\V{a}_2$};
\end{tikzpicture}
\\[44pt]
$\det A < 0$ &
\begin{tikzpicture}[scale=.5,baseline=30pt]
\draw[->] (-1,0) -- (10,0);
\draw[->] (0,-.2) -- (0,5);
\draw[fill=gray!35] (0,0) -- (6,1) -- (8,4) -- (2,3) -- cycle;
\draw[->] (0,0) -- (6,1);
\draw[->] (0,0) -- (2,3);
\draw (2,3) -- (8,4) -- (6,1);
\node at (4,2) {$- \det A$};
\node[anchor=west] at (6,1) {$\V{a}_2$};
\node[anchor=south] at (2,3) {$\V{a}_1$};
\end{tikzpicture}
\\[44pt]
$\det A = 0$ &
\begin{tikzpicture}[scale=.5,baseline=30pt]
\draw[->] (-1,0) -- (10,0);
\draw[->] (0,-.2) -- (0,5);
\draw[->] (0,0) -- (4,1);
\draw[->] (0,0) -- (6,1.5);
\draw (0,0) -- (8,2);
\node[anchor=south] at (4,1) {$\V{a}_1$};
\node[anchor=south] at (6,1.5) {$\V{a}_2$};
%\node[anchor=west] at (8,2) {$\V{a}_1 + \V{a}_2$};
\end{tikzpicture}
\\[44pt]
$\det A = 0$ &
\begin{tikzpicture}[scale=.5,baseline=30pt]
\draw[->] (-1,0) -- (10,0);
\draw[->] (0,-.2) -- (0,5);
\draw[->] (0,0) -- (5,2);
\filldraw (0,0) circle [radius=2pt];
\node[anchor=west] at (5,2) {$\V{a}_2$};
\node[anchor=north west] at (0,0) {$\V{a}_1$};
\end{tikzpicture}
\end{tabular}
\end{center}

I den første figuren har vi en matrise med positiv determinant.  Da
gjør transformasjonen $\V{v} \mapsto A\V{v}$ at enhetskvadratet
skaleres til et parallellogram med areal $\det A$.  Hvis $\det A > 1$
betyr dette at planet «blåses opp»; hvis $0 < \det A < 1$ betyr det at
planet «krympes sammen».

I den andre figuren har vi en matrise med negativ determinant.  Da er
situasjonen helt lik som i den første figuren, bortsett fra at det er
$(- \det A)$ som er arealet.  Da får vi at med $\det A < -1$ blir
planet «blåst opp», og med $-1 < \det A < 0$ blir det «krympet
sammen».

I den tredje og den fjerde figuren har vi situasjoner der
determinanten er~$0$.  Det vil si at parallellogrammet utspent av
kolonnene i~$A$ har areal~$0$.  Det blir altså ikke et virkelig
parallellogram i disse tilfellene; det har kollapset til et
«degenerert» parallellogram som er bare en linje.  På samme måte vil
transformasjonen $\V{v} \mapsto A \V{v}$ i disse tilfellene kollapse
hele planet ned til linjen utspent av $\V{a}_1$ og~$\V{a}_2$.


\section*{Determinanter for $3 \times 3$-matriser}

For en $3 \times 3$-matrise
\[
A =
\begin{bmatrix}
a_{11} & a_{12} & a_{13} \\
a_{21} & a_{22} & a_{23} \\
a_{31} & a_{32} & a_{33}
\end{bmatrix}
\]
er determinanten definert ved:
\begin{align*}
\det A &=
\begin{vmatrix}
a_{11} & a_{12} & a_{13} \\
a_{21} & a_{22} & a_{23} \\
a_{31} & a_{32} & a_{33}
\end{vmatrix}
\\
&=
a_{11}
\begin{vmatrix}
a_{22} & a_{23} \\
a_{32} & a_{33}
\end{vmatrix}
- a_{12}
\begin{vmatrix}
a_{21} & a_{23} \\
a_{31} & a_{33}
\end{vmatrix}
+ a_{13}
\begin{vmatrix}
a_{21} & a_{22} \\
a_{31} & a_{32}
\end{vmatrix}
\end{align*}
Her er det også en geometrisk tolkning.  Vi kan tegne opp kolonnene
i~$A$ som piler i~$\R^3$, og lage et parallell\-epiped med disse pilene
som tre av sidene.  Dette kaller vi for parallell\-epipedet utspent av
de tre vektorene.  Da har vi at volumet av parallell\-epipedet utspent
av kolonnene i~$A$ er lik absoluttverdien av determinanten til~$A$.


\section*{Den generelle definisjonen}

Vi definerer determinanten til en vilkårlig stor kvadratisk matrise
etter samme mønster som determinanten til en $3 \times 3$-matrise.

\begin{defn}
La
\[
A =
\begin{bmatrix}
a_{11} & a_{12} & \cdots & a_{1n} \\
a_{21} & a_{22} & \cdots & a_{2n} \\
\vdots & \vdots & \ddots & \vdots \\
a_{n1} & a_{n2} & \cdots & a_{nn}
\end{bmatrix}
\]
være en $n \times n$-matrise.  \defterm{Determinanten} til~$A$, som
har notasjonen $\det A$, defineres på følgende måte.
\begin{enumerate}
\item Hvis $n = 1$, så har vi at
$A = \begin{bmatrix} a_{11} \end{bmatrix}$, og da definerer vi at
$\det A = a_{11}$.
\item Hvis $n > 1$, innfører vi først noen hjelpevariabler.  For hver
$i$ og~$j$ fra $1$ til~$n$ setter vi $A_{ij}$ til å være
$(n-1) \times (n-1)$-matrisen vi får ved å fjerne rad~$i$ og
kolonne~$j$ fra~$A$, og vi setter
\[
C_{ij} = (-1)^{i + j} \det A_{ij}
\]
til å være determinanten til denne matrisen, med et fortegn som
avhenger av $i$ og~$j$.  Determinanten til~$A$ defineres ved:
\[
\det A = \sum_{j=1}^n a_{1j} C_{1j}\qedhere
\]
\end{enumerate}
\end{defn}

Tallene $C_{ij}$ i definisjonen kalles \defterm{kofaktorer} av~$A$.

Det er ikke vanskelig å se at hvis vi setter inn en
$2 \times 2$-matrise eller en $3 \times 3$-matrise i denne generelle
definisjonen, så får vi bare de vanlige reglene for determinanter av
$2 \times 2$- og $3 \times 3$-matriser.

For å få litt erfaring med å bruke definisjonen på større matriser
regner vi ut determinanten av en $4 \times 4$-matrise.

\begin{ex}
\label{ex:det-4x4}
La $A$ være følgende $4 \times 4$-matrise:
\[
A =
\begin{bmatrix}
3 & 0 & 2 & 4 \\
1 & 2 & 1 & 0 \\
2 & 0 & 1 & 0 \\
7 & 0 & 5 & 3
\end{bmatrix}
\]
Vi regner ut determinanten til~$A$.  Fra definisjonen får vi:
\begin{align*}
\det A &=
3 \begin{vmatrix}
2 & 1 & 0 \\
0 & 1 & 0 \\
0 & 5 & 3
\end{vmatrix}
- 0 \begin{vmatrix}
1 & 1 & 0 \\
2 & 1 & 0 \\
7 & 5 & 3
\end{vmatrix}
\\
&\qquad
+ 2 \begin{vmatrix}
1 & 2 & 0 \\
2 & 0 & 0 \\
7 & 0 & 3
\end{vmatrix}
- 4 \begin{vmatrix}
1 & 2 & 1 \\
2 & 0 & 1 \\
7 & 0 & 5
\end{vmatrix}
\end{align*}
Vi regner ut hver av $3 \times 3$-determinantene som trengs (merk at
vi ikke trenger å regne ut den andre, for den skal uansett ganges
med~$0$):
\begin{align*}
\begin{vmatrix}
2 & 1 & 0 \\
0 & 1 & 0 \\
0 & 5 & 3
\end{vmatrix}
&=
  2 \begin{vmatrix} 1 & 0 \\ 5 & 3 \end{vmatrix}
- 1 \begin{vmatrix} 0 & 0 \\ 0 & 3 \end{vmatrix}
+ 0 \begin{vmatrix} 0 & 1 \\ 0 & 5 \end{vmatrix}
 = 6 \\
\begin{vmatrix}
1 & 2 & 0 \\
2 & 0 & 0 \\
7 & 0 & 3
\end{vmatrix}
&=
  1 \begin{vmatrix} 0 & 0 \\ 0 & 3 \end{vmatrix}
- 2 \begin{vmatrix} 2 & 0 \\ 7 & 3 \end{vmatrix}
+ 0 \begin{vmatrix} 2 & 0 \\ 7 & 0 \end{vmatrix}
 = -12 \\
\begin{vmatrix}
1 & 2 & 1 \\
2 & 0 & 1 \\
7 & 0 & 5
\end{vmatrix}
&=
  1 \begin{vmatrix} 0 & 1 \\ 0 & 5 \end{vmatrix}
- 2 \begin{vmatrix} 2 & 1 \\ 7 & 5 \end{vmatrix}
+ 1 \begin{vmatrix} 2 & 0 \\ 7 & 0 \end{vmatrix}
 = -6
\end{align*}
Ved å sette inn disse i uttrykket for $\det A$ får vi:
\[
\det A = 3 \cdot 6 - 0 + 2 \cdot (-12) - 4 \cdot (-6) = 18
\]
Vi har altså regnet ut at $\det A = 18$.
\end{ex}

Ved hjelp av definisjonen kan vi regne ut determinanten til en hvilken
som helst kvadratisk matrise, men det kan bli veldig mye jobb.  I
eksempelet så vi at determinanten til en $4 \times 4$-matrise er
definert ut fra determinantene til fire $3 \times 3$-matriser, og hver
av disse er igjen definert ut fra determinantene til tre
$2 \times 2$-matriser.  Hvis vi går til større matriser, blir
arbeidsmengden fort enormt stor.

I løpet av dette kapitlet skal vi se på noen lure teknikker for å
regne ut determinanter på mindre arbeidskrevende måter.


\section*{Kofaktorekspansjon}

I definisjonen av determinanten går vi gjennom første rad i matrisen,
og ser på tallene
\[
a_{11},\ a_{12},\ \ldots,\ a_{1n}.
\]
Hvert tall $a_{1j}$ ganges med den tilhørende kofaktoren~$C_{1j}$, og
til slutt summerer vi alle disse produktene.

Det er imidlertid ikke nødvendig å gå langs første rad når vi gjør
dette.  Det fungerer like bra å gå langs en annen rad og følge det
samme systemet, og resultatet blir det samme.  Det går dessuten an å
gå langs en hvilken som helst kolonne med samme system.  Vi
oppsummerer dette i følgende teorem.

\begin{thm}
La $A$ være en $n \times n$-matrise, der $n > 1$, og la $A_{ij}$
og~$C_{ij}$ være som i definisjonen av determinant.  Da har vi
\[
\det A
= \sum_{j=1}^n a_{kj} C_{kj}
= \sum_{i=1}^n a_{il} C_{il}
\]
for alle $k$ og~$l$ slik at $1 \le k \le n$ og $1 \le l \le n$.
\end{thm}

Å regne ut determinanten ved å beregne en sum av tall fra matrisen
ganget med kofaktorer slik som i teoremet kalles
\defterm{kofaktor\-ekspansjon}.  Vi sier at vi gjør kofaktorekspansjon
langs rad~$k$ eller langs kolonne~$l$.

La oss nå regne ut den samme determinanten som i
eksempel~\ref{ex:det-4x4}, men på en lurere måte.

\begin{ex}
Vi lar igjen $A$ være denne $4 \times 4$-matrisen:
\[
A =
\begin{bmatrix}
3 & 0 & 2 & 4 \\
1 & 2 & 1 & 0 \\
2 & 0 & 1 & 0 \\
7 & 0 & 5 & 3
\end{bmatrix}
\]
Vi kan observere at den andre kolonnen inneholder nesten bare nuller,
så det er lurt å gjøre kofaktor\-ekspansjon langs den.  Da får vi:
\begin{align*}
\det A &=
- 0 \begin{vmatrix}
1 & 1 & 0 \\
2 & 1 & 0 \\
7 & 5 & 3
\end{vmatrix}
+ 2 \begin{vmatrix}
3 & 2 & 4 \\
2 & 1 & 0 \\
7 & 5 & 3
\end{vmatrix}
\\
&\qquad
- 0 \begin{vmatrix}
3 & 2 & 4 \\
1 & 1 & 0 \\
7 & 5 & 3
\end{vmatrix}
+ 0 \begin{vmatrix}
3 & 2 & 4 \\
1 & 1 & 0 \\
2 & 1 & 0
\end{vmatrix}
\\
&= 2 \cdot \left( 4 \cdot \begin{vmatrix} 2 & 1 \\ 7 & 5 \end{vmatrix} +
                  3 \cdot \begin{vmatrix} 3 & 2 \\ 2 & 1 \end{vmatrix} \right)
\\
&= 2 \cdot (4 \cdot 3 + 3 \cdot (-1))
 = 18
\end{align*}
Vi fikk samme resultat som i eksempel~\ref{ex:det-4x4}, men med mindre
arbeid, siden vi bare trengte å regne ut én $3 \times 3$-determinant.
\end{ex}

Vi må passe på at vi får fortegnene i kofaktorene riktig.  Når vi gjør
kofaktorekspansjon langs første rad (slik som i definisjonen) eller
langs første kolonne, starter vi alltid med positivt fortegn i det
første leddet.  Men når vi ekspanderer langs en annen rad eller
kolonne, kan det hende vi må starte med negativt fortegn.  Det kan
være nyttig å bruke følgende diagram som en huskeregel for hvilket
fortegn vi skal ha i de forskjellige kofaktorene:
\[
\begin{bmatrix}
+ & - & + & - & \cdots \\
- & + & - & + & \cdots \\
+ & - & + & - & \cdots \\
- & + & - & + & \cdots \\
\vdots & \vdots & \vdots & \vdots & \ddots
\end{bmatrix}
\]


\section*{Determinanter og radoperasjoner}

Hvis vi utfører en radoperasjon på en matrise, så får vi en ny
matrise.  Den matrisen har ikke nødvendigvis samme determinant som den
opprinnelige, men det viser seg at determinanten endrer seg på ganske
kontrollerte måter når vi utfører radoperasjoner.  Dette kan vi
utnytte for å spare oss for en del arbeid når vi skal regne ut
determinanter, spesielt hvis vi har store matriser.

\begin{thm}
\label{thm:det-radop}
La $A$ være en $n \times n$-matrise, og la $B$ være en matrise vi får
ved å utføre en radoperasjon på~$A$.  Da har vi følgende sammenheng
mellom determinantene til $A$ og~$B$, basert på hvilken type
radoperasjon vi utførte:
\[
\begin{array}{ll}
\text{Radoperasjon} & \text{Resultat} \\\hline
\textup{Gange en rad med et tall~$k$} &
\det B = k \cdot \det A \\\hline
\textup{Legge til et multiplum}\\\textup{av én rad i en annen} &
\det B = \det A \\\hline
\textup{Bytte om to rader} &
\det B = - \det A \\\hline
\end{array}
\]
\end{thm}

\medskip
La oss bruke dette teoremet til å beregne en determinant.

\begin{ex}
\label{ex:det-radop}
Vi regner ut $\det A$, der
\[
A =
\begin{bmatrix}
3 & 3 & 12 \\
2 & 2 & 13 \\
4 & 2 & 19
\end{bmatrix}
\]
Vi får:
\begin{align*}
\det A &=
\begin{vmatrix}
3 & 3 & 12 \\
2 & 2 & 13 \\
4 & 2 & 19
\end{vmatrix}
=
3 \cdot
\begin{vmatrix}
1 & 1 &  4 \\
2 & 2 & 13 \\
4 & 2 & 19
\end{vmatrix}
\\
&=
3 \cdot
\begin{vmatrix}
1 &  1 &  4 \\
0 &  0 &  5 \\
0 & -2 &  3
\end{vmatrix}
=
- 3 \cdot
\begin{vmatrix}
1 &  1 &  4 \\
0 & -2 &  3 \\
0 &  0 &  5
\end{vmatrix}
\\
&= -3 \cdot 1 \cdot \begin{vmatrix} -2 & 3 \\ 0 & 5 \end{vmatrix}
\\
&= -3 \cdot 1 \cdot (-2) \cdot 5
= 30
\end{align*}
Her startet vi med å gjøre radoperasjoner på matrisen, samtidig som vi
holdt styr på hvordan determinanten endret seg.

Først ganget vi øverste rad med~$1/3$.  Det medførte at determinanten
til den nye matrisen ble $1/3$ ganger determinanten til~$A$, så vi
måtte gange den med~$3$ for at tallene skal bli like.  En hendig måte
å huske hvordan dette fungerer er å tenke på det som å sette et tall
utenfor parentes.  På samme måte som vi kan trekke ut et $3$-tall fra
en parentes og få
\[
(3 + 3 + 12) = 3 \cdot (1 + 1 + 12),
\]
kan vi trekke ut et $3$-tall fra en rad i en determinant.

Etterpå trakk vi fra multipler av første rad i de to andre radene, men
det medførte ingen endring av determinanten.

Så byttet vi de to nederste radene, og det gjorde at determinanten
skifter fortegn.

Til slutt gjorde vi kofaktorekspansjon langs den første kolonnen.
Siden vi ved å utføre radoperasjoner hadde sørget for å få bare nuller
under diagonalen, ble kofaktorekspansjonen enkel og grei.
\end{ex}


\section*{Triangulære matriser}

Vi sier at en $n \times n$-matrise er \defterm{øvre triangulær} hvis
alle tall under diagonalen er~$0$, altså hvis den er på følgende form:
\[
\begin{bmatrix}
a_{11} & a_{12} & a_{13} & \cdots & a_{1n} \\
0      & a_{22} & a_{23} & \cdots & a_{2n} \\
0      & 0      & a_{33} & \cdots & a_{3n} \\
\vdots & \vdots & \vdots & \ddots & \vdots \\
0      & 0      & 0      & \cdots & a_{nn}
\end{bmatrix}
\]
Tilsvarende sier vi at en $n \times n$-matrise er \defterm{nedre
  triangulær} hvis alle tall over diagonalen er~$0$, altså hvis den
er på følgende form:
\[
\begin{bmatrix}
a_{11} & 0      & 0      & \cdots & 0      \\
a_{21} & a_{22} & 0      & \cdots & 0      \\
a_{31} & a_{32} & a_{33} & \cdots & 0      \\
\vdots & \vdots & \vdots & \ddots & \vdots \\
a_{n1} & a_{n2} & a_{n3} & \cdots & a_{nn}
\end{bmatrix}
\]

\begin{ex}
I eksempel~\ref{ex:det-radop} brukte vi radoperasjoner til å skrive om
matrisen vår til følgende:
\[
\begin{bmatrix}
1 &  1 &  4 \\
0 & -2 &  3 \\
0 &  0 &  5
\end{bmatrix}
\]
Denne matrisen er øvre triangulær.
\end{ex}

Hvis vi skal finne determinanten til en øvre triangulær matrise er det
praktisk å kofaktorekspandere langs første kolonne.  Da får vi bare
ett ledd i ekspansjonen, nemlig tallet øverst til venstre i matrisen
ganget med determinanten til matrisen der første rad og kolonne er
fjernet.  Denne matrisen er igjen øvre triangulær.  Så fortsetter vi
med kofaktorekspansjon langs første kolonne i hvert steg nedover.  Det
vi ender opp med til slutt er å bare gange sammen alle tallene på
diagonalen.

Vi oppsummerer dette i et teorem.

\begin{thm}
\label{thm:det-diag}
La $A$ være en (øvre eller nedre) triangulær $n \times n$-matrise.  Da
er determinanten til~$A$ lik produktet av tallene på diagonalen i~$A$:
\[
\det A = a_{11} \cdot a_{22} \cdot a_{33} \cdots a_{nn}\qedhere
\]
\end{thm}

\begin{ex}
Ved å bruke teoremet kan vi skrive opp determinanten til en triangulær
matrise direkte, uten å måtte beregne andre determinanter først:
\[
\begin{vmatrix}
1 &  1 &  4 \\
0 & -2 &  3 \\
0 &  0 &  5
\end{vmatrix}
= 1 \cdot (-2) \cdot 5
\qedhere
\]
\end{ex}

Hvis vi skal beregne determinanten til en stor og stygg matrise, er
det en god strategi å bruke radoperasjoner for å skrive om matrisen
til øvre triangulær form (og holde orden på hvordan determinanten
endrer seg ved hjelp av teorem~\ref{thm:det-radop}), og så regne ut
determinanten til den triangulære matrisen ved hjelp av
teorem~\ref{thm:det-diag}.


\section*{Flere regneregler}

Vi tar med et par regneregler til for determinanter.

\begin{thm}
\label{thm:det-produkt}
Determinanten til et produkt av to matriser er produktet av
determinantene.  Altså: Hvis $A$ og~$B$ er to $n \times n$-matriser,
så er
\[
\det (AB) = (\det A) (\det B).
\]
\end{thm}

\begin{thm}
\label{thm:det-tr}
Determinanten endrer seg ikke når vi transponerer matrisen.  Altså:
Hvis $A$ er en $n \times n$-matrise, så er
\[
\det A = \det A\tr.
\]
\end{thm}


\section*{Karakterisering av inverterbarhet}

Vi kan bruke determinanten til å sjekke om en matrise er inverterbar
eller ikke.  Dette kan dessuten knyttes sammen med hvorvidt kolonnene
i matrisen vår er lineært uavhengige, og hva det utspenner.

Følgende teorem gir en presis sammenheng mellom inverterbarhet,
determinant, lineær uavhengighet og mengden utspent av kolonnene.

\begin{thm}
\label{thm:karakterisering-inverterbar}
La $A$ være en $n \times n$-matrise.  Følgende påstander er
ekvivalente:
\begin{enumerate}
\item $A$ er inverterbar.
\item $\det A \ne 0$.
\item Kolonnene i~$A$ er lineært uavhengige.
\item Kolonnene i~$A$ utspenner~$\R^n$.
\end{enumerate}
\end{thm}


\kapittelslutt

