% -*- TeX-master: "oving07"; -*-
\oppgaver{9}


\begin{oppgave}
Sjekk om følgende funksjoner er lineære. Hvis ja, sjekk om den er injektiv og surjektiv.

\begin{punkt}
$T(\begin{bmatrix}
x\\
y\\
z
\end{bmatrix})=\begin{bmatrix}
8x-7y\\
3z-8x-7y\\
5y-4x-8z\\
6y-6x-4z
\end{bmatrix}.$
\end{punkt}

\begin{punkt}
$T(\begin{bmatrix}
x\\
y\\
z\\
w
\end{bmatrix})=\begin{bmatrix}
x\\
y\\
z\\
w
\end{bmatrix}\boldsymbol{\cdot }\begin{bmatrix}
x\\
y\\
z\\
w
\end{bmatrix}.$

\end{punkt}

\begin{punkt}
$T(\begin{bmatrix}
x\\
y\\
z\\
w
\end{bmatrix})=\begin{bmatrix}
x\\
y\\
z\\
w
\end{bmatrix}\boldsymbol{\cdot }\begin{bmatrix}
1\\
2\\
3\\
4
\end{bmatrix}.$

\end{punkt}

\begin{punkt}
$T(\begin{bmatrix}
x_1\\
x_2\\
x_3\\
x_4\\
x_5
\end{bmatrix})=\begin{bmatrix}
2x_1-5x_2+3x_3+4x_4\\
2x_1-4x_2+7x_3+2x_4+x_5\\
-6x_1+15x_2-9x_3-12x_4+x_5\\
4x_1-8x_2+14x_3+5x_4-6x_5\\
-2x_1+5x_2+4x_3+5x_4-4x_5
\end{bmatrix}.$

\end{punkt}


\end{oppgave}

\begin{losning}


\begin{punkt}
$$\begin{bmatrix}
8 & -7 & 0\\
-8 & -7 & 3\\
-4 & 5 & -8\\
-6 & 6 & -4
\end{bmatrix}$$
Injektiv, ikke surjektiv.
\end{punkt}

\begin{punkt}
Ikke lineær.
\end{punkt}

\begin{punkt}
$$\begin{bmatrix}
1 & 2 & 3 & 4
\end{bmatrix}$$
Surjektiv, ikke injektiv.
\end{punkt}

\begin{punkt}
$$
\begin{bmatrix}
2 & -5 & 3 & 4 & 0 \\
2 & -4 & 7 & 2 & 1 \\
-6 & 15 & -9 & -12 & 1 \\
4 & -8 & 14 & 5 & -6 \\
-2 & 5 & 4 & 5 & -4
\end{bmatrix}
$$
Injektiv og surjektiv.

\noindent 
Husk: Du har regnet ut determinanten til denne matrisen tidligere.

\end{punkt}

\end{losning}



\begin{oppgave}
\label{oppg:1dlintrans}
Vi skal se på funksjoner $\mathbb{R}\rightarrow \mathbb{R}$. Begrunn om følgende funksjoner er lineære eller ikke:

\begin{punkt}
$f(x) = 2x$
\end{punkt}

\begin{punkt}
$f(x) = 3x+1$
\end{punkt}

\begin{punkt}
$f(x) = 3x^2$
\end{punkt}

\begin{punkt}
$f(x)= -x$
\end{punkt}

\begin{punkt}
$f(x)=e^x$
\end{punkt}

\begin{punkt}
$f(x)=\cos (x)$
\end{punkt}
\end{oppgave}

\begin{losning}

\textbf{a)} og \textbf{d)} er lineære; \textbf{b)}, \textbf{c)}, \textbf{e)} og \textbf{f} er ikke.

\noindent
Hint: Hva er definisjonen av en lineærtransformasjon. Husk $e^{x+y}=e^x\cdot e^y$ for del \textbf{e)} og addisjonsformelen for cosinus i del \textbf{f)}.

\end{losning}


\begin{oppgave}
Vis at en lineærtransformasjon $f:\mathbb{R}\rightarrow \mathbb{R}$ er entydig bestemt av et reelt tall.

\noindent 
Hint: Hvordan ser de lineære funksjonene i forrige oppgave ut?
\end{oppgave}


\begin{losning}
Hint: $f(x)=x\cdot f(1)$ hvor $f(1)$ er et reelt tall. Kan du fullføre oppgaven?
\end{losning}


\begin{oppgave}
\begin{punkt}
La $f$ og $g$ være funksjoner $\mathbb{R}\rightarrow \mathbb{R}$ gitt ved $$f(x)=2^x \quad \text{og} \quad g(x)=x^3-x.$$ Vis at $f$ er injektiv men ikke surjektiv; $g$ er surjektiv men ikke injektiv.

\end{punkt}
\begin{punkt}
Kan en eller begge tilfellene i del \textbf{a)} skje for en lineærtrasnformasjon $L$ fra $\mathbb{R}^n$ til $\mathbb{R}^n$? Hva med $\mathbb{R}^n$ til $\mathbb{R}^m$?
\end{punkt}

\end{oppgave}

\begin{losning}

\begin{punkt}
Hint: Du kan løse oppgaven ved å bruke definisjonene for injektiv og surjektiv direkte.
\end{punkt}

\begin{punkt}
Nei i tilefellet $\mathbb{R}^n\rightarrow \mathbb{R}^n$; injektiv, surjektiv og inverterbar er det samme for kvadratiske matriser. Ja i tilfellet $\mathbb{R}^n\rightarrow \mathbb{R}^m$.


\noindent
Merk: Dette er altså en helt spesiell egenskap for lineære funksjoner som er gitt av \emph{kvadratiske} matriser.
\end{punkt}


\end{losning}

\begin{oppgave}
\begin{punkt}
La $C^\infty$ være undermengden av alle funksjoner som er uendelig mange ganger deriverbar; kan deriveres et vilkårlig antal ganger. Vis at dette er et underrom av vektorrommet av alle kontinuerlige funksjoner.
\end{punkt}

\begin{punkt}
Vis at funksjonen $T:C^\infty\rightarrow C^\infty$ som er gitt ved derivasjon, $$T(f)=f',$$ er en lineærtransformasjon.

\noindent
Hint: I matte 1 lærte vi regneregler for derivasjon av i) en sum av to funksjoner og ii) en funksjon multiplisert med en konstant.
\end{punkt}

\begin{punkt}
Vis at $T$ har uendelig mange egenverdier.

\noindent 
Hint: Husker du hvilken funksjon fra matte 1 som ikke endrer seg ved derivasjon? Hva betyr dette mtp. oppgaven? Kan du modifisere denne litt for å løse oppgaven?
\end{punkt}
\end{oppgave}
