% -*- TeX-master: "oving07"; -*-
\oppgaver{9}

\begin{oppgave}
\label{oppg:1dlintrans}
Vi skal se på funksjoner $\mathbb{R}\rightarrow \mathbb{R}$. Begrunn om følgende funksjoner er lineære eller ikke:

\begin{punkt}
$f(x) = 2x$
\end{punkt}

\begin{punkt}
$f(x) = 3x+1$
\end{punkt}

\begin{punkt}
$f(x) = 3x^2$
\end{punkt}

\begin{punkt}
$f(x)= -x$
\end{punkt}

\begin{punkt}
$f(x)=e^x$
\end{punkt}

\begin{punkt}
$f(x)=\cos (x)$
\end{punkt}
\end{oppgave}

\begin{losning}

\textbf{a)} og \textbf{d)} er lineære; \textbf{b)}, \textbf{c)}, \textbf{e)} og \textbf{f} er ikke.

\noindent
Hint: Hva er definisjonen av en lineærtransformasjon. Husk $e^{x+y}=e^x\cdot e^y$ for del \textbf{e)} og addisjonsformelen for cosinus i del \textbf{f)}.

\end{losning}


\begin{oppgave}
Vis at en lineærtransformasjon $f:\mathbb{R}\rightarrow \mathbb{R}$ er entydig bestemt av et reelt tall.

\noindent 
Hint: Hvordan ser de lineære funksjonene i oppgave \ref{oppg:1dlintrans} ut?
\end{oppgave}


\begin{losning}
Hint: $f(x)=x\cdot f(1)$ hvor $f(1)$ er et reelt tall. Kan du fullføre oppgaven?
\end{losning}