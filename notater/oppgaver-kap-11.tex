% -*- TeX-master: "oving09"; -*-
\oppgaver{11}

\begin{oppgave}
Løs likningssystemet
\begin{align*}
(1+i) z - w   &= i \\
(1-i) z + (1+i) w &= 1.
\end{align*}
\end{oppgave}

\begin{losning}
Ligningssystemet 
\begin{equation*}
\begin{matrix}
(1+i)z &- w &= i \\
(1-i)z &+ (1+i)w &= 1
\end{matrix}
\end{equation*}
kan skrives som matriseligningen
\begin{equation*}
\begin{bmatrix}	
1+i & -1 \\
1-i & 1+i
\end{bmatrix}
\begin{bmatrix}
z \\ w
\end{bmatrix}	
=
\begin{bmatrix}
i \\ 1
\end{bmatrix}
\end{equation*}
Ligningen har entydig løsning
%Vi regner ut inversen til en $2\times 2$-matrise og multipliserer på begge sider med den og får
\begin{equation*}
\begin{bmatrix}
z \\ w
\end{bmatrix}
=
%\begin{pmatrix}
%1 + i & \frac{1}{2}(1 - i) \\
%i& 1
%\end{pmatrix}
%\begin{pmatrix}
%i \\ 1
%\end{pmatrix}
\frac{1}{2}
\begin{bmatrix}
1 + i \\ 0
\end{bmatrix}.
\end{equation*}

\end{losning}


\begin{oppgave}
Er kolonnene lineært avhengige? Hvis ja, finn nullrommet til matrisen.
\begin{punkt}
\[
\begin{bmatrix}
2i & 3i & 4i \\ 3 & 4 & 5 \\ 4 & 5 & 6 
\end{bmatrix}
\]
\end{punkt}

\begin{punkt}
\[
\begin{bmatrix}
2i & 3 & 4 \\ 3 & 4i & 5 \\ 4 & 5 & 6i 
\end{bmatrix}
\]
\end{punkt}

\begin{punkt}
\[
\begin{bmatrix}
2i & 3 & 4 \\ 3i & 4 & 5 \\ 4i & 5 & 6 
\end{bmatrix}
\]
\end{punkt}
\end{oppgave}

\begin{losning}
Vi vet at kolonenne til en kvadratisk matrise er lineært avhengige hvis, og bare hvis, determinanten er $0$.
\begin{punkt}
	Her er $\det(A) = 0$ så vi vet at kolonnene er lineært avhengige. Radreduserer vi matrisen får vi systemet
	\begin{equation*}
		\begin{bmatrix}
			1 & 0 & -1 \\
			0 & 1 & 2 \\
			0 & 0 & 0 
		\end{bmatrix}
		\begin{bmatrix}
			x \\ y \\ z
		\end{bmatrix}
		=
		\begin{bmatrix}
			0 \\ 0 \\ 0
		\end{bmatrix}
	\end{equation*}
	Her har vi én nullrad som betyr at nullrommet har dimensjon 1. Vi velger $z=t$ som fri kompleks variabel og får at $x=t$ og $y = -2t$ som betyr at nullrommet er
	\begin{equation*}
		\mathrm{null}(A) = \mathrm{Sp}\left\{\begin{bmatrix}1 \\ -2 \\ 1\end{bmatrix}\right\} = t\begin{bmatrix}1 \\ -2 \\ 1\end{bmatrix}
	\end{equation*}
\end{punkt}
\begin{punkt}
	Matrisen har determinant ulik 0. Derfor er kolonnene lineært uavhengige, og nullrommet inneholder kun vektoren $\vvv{0}{0}{0}$.
\end{punkt}
\begin{punkt}
	Her er determinanten lik 0 så kolonnene er lineært avhengige. Vi finner radredusert form
	\begin{equation*}
		\begin{bmatrix}
			1 & 0 & -1 \\
			0 & 1 & 2 \\
			0 & 0 & 0 
		\end{bmatrix}
	\end{equation*}
	for så å finne nullrommet. Dette er den samme matrisen som den vi fikk i a) og nullrommet blir derfor likt. %Vi kan også observere at matrisene i a) og c) er transponerte av hverandre.
\end{punkt}
\end{losning}



\begin{oppgave}
Finn hver matrises determinant, egenverdier og tilhørende egenrom. 
\begin{punkt}
\[
\begin{bmatrix}
0 & 0 & -1 \\ 1 & -2 & 2 \\ 1 & 0 & 0 
\end{bmatrix}
\]
\end{punkt}

\begin{punkt}
\[
\begin{bmatrix}
0 & 1 & 1 \\ 1 & 0 & 1 \\ 1 & 1 & 0 
\end{bmatrix}
\]
\end{punkt}


\begin{punkt}
\[
\begin{bmatrix}
3 & 1 & 1 \\ 1 & 3 & 1 \\ 1 & 1 & 3 
\end{bmatrix}
\]
\end{punkt}

\begin{punkt}
\[
\begin{bmatrix}
3 & 1 & 0 \\ 0 & 3 & 1 \\ 0 & 0 & 3 
\end{bmatrix}
\]
\end{punkt}

\end{oppgave}


\begin{losning}
	\begin{punkt}
		Determinanten er $-2$ og egenverdiene er $-2$, $i$ og $-i$. Egenrommene er utspent av de korresponderende egenvektorene
		\begin{equation*}
			\begin{bmatrix}
				0 \\ 1 \\ 0
			\end{bmatrix}
			,
			\begin{bmatrix}
				i \\ 1 \\ 1
			\end{bmatrix}
			,
			\begin{bmatrix}
				-i \\ 1 \\ 1
			\end{bmatrix}
		\end{equation*}
	\end{punkt}
	\begin{punkt}
		Her er determinanten $2$ og vi har karakteristisk polynom
		\begin{equation*}
			(\lambda + 1)^2(\lambda - 2)
		\end{equation*}
		som gir egenverdier $\lambda  = -1$ og $\lambda = 2$.
		
	\end{punkt}

	\begin{punkt}
		Determinanten er $20$ og karakteristisk polynom er 
		\begin{equation*}
			(\lambda - 2)^2(\lambda - 5)
		\end{equation*}
	\end{punkt}

	\begin{punkt}
		Determinanten er $27$ og egenverdiene er $\lambda = 3$ med multiplisitet 3
	\end{punkt}

\end{losning}

\begin{oppgave}
Beregn produktet av egenverdiene for hver matrise i forrige oppgave.
\end{oppgave}

\begin{losning}
	Hvis vi ganger sammen egenverdiene \emph{med} multiplisitet får vi
	\begin{align*}
		(-2)i(-i) &= -2 \\
		(-1)^22 &= 2 \\
		2^2\cdot 5 &= 20 \\
		3^3 & = 27
	\end{align*}
	altså nøyaktig determinantene til de respektive matrisene.
\end{losning}

\begin{oppgave}
Finn en vektor $\V w$ slik at
\[
\V w,
\quad 
\text{}
\quad 
\begin{bmatrix}
-3 \\ 2i \\ 8 
\end{bmatrix}
\quad 
\text{og}
\quad 
\begin{bmatrix}
1+i \\ 0 \\ -2 
\end{bmatrix}
\]
spenner ut $\C^3$.
\end{oppgave}

\begin{losning}
%	Siden sannsynligheten for at en tilfeldig valgt vektor i $\mathbb{C}^3$ ligger i planet utspent av de to oppgitte vektorene er forsvinnende liten kan vi prøve å bare gjette en vektor og se om den er en lineær kombinasjon av de to andre. Vi gjetter
Ved samme fremgangsmåte som kapittel 3, oppgave 5, kan vi f. eks se at
	\begin{equation*}
		\V w= \begin{bmatrix} 1 \\ 1 \\ 1 \end{bmatrix}
	\end{equation*}
er lineært uavhengig av vektorene i oppgaveteksten. Videre kan du sjekke at de to gitte vektorene også er lineært uavhengige. Vi har altså tre lineært uavhengige vektorer i $\mathbb{C}^3$, som derfor spenner ut $\mathbb{C}^3$.
%	og regner ut determinanten til matrisen
%	\begin{equation*}
%		\begin{bmatrix}
%			1 & -3 & 1 + i 	\\
%			1 & 2i & 0 		\\
%			1 & 8 & -2 		
%		\end{bmatrix}
%	\end{equation*}
%	som blir ulik null. Dette skjer hvis, og bare hvis, kolonenne er linær uavhengige. Det vil si at vårt valg av vektoren $\V w$ ikke er en lineær kombinasjon av de to andre, som betyr at de tre vektorene sammen spenner ut $\mathbb{C}$.
\end{losning}


\begin{oppgave}
Finn $a$ slik at vektoren 
\[
\begin{bmatrix}
a \\ 6-6i \\ -12 
\end{bmatrix}
\]
ligger i planet utspent av 
\[
\begin{bmatrix}
-3 \\ 2i \\ 8 
\end{bmatrix}
\quad 
\text{og}
\quad 
\begin{bmatrix}
1 \\ 1 \\ 2 
\end{bmatrix}.
\]

\end{oppgave}


\begin{losning}
%	Her blir det motsatt av forrige oppgave: Hvis vi gjetter $a$ vil det svært liten sannsynlighet at den ligger i planet utspent av de to andre vektorene. Det at vektoren ligger i dette planet betyr at det finnes $b$ og $c$, muligens komplekse, slik at
Vi ønsker å velge $a$ slik at vi har en lineærkombinasjon
	\begin{equation*}
		\begin{bmatrix}
			a\\ 6 - 6i \\ -12
		\end{bmatrix}
		= b
		\begin{bmatrix}
			-3 \\ 2i \\ 8
		\end{bmatrix}
		+ c
		\begin{bmatrix}
			1 \\ 1 \\ 2
		\end{bmatrix},
	\end{equation*}
\end{losning}
ekvivalent har matriseligningen
\begin{equation*}
	\begin{bmatrix}
		a \\ 6-6i \\ -12
	\end{bmatrix}
	=
	\begin{bmatrix}
		-3 & 1 \\
		2i & 1 \\
		8 & 2
	\end{bmatrix}
	\begin{bmatrix}
		b \\ c
	\end{bmatrix}
\end{equation*}
en løsning. Dette gir at $b = -3$ og $c = 6$ slik at 
\begin{equation*}
	a = 3b + c = 15
\end{equation*}
\begin{oppgave}
Vis at dersom matrisen $A$ har egenverdi $\lambda$, har $A^2$ egenverdi $\lambda^2$.
\end{oppgave}


\begin{losning}
Vi antar at $A$ har egenverdi $\lambda$. Dette betyr at det finnes en $\mathbf{v}\not = \mathbf{0}$ slik at
\begin{equation*}
A \mathbf{v} = \lambda \mathbf{v}.
\end{equation*}
Multipliser begge sidene av ligningen med $A$ for å få
\begin{equation*}
A(A\mathbf{v}) = A(\lambda\mathbf{v}).
\end{equation*}
Vi kan flytte på paranteser, dra ut konstanter og bruke at $\mathbf{v}$ er en egenverdi til $A$ for å se at
\begin{equation*}
A^2\mathbf{v} = A(A\mathbf{v}) =A( \lambda \mathbf{v})=  \lambda (A \mathbf{v}) =\lambda (\lambda \mathbf{v})= \lambda^2\mathbf{v}.
\end{equation*}
Dette betyr at $\lambda^2$ er en egenverdi til matrisen $A^2$.
\end{losning}

\begin{oppgave}
Finn egenverdiene til rotasjonsmatrisen
\[
T_{\theta}=
\begin{bmatrix}
\cos \theta & -\sin \theta  \\ \sin \theta & \cos \theta  
\end{bmatrix}.
\]
Hva er egenverdiene til $T_{2 \theta}$?
\end{oppgave}


\begin{losning}
Vi setter som vanlig opp matrisen
\begin{equation*}
	T_\theta - \lambda I = 
	\begin{bmatrix}
	\cos \theta - \lambda & -\sin\theta \\
	\sin\theta & \cos\theta - \lambda
\end{bmatrix}
\end{equation*}
og beregner determinanten
\begin{equation*}
	\det(T_\theta - \lambda I) = \lambda^2 - 2\cos\theta + 1
\end{equation*}
Ved nå å bruke abc-formelen får vi at 
\begin{equation*}
	\lambda = \frac{2\cos\theta \pm 2\sqrt{\cos ^2\theta - 1}}{2}
\end{equation*}	
som blir
\begin{equation*}
	\lambda = \cos\theta \pm i\sin\theta
\end{equation*}
Hvis vi ser på denne komplekse egenverdien som en vektor i planet er det nøyaktiv vektoren som peker i samme retning som $\V e_1$ rotert med/mot klokken med $\theta$ radianer. Fra tidligere øvinger vet vi at $T_{2\theta} = T_\theta\cdot T_\theta = T_\theta^2$. Fra oppgave 7 vet vi derfor at egenverdiene til  $T_{2\theta}$ er $\lambda^2$
\begin{equation*}
	\lambda^2 = (\cos\theta \pm i\sin\theta)^2 = \cos^2\theta - \sin^2\theta \pm \sin2\theta
\end{equation*}
og så husker vi at $\cos^2\theta - \sin^2\theta = \cos 2\theta$ for å få det vi forventet, altså at hvis vi dobler vinkelen i $T_\theta$ dobler vi vinkelen i egenverdien.
\end{losning}


\begin{oppgave}
Beregn determinanten. Følg nøye med,  det kan være det ikke er så mye jobb som det ser ut.
\begin{punkt}
\[
\begin{bmatrix}
 i & 0 & 0 \\  0 & i & 1 \\  0& 1  &i  
\end{bmatrix}
\]
\end{punkt}
\begin{punkt}
\[
\begin{bmatrix}
i & 1 & 0 & 0 \\ 1 & i & 0 & 0 \\ 0 & 0 & i & 1 \\ 0 & 0& 1  &i  
\end{bmatrix}
\]
\end{punkt}

\begin{punkt}
\[
\begin{bmatrix}
i & 1 & 1 & 1  \\ 1 & i & 1 & 1 \\ 0 & 0 & i & 1 \\ 0 & 0& 1  &i 
\end{bmatrix}
\]
\end{punkt}

\begin{punkt}
\[
\begin{bmatrix}
i & 1 & 1 & 1 & 1 & 1 \\ 1 & i & 1 & 1 & 1 & 1\\ 0 & 0 & i & 1& 1 & 1 \\ 0 & 0& 1  &i & 1 & 1 \\ 0 & 0 & 0 & 0 & i & 1\\ 0 & 0 & 0 & 0 & 1 & i 
\end{bmatrix}
\]
\end{punkt}

\begin{punkt}
\[
\begin{bmatrix}
1 & 1 & 1 & 1 & 1 & 1 \\ 1 & 1 & 1 & 1 & 1 & 1\\ 0 & 0 & 1 & 1& 1 & 1 \\ 0 & 0& 1  &1 & 1 & 1 \\ 0 & 0 & 0 & 0 & 1 & 1\\ 0 & 0 & 0 & 0 & 1 & 1 
\end{bmatrix}
\]
\end{punkt}

\end{oppgave}

\begin{losning}
	Her må vi finne den beste raden/kolonnen for å regne ut determinanten.
	\begin{punkt}	% a)
		Regner vi ut determinanten langs den første raden får vi determinanten
		\begin{equation*}
			i(i^2 - 1^2)  - 0 + 0 = -2i
		\end{equation*}
	\end{punkt}
	\begin{punkt}	% b)
		Her kan vi velge første rad eller første kolonne. Vi velger første kolonne og får determinanten
		\begin{equation*}
			i\det 
			\begin{bmatrix}
				i & 0 & 0 \\
				0 & i & 1 \\
				0 & 1 & i
			\end{bmatrix}
			-1\det
			\begin{bmatrix}
				1 & 0 & 0 \\
				0 & i & 1 \\
				0 & 1 & i
			\end{bmatrix}
		\end{equation*}
		Den første determinanten er samme som i oppg a) og den andre ligner. Vi får at determinanten er
		\begin{equation*}
			2 + 2i
		\end{equation*}
	\end{punkt}

	\begin{punkt}
		Vi velger å regne ut determinanten langs første kolonne. Det gir
		\begin{equation*}
			i\det\begin{bmatrix}
				i & 1 & 1 \\
				0 & i & 1 \\
				0 & 1 & i
			\end{bmatrix}
			-1\det\begin{bmatrix}
				1 & 1 & 1 \\
				0 & i & 1 \\
				0 & 1 & i
			\end{bmatrix}
		\end{equation*}
		Her ser vi at den første determinanten er kjent, og den andre regner vi ut langs første kolonne og får at determinanten til slutt blir $2 + 2i$ som i b)
	\end{punkt}
	Oppgave d) og e) er tilsvarende: nøst deg nedover ved å velge en lur rad/kolonne å beregne determinanten fra.
\end{losning}

\begin{oppgave}
Sett sammen egenvektorene til 
\[
A=
\begin{bmatrix}
0 & 0 & -1 \\ 1 & -2 & 2 \\ 1 & 0 & 0 
\end{bmatrix}
\]
i en $3 \times 3$-matrise $P$ der egenvektorene er kolonner, og beregn $P^{-1} A P$.
\end{oppgave}

\begin{losning}
	Vi fant egenverdiene og egenvektorene i oppgave 3. Hvis vi setter disse opp i matrisen $P$ får vi
	\begin{equation*}
		P = \begin{bmatrix}
				0 & i & -i \\
				1 & 1 & 1 \\
				0 & 1 & 1 
			\end{bmatrix}
			\quad P^{-1} = \frac{1}{2}
			\begin{bmatrix}
				0 & 2 & -2 \\
				 -i & 0 & 1 \\
				 i & 0 & 1
			\end{bmatrix}
	\end{equation*}
	som gir oss
	\begin{equation*}
		P^{-1}AP = \begin{bmatrix}
			-2 & 0 & 0 \\
			0 & i & 0 \\
			0 & 0 & -i 
		\end{bmatrix}
	\end{equation*}
	som er en diagonalmatrise hvor elementene er egenverdiene til $A$.
\end{losning}

\begin{oppgave}
Finn egenverdiene til matrisen 
\[
\begin{bmatrix}
2 & 3  \\ 4 & 6 
\end{bmatrix}.
\]
\end{oppgave}

\begin{losning}
	Vi regner ut det karakteristiske polynomet som vanlig og får 
	\begin{equation*}
		\lambda^2 - 8\lambda = \lambda(\lambda - 8)
	\end{equation*}
	som betyr at egenverdiene er $0$ og $8$. Her kan vi observere at determinanten til matrisen er $0$ siden den første raden er halvparten av den andre.
\end{losning}

\begin{oppgave}
Vis at dersom en matrise har egenverdien 0, er den singulær.
\end{oppgave}

\begin{losning} Anta at null er en egenverdi til en matrise $A$.	
\\
\noindent
Ønsker: $A$ er singulær. Vi har sett mange ekvivalente formuleringer av singulær. En av dem er at det finnes en ikke-null løsning av ligningen $A\mathbf{x}=\mathbf{0}$.
\\
\noindent
Antagelsen: Det finnes en ikke-null vektor $\mathbf{v}$ slik at $A\mathbf{v}=0 \mathbf{v}=\mathbf{0}$. 
\\
\noindent
Observasjon: Vi ser at egenvektoren $\mathbf{v}$ er en ikke-null løsning av $A\mathbf{x}=\mathbf{0}$.
\\
\noindent
Konklusjon: $A$ er singulær, som er det vi ønsket å vise.

%	Vi viser dette med en selvmotsigelse. Anta at $A$ ikke er singulær, altså at $A\V y = 0$ kun har den trivielle løsningen. Vi vet at egenverdiene til en matrise $A$ er slik at det finnes en ikke-null vektor $\V v$ slik at
%	\begin{equation*}
%		A\V v = \lambda\V v
%	\end{equation*}
%	Siden vi antar at en av egenverdiene er $\lambda = 0$ vet vi da at
%	\begin{equation*}
%		A\V v = 0		
%	\end{equation*}
%	for en \emph{ikke-null} vektor $\V v$. Altså har ligningssystemet
%	\begin{equation*}
%		A\V y = 0
%	\end{equation*}
%	minst 2 løsninger: $\V v$ og den trivielle løsningen $0$. Men vi antok at $A$ kun hadde den trivielle løsningen, en selvmotsigelse. Altså må $A$ være singulær.
\end{losning}


%
%\begin{oppgave}
%Beregn determinanten.
%\begin{punkt}
%\[
%\begin{bmatrix}
%i & 0 & 0 & 0 & 0 & 0 \\ 0 & i & 0 & 0 & 0 & 0\\ 0 & 0 & i & 0& 0 & 0 \\ 0 & 0& 0  &i & 0 & 0 \\ 0 & 0 & 0 & 0 & i & 0\\ 0 & 0 & 0 & 0 & 0 & i- 
%\end{bmatrix}.
%\]
%\end{punkt}
%\begin{punkt}
%\[
%\begin{bmatrix}
%i & 1 & 0 & 0 & 0 & 0 \\ 0 & i & 0 & 0 & 0  & 0 \\ 0 & 0 & i & 1& 0 & 0 \\ 0 & 0& 0  &i & 0 & 0 \\ 0 & 0 & 0 & 0 & i & 1\\ 0 & 0 & 0 & 0 & 0 & i 
%\end{bmatrix}.
%\]
%\end{punkt}
%\begin{punkt}
%\[
%\begin{bmatrix}
%i & 1 & 0 & 0 & 0 & 0 \\ 1 & i & 0 & 0 & 0  & 0 \\ 0 & 0 & i & 1& 0 & 0 \\ 0 & 0& 1  &i & 0 & 0 \\ 0 & 0 & 0 & 0 & i & 1\\ 0 & 0 & 0 & 0 & 1 & i 
%\end{bmatrix}.
%\]
%\end{punkt}
%
%\begin{punkt}
%\[
%\begin{bmatrix}
%i & 1 & 1 & 1 & 1 & 1 \\ 1 & i & 1 & 1 & 1 & 1\\ 0 & 0 & i & 1& 1 & 1 \\ 0 & 0& 1  &i & 1 & 1 \\ 0 & 0 & 0 & 0 & i & 1\\ 0 & 0 & 0 & 0 & 1 & i 
%\end{bmatrix}.
%\]
%\end{punkt}
%
%\begin{punkt}
%\[
%\begin{bmatrix}
%1 & 1 & 1 & 1 & 1 & 1 \\ 1 & 1 & 1 & 1 & 1 & 1\\ 0 & 0 & 1 & 1& 1 & 1 \\ 0 & 0& 1  &1 & 1 & 1 \\ 0 & 0 & 0 & 0 & 1 & 1\\ 0 & 0 & 0 & 0 & 1 & 1 
%\end{bmatrix}.
%\]
%\end{punkt}
%\end{oppgave}
%
%\begin{losning}
%TODO
%\end{losning}



%\begin{oppgave}
%\begin{punkt}
%Betrakt $A=\begin{bmatrix}
%0 & -1\\
%1 & 0
%\end{bmatrix}$ som en matrise med komplekse tall. Finn alle egenverdiene og egenvektorene til $A$.
%\end{punkt}
%
%\begin{punkt}
%Gi en geometrisk forklaring på hvorfor $A$ ikke har noen reelle egenverdier. Hvorfor gjelder ikke dette for komplekse egenverdier?
%\end{punkt}
%
%\end{oppgave}
%
%\begin{losning}
%
%\begin{punkt}
%Den karakteristiske likningen blir $\lambda^2+1=0$ og har løsning $\pm i$. Egenverdiene er derfor $\pm i$.
%
%Egenvektorer: TODO
%
%\end{punkt}
%
%\begin{punkt}
%Matrisen roterer alle vektorer med 90 grader. Derfor skaleres ingen vektorer med en reell skalar.
%
%\noindent
%Dersom vi tillater komplekse egenverdier har vi 'snurremultiplikasjon'. Å multiplisere hver komponent i en vektor med $i$ roterer er jo nøyaktig å rotere vektoren 90 grader.
%\end{punkt}
%\end{losning}
%
%\begin{oppgave}
%\begin{punkt}
%Forrige øving viste vi at $\mathbb{C}$ er et vektorrom over $\mathbb{R}$ (reelt vektorrom). Gi en basis. Hva er dimensjonen til $\mathbb{C}$? Hva om vi ser på $\mathbb{C}$ som et komplekst vektorrom?
%\end{punkt}
%\begin{punkt}
%Kan du gi en geometrisk forklaring på svaret ditt i \textbf{a)}?
%\end{punkt}
%\end{oppgave}
%
%\begin{losning}
%\begin{punkt}
%Reelt vektorrom: En basis er gitt ved $\B=(1,i)$ slik at dimensjonen er 2.
%
%\noindent
%Kompleks vektorrom: En basis er gitt ved $1$ slik at dimensjonen er 1.
%\end{punkt}
%
%\begin{punkt}
%Hvis vi innfører koordinater med hensyn på $\B$ (se løsningen til del \textbf{a)}) for $\mathbb{C}$ som reelt vektorrom har vi at $[a+ib]=\vv{a}{b}$. Dette betyr at $\mathbb{C}$ oppfører seg helt likt som $\mathbb{R}^2$; skalarmultiplikasjon $t[a+ib]=\vv{ta}{tb}=[(ta)+i(tb)]$ hvor $t$ er et reelt tall. En vektor $a+ib$ kan altså bare skaleres til å gi alle punktene på linjen $t(a+ib)$, $t\in\mathbb{R}$, i det komplekse planet. Men dersom vi betrakter $\mathbb{C}$ som et komplekst vektorrom, tillater vi komplekse skalarer. En vektor $a+ib$ spenner nå ut $w(a+ib)$, $w\in \mathbb{C}$, som er hele det komplekse planet dersom $a+ib\neq 0$ fordi vi har 'snurreganging' ($c+id=\frac{c+id}{a+ib}(a+ib)$).
%
%\noindent
%Moral: En linje i et komplekst vektorrom svarer -- rent geometrisk -- til et plan i et reelt vektorrom. 
%
%\noindent
%Merknad: Til tross for at $\mathbb{C}^n$ (som komplekst vektorrom) og $\mathbb{R}^n$ (som reelt vektorrom) ser ulike ut, må du ikke la deg lure. De oppfører seg helt likt! Derfor burde du tenke på $\mathbb{C}^n$ som $\mathbb{R}^n$ når du jobber med dem i praksis. 
%\end{punkt}
%
%\end{losning}
