% -*- TeX-master: "oving09"; -*-
\oppgaver{11}

\begin{oppgave}
Er kolonnene lineært uavhengige? Hvis ja, finn nullrommet til matrisen.
\begin{punkt}
\[
\begin{pmatrix}
2i & 3i & 4i \\ 3 & 4 & 5 \\ 4 & 5 & 6 
\end{pmatrix}
\]
\end{punkt}

\begin{punkt}
\[
\begin{pmatrix}
2i & 3 & 4 \\ 3 & 4i & 5 \\ 4 & 5 & 6i 
\end{pmatrix}
\]
\end{punkt}

\begin{punkt}
\[
\begin{pmatrix}
2i & 3 & 4 \\ 3i & 4 & 5 \\ 4i & 5 & 6 
\end{pmatrix}
\]
\end{punkt}
\end{oppgave}

\begin{losning}
TODO
\end{losning}


\begin{oppgave}
Vis at egenverdiene til rotasjonsmatrisen
\[
\begin{pmatrix}
\cos \theta & -\sin \theta  \\ \sin \theta & \cos \theta  
\end{pmatrix}
\]
er komplekse.
\end{oppgave}

\begin{losning}
TODO
\end{losning}

\begin{oppgave}
Finn egenverdier og tilhørende egenrom til matrisen 
\[
\begin{pmatrix}
0 & 1 & 1 \\ 1 & 0 & 1 \\ 1 & 1 & 0 
\end{pmatrix}.
\]
\end{oppgave}

\begin{losning}
TODO
\end{losning}

\begin{oppgave}
Finn $a$ slik at vektoren 
\[
\begin{pmatrix}
a \\ -6i \\ -12 
\end{pmatrix}
\]
ligger i 'planet' utspent av 
\[
\begin{pmatrix}
-3 \\ 2i \\ 8 
\end{pmatrix}
\quad 
\text{og}
\quad 
\begin{pmatrix}
1 \\ 0 \\ -2 
\end{pmatrix}.
\]

\end{oppgave}

\begin{losning}
TODO
\end{losning}


\begin{oppgave}
En oppgave a la oppgave 11 i øving 5, men kompleks
\end{oppgave}

\begin{losning}
TODO
\end{losning}

\begin{oppgave}
En oppgave a la oppgave 11 i øving 5, men symmetrisk matrise
\end{oppgave}

\begin{losning}
TODO
\end{losning}


%\begin{oppgave}
%\begin{punkt}
%Betrakt $A=\begin{bmatrix}
%0 & -1\\
%1 & 0
%\end{bmatrix}$ som en matrise med komplekse tall. Finn alle egenverdiene og egenvektorene til $A$.
%\end{punkt}
%
%\begin{punkt}
%Gi en geometrisk forklaring på hvorfor $A$ ikke har noen reelle egenverdier. Hvorfor gjelder ikke dette for komplekse egenverdier?
%\end{punkt}
%
%\end{oppgave}
%
%\begin{losning}
%
%\begin{punkt}
%Den karakteristiske likningen blir $\lambda^2+1=0$ og har løsning $\pm i$. Egenverdiene er derfor $\pm i$.
%
%Egenvektorer: TODO
%
%\end{punkt}
%
%\begin{punkt}
%Matrisen roterer alle vektorer med 90 grader. Derfor skaleres ingen vektorer med en reell skalar.
%
%\noindent
%Dersom vi tillater komplekse egenverdier har vi 'snurremultiplikasjon'. Å multiplisere hver komponent i en vektor med $i$ roterer er jo nøyaktig å rotere vektoren 90 grader.
%\end{punkt}
%\end{losning}
%
%\begin{oppgave}
%\begin{punkt}
%Forrige øving viste vi at $\mathbb{C}$ er et vektorrom over $\mathbb{R}$ (reelt vektorrom). Gi en basis. Hva er dimensjonen til $\mathbb{C}$? Hva om vi ser på $\mathbb{C}$ som et komplekst vektorrom?
%\end{punkt}
%\begin{punkt}
%Kan du gi en geometrisk forklaring på svaret ditt i \textbf{a)}?
%\end{punkt}
%\end{oppgave}
%
%\begin{losning}
%\begin{punkt}
%Reelt vektorrom: En basis er gitt ved $\B=(1,i)$ slik at dimensjonen er 2.
%
%\noindent
%Kompleks vektorrom: En basis er gitt ved $1$ slik at dimensjonen er 1.
%\end{punkt}
%
%\begin{punkt}
%Hvis vi innfører koordinater med hensyn på $\B$ (se løsningen til del \textbf{a)}) for $\mathbb{C}$ som reelt vektorrom har vi at $[a+ib]=\vv{a}{b}$. Dette betyr at $\mathbb{C}$ oppfører seg helt likt som $\mathbb{R}^2$; skalarmultiplikasjon $t[a+ib]=\vv{ta}{tb}=[(ta)+i(tb)]$ hvor $t$ er et reelt tall. En vektor $a+ib$ kan altså bare skaleres til å gi alle punktene på linjen $t(a+ib)$, $t\in\mathbb{R}$, i det komplekse planet. Men dersom vi betrakter $\mathbb{C}$ som et komplekst vektorrom, tillater vi komplekse skalarer. En vektor $a+ib$ spenner nå ut $w(a+ib)$, $w\in \mathbb{C}$, som er hele det komplekse planet dersom $a+ib\neq 0$ fordi vi har 'snurreganging' ($c+id=\frac{c+id}{a+ib}(a+ib)$).
%
%\noindent
%Moral: En linje i et komplekst vektorrom svarer -- rent geometrisk -- til et plan i et reelt vektorrom. 
%
%\noindent
%Merknad: Til tross for at $\mathbb{C}^n$ (som komplekst vektorrom) og $\mathbb{R}^n$ (som reelt vektorrom) ser ulike ut, må du ikke la deg lure. De oppfører seg helt likt! Derfor burde du tenke på $\mathbb{C}^n$ som $\mathbb{R}^n$ når du jobber med dem i praksis. 
%\end{punkt}
%
%\end{losning}
