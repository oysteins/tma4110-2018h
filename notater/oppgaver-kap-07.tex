% -*- TeX-master: "oving05"; -*-
\oppgaver{7}

\begin{oppgave}

Finn egenverdier og tilhørende egenvektorer til følgende matriser.

\begin{punkt}
$$\begin{bmatrix}
1 & 2\\
2 & 1
\end{bmatrix}$$
\end{punkt}

\begin{punkt}
$$\begin{bmatrix}
1 & 2 & 0\\
2 & 1 & 0\\
0 & 0 & 0
\end{bmatrix}$$
\end{punkt}


\begin{punkt}
$$\begin{bmatrix}
4 & 0 & 3\\
-2 & 1 & 3\\
-2 & 0 & 4
\end{bmatrix}$$
\end{punkt}


\begin{punkt}
$$\begin{bmatrix}
4 & 2 & 3\\
-1 & 1 & -3\\
2 & 4 & 9
\end{bmatrix}$$

\noindent
Hint: Hvis ikke $\lambda=1$ fungerer, prøv $\lambda=2$. Hvis ikke $\lambda=2$ fungerer, prøv $\lambda=3$\ldots
\end{punkt}


\end{oppgave}


\begin{losning}
I svarene nedenfor vil vi liste opp egenvektorer, og deretter egenrommet til hver egenverdi i samme rekkefølge.
\begin{punkt}
Egenverdier: 3, $-1$. Egenrom: linjen utspent av $\vv{1}{1}$, $\vv{-1}{1}$ utenom origo.
\end{punkt}

\begin{punkt}
Egenverdier: 3, $-1$, 0. Egenrom: Linjen utspent av $\begin{bmatrix} 
1\\
1\\
0
\end{bmatrix}$, $\begin{bmatrix} 
-1\\
1\\
0
\end{bmatrix}$, $\begin{bmatrix} 
0\\
0\\
1
\end{bmatrix}$ utenom origo.

\end{punkt}

\begin{punkt}
Egenverdier: 1, 2, 3. Egenrom: Linjen utspent av $\begin{bmatrix} 
0\\
1\\
0
\end{bmatrix}$, $\begin{bmatrix} 
-1\\
2\\
2
\end{bmatrix}$, $\begin{bmatrix} 
-1\\
1\\
1
\end{bmatrix}$ utenom origo.
\end{punkt}

\begin{punkt}
Egenverdi: 3, 8. Egenrom: Planet utspent av $\begin{bmatrix} 
-2\\
1\\
0
\end{bmatrix}$ og $\begin{bmatrix} 
-3\\
0\\
1
\end{bmatrix}$ utenom origo, linjen utspent av $\begin{bmatrix} 
1\\
-1\\
2
\end{bmatrix}$ utenom origo.
\end{punkt}



\end{losning}




\begin{oppgave}

\begin{punkt}
Regn ut egenverdiene til $$\begin{bmatrix}
1 & 0\\
0 & 0
\end{bmatrix},
$$ og finn tilhørende egenrom. 
\end{punkt}

\begin{punkt}
Skissér egenrommene.

\noindent 

Hint: Du har løst denne oppgaven tidligere.
\end{punkt}

\end{oppgave}

\begin{losning}

\begin{punkt}
Egenverdi: 0,1. Egenrom: $x$-aksen og $y$-aksen (uten origo).
\end{punkt}

\begin{punkt}
Vektorer langs $y$-aksen blir null-vektoren ved multiplikasjon av $A$; vektorer langs $x$-aksen er uendret ved muliplikasjon av $A$.
\end{punkt}

\end{losning}


\begin{oppgave}
\begin{punkt}
Regn ut egenvektorene til $$\frac{1}{2}\begin{bmatrix}
1 & 1\\
1 & 1
\end{bmatrix},
$$ og finn tilhørende egenrom.

\end{punkt}

\begin{punkt}
Skissér egenrommene.
\end{punkt}

\end{oppgave}


\begin{losning}

\begin{punkt}
Egenverdi: 0,1. Egenrom: linjene utspent av $(1,1)$ og $(-1,1)$ (uten origo).
\end{punkt}

\begin{punkt}
Vektorer langs $(1,-1)$-linjen blir null-vektoren ved multiplikasjon; vektorer langs $(-1,1)$-linjen er uendret ved muliplikasjon.

\noindent
Merk: Man kan tolke $x$-aksen som linjen utspent av $(1,0)$ og $y$-aksen som linjen utspent av $(-1,1)$. Dette er altså helt lik situasjonen som i forrige oppgave, men nå er egenrommene rotert med 45 grader.
\end{punkt}

\end{losning}



\begin{oppgave}

\begin{punkt}
Vis at $$\frac{1}{2}\begin{bmatrix}
0 & -1\\
1 & 0
\end{bmatrix},
$$ ikke har noen egenverdier.

\end{punkt}

\begin{punkt}
Gi en geometrisk forklaring på del \textbf{a)}.

\noindent
Hint: Du har sett på de geometriske egenskapene til matrisen tidligere.
\end{punkt}

\end{oppgave}


\begin{losning}

\begin{punkt}
Dersom vi prøver å løse $\text{det}(A-\lambda I)=0$ får vi polynomlikningen $\lambda^2+1=0$. Denne har ingen (reelle) løsninger.
\end{punkt}

\begin{punkt}
Matrisen roterer vektorer med 90 grader. Men en likning på formen $A\V{x}=c\V{x}$ betyr at $A$ skalerer $\V{x}$ med en faktor $c$ uten at $\V{x}$ roteres.
\end{punkt}

\end{losning}


\begin{oppgave}
\begin{punkt}
Regn ut egenverdiene til $$A=\begin{bmatrix}
1 & 0 & 0 & 0\\
0 & 2 & 0 & 0\\
0 & 0 & -5 & 0\\
0 & 0 & 0 & 77
\end{bmatrix},
$$ og finn tilhørende egenrom.

\end{punkt}

\begin{punkt}
Gi en geometrisk tolkning av egenrommene.
\end{punkt}

\begin{punkt}
$A$ er en $4\times 4$-matrise. Er det alltid enkelt å finne egenverdiene til en $4\times 4$-matrise? Mer generelt, er det alltid enkelt å finne egenverdiene til $n\times n$-matriser?
\end{punkt}

\end{oppgave}


\begin{losning}

\begin{punkt}
Egenverdi 1, 2, $-5$, 77. Egenrom: $x_1$-aksen, $x_2$-aksen, $x_3$-aksen, $x_4$-aksen.
\end{punkt}

\begin{punkt}
Vektorer på $x_1$-aksen endres ikke; vektorer på $x_2$-aksen skaleres med 2; vektorer på $x_3$-aksen flippes (multiplikasjon $-1$) og skaleres med 5; vektorer langs $x_4$-aksen skaleres med 77.
\end{punkt}

\begin{punkt}
Nei. Vi må løse en fjerdegradslinking, som kan bli meget vanskelig. For generell $n$: Nei; vi må løse $n$-tegradslikninger.
\end{punkt}

\end{losning}



\begin{oppgave}
La $A$ være en $n\times n$-matrise slik at $A^2=A$. Vis at $A$ kun kan ha null og en som egenverdier.

\noindent
Hint: $A\V{x}=c\V{x}$ og $A^2\V{x}=c^2\V{x}$, hva antar vi?
\end{oppgave}

\begin{losning}
Full løsning: Hvis $c$ er en egenverdi tilfredstiller den -- per definisjon -- $A\V{x}=c\V{x}$ for en ikke-null vektor $\V{x}$. Kombiner dette med antagelsen om at $A^2\V{x}=A\V{x}$ for å se at $\V{x}$ tilfredstiller $c^2\V{x}=c\V{x}$. Omformuler denne likningen til $(c^2-c)\V{x}=0$. Ettersom $\V{x}$ ikke er null-vektoren, må en komponent i $\V{x}$ ikke være lik null (hvorfor?), og derfor må $c^2-c=0$. Dette er en andregradslikning med løsning $c=0$ eller $c=1$.
\end{losning}

\begin{oppgave}

\begin{punkt}
Finn vektorene som svarer til at $$\V{e}_1\vv{1}{0}\quad \text{og}\quad\V{e}_2= \vv{0}{1}$$ er blitt rotert med $\theta$ radianer.
\end{punkt}

\begin{punkt}
Utled formelen for $2 \times 2$-matrisen $T_\theta$ som roterer vektorer $\theta$ radianer mot klokken ved multiplikasjon.

\noindent
Hint: Hva skjer når du multipliserer $T_\theta$ med $\V{e}_1$ og $\V{e}_2$?
\end{punkt}

\begin{punkt}
For hvilke verdier av $\theta$ har $T_\theta$ en egenverdi? Gi en geometrisk forklaring.
\end{punkt}


\end{oppgave}


\begin{losning}

\begin{punkt}
$\vv{\cos\theta}{\sin\theta}$, $\vv{-\sin\theta}{\cos\theta}$
\end{punkt}

\begin{punkt}
Matrisen er akkurat den som har svaret i del \textbf{a)} som kolonner:
$$T_\theta=\begin{bmatrix}
\cos\theta & -\sin\theta\\
\sin\theta & \cos\theta
\end{bmatrix}$$
\end{punkt}

\begin{punkt}
Vi må ha $\theta=0$; ingen rotasjon.
\end{punkt}

\end{losning}

\begin{oppgave}
Avgjør om følgende påstander er sanne eller ikke. Begrunn svaret ditt.

\begin{punkt}
En $n\times n$-matrise har alltid $n$ egenverdier.
\end{punkt}

\begin{punkt}
Dersom $A$ har en ikke-null egenverdi $c$, så kan ikke $A$ være lik null-matrisen.
\end{punkt}

\begin{punkt}
To egenvektorer til en matrise $A$ som svarer til to ulike egenverdier kan være lineært avhengige.
\end{punkt}

\begin{punkt}
To egenvektorer til en matrise $A$ som svarer til to like egenverdier kan være lineært uavhengige.
\end{punkt}



\end{oppgave}

\begin{losning}

\begin{punkt}
Feil. Vi får generelt et $n$-tegradspolynom som kan ha null til $n$ løsninger (Det er altså maksimalt $n$ egenverdier). Det har vært mange eksempler på dette tidligere i øvingen.
\end{punkt}

\begin{punkt}
Sant. Vi har -- per definisjon -- en ikke-null vektor $\V{x}$ som tilfredstiller $A\V{x}=c\V{x}$ hvor $c\neq 0$. Derfor har vi funnet en vektor $\V{x}$ slik at $A\V{x}$ ikke er lik null-vektoren. Men da kan $A$ umulig være null-matrisen; hvis alle elementene i $A$ var lik null ville $A\V{x}$ vært lik null for alle valg av $\V{x}$.
\end{punkt}

\begin{punkt}
Feil. To egenvektorer som har ulike egenverdier er alltid lineært uavhengige. 

\noindent
Hint: Anta at $A\V{v}_1=c_1\V{v}_1$ og $A\V{v}_1=c_1\V{v}_1$ hvor $c_1\neq c_2$. Hvis også $\V{v}_1=a\V{v}_2$ for en konstant $a$ kan vi kombinere likningene for å få at $$A\V{v}_1=a(A\V{v}_2)=ac_2\V{v}_2=c_2\V{v}_1.$$ Men nå er $A\V{v}_1$ lik $c_1\V{v}_1$ og $c_2\V{v}_2$, som gir $(c_1-c_2)\V{v}_1=\V{0}$. Vektoren $\V{v}_1$ er ikke null-vektoren -- fordi den er en egenvektor -- og derfor må $c_1=c_2$. Dette er altså umulig.

\noindent
Intuisjon: $A$ skalerer en egenvektor med tilhørende egenverdi, og på grunn av egenskapen $A(c\V{x})=c(A\V{x})=c\cdot\text{egenverdi}\cdot\V{x}=\text{egenverdi}\cdot(c\V{x})$, skaleres også alle vektorene på linjen utspent av $\V{x}$ med denne egenverdien. To lineært avhengige vektorer ligger på samme linje -- i samme egenrom -- altså må $A$ skalere dem med samme egenverdi. Det er akkurat dette som er skrevet ut matematisk i hintet ovenfor!
\end{punkt}

\begin{punkt}
Sant. Det er et eksempel i en tidligere oppgave.
\end{punkt}

\end{losning}

\begin{oppgave}
La $A$ være en $n\times n$-matrise. Vis at $A$ og dens transponerte $A\tr$ har like egenverdier.

\noindent
Hint: Husk at determinanten til en matrise $B$ og dens transponerte $B\tr$ er like.
\end{oppgave}

\begin{losning}
Hint: Egenverdiene til $A$ er løsninger på polynomet $\text{det}(A-\lambda I)=0$. Tilsvarende er egenverdiene til $A\tr$ løsninger på polynomet $\text{det}(A\tr-\lambda I)=0$. Observer at $(A-\lambda I)\tr=A\tr-\lambda I$. Bruk hintet i oppgaveteksten på matrisen $B=A-\lambda\tr$.
\end{losning}
