% -*- TeX-master: "oving05"; -*-
\oppgaver{7}

\begin{oppgave}
Finn egenverdier og tilhørende egenvektorer til følgende matriser.
\\[-8pt]
\begin{minipage}[t]{.20\textwidth}
\begin{punkt}
$
\begin{bmatrix}
1 & 2\\
2 & 1
\end{bmatrix}$
\end{punkt}

\begin{punkt}
$\begin{bmatrix}
1 & 2 & 0\\
2 & 1 & 0\\
0 & 0 & 0
\end{bmatrix}$
\end{punkt}
\end{minipage}
\begin{minipage}[t]{.20\textwidth}
\begin{punkt}
$\begin{bmatrix}
0 & 1 \\
0 & 0 
\end{bmatrix}$

\end{punkt}


\begin{punkt}
$\begin{bmatrix}
4 & 2 & 3\\
-1 & 1 & -3\\
2 & 4 & 9
\end{bmatrix}$
\end{punkt}
\end{minipage}
\\[5pt]
Hint for del~\textbf{d)}: Polynomdivisjon. Hvis ikke $\lambda=1$ fungerer, prøv
$\lambda=2$. Hvis ikke $\lambda=2$ fungerer, prøv $\lambda=3$, \ldots
\end{oppgave}


\begin{losning}
I svarene nedenfor vil vi liste opp egenvektorer, og deretter egenrommet til hver egenverdi i samme rekkefølge.

\begin{punkt}
Egenverdier: 3, $-1$. Egenrom: linjen utspent av $\vv{1}{1}$, $\vv{-1}{1}$.
\end{punkt}

\begin{punkt}
Egenverdier: 3, $-1$, 0. Egenrom: Linjen utspent av $\begin{bmatrix} 
1\\
1\\
0
\end{bmatrix}$, $\begin{bmatrix} 
-1\\
1\\
0
\end{bmatrix}$, $\begin{bmatrix} 
0\\
0\\
1
\end{bmatrix}$.

\end{punkt}

\begin{punkt}
Egenverdi: 0. Egenrom: Linjen utspent av $\begin{bmatrix} 
1\\
0
\end{bmatrix}$.

\noindent
Merk: Dette er et eksempel på at geometrisk multiplisitet er strengt mindre enn algebraisk multiplisitet.
\end{punkt}

\begin{punkt}
Egenverdi: 3, 8. Egenrom: Planet utspent av $\begin{bmatrix} 
-2\\
1\\
0
\end{bmatrix}$ og $\begin{bmatrix} 
-3\\
0\\
1
\end{bmatrix}$, linjen utspent av $\begin{bmatrix} 
1\\
-1\\
2
\end{bmatrix}$.

\noindent:
Merk: Dette er et eksempel på at egenvektorer med lik egenverdi kan være lineært uavhengige.
\end{punkt}



\end{losning}




\begin{oppgave}

\begin{punkt}
Regn ut egenverdiene til $$\begin{bmatrix}
1 & 0\\
0 & 0
\end{bmatrix},
$$ og finn tilhørende egenrom. 
\end{punkt}

\begin{punkt}
Skissér egenrommene.

\noindent 
Hint: Du har løst denne oppgaven tidligere.
\end{punkt}

\end{oppgave}

\begin{losning}

\begin{punkt}
Egenverdi: 0,1. Egenrom: $x$-aksen og $y$-aksen.
\end{punkt}

\begin{punkt}
Vektorer langs $y$-aksen blir null-vektoren ved multiplikasjon av $A$; vektorer langs $x$-aksen er uendret ved muliplikasjon av $A$.
\end{punkt}

\end{losning}


\begin{oppgave}
\begin{punkt}
Regn ut egenvektorene til $$\frac{1}{2}\begin{bmatrix}
1 & 1\\
1 & 1
\end{bmatrix},
$$ og finn tilhørende egenrom.

\end{punkt}

\begin{punkt}
Skissér egenrommene.
\end{punkt}

\end{oppgave}


\begin{losning}

\begin{punkt}
Egenverdi: 0,1. Egenrom: linjene utspent av $(1,1)$ og $(-1,1)$.
\end{punkt}

\begin{punkt}
Vektorer langs $(1,-1)$-linjen blir null-vektoren ved multiplikasjon; vektorer langs $(-1,1)$-linjen er uendret ved muliplikasjon.

\noindent
Merk: Man kan tolke $x$-aksen som linjen utspent av $(1,0)$ og $y$-aksen som linjen utspent av $(-1,1)$. Dette er altså helt lik situasjonen som i forrige oppgave, men nå er egenrommene rotert med 45 grader.
\end{punkt}

\end{losning}



\begin{oppgave}

\begin{punkt}
Vis at matrisen
\[
\begin{bmatrix}
 0 & 1\\
-1 & 0
\end{bmatrix},
\]
ikke har noen egenverdier.
\end{punkt}

\begin{punkt}
Gi en geometrisk forklaring på del \textbf{a)}.
\end{punkt}

\end{oppgave}


\begin{losning}

\begin{punkt}
Dersom vi prøver å løse $\text{det}(A-\lambda I)=0$ får vi polynomlikningen $\lambda^2+1=0$. Denne har ingen (reelle) løsninger.
\end{punkt}

\begin{punkt}
Matrisen roterer vektorer med $-90$ grader. Men en likning på formen $A\V{x}=c\V{x}$ betyr at $A$ skalerer $\V{x}$ med en faktor $c$ uten at $\V{x}$ roteres.
\end{punkt}

\end{losning}



\begin{oppgave}

\begin{punkt}
Finn vektorene som svarer til at $$\V{e}_1 = \vv{1}{0}\quad \text{og}\quad\V{e}_2= \vv{0}{1}$$ er blitt rotert med $\theta$ radianer.
\end{punkt}

\begin{punkt}
Utled formelen for $2 \times 2$-matrisen $T_\theta$ som roterer vektorer $\theta$ radianer mot klokken ved multiplikasjon.

\noindent
Hint: Hva skjer når du ganger $T_\theta$ med $\V{e}_1$ og $\V{e}_2$?
\end{punkt}

\begin{punkt}
For hvilke verdier av $\theta$ har $T_\theta$ en egenverdi? Gi en geometrisk forklaring.
\end{punkt}


\end{oppgave}


\begin{losning}

\begin{punkt}
$\vv{\cos\theta}{\sin\theta}$, $\vv{-\sin\theta}{\cos\theta}$
\end{punkt}

\begin{punkt}
Matrisen er akkurat den som har svaret i del \textbf{a)} som kolonner:
$$T_\theta=\begin{bmatrix}
\cos\theta & -\sin\theta\\
\sin\theta & \cos\theta
\end{bmatrix}$$
\end{punkt}

\begin{punkt}
Vi må ha $\theta=0$; ingen rotasjon.
\end{punkt}

\end{losning}

\begin{oppgave}
Avgjør om følgende påstander er sanne eller ikke. Begrunn svaret ditt.

\begin{punkt}
En $n\times n$-matrise har alltid $n$ egenverdier.
\end{punkt}

\begin{punkt}
Dersom $A$ har en ikke-null egenverdi $c$, så kan ikke $A$ være lik null-matrisen.
\end{punkt}


\begin{punkt}
To egenvektorer til en matrise $A$ som svarer til samme egenverdi kan være lineært uavhengige.
\end{punkt}



\end{oppgave}

\begin{losning}

\begin{punkt}
Feil. Vi får generelt et $n$-tegradspolynom som kan ha alt fra null til $n$ løsninger (det er maksimalt $n$ egenverdier). Det har vært mange eksempler på dette tidligere i øvingen.
\end{punkt}

\begin{punkt}
Sant. Vi har -- per definisjon -- en ikke-null vektor $\V{x}$ som tilfredsstiller $A\V{x}=c\V{x}$ hvor $c\neq 0$. Derfor har vi funnet en vektor $\V{x}$ slik at $A\V{x}$ ikke er lik null-vektoren. Men da kan $A$ umulig være null-matrisen; hvis alle elementene i $A$ var lik null ville $A\V{x}$ vært lik null for alle valg av $\V{x}$.
\end{punkt}


\begin{punkt}
Sant. Det er et eksempel i en tidligere oppgave.
\end{punkt}

\end{losning}

\begin{oppgave}
La $A$ være en $n\times n$-matrise. Vis at $A$ og dens transponerte $A\tr$ har like egenverdier.

\noindent
Hint: Husk at determinanten til en matrise $B$ og dens transponerte $B\tr$ er like.
\end{oppgave}

\begin{losning}
Hint: Egenverdiene til $A$ er løsninger på polynomet $\text{det}(A-\lambda I)=0$. Tilsvarende er egenverdiene til $A\tr$ løsninger på polynomet $\text{det}(A\tr-\lambda I)=0$. Observer at $(A-\lambda I)\tr=A\tr-\lambda I$. Bruk hintet i oppgaveteksten på matrisen $B=A-\lambda\tr$.
\end{losning}



\begin{oppgave}
\begin{punkt}
Regn ut egenverdiene til $$A=\begin{bmatrix}
1 & 2 & 3 & 4\\
0 & 2 & 3 & 4\\
0 & 0 & -5 & 0\\
0 & 0 & 0 & 77
\end{bmatrix}.
$$

\end{punkt}

\begin{punkt}
Finn egenrommene til de ulike egenverdiene.
\end{punkt}

\begin{punkt}
$A$ er en $4\times 4$-matrise. Er det alltid enkelt å finne egenverdiene til en $4\times 4$-matrise? Mer generelt, er det alltid enkelt å finne egenverdiene til $n\times n$-matriser?
\end{punkt}

\end{oppgave}


\begin{losning}

\begin{punkt}
Egenverdier: 1, 2, $-5$, 77. 

\noindent
Hint: Hva er determinanten til en matrise på trappeform?
\end{punkt}

\begin{punkt}

1: Linjen utspent av $\begin{bmatrix}
1\\
0\\
0\\
0
\end{bmatrix}$; 2: Linjen utspent av $\begin{bmatrix}
2\\
1\\
0\\
0
\end{bmatrix}$; $-5$: linjen utspent av $\begin{bmatrix}
-5\\
-6\\
14\\
0
\end{bmatrix}$; 77: linjen utspent av $\begin{bmatrix}
-5\\
-6\\
14\\
0
\end{bmatrix}$.

\end{punkt}

\begin{punkt}
Nei. Vi må løse en fjerdegradslinking, som kan bli meget vanskelig. For generell $n$: Nei; vi må løse $n$-tegradslikninger.
\end{punkt}

\end{losning}


\begin{oppgave}
La $A$ være følgende matrise:
\[
A =
\begin{bmatrix}
 28 & 30 & -20 & -2 \\
 6 & 40 & -10 & -4 \\
 4 & 10 & 20 & -6 \\
 2 & 20 & -30 & 32
\end{bmatrix}
\]
\begin{punkt}
Hvilke av vektorene
\[
\vvvv{0}{0}{0}{0},\ %
\vvvv{1}{2}{3}{4},\ %
\vvvv{3}{2}{2}{1},\ %
\vvvv{1}{0}{0}{0},\ %
\vvvv{2}{1}{2}{3},\ %
\vvvv{3}{2}{1}{2}\ \text{og}\ %
\vvvv{4}{3}{2}{1}
\]
er egenvektorer for~$A$?
\end{punkt}
\begin{punkt}
Finn alle egenverdiene til~$A$, og de tilhørende egenrommene.
\end{punkt}
\end{oppgave}

\begin{losning}
\begin{punkt}
Vi vet at nullvektoren per definisjon ikke er en egenvektor.  Vi
ganger $A$ med hver av de andre vektorene, og ser hva vi får:
\begin{align*}
A \vvvv{1}{2}{3}{4} &= \vvvv{20}{40}{60}{80} = 20 \vvvv{1}{2}{3}{4} \\
A \vvvv{3}{2}{2}{1} &= \vvvv{102}{74}{66}{18} \\
A \vvvv{1}{0}{0}{0} &= \vvvv{28}{6}{4}{2} \\
A \vvvv{2}{1}{2}{3} &= \vvvv{40}{20}{40}{60} = 20 \vvvv{2}{1}{2}{3} \\
A \vvvv{3}{2}{1}{2} &= \vvvv{120}{80}{40}{80} = 40 \vvvv{3}{2}{1}{2} \\
A \vvvv{4}{3}{2}{1} &= \vvvv{160}{120}{80}{40} = 40 \vvvv{4}{3}{2}{1}
\end{align*}
Vi fant fire egenvektorer, tilhørende de to egenverdiene $20$ og~$40$.
Vi ser ganske lett at de to vektorene
\[
\vvvv{3}{2}{2}{1}
\quad\text{og}\quad\vvvv{1}{0}{0}{0}
\]
ikke er egenvektorer.
\end{punkt}
\begin{punkt}
La
\[
\V{v}_1 = \vvvv{1}{2}{3}{4}
\quad\text{og}\quad
\V{v}_1 = \vvvv{2}{1}{2}{3}
\]
være egenvektorene fra del~(a) som hører til egenverdien~$20$, og la
\[
\V{w}_1 = \vvvv{3}{2}{1}{2}
\quad\text{og}\quad
\V{w}_1 = \vvvv{4}{3}{2}{1}
\]
være egenvektorene fra del~(a) som hører til egenverdien~$40$.  Du kan
sjekke at $\V{v}_1$ og~$\V{v}_2$ er lineært uavhengige (ingen av dem
er en skalar ganger den andre) og at $\V{w}_1$ og~$\V{w}_2$ er lineært
uavhengige.  Det betyr at
\[
\V{v}_1,\ \V{v}_2,\ \V{w}_1\ \text{og}\ \V{w}_2
\]
er lineært uavhengige, siden vi vet at det ikke kan finnes lineære
avhengigheter mellom egenvektorer som hører til forskjellige
egenverdier (teorem~\ref{thm:egenvektorer-lin-uavh}~(a)).

(Merk at det er nødvendig å først sjekke at $\V{v}_1$ og~$\V{v}_2$ er
lineært uavhengige og at $\V{w}_1$ og~$\V{w}_2$ er lineært uavhengige.
Du kan ikke bare bruke teorem~\ref{thm:egenvektorer-lin-uavh}~(a)
direkte på alle de fire vektorene, fordi de ikke hører til fire
forskjellige egenverdier.)

Hvis det nå finnes en vektor $\V{u}$ som enten
\begin{enumerate}
\item \ldots{} er en egenvektor tilhørende~$20$ som ikke ligger i
$\Sp \{ \V{v}_1, \V{v}_2 \}$, eller
\item \ldots{} er en egenvektor tilhørende~$40$ som ikke ligger i
$\Sp \{ \V{w}_1, \V{w}_2 \}$, eller
\item \ldots{} er en egenvektor som tilhører en annen egenverdi enn
$20$ eller~$40$,
\end{enumerate}
så får vi at
\[
\V{v}_1,\ \V{v}_2,\ \V{w}_1\ \V{w}_2\ \text{og}\ \V{u}
\]
er fem lineært uavhengige vektorer i~$\R^4$.  Det er ikke mulig, så
det kan ikke finnes noen slik vektor~$\V{u}$.

Dette betyr at vi har funnet alle egenverdier og egenvektorer for~$A$,
og de er:
\begin{align*}
&\text{Egenverdien~$20$ med egenrom $\Sp \{ \V{v}_1, \V{v}_2 \}$} \\
&\text{Egenverdien~$40$ med egenrom $\Sp \{ \V{w}_1, \V{w}_2 \}$}
\end{align*}
\end{punkt}
\end{losning}



\begin{oppgave}
La $A$ være en $n\times n$-matrise slik at $A^2=A$.
Hva kan du da si om egenverdiene til~$A$?
%Vis at $A$ kun kan ha null og en som egenverdier.

\noindent
Hint:
Prøv å finne noen forskjellige matriser~$A$ som er slik at $A^2 = A$.
Kan du finne en slik matrise som ikke har noen egenverdier?
En som har én egenverdi?  To egenverdier?  Flere enn to?
% $A\V{x}=c\V{x}$ og $A^2\V{x}=c^2\V{x}$, hva antar vi?
\end{oppgave}

\begin{losning}
Full løsning: Hvis $c$ er en egenverdi tilfredsstiller den -- per definisjon -- $A\V{x}=c\V{x}$ for en ikke-null vektor $\V{x}$. Kombiner dette med antagelsen om at $A^2\V{x}=A\V{x}$ for å se at $\V{x}$ tilfredsstiller $c^2\V{x}=c\V{x}$. Omformuler denne likningen til $(c^2-c)\V{x}=0$. Ettersom $\V{x}$ ikke er null-vektoren, må en komponent i $\V{x}$ ikke være lik null (hvorfor?), og derfor må $c^2-c=0$. Dette er en andregradslikning med løsning $c=0$ eller $c=1$.
\end{losning}



\begin{oppgave}
La $A$ være en $n \times n$-matrise som har $n$ forskjellige
egenverdier
$\lambda_1$, $\lambda_2$, \ldots, $\lambda_n$.
Lag en $n \times n$-matrise
\[
V = \begin{bmatrix} \V{v}_1 & \V{v}_2 & \cdots & \V{v}_n \end{bmatrix},
\]
der $\V{v}_1$ er en egenvektor som hører til egenverdien~$\lambda_1$,
og $\V{v}_2$ er en egenvektor som hører til egenverdien~$\lambda_2$,
og så videre.
\begin{punkt}
Kan du finne ut om matrisen~$V$ er inverterbar eller ikke?
\end{punkt}
\begin{punkt}
Dersom $V$ er inverterbar, hvordan ser matrisen $V^{-1} A V$ ut?
\end{punkt}
\begin{punkt}
Finn en $3 \times 3$-matrise som har egenverdier $1$, $2$ og~$3$, med
tilhørende egenvektorer
\[
\vvv{1}{3}{2},\quad
\vvv{2}{6}{5}\quad\text{og}\quad
\vvv{1}{4}{2}.
\]
\end{punkt}
\end{oppgave}

\begin{losning}
\begin{punkt}
Vi vet at egenvektorer som hører til forskjellige egenverdier er
lineært uavhengige, og da følger det fra
teorem~\ref{thm:karakterisering-inverterbar} at matrisen~$V$ er
inverterbar.
\end{punkt}
\begin{punkt}
Vi regner ut
\begin{align*}
V^{-1} A V
&= V^{-1} \begin{bmatrix} A \V{v}_1 & A \V{v}_2 & \cdots & A \V{v}_n \end{bmatrix} \\
&= V^{-1} \begin{bmatrix} \lambda_1 \V{v}_1 & \lambda_2 \V{v}_2 & \cdots & \lambda_n \V{v}_n \end{bmatrix} \\
&= \begin{bmatrix} V^{-1} \lambda_1 \V{v}_1 & V^{-1} \lambda_2 \V{v}_2 & \cdots & V^{-1} \lambda_n \V{v}_n \end{bmatrix} \\
&= \begin{bmatrix} \lambda_1 V^{-1} \V{v}_1 & \lambda_2 V^{-1} \V{v}_2 & \cdots & \lambda_n V^{-1} \V{v}_n \end{bmatrix} \\
&=
D
\begin{bmatrix} V^{-1} \V{v}_1 & V^{-1} \V{v}_2 & \cdots & V^{-1} \V{v}_n \end{bmatrix} \\
&= D V^{-1} V \\
&= D,
\end{align*}
der
\[
D =
\begin{bmatrix}
\lambda_1 & 0         & \cdots & 0 \\
0         & \lambda_2 & \cdots & 0 \\
\vdots    & \vdots    & \ddots & \vdots \\
0         & 0         & \cdots & \lambda_n
\end{bmatrix}
\]
er diagonalmatrisen bestående av egenverdiene til~$A$.

Da kan vi dessuten legge merke til at vi har
\[
A = V V^{-1} A V V^{-1} = V D V^{-1}.
\]
(Oppgaven spurte ikke om dette, men det er likevel en interessant
observasjon, og den hjelper oss med å løse neste deloppgave.)
\end{punkt}
\begin{punkt}
Lag en diagonalmatrise~$D$ med egenverdiene på diagonalen, og en
matrise~$V$ med egenvektorene som kolonner:
\[
D =
\begin{bmatrix}
1 & 0 & 0 \\
0 & 2 & 0 \\
0 & 0 & 3
\end{bmatrix}
\quad\text{og}\quad
V =
\begin{bmatrix}
1 & 2 & 1 \\
3 & 6 & 4 \\
2 & 5 & 2
\end{bmatrix}
\]
Da ser du fra del~(b) at matrisen $A = VDV^{-1}$ oppfyller kravene i
oppgaven.  Regn ut inversen til~$V$ på vanlig måte; da får du:
\[
V^{-1} =
\begin{bmatrix}
 8 & -1 & -2 \\
-2 &  0 &  1 \\
-3 &  1 &  0
\end{bmatrix}
\]
Nå kan du gange sammen matrisene og ende opp med:
\[
A = VDV^{-1} =
\begin{bmatrix}
 -9 & 2 & 2 \\
-36 & 9 & 6 \\
-22 & 4 & 6
\end{bmatrix}
\]
\end{punkt}
\end{losning}
