\ifx\inkludert\undefined
\documentclass[norsk,a4paper,twocolumn,oneside]{memoir}

\usepackage[utf8]{inputenc}
\usepackage{babel}
\usepackage{amsmath,amssymb,amsthm}
\usepackage[total={17cm,27cm}]{geometry}
\usepackage[table]{xcolor}
%\usepackage{tabularx}
\usepackage{systeme}
%\usepackage{hyperref}
%\usepackage{enumerate}

%\usepackage{sectsty}
\setsecheadstyle{\bfseries\large}
%\subsectionfont{\bf\normalsize}

\usepackage{tikz}
\usetikzlibrary{arrows.meta}

\newcommand{\defterm}[1]{\emph{#1}}

\newcommand{\N}{\mathbb{N}}
\newcommand{\Z}{\mathbb{Z}}
\newcommand{\Q}{\mathbb{Q}}
\newcommand{\R}{\mathbb{R}}

\newcommand{\abs}[1]{|#1|}

\newcommand{\roweq}{\sim}
\DeclareMathOperator{\Span}{Span}

\newcommand{\V}[1]{\mathbf{#1}}
\newcommand{\vv}[2]{\begin{bmatrix} #1 \\ #2 \end{bmatrix}}
\newcommand{\vvv}[3]{\begin{bmatrix} #1 \\ #2 \\ #3 \end{bmatrix}}
\newcommand{\vvvv}[4]{\begin{bmatrix} #1 \\ #2 \\ #3 \\ #4 \end{bmatrix}}
\newcommand{\vn}[2]{\vvvv{#1_1}{#1_2}{\vdots}{#1_#2}}

\newenvironment{amatrix}[1]{% "augmented matrix"
  \left[\begin{array}{*{#1}{c}|c}
}{%
  \end{array}\right]
}

% \newcounter{notatnr}
% \newcommand{\notatnr}[2]
% {\setcounter{notatnr}{#1}%
%  \setcounter{page}{#2}%
% }

\newtheorem{thm}{Teorem}[chapter]
\newtheorem*{thm-nn}{Teorem}
\newtheorem{cor}[thm]{Korollar}
\newtheorem{lem}[thm]{Lemma}
\newtheorem{prop}[thm]{Proposisjon}
\theoremstyle{definition}
\newtheorem{exx}[thm]{Eksempel}
\newtheorem*{defnx}{Definisjon}
\newtheorem*{oppg}{Oppgave}
\newtheorem*{merkx}{Merk}
\newtheorem*{spmx}{Spørsmål}

\newenvironment{defn}
  {\pushQED{\qed}\renewcommand{\qedsymbol}{$\triangle$}\defnx}
  {\popQED\enddefnx}
\newenvironment{ex}
  {\pushQED{\qed}\renewcommand{\qedsymbol}{$\triangle$}\exx}
  {\popQED\endexx}
\newenvironment{merk}
  {\pushQED{\qed}\renewcommand{\qedsymbol}{$\triangle$}\merkx}
  {\popQED\endmerkx}
\newenvironment{spm}
  {\pushQED{\qed}\renewcommand{\qedsymbol}{$\triangle$}\spmx}
  {\popQED\endspmx}

\setlength{\columnsep}{26pt}

\newcommand{\Tittel}[2]{%
\twocolumn[
\begin{center}
\Large
\begin{tabularx}{\textwidth}{cXr}
\cellcolor{black}\color{white}%
\bf {#1} &
#2
\hfill &
\footnotesize TMA4110 høsten 2018
\\ \hline
\end{tabularx}
\end{center}
]}

\newcommand{\tittel}[1]{\Tittel{\arabic{notatnr}}{#1}}

\newcommand{\linje}{%
\begin{center}
\rule{.8\linewidth}{0.4pt}
\end{center}
}


\newcommand{\chapternumber}{}

\makechapterstyle{tma4110}{%
 \renewcommand*{\chapterheadstart}{}
 \renewcommand*{\printchaptername}{}
 \renewcommand*{\chapternamenum}{}
 \renewcommand*{\printchapternum}{\renewcommand{\chapternumber}{\thechapter}}
 \renewcommand*{\afterchapternum}{}
 \renewcommand*{\printchapternonum}{\renewcommand{\chapternumber}{}}
 \renewcommand*{\printchaptertitle}[1]{
\LARGE
\begin{tabularx}{\textwidth}{cXr}
\cellcolor{black}\color{white}%
\textbf{\chapternumber} &
\textbf{##1}
\hfill &
%\footnotesize TMA4110 høsten 2018
\\ \hline
\end{tabularx}%
}
 \renewcommand*{\afterchaptertitle}{\par\nobreak\vskip \afterchapskip}
 % \newcommand{\chapnamefont}{\normalfont\huge\bfseries}
 % \newcommand{\chapnumfont}{\normalfont\huge\bfseries}
 % \newcommand{\chaptitlefont}{\normalfont\Huge\bfseries}
 \setlength{\beforechapskip}{0pt}
 \setlength{\midchapskip}{0pt}
 \setlength{\afterchapskip}{10pt}
}


\newcounter{oppgnr}[chapter]
\newcounter{punktnr}[oppgnr]
\newenvironment{oppgave}
 {\par\noindent\stepcounter{oppgnr}\textbf{{\arabic{oppgnr}}.}}
 {\par\bigskip}
\newenvironment{punkt}
 {\par\smallskip\noindent\stepcounter{punktnr}\textbf{\alph{punktnr})} }
 {\par}

\newcommand{\oppgaver}{\linje\section*{Oppgaver}}

\usepackage{xr}
\externaldocument{tma4110-2018h}
\newcommand{\kapittel}[2]{\setcounter{chapter}{#1}\addtocounter{chapter}{-1}\chapter{#2}}
\newcommand{\kapittelslutt}{\enddocument}
\begin{document}
\chapterstyle{tma4110}
\pagestyle{plain}
\fi


\kapittel{10}{Projeksjon}
\label{ch:projeksjon}
\section*{Skalarproduktet}
La $\V{v}$ og $\V{w}$ være kolonnevektorer i $\R^n$. Skalarproduktet mellom dem er definert som
\[
 \V{v}\cdot  \V{w} = v_1w_1 + v_2w_2 + \cdots +v_nw_n=\V{v}^T\V{w}.
 \] 
 Resultatet blir en skalar, derav navnet. Vi definerer lengden til en vektor $\V v$ som 
 \[
 \|v\|=\sqrt{\V{v}^T\V{v}},
 \]
 og vi sier at $\V{v}$ og $\V{w}$ er ortogonale dersom 
\[
 \V{v}^T  \V{w} = 0.
 \]
I $\R^2$ og $\R^3$ er vinkelen $\pi/2$ mellom ortogonale vektorer. Vi kan også definere skalarprodukt for rekkevektorer
 \[
 \V{v}\cdot  \V{w} =\V{v}\V{w}^T.
 \] 
 \begin{ex}
 Vektorene $\V{v}=[1,1,0,0]$ og $\V{w}=[0,0,1,1]$ er ortogonale. 
 \end{ex}

 \begin{ex}
 Vektoren $\V{v}=\frac{\V{w}}{\|\V{w}\|}$ har lengde 1 for alle $\V w \neq 0$.
 \end{ex}
 
 \section*{Projeksjon i $\R^2$}
Vi bruker skalarproduktet til å projisere vektorer på hverandre. Det sentrale spørsmålet er: hvor lange er de røde og blå lengdene i figuren under? 
 \begin{center}
\begin{tikzpicture}[scale=.42]
\draw[-latex,thick] (0,0) -- (7,1);
\draw[-latex,thick] (0,0) -- (3,6);
\draw[-, thick, red] (0,0) -- (27/50*7,27/50*1);
\draw[-,thick, blue] (27/50*7,27/50*1) -- (3,6);
\draw[-]  (27/50*7-1/7,27/50*1+1-1/77) -- (27/50*7+1-1/7,27/50*1+8/7-1/77);
\draw[-]  (27/50*7+1,27/50*1+1/7) -- (27/50*7+1-1/7,27/50*1+8/7-1/77);
\node[anchor=east] at (9,1.2) {\footnotesize $\V{v}$};
\node[anchor=south] at (3.5,7) {\footnotesize $\V{w}$};
\node[anchor=east,red] at (3,-.5) {\footnotesize $w_{\V{v}}$};
\node[anchor=east,blue] at (5.5,3) {\footnotesize $w_{\V{v}^{\perp}}$};
%\foreach \x in {-4,-3,-2,-1,1,2,3,4,5,6}
%\draw (\x,5pt) -- (\x,-5pt);
%\foreach \y in {-4,-3,-2,-1,1,2,3,4,5}
%\draw (5pt,\y) -- (-5pt,\y);
%\filldraw (2,3) circle [radius=3pt] node[anchor=west] {$z=2+3i$};
%\filldraw (2,-3) circle [radius=3pt] node[anchor=west] {$\overline z=2-3i$};
%\filldraw (4,5) circle [radius=3pt] node[anchor=west] {$w=4+5i$};
%\filldraw (0,1) circle [radius=3pt] node[anchor=east] {$\V{e}_2$};
%\filldraw (-1,-2) circle [radius=3pt] node[anchor=east] {$\V{u}$};
%\filldraw (3,2) circle [radius=3pt] node[anchor=east] {$\V{v}$};
%\filldraw (1,4) circle [radius=3pt] node[anchor=south] {$A \V{e}_1$};
%\filldraw (3,-3) circle [radius=3pt] node[anchor=north] {$A \V{e}_2$};
%\filldraw (-7,2) circle [radius=3pt] node[anchor=east] {$A \V{u}$};
%\filldraw (9,6) circle [radius=3pt] node[anchor=north] {$A \V{v}$};
%\draw[->,shorten <=4pt,shorten >=4pt] (1,0) to[bend right=20] (1,4);
%\draw[->,shorten <=4pt,shorten >=4pt] (0,1) to[bend right=30] (3,-3);
%\draw[->,shorten <=4pt,shorten >=4pt] (-1,-2) to[bend right=20] (-7,2);
%\draw[->,shorten <=4pt,shorten >=4pt] (3,2) to[bend left=20] (9,6);
\end{tikzpicture}
\\
{\small \textit{Hva er projeksjon?}}
\end{center}
I to dimensjoner er det lett å vise at skalarproduktet også kan beregnes som 
\[
 \V{v}\cdot  \V{w} = \|\V v\| \|\V w\| \cos \theta,
 \] 
der $\theta$ er vinkelen mellom $\V{v}$ og $\V{w}$. Vet man det, kan man beregne den røde lengden:
\[
w_{\V{v}}=\|\V w\| \cos \theta=\frac{\|\V v\|}{\|\V v\|} \|\V w\| \cos \theta =\frac{\V v \cdot \V w}{\|\V v\|}.
\]
Den blå lengden kan så beregnes med pytagoras:
\[
w_{\V{v}^{\perp}}=\sqrt{\|\V w\|^2-\left(\frac{\V v \cdot \V w}{\|\V v\|}\right)^2}.
\]
Denne lengden kalles $\V w$ sin skalarprojeksjon på normalen til $\V w$. Hvis vi tenker at de røde og blå lengdene er vektorer, kan de skrives
\[
\V w_{\V{v}}=w_{\V{v}}\frac{\V v}{\|\V v\|}=\frac{\V v \cdot \V w}{\|\V v\|^2}\V v=\frac{\V v \cdot \V w}{\V v \cdot \V v}\V v
\]
og
\[
\V w - \V w_{\V{v}}.
\]
 \begin{center}
\begin{tikzpicture}[scale=.42]
\draw[-latex,thick] (0,0) -- (7,1);
\draw[-latex,thick] (0,0) -- (3,6);
\draw[-latex, thick, red] (0,0) -- (27/50*7,27/50*1);
\draw[-latex,thick, blue] (27/50*7,27/50*1) -- (3,6);
\draw[-]  (27/50*7-1/7,27/50*1+1-1/77) -- (27/50*7+1-1/7,27/50*1+8/7-1/77);
\draw[-]  (27/50*7+1,27/50*1+1/7) -- (27/50*7+1-1/7,27/50*1+8/7-1/77);
\node[anchor=east] at (9,1.2) {\footnotesize $\V{v}$};
\node[anchor=south] at (3.5,7) {\footnotesize $\V{w}$};
\node[anchor=east,red] at (3,-.5) {\footnotesize $\V w_{\V{v}}$};
\node[anchor=east,blue] at (5.5,3) {\footnotesize $\V w_{\V{v}^{\perp}}$};
%\foreach \x in {-4,-3,-2,-1,1,2,3,4,5,6}
%\draw (\x,5pt) -- (\x,-5pt);
%\foreach \y in {-4,-3,-2,-1,1,2,3,4,5}
%\draw (5pt,\y) -- (-5pt,\y);
%\filldraw (2,3) circle [radius=3pt] node[anchor=west] {$z=2+3i$};
%\filldraw (2,-3) circle [radius=3pt] node[anchor=west] {$\overline z=2-3i$};
%\filldraw (4,5) circle [radius=3pt] node[anchor=west] {$w=4+5i$};
%\filldraw (0,1) circle [radius=3pt] node[anchor=east] {$\V{e}_2$};
%\filldraw (-1,-2) circle [radius=3pt] node[anchor=east] {$\V{u}$};
%\filldraw (3,2) circle [radius=3pt] node[anchor=east] {$\V{v}$};
%\filldraw (1,4) circle [radius=3pt] node[anchor=south] {$A \V{e}_1$};
%\filldraw (3,-3) circle [radius=3pt] node[anchor=north] {$A \V{e}_2$};
%\filldraw (-7,2) circle [radius=3pt] node[anchor=east] {$A \V{u}$};
%\filldraw (9,6) circle [radius=3pt] node[anchor=north] {$A \V{v}$};
%\draw[->,shorten <=4pt,shorten >=4pt] (1,0) to[bend right=20] (1,4);
%\draw[->,shorten <=4pt,shorten >=4pt] (0,1) to[bend right=30] (3,-3);
%\draw[->,shorten <=4pt,shorten >=4pt] (-1,-2) to[bend right=20] (-7,2);
%\draw[->,shorten <=4pt,shorten >=4pt] (3,2) to[bend left=20] (9,6);
\end{tikzpicture}
\\
{\small \textit{Hva er projeksjon II?}}
\end{center}
Tenker man på projeksjonene til $\V w$ på $\V v$ og $\V v^{\perp}$ som en lengder, kalles det skalarprojeksjon, og tenker man på det som vektorer, kalles det vektorprojeksjon. Det er egentlig ett fett hva man foretrekker.

%Vi skal ta dette for god fisk, og definere
%\[
% \theta = \cos^{-1} \frac{ \V{v}\cdot  \V{w}}{\|v\| \|w\|} 
% \] 
%som vinkelen mellom vektorer i $\R^n$.
 


% \section*{Projeksjon i $\R^3$}

 \section*{Projeksjon i $\R^n$}
En projeksjon er en lineærtransformasjon $P$ som tilfredsstiller
\[
 P=P^2.
 \]
 Denne ligningen sier at intet nytt skjer om du benytter lineærtransformasjonen for andre gang, og det er nettopp den egenskapen vi er ute etter når vi projiserer. Hvis du projiserer $\V w_{\V v}$ på $\V v$, får du $\V w_{\V v}$. Kanskje vi skal definere
 \[
 P_{\V v}=
 \]
 og 
 \[
 P_{\V v^{\perp}}=
 \]
 \begin{center}
\begin{tikzpicture}[scale=.42]
\draw[-latex,thick] (0,0) -- (7,1);
\draw[-latex,thick] (0,0) -- (3,6);
\draw[-latex, thick, red] (0,0) -- (27/50*7,27/50*1);
\draw[-latex,thick, blue] (27/50*7,27/50*1) -- (3,6);
\draw[-]  (27/50*7-1/7,27/50*1+1-1/77) -- (27/50*7+1-1/7,27/50*1+8/7-1/77);
\draw[-]  (27/50*7+1,27/50*1+1/7) -- (27/50*7+1-1/7,27/50*1+8/7-1/77);
\node[anchor=east] at (9,1.2) {\footnotesize $\V{v}$};
\node[anchor=south] at (3.5,7) {\footnotesize $\V{w}$};
\node[anchor=east,red] at (3.5,-.8) {\footnotesize $P_{\V{v}}(\V w)$};
\node[anchor=east,blue] at (7,3) {\footnotesize $P_{\V{v}^{\perp}}(\V w)$};
%\foreach \x in {-4,-3,-2,-1,1,2,3,4,5,6}
%\draw (\x,5pt) -- (\x,-5pt);
%\foreach \y in {-4,-3,-2,-1,1,2,3,4,5}
%\draw (5pt,\y) -- (-5pt,\y);
%\filldraw (2,3) circle [radius=3pt] node[anchor=west] {$z=2+3i$};
%\filldraw (2,-3) circle [radius=3pt] node[anchor=west] {$\overline z=2-3i$};
%\filldraw (4,5) circle [radius=3pt] node[anchor=west] {$w=4+5i$};
%\filldraw (0,1) circle [radius=3pt] node[anchor=east] {$\V{e}_2$};
%\filldraw (-1,-2) circle [radius=3pt] node[anchor=east] {$\V{u}$};
%\filldraw (3,2) circle [radius=3pt] node[anchor=east] {$\V{v}$};
%\filldraw (1,4) circle [radius=3pt] node[anchor=south] {$A \V{e}_1$};
%\filldraw (3,-3) circle [radius=3pt] node[anchor=north] {$A \V{e}_2$};
%\filldraw (-7,2) circle [radius=3pt] node[anchor=east] {$A \V{u}$};
%\filldraw (9,6) circle [radius=3pt] node[anchor=north] {$A \V{v}$};
%\draw[->,shorten <=4pt,shorten >=4pt] (1,0) to[bend right=20] (1,4);
%\draw[->,shorten <=4pt,shorten >=4pt] (0,1) to[bend right=30] (3,-3);
%\draw[->,shorten <=4pt,shorten >=4pt] (-1,-2) to[bend right=20] (-7,2);
%\draw[->,shorten <=4pt,shorten >=4pt] (3,2) to[bend left=20] (9,6);
\end{tikzpicture}
\\
{\small \textit{Hva er projeksjon III?}}
\end{center}
 

\section*{Ortogonal basis}
En ortogonal mengde er en mengde vekorer $\V v_1$, $\V v_2$, ...,$\V v_n$, slik at
\[
\V v_i \cdot \V v_k
\]
for alle vektorer $\V v_i$ og $\V v_k$ i mengden. 
\begin{ex}
Den vanlige basisen $\V e_1$, $\V e_2$,...,$\V e_n$ for $\R^n$ er en ortogonal mengde.
\end{ex}

\begin{ex}
Gram-Schmidt
\end{ex}

\section*{Minste kvadraters metode}
Dette er en teknikk for å 'løse' systemer med flere likninger enn ukjente. Kort fortalt projiserer den høyresiden ned i kolonnerommet til matrisen, for å finne punktet med kortest avstand til høyresiden.

\kapittelslutt
