\documentclass[norsk,a4paper,twocolumn,oneside]{memoir}

\usepackage[utf8]{inputenc}
\usepackage{babel}
\usepackage{amsmath,amssymb,amsthm}
\usepackage{mathrsfs}
\usepackage[total={17cm,27cm}]{geometry}
\usepackage[table]{xcolor}
%\usepackage{tabularx}
\usepackage{systeme}
%\usepackage{hyperref}
%\usepackage{enumerate}
\usepackage{ifthen}
\usepackage{textgreek}
\usepackage{multirow}
\usepackage{placeins}

%\usepackage{sectsty}
\setsecheadstyle{\bfseries\large}
%\subsectionfont{\bf\normalsize}

\usepackage{tikz,pgfplots}
\usetikzlibrary{calc}
\usetikzlibrary{arrows.meta}
\def\centerarc[#1](#2)(#3:#4:#5)% Syntax: [draw options] (center) (initial angle:final angle:radius)
    { \draw[#1] ($(#2)+({#5*cos(#3)},{#5*sin(#3)})$) arc (#3:#4:#5); }
\usepackage{pgfornament}

\newcommand{\defterm}[1]{\emph{#1}}

\newcommand{\N}{\mathbb{N}}
\newcommand{\Z}{\mathbb{Z}}
\newcommand{\Q}{\mathbb{Q}}
\newcommand{\R}{\mathbb{R}}

\newcommand{\M}{\mathcal{M}} % vektorrom av matriser
\newcommand{\C}{\mathcal{C}} % vektorrom av kontinuerlige funksjoner
\renewcommand{\P}{\mathcal{P}} % vektorrom av polynomer
\newcommand{\B}{\mathscr{B}} % basis

\renewcommand{\Im}{\operatorname{Im}}
\renewcommand{\Re}{\operatorname{Re}}

\newcommand{\abs}[1]{|#1|}
\newcommand{\intersect}{\cap}
\newcommand{\union}{\cup}
\newcommand{\fcomp}{\circ}
\newcommand{\iso}{\cong}

\newcommand{\roweq}{\sim}
\DeclareMathOperator{\Sp}{Sp}
\DeclareMathOperator{\Null}{Null}
\DeclareMathOperator{\Col}{Col}
\DeclareMathOperator{\Row}{Row}
\DeclareMathOperator{\rank}{rank}
\DeclareMathOperator{\im}{im}
\DeclareMathOperator{\id}{id}
\DeclareMathOperator{\Hom}{Hom}
\newcommand{\tr}{^\top}
\newcommand{\koord}[2]{[\,{#1}\,]_{#2}} % koordinater mhp basis

\newcommand{\V}[1]{\mathbf{#1}}
\newcommand{\vv}[2]{\begin{bmatrix} #1 \\ #2 \end{bmatrix}}
\newcommand{\vvS}[2]{\left[ \begin{smallmatrix} #1 \\ #2 \end{smallmatrix} \right]}
\newcommand{\vvv}[3]{\begin{bmatrix} #1 \\ #2 \\ #3 \end{bmatrix}}
\newcommand{\vvvv}[4]{\begin{bmatrix} #1 \\ #2 \\ #3 \\ #4 \end{bmatrix}}
\newcommand{\vvvvv}[5]{\begin{bmatrix} #1 \\ #2 \\ #3 \\ #4 \\ #5 \end{bmatrix}}
\newcommand{\vn}[2]{\vvvv{#1_1}{#1_2}{\vdots}{#1_#2}}

\newcommand{\e}{\V{e}}
\renewcommand{\u}{\V{u}}
\renewcommand{\v}{\V{v}}
\newcommand{\w}{\V{w}}
\renewcommand{\b}{\V{b}}
\newcommand{\x}{\V{x}}
\newcommand{\0}{\V{0}}

\newenvironment{amatrix}[1]{% "augmented matrix"
  \left[\begin{array}{*{#1}{c}|c}
}{%
  \end{array}\right]
}

\newcommand{\boks}[1]{\framebox{\strut $#1$}}

% \newcounter{notatnr}
% \newcommand{\notatnr}[2]
% {\setcounter{notatnr}{#1}%
%  \setcounter{page}{#2}%
% }

\newtheorem{thm}{Teorem}[chapter]
\newtheorem*{thm-nn}{Teorem}
\newtheorem{cor}[thm]{Korollar}
\newtheorem{lem}[thm]{Lemma}
\newtheorem{prop}[thm]{Proposisjon}
\theoremstyle{definition}
\newtheorem{exx}[thm]{Eksempel}
\newtheorem*{defnx}{Definisjon}
\newtheorem*{oppg}{Oppgave}
\newtheorem*{merkx}{Merk}
\newtheorem*{spmx}{Spørsmål}

\newenvironment{defn}
  {\pushQED{\qed}\renewcommand{\qedsymbol}{$\triangle$}\defnx}
  {\popQED\enddefnx}
\newenvironment{ex}
  {\pushQED{\qed}\renewcommand{\qedsymbol}{$\triangle$}\exx}
  {\popQED\endexx}
\newenvironment{merk}
  {\pushQED{\qed}\renewcommand{\qedsymbol}{$\triangle$}\merkx}
  {\popQED\endmerkx}
\newenvironment{spm}
  {\pushQED{\qed}\renewcommand{\qedsymbol}{$\triangle$}\spmx}
  {\popQED\endspmx}

\setlength{\columnsep}{26pt}

\newcommand{\Tittel}[2]{%
\twocolumn[
\begin{center}
\Large
\begin{tabularx}{\textwidth}{cXr}
\cellcolor{black}\color{white}%
\bf {#1} &
#2
\hfill &
\footnotesize TMA4110 høsten 2018
\\ \hline
\end{tabularx}
\end{center}
]}

\newcommand{\tittel}[1]{\Tittel{\arabic{notatnr}}{#1}}

\newcommand{\linje}{%
\begin{center}
\rule{.8\linewidth}{0.4pt}
\end{center}
}


\newcommand{\kapittelemnenavn}{TMA4110 høsten 2018}
\newcommand{\chapternumber}{}

\makechapterstyle{tma4110}{%
 \renewcommand*{\chapterheadstart}{}
 \renewcommand*{\printchaptername}{}
 \renewcommand*{\chapternamenum}{}
 \renewcommand*{\printchapternum}{\renewcommand{\chapternumber}{\thechapter}}
 \renewcommand*{\afterchapternum}{}
 \renewcommand*{\printchapternonum}{\renewcommand{\chapternumber}{}}
 \renewcommand*{\printchaptertitle}[1]{
\LARGE
\begin{tabularx}{\textwidth}{cXr}
\cellcolor{black}\color{white}%
\textbf{\chapternumber} &
\textbf{##1}
\hfill &
\footnotesize\kapittelemnenavn
\\ \hline
\end{tabularx}%
}
 \renewcommand*{\afterchaptertitle}{\par\nobreak\vskip \afterchapskip}
 % \newcommand{\chapnamefont}{\normalfont\huge\bfseries}
 % \newcommand{\chapnumfont}{\normalfont\huge\bfseries}
 % \newcommand{\chaptitlefont}{\normalfont\Huge\bfseries}
 \setlength{\beforechapskip}{0pt}
 \setlength{\midchapskip}{0pt}
 \setlength{\afterchapskip}{10pt}
}
\chapterstyle{tma4110}
\pagestyle{plain}


\newboolean{vis-oppgaver}
\newboolean{vis-losninger}
\setboolean{vis-oppgaver}{true}
\setboolean{vis-losninger}{false}

\newcounter{oppg-kap} % kapittelnummerering for oppgaver
\newcounter{oppgnr}[oppg-kap]
\newcounter{punktnr}[oppgnr]

\newenvironment{oppgave}%
 {\ifthenelse{\boolean{vis-oppgaver}}%
             {\par\noindent\stepcounter{oppgnr}\textbf{\arabic{oppgnr}.}}%
             {\expandafter\comment}}%
 {\ifthenelse{\boolean{vis-oppgaver}}%
             {\par\bigskip}%
             {\expandafter\endcomment}}

\newenvironment{losning}%
 {\ifthenelse{\boolean{vis-losninger}}%
             {\par\noindent\stepcounter{oppgnr}\textbf{\arabic{oppg-kap}.\arabic{oppgnr}.}}%
             {\expandafter\comment}}%
 {\ifthenelse{\boolean{vis-losninger}}%
             {\par\bigskip}%
             {\expandafter\endcomment}}

\newenvironment{punkt}
 {\par\smallskip\noindent\stepcounter{punktnr}\textbf{\alph{punktnr})} }
 {\par}

\newcommand{\kap}[1]{\setcounter{oppg-kap}{#1}\addtocounter{oppg-kap}{-1}\stepcounter{oppg-kap}}

\newcommand{\oppgaver}[1]{%
  \kap{#1}%
  \ifthenelse{\boolean{vis-oppgaver}}%
             {\linje\section*{Oppgaver}}%
             {}}

\usepackage{tikz}
\newcommand*\circled[1]{\tikz[baseline=(char.base)]{
            \node[shape=circle,draw,inner sep=2pt] (char) {#1};}}

\begin{document}

\notatnr{3}{5} % TODO sett riktig startsidetall ut fra lengden på forrige notat
\tittel{Gausseliminasjon}

\noindent%
I dette notatet skal vi formalisere det vi gjorde i forrige notat.  Vi
skal se hvordan vi kan løse et hvilket som helst lineært
likningssystem ved å skrive om totalmatrisen til systemet etter
bestemte regler.

Reglene for hvordan totalmatrisen kan skrives om kalles
\emph{radoperasjoner}, og vi vil ende opp med at matrisen er på
\emph{trappeform}.  Hele prosessen kalles \emph{gausseliminasjon}.

% \section*{Matriser}

% \[
% \text{$m \times n$-matrise}
% \begin{bmatrix}

% \end{bmatrix}
% \]


\section*{Radoperasjoner}

% I eksempel~2.5 i forrige notat løste vi dette systemet:
% \[
% \systeme{
%    x + 2y - 2z = -5,
%    x + 3y +  z = 5,
%   2x + 7y + 7z = 28
% }
% \]
% Totalmatrisen til systemet er:
% \[
% \begin{amatrix}{3}

% \end{amatrix}
% \]

Følgende tre måter å endre en matrise på kalles
\defterm{radoperasjoner}:
\begin{enumerate}
\item Gange alle tallene i en rad med det samme tallet (ikke~$0$)
\item Legge til (et multiplum av) en rad i en annen
\item Bytte rekkefølge på radene
\end{enumerate}

Vi sier at to matriser er \defterm{radekvivalente} hvis vi kan komme
fra den ene til den andre ved å utføre en eller flere radoperasjoner.
Vi bruker notasjonen $M \roweq N$ for å si at to matriser $M$ og~$N$
er radekvivalente.

\begin{ex}
Disse matrisene er radekvivalente, siden vi får den andre matrisen fra
den første ved å gange øverste rad med~$4$:
\begin{align*}
\begin{bmatrix}
 2 & 5 \\
 1 & 7
\end{bmatrix}
&\roweq
\begin{bmatrix}
 8 & 20 \\
 1 &  7
\end{bmatrix}
\end{align*}
Merk at vi også kan gå motsatt vei: Ved å gange øverste rad i den
andre matrisen med $1/4$ får vi tilbake den første matrisen.

Disse to matrisene er også radekvivalente:
\begin{align*}
\begin{bmatrix}
 3 & 1 \\
 9 & 5
\end{bmatrix}
&\roweq
\begin{bmatrix}
 3 & 1 \\
 0 & 2
\end{bmatrix}
\end{align*}
Her har vi brukt den andre typen radoperasjon: Vi la til $-3$ ganger
øverste rad i nederste rad for å komme fra den første matrisen til den
andre.  Merk igjen at vi også kan gå motsatt vei: Ved å legge til $3$
ganger øverste rad i nederste rad, kommer vi fra den andre matrisen
til den første.
\end{ex}

Hele poenget med radoperasjoner er at det å utføre en radoperasjon på
en totalmatrise tilsvarer å skrive om likningssystemet til et nytt
system som er ekvivalent med det opprinnelige.  Vi formulerer dette
som et teorem:

\begin{thm}
Hvis to likningssystemer har radekvivalente totalmatriser, så er de to
likningssystemene ekvivalente.
\end{thm}

\begin{ex}
Vi gjentar regningen i eksempel~2.5 fra forrige notat, denne gangen
ved å utføre radoperasjoner på totalmatrisen til likningssystemet:
\begin{align*}
\begin{amatrix}{3}
1 & 2 & -2 & -5 \\
1 & 5 &  9 & 33 \\
2 & 5 & -1 &  0
\end{amatrix}
&\roweq
\begin{amatrix}{3}
1 & 2 & -2 & -5 \\
0 & 3 & 11 & 38 \\
2 & 5 & -1 &  0
\end{amatrix}
\\
&\roweq
\begin{amatrix}{3}
1 & 2 & -2 & -5 \\
0 & 3 & 11 & 38 \\
0 & 1 &  3 & 10
\end{amatrix}
\\
&\roweq
\begin{amatrix}{3}
1 & 2 & -2 & -5 \\
0 & 1 &  3 & 10 \\
0 & 3 & 11 & 38
\end{amatrix}
\\
&\roweq
\begin{amatrix}{3}
1 & 2 & -2 & -5 \\
0 & 1 &  3 & 10 \\
0 & 0 &  2 &  8
\end{amatrix}
\end{align*}
Her gjorde vi følgende radoperasjoner: Legge til $-1$ ganger første
rad i andre rad, legge til $-2$ ganger første rad i tredje rad, bytte
andre og tredje rad, og legge til $-3$ ganger andre rad i tredje rad.

Den siste matrisen her er på det som kalles trappeform, og da er det
(som vi så i eksempel~2.5) lett å lese av løsningen.  Hvis vi vil
gjøre det enda lettere, kan vi fortsette med radoperasjoner til vi
oppnår det som kalles \emph{redusert trappeform}:
\begin{align*}
\begin{amatrix}{3}
1 & 2 & -2 & -5 \\
0 & 1 &  3 & 10 \\
0 & 0 &  2 &  8
\end{amatrix}
&\roweq
\begin{amatrix}{3}
1 & 2 & -2 & -5 \\
0 & 1 &  3 & 10 \\
0 & 0 &  1 &  4
\end{amatrix}
\\
&\roweq
\begin{amatrix}{3}
1 & 2 &  0 &  3 \\
0 & 1 &  0 & -2 \\
0 & 0 &  1 &  4
\end{amatrix}
\\
&\roweq
\begin{amatrix}{3}
1 & 0 &  0 &  7 \\
0 & 1 &  0 & -2 \\
0 & 0 &  1 &  4
\end{amatrix}
\end{align*}
Den siste totalmatrisen her svarer til følgende likningssystem:
\[
\systeme*{
x = 7,
y = -2,
z = 4
}
\]
Her har vi altså kommet helt frem til løsningen.
\end{ex}


\section*{Trappeform}

% I det siste eksempelet i forrige notat løste vi et likningssystem med
% tre likninger og tre ukjente.  Vi erstattet systemet med nye
% ekvivalente systemer frem til det så slik ut:
% \[
% \systeme{
%    x + 2y - 2z = -5,
%         y + 3z = 10,
%             2z =  8
% }
% \]
% Legg merke til den spesielle fasongen på dette systemet som gjør at
% det er lett å finne løsningen: $x$ og~$y$ er ikke med i den siste
% likningen, og $x$ er ikke med i den midterste.  Vi sier at dette
% systemet er på \emph{trappeform}, siden den tomme plassen til venstre
% under likningene har fasong som en trapp.

% Totalmatrisen for systemet ser slik ut:
% \[
% \begin{amatrix}{3}
% 1 & 2 & -2 & -5 \\
% 0 & 1 &  3 & 10 \\
% 0 & 0 &  2 &  8
% \end{amatrix}
% \]
% Her er det de tre nullene nede til venstre som gjør at vi ser at
% systemet er på trappeform.

Vi vil nå gi en presis definisjon av begrepet «trappeform».  Da
trenger vi også et annet begrep, nemlig «lederelement».

\begin{defn}
Det første tallet i en rad i en matrise som ikke er~$0$ kalles
\defterm{lederelementet} for den raden.  (En rad med bare nuller har
ikke noe lederelement.)
\end{defn}

\begin{ex}
\label{ex:lederelement}
Se på følgende matrise:
\[
\begin{bmatrix}
3 & -2 & 0 & 2 \\
0 &  0  & 5 & 12 \\
1 &  8 & 3 & 7 \\
0 &  0  & 0 & 0
\end{bmatrix}
\]
Lederelementene her er tallet~$3$ i den øverste raden, tallet~$5$ i
den andre raden og tallet~$1$ i den tredje raden.  Den siste raden
består av bare nuller, og har derfor ikke noe lederelement.
\end{ex}

\begin{defn}
En matrise er på \defterm{trappeform} dersom hvert lederelement er til
høyre for alle lederelementer i tidligere rader, og eventuelle
nullrader er helt nederst.
\end{defn}

\begin{ex}
\label{ex:trappeform}
Denne matrisen er på trappeform:
\[
\begin{bmatrix}
3 & 7 & 6 \\
0 & 1 & -2 \\
0 & 0 & 2
\end{bmatrix}
\]
Lederelementene er $3$, $1$ og~$2$, og hvert av dem er til høyre for
alle de tidligere lederelementene.

Denne matrisen er også på trappeform:
\[
\begin{bmatrix}
-2 & 7 & 1 & 5 \\
 0 & 0 & 4 & 9 \\
 0 & 0 & 0 & 0
\end{bmatrix}
\]

Denne matrisen er ikke på trappeform fordi nullradene ikke er samlet
nederst:
\[
\begin{bmatrix}
3 & 8 & 2 & 0 \\
0 & 0 & 0 & 0 \\
0 & 2 & 1 & 4 \\
0 & 0 & 0 & 0
\end{bmatrix}
\]

Matrisen i eksempel~\ref{ex:lederelement} er ikke på trappeform fordi
lederelementet i tredje rad ikke er til høyre for lederelementene i
første og andre rad.

TODO: mer eksempel
\end{ex}

\begin{defn}
En matrise er på \defterm{redusert trappeform} hvis den er på
trappeform og dessuten oppfyller følgende:
\begin{itemize}
\item Alle lederelementene er~$1$.
\item Alle tall som står over lederelementer er~$0$.
\end{itemize}
\end{defn}


\section*{Gausseliminasjon}

\begin{ex}
\begin{align*}
\begin{amatrix}{3}
1 & 2 & -2 & -5 \\
1 & 3 & 1  & 5 \\
2 & 7 & 7  & 28
\end{amatrix}
&\roweq
\begin{amatrix}{3}
1 & 2 & -2 & -5 \\
0 & 1 & 3  & 10 \\
2 & 7 & 7  & 28
\end{amatrix}
\\
&\roweq
\begin{amatrix}{3}
1 & 2 & -2 & -5 \\
0 & 1 & 3  & 10 \\
0 & 3 & 11 & 38
\end{amatrix}
\\
&\roweq
\begin{amatrix}{3}
1 & 2 & -2 & -5 \\
0 & 1 & 3  & 10 \\
0 & 0 & 2  & 8
\end{amatrix}
\end{align*}
\end{ex}


\end{document}
