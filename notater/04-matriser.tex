\ifx\inkludert\undefined
\documentclass[norsk,a4paper,twocolumn,oneside]{memoir}

\usepackage[utf8]{inputenc}
\usepackage{babel}
\usepackage{amsmath,amssymb,amsthm}
\usepackage[total={17cm,27cm}]{geometry}
\usepackage[table]{xcolor}
%\usepackage{tabularx}
\usepackage{systeme}
%\usepackage{hyperref}
%\usepackage{enumerate}

%\usepackage{sectsty}
\setsecheadstyle{\bfseries\large}
%\subsectionfont{\bf\normalsize}

\usepackage{tikz}
\usetikzlibrary{arrows.meta}

\newcommand{\defterm}[1]{\emph{#1}}

\newcommand{\N}{\mathbb{N}}
\newcommand{\Z}{\mathbb{Z}}
\newcommand{\Q}{\mathbb{Q}}
\newcommand{\R}{\mathbb{R}}

\newcommand{\abs}[1]{|#1|}

\newcommand{\roweq}{\sim}
\DeclareMathOperator{\Span}{Span}

\newcommand{\V}[1]{\mathbf{#1}}
\newcommand{\vv}[2]{\begin{bmatrix} #1 \\ #2 \end{bmatrix}}
\newcommand{\vvv}[3]{\begin{bmatrix} #1 \\ #2 \\ #3 \end{bmatrix}}
\newcommand{\vvvv}[4]{\begin{bmatrix} #1 \\ #2 \\ #3 \\ #4 \end{bmatrix}}
\newcommand{\vn}[2]{\vvvv{#1_1}{#1_2}{\vdots}{#1_#2}}

\newenvironment{amatrix}[1]{% "augmented matrix"
  \left[\begin{array}{*{#1}{c}|c}
}{%
  \end{array}\right]
}

% \newcounter{notatnr}
% \newcommand{\notatnr}[2]
% {\setcounter{notatnr}{#1}%
%  \setcounter{page}{#2}%
% }

\newtheorem{thm}{Teorem}[chapter]
\newtheorem*{thm-nn}{Teorem}
\newtheorem{cor}[thm]{Korollar}
\newtheorem{lem}[thm]{Lemma}
\newtheorem{prop}[thm]{Proposisjon}
\theoremstyle{definition}
\newtheorem{exx}[thm]{Eksempel}
\newtheorem*{defnx}{Definisjon}
\newtheorem*{oppg}{Oppgave}
\newtheorem*{merkx}{Merk}
\newtheorem*{spmx}{Spørsmål}

\newenvironment{defn}
  {\pushQED{\qed}\renewcommand{\qedsymbol}{$\triangle$}\defnx}
  {\popQED\enddefnx}
\newenvironment{ex}
  {\pushQED{\qed}\renewcommand{\qedsymbol}{$\triangle$}\exx}
  {\popQED\endexx}
\newenvironment{merk}
  {\pushQED{\qed}\renewcommand{\qedsymbol}{$\triangle$}\merkx}
  {\popQED\endmerkx}
\newenvironment{spm}
  {\pushQED{\qed}\renewcommand{\qedsymbol}{$\triangle$}\spmx}
  {\popQED\endspmx}

\setlength{\columnsep}{26pt}

\newcommand{\Tittel}[2]{%
\twocolumn[
\begin{center}
\Large
\begin{tabularx}{\textwidth}{cXr}
\cellcolor{black}\color{white}%
\bf {#1} &
#2
\hfill &
\footnotesize TMA4110 høsten 2018
\\ \hline
\end{tabularx}
\end{center}
]}

\newcommand{\tittel}[1]{\Tittel{\arabic{notatnr}}{#1}}

\newcommand{\linje}{%
\begin{center}
\rule{.8\linewidth}{0.4pt}
\end{center}
}


\newcommand{\chapternumber}{}

\makechapterstyle{tma4110}{%
 \renewcommand*{\chapterheadstart}{}
 \renewcommand*{\printchaptername}{}
 \renewcommand*{\chapternamenum}{}
 \renewcommand*{\printchapternum}{\renewcommand{\chapternumber}{\thechapter}}
 \renewcommand*{\afterchapternum}{}
 \renewcommand*{\printchapternonum}{\renewcommand{\chapternumber}{}}
 \renewcommand*{\printchaptertitle}[1]{
\LARGE
\begin{tabularx}{\textwidth}{cXr}
\cellcolor{black}\color{white}%
\textbf{\chapternumber} &
\textbf{##1}
\hfill &
%\footnotesize TMA4110 høsten 2018
\\ \hline
\end{tabularx}%
}
 \renewcommand*{\afterchaptertitle}{\par\nobreak\vskip \afterchapskip}
 % \newcommand{\chapnamefont}{\normalfont\huge\bfseries}
 % \newcommand{\chapnumfont}{\normalfont\huge\bfseries}
 % \newcommand{\chaptitlefont}{\normalfont\Huge\bfseries}
 \setlength{\beforechapskip}{0pt}
 \setlength{\midchapskip}{0pt}
 \setlength{\afterchapskip}{10pt}
}


\newcounter{oppgnr}[chapter]
\newcounter{punktnr}[oppgnr]
\newenvironment{oppgave}
 {\par\noindent\stepcounter{oppgnr}\textbf{{\arabic{oppgnr}}.}}
 {\par\bigskip}
\newenvironment{punkt}
 {\par\smallskip\noindent\stepcounter{punktnr}\textbf{\alph{punktnr})} }
 {\par}

\newcommand{\oppgaver}{\linje\section*{Oppgaver}}

\usepackage{xr}
\externaldocument{tma4110-2018h}
\newcommand{\kapittel}[2]{\setcounter{chapter}{#1}\addtocounter{chapter}{-1}\chapter{#2}}
\newcommand{\kapittelslutt}{\enddocument}
\begin{document}
\chapterstyle{tma4110}
\pagestyle{plain}
\fi


\kapittel{4}{Matriser}
\label{ch:matriser}



\section*{Definisjoner og notasjon}

%% TODO: definer kolonnevektor (søylevektor) og radvektor

%% TODO: notasjon (v_1, ..., v_n) for kolonnevektor

En \defterm{$m \times n$-matrise} er en rektangulær tabell med tall
som har $m$~tall i høyden og $n$~tall i bredden:
\[
\begin{bmatrix}
a_{11} & a_{12} & \cdots & a_{1n} \\
a_{21} & a_{22} & \cdots & a_{2n} \\
\vdots & \vdots & \ddots & \vdots \\
a_{m1} & a_{m2} & \cdots & a_{mn}
\end{bmatrix}
\]
\defterm{Kolonnene} i matrisen er følgende kolonnevektorer:
\[
\vvvv{a_{11}}{a_{21}}{\vdots}{a_{m1}}\qquad
\vvvv{a_{12}}{a_{22}}{\vdots}{a_{m2}}\qquad
\cdots\qquad
\vvvv{a_{1n}}{a_{2n}}{\vdots}{a_{mn}}\qquad
\]
\defterm{Radene} i matrisen er følgende radvektorer:
\begin{gather*}
\begin{bmatrix}
a_{11} & a_{12} & \cdots & a_{1n}
\end{bmatrix}
\\
\begin{bmatrix}
a_{21} & a_{22} & \cdots & a_{2n}
\end{bmatrix}
\\
\vdots
\\
\begin{bmatrix}
a_{m1} & a_{m2} & \cdots & a_{mn}
\end{bmatrix}
\end{gather*}

\begin{ex}
Her er et eksempel på en $2 \times 3$-matrise:
\[
\begin{bmatrix}
5 & 0 & -2 \\
3 & 1 &  4
\end{bmatrix}
\]
Kolonnene i denne matrisen er:
\[
\vv{5}{3},\quad
\vv{0}{1}\quad\text{og}\quad
\vv{-2}{4}
\]
Radene er:
\[
\begin{bmatrix}
5 & 0 & -2
\end{bmatrix}
\quad\text{og}\quad
\begin{bmatrix}
3 & 1 &  4
\end{bmatrix}
\qedhere
\]
\end{ex}

Noen ganger har vi en liste med kolonnevektorer, si
\[
\V{v}_1,\ \V{v}_2,\ \ldots,\ \V{v}_n,
\]
og vil lage en matrise som har disse vektorene som kolonner.
Den matrisen kan vi skrive slik:
\[
\begin{bmatrix}
\V{v}_1 & \V{v}_2 & \cdots & \V{v}_n
\end{bmatrix}
\]
Hvis vektorene ligger i~$\R^m$, blir dette en $m \times n$-matrise.

\begin{ex}
La
\[
\V{v}_1 = \vvv{2}{0}{4}
\qquad\text{og}\qquad
\V{v}_2 = \vvv{1}{1}{2}
\]
være to vektorer i~$\R^3$.  Matrisen
$\begin{bmatrix} \V{v}_1 & \V{v}_2 \end{bmatrix}$ med disse vektorene
som kolonner blir da følgende $3 \times 2$-matrise:
\[
\begin{bmatrix} \V{v}_1 & \V{v}_2 \end{bmatrix}
=
\begin{bmatrix}
2 & 1 \\
1 & 0 \\
2 & 4
\end{bmatrix}
\qedhere
\]
\end{ex}

På samme måte kan vi, hvis vi har en liste med radvektorer
\[
\V{r}_1,\ \V{r}_2,\ \ldots,\ \V{r}_m,
\]
lage en matrise
\[
\begin{bmatrix}
\V{r}_1 \\
\V{r}_2 \\
\vdots \\
\V{r}_m
\end{bmatrix}
\]
med disse vektorene som rader.

\begin{ex}
La
\[
\V{r}_1 = \begin{bmatrix} 2 & 1 \end{bmatrix},\quad
\V{r}_2 = \begin{bmatrix} 1 & 0 \end{bmatrix}
\quad\text{og}\quad
\V{r}_3 = \begin{bmatrix} 2 & 4 \end{bmatrix}
\]
være tre radvektorer.  Da har vi:
\[
\begin{bmatrix}
\V{r}_1 \\
\V{r}_2 \\
\V{r}_3
\end{bmatrix}
=
\begin{bmatrix}
2 & 1 \\
1 & 0 \\
2 & 4
\end{bmatrix}
\qedhere
\]
\end{ex}


\section*{Produkt av matrise og vektor}

La
\[
A =
\begin{bmatrix}
\V{a}_1 & \V{a}_2 & \cdots & \V{a}_n
\end{bmatrix}
\]
være en $m \times n$-matrise med vektorene $\V{a}_1$, $\V{a}_2$,
\ldots, $\V{a}_n$ som kolonner, og la
\[
\V{v} = \vn{v}{n}
\]
være en vektor i~$\R^n$.  Vi definerer produktet $A\V{v}$ av $A$
og~$\V{v}$ som lineærkombinasjonen av kolonnene i~$A$ med tallene
i~$\V{v}$ som vekter:
\[
A \V{v} = \V{a}_1 v_1 + \V{a}_2 v_2 + \cdots + \V{a}_n v_n
\]
Merk at produktet~$A \V{v}$ bare er definert når bredden av matrisen
$A$ er lik høyden av vektoren~$\V{v}$.

\begin{ex}
\label{ex:Av}
Vi regner ut produktet av en $2 \times 3$-matrise og en vektor
i~$\R^3$:
\begin{align*}
\begin{bmatrix}
5 & 0 & -2 \\
3 & 1 &  4
\end{bmatrix}
\vvv{2}{-1}{3}
&=
\vv{5}{3} \cdot 2 +
\vv{0}{1} \cdot (-1) +
\vv{-2}{4} \cdot 3
\\
&= \vv{5 \cdot 2 + 0 \cdot (-1) + (-2) \cdot 3}
      {3 \cdot 2 + 1 \cdot (-1) +   4  \cdot 3}
\\
&= \vv{4}{17}
\end{align*}
Merk at resultatet blir en vektor i~$\R^2$.
\end{ex}

Når vi skal regne ut et produkt $A \V{v}$ av en matrise og en vektor,
trenger vi ikke egentlig å skrive opp lineærkombinasjonen av kolonnene
i $A$ med vekter fra $\V{v}$ slik som vi gjorde i eksempelet over.
Den andre linjen av utregningen i eksempelet viser en mer direkte måte
å komme frem på: Tallet som skal være på første posisjon i
resultatvektoren får vi ved å gange tallene fra første rad i~$A$ med
tallene i~$\V{v}$, og legge sammen resultatene.  Tallet på andre
posisjon i resultatvektoren får vi på samme måte fra andre rad i~$A$.

Generelt har vi at dersom
\[
A =
\begin{bmatrix}
a_{11} & a_{12} & \cdots & a_{1n} \\
a_{21} & a_{22} & \cdots & a_{2n} \\
\vdots & \vdots & \ddots & \vdots \\
a_{m1} & a_{m2} & \cdots & a_{mn}
\end{bmatrix}
\quad\text{og}\quad
\V{v} = \vn{v}{n},
\]
så kan vi regne ut produktet $A \V{v}$ på følgende måte:
\[
A \V{v} =
\vvvv{a_{11} v_1 + a_{12} v_2 + \cdots + a_{1n} v_n}
     {a_{21} v_1 + a_{22} v_2 + \cdots + a_{2n} v_n}
     {\vdots}
     {a_{m1} v_1 + a_{m2} v_2 + \cdots + a_{mn} v_n}
\]
(Det er ikke vanskelig å se at dette følger fra definisjonen
av~$A \V{v}$.)



\begin{thm}
\label{thm:Av-lin-komb}
Hvis $A$ er en $m \times n$-matrise, $\V{v}$ og~$\V{w}$ er vektorer
i~$\R^n$ og $c$ er et tall, så har vi følgende likheter:
\[
A (\V{v} + \V{w}) = A \V{v} + A \V{w}
\qquad\text{og}\qquad
A (c \V{v}) = c (A \V{v})
\]
\end{thm}

\begin{ex}
La $A$ og~$\V{v}$ være følgende matrise og vektor:
\[
A =
\begin{bmatrix}
5 & 0 & -2 \\
3 & 1 &  4
\end{bmatrix}
\quad\text{og}\quad
\V{v} = \vvv{2}{-1}{3}
\]
I eksempel~\ref{ex:Av} regnet vi ut produktet $A \V{v}$.
Men hva om vi har lyst til å gange $A$ med vektoren $(8, -4, 12)$?
Siden denne vektoren er lik $4 \cdot \V{v}$ kan vi bruke
teorem~\ref{thm:Av-lin-komb} og få:
\[
A \vvv{8}{-4}{12}
= A \cdot (4 \cdot \V{v})
= 4 \cdot (A \cdot \V{v})
= 4 \cdot \vv{4}{17}
= \vv{16}{68}
\]
Altså: $A$ ganget med vektoren $4 \V{v}$ er det samme som $4$ ganger
vektoren $A \V{v}$.
% \[
% \begin{bmatrix}
% 5 & 0 & -2 \\
% 3 & 1 &  4
% \end{bmatrix}
% \vvv{2}{-1}{3}
% = \vv{4}{17}
% \]
% Hva om vi vil gange den samme matrisen med vektoren $(8, -4, 12)$?
% Siden denne vektoren er lik $4 \cdot (2,-1,3)$, kan vi bruke
% teorem~\ref{thm:Av-lin-komb} og få:
% \begin{align*}
% \begin{bmatrix}
% 5 & 0 & -2 \\
% 3 & 1 &  4
% \end{bmatrix}
% \vvv{8}{-4}{12}
% &= 
% \begin{bmatrix}
% 5 & 0 & -2 \\
% 3 & 1 &  4
% \end{bmatrix}
% \left( 4 \cdot \vvv{2}{-1}{3} \right)
% \\
% &=
% 4 \cdot \left(
% \begin{bmatrix}
% 5 & 0 & -2 \\
% 3 & 1 &  4
% \end{bmatrix}
% \vvv{2}{-1}{3} \right)
% \\
% &=
% 4 \cdot \vv{4}{17}
% = \vv{16}{68}
% \end{align*}
\end{ex}

Det er verdt å merke seg hva som skjer hvis vi ganger en matrise med
en vektor der nøyaktig ett av tallene er~$1$, og resten er~$0$.  La
oss teste dette med en eksempelmatrise:

\begin{ex}
Vi ganger $2 \times 3$-matrisen
\[
A =
\begin{bmatrix}
5 & 0 & -2 \\
3 & 1 &  4
\end{bmatrix}
\]
med de tre vektorene
\[
\V{e}_1 = \vvv{1}{0}{0},\quad
\V{e}_2 = \vvv{0}{1}{0}\quad\text{og}\quad
\V{e}_3 = \vvv{0}{0}{1}
\]
og får følgende:
\begin{align*}
A \V{e}_1 &= \vv{5}{3} \\
A \V{e}_2 &= \vv{0}{1} \\
A \V{e}_3 &= \vv{-2}{4}
\end{align*}
Resultatene ble altså de tre kolonnene i~$A$.
\end{ex}

Generelt har vi at hvis $A$ er en $m \times n$-matrise, og $\V{e}_1$,
$\V{e}_2$, \ldots, $\V{e}_n$ er vektorene i~$\R^n$ som er slik at
$\V{e}_i$ har et $1$-tall i sin $i$-te koordinat og bare $0$-er
ellers, så er
\[
A \V{e}_1,\quad
A \V{e}_2,\quad
\ldots,\quad
A \V{e}_n
\]
nøyaktig samme vektorer som kolonnene i~$A$.


\section*{Sum og skalering av matriser}

La $A$ og~$B$ være to $m \times n$-matriser:
\[
A =
\begin{bmatrix}
a_{11} & \cdots & a_{1n} \\
a_{21} & \cdots & a_{2n} \\
\vdots & \ddots & \vdots \\
a_{m1} & \cdots & a_{mn}
\end{bmatrix}
\qquad
B =
\begin{bmatrix}
b_{11} & \cdots & b_{1n} \\
b_{21} & \cdots & b_{2n} \\
\vdots & \ddots & \vdots \\
b_{m1} & \cdots & b_{mn}
\end{bmatrix}
\]
% \[
% A =
% \begin{bmatrix}
% a_{11} & a_{12} & \cdots & a_{1n} \\
% a_{21} & a_{22} & \cdots & a_{2n} \\
% \vdots & \vdots & \ddots & \vdots \\
% a_{m1} & a_{m2} & \cdots & a_{mn}
% \end{bmatrix}
% \qquad
% B =
% \begin{bmatrix}
% b_{11} & b_{12} & \cdots & b_{1n} \\
% b_{21} & b_{22} & \cdots & b_{2n} \\
% \vdots & \vdots & \ddots & \vdots \\
% b_{m1} & b_{m2} & \cdots & b_{mn}
% \end{bmatrix}
% \]
Vi definerer summen $A+B$ på den mest åpenbare måten -- vi legger
sammen tallene fra de to matrisene i hver posisjon, og får en ny
$m \times n$-matrise:
\[
A + B =
\begin{bmatrix}
a_{11} + b_{11} & a_{12} + b_{12} & \cdots & a_{1n} + b_{1n} \\
a_{21} + b_{21} & a_{22} + b_{22} & \cdots & a_{2n} + b_{2n} \\
\vdots          & \vdots          & \ddots & \vdots          \\
a_{m1} + b_{m1} & a_{m2} + b_{m2} & \cdots & a_{mn} + b_{mn} \\
\end{bmatrix}
\]

Produktet av et tall og en matrise defineres også på den åpenbare
måten -- vi ganger med tallet på hver plass i matrisen:
\[
c A =
\begin{bmatrix}
c \cdot a_{11} & c \cdot a_{12} & \cdots & c \cdot a_{1n} \\
c \cdot a_{21} & c \cdot a_{22} & \cdots & c \cdot a_{2n} \\
\vdots         & \vdots         & \ddots & \vdots         \\
c \cdot a_{m1} & c \cdot a_{m2} & \cdots & c \cdot a_{mn}
\end{bmatrix}
\]

% \begin{ex}
% TODO: nødvendig med eksempel?
% \end{ex}

\begin{thm}
Hvis $A$ og~$B$ er $m \times n$-matriser, $\V{v}$ er en vektor
i~$\R^n$ og $c$ er et tall, så har vi følgende likheter:
\[
(A + B) \V{v} = A \V{v} + B \V{v}
\qquad\text{og}\qquad
(c A) \V{v} = c (A \V{v})
\]
\end{thm}


\section*{Vektorer som matriser}

Hittil har vi snakket om vektorer og matriser som to forskjellige
slags ting, men vi kan også velge å se på vektorer som et
spesialtilfelle av matriser der enten høyden eller bredden er~$1$.

En kolonnevektor
\[
\vn{v}{m}
\]
kan vi tenke på som en $m \times 1$-matrise, og en radvektor
\[
\begin{bmatrix} w_1 & w_2 & \cdots & w_n \end{bmatrix}
\]
kan vi tenke på som en $1 \times n$-matrise.

\begin{ex}
La $\V{v}$ og~$\V{w}$ være henholdsvis en kolonnevektor og en
radvektor:
\[
\V{v} = \vvv{5}{-2}{4}
\qquad
\V{w} = \begin{bmatrix} 2 & 3 & 1 \end{bmatrix}
\]
Vektoren~$\V{v}$ kan vi også se på som en $3 \times 1$-matrise, og
vektoren~$\V{w}$ kan vi se på som en $1 \times 3$-matrise.

Hvis vi velger å tenke på $\V{w}$ som en matrise og $\V{v}$ som en
vektor, så kan vi gange dem sammen med den vanlige regelen for produkt
av matrise og vektor:
\begin{align*}
\V{w} \cdot \V{v}
&= \begin{bmatrix} 2 & 3 & 1 \end{bmatrix} \cdot \vvv{5}{-2}{4} \\
&= \begin{bmatrix} 2 \cdot 5 + 3 \cdot (-2) + 1 \cdot 4 \end{bmatrix}
 = \begin{bmatrix} 8 \end{bmatrix}
\end{align*}
Resultatet blir vektoren~$\begin{bmatrix} 8 \end{bmatrix}$ i~$\R^1$.
En vektor i~$\R^1$ består av kun ett tall, og vi vil vanligvis si at
en slik vektor er det samme som det ene tallet.  Med andre ord kan vi
sløyfe klammene og ganske enkelt skrive:
\[
\V{w} \cdot \V{v} = 8.\qedhere
\]
\end{ex}

Generelt har vi at produktet av en radvektor og en kolonnevektor er
gitt ved følgende uttrykk:
\[
\begin{bmatrix} w_1 & w_2 & \cdots & w_n \end{bmatrix}
\cdot \vn{v}{n}
= w_1 v_1 + w_2 v_2 + \cdots + w_n v_n
\]
(Merk at det må være like mange tall i radvektoren som i
kolonnevektoren for at vi skal kunne gange dem sammen.)


\section*{Matrisemultiplikasjon}

Nå kommer vi til den mest spennende av de aritmetiske operasjonene vi
kan gjøre med matriser, nemlig \emph{matrisemultiplikasjon}.

Det er litt mer komplisert å beskrive hvordan vi multipliserer
matriser enn hvordan vi summerer dem.  Det er imidlertid en god grunn
til at det er slik.  Vi kunne valgt å definere multiplikasjon av
matriser på tilsvarende måte som sum, men det ville ikke blitt
spesielt nyttig.  Vi vil nemlig at matrisemultiplikasjon skal oppføre
seg pent sammen med multiplikasjon av matriser med vektorer.  Spesielt
vil vi at følgende likhet skal holde:
\[
(AB) \V{v} = A (B \V{v})
\]
Vi vil altså at vi fritt skal kunne flytte parentesene, akkurat slik
vi kan gjøre med et produkt av tre tall.

Hvordan kan vi definere produkt av matriser slik at dette fungerer?
La oss først se på et eksempel.

\begin{ex}
La $A$ og~$B$ være følgende to $2 \times 2$-matriser:
\[
A =
\begin{bmatrix}
3 & -5 \\
2 &  7
\end{bmatrix}
\qquad
B =
\begin{bmatrix}
4 & 3 \\
2 & 1
\end{bmatrix}
\]
Vi har lyst til å finne matrisen $AB$ som skal være slik at
$(AB) \V{v} = A(B(\V{v}))$ for alle vektorer $\V{v}$ i~$\R^2$.

Nå er det lurt å se på de to spesielle vektorene
\[
\vv{1}{0}
\quad\text{og}\quad
\vv{0}{1}.
\]
Vi vet fra tidligere at hvis vi ganger en $2 \times 2$-matrise med en
av disse vektorene, så får vi ut den første eller den andre kolonnen i
matrisen.

Vi regner ut:
\begin{align*}
B \vv{1}{0} &= \vv{4}{2} \\
A \left( B \vv{1}{0} \right)
&=
\begin{bmatrix}
3 & -5 \\
2 &  7
\end{bmatrix}
\vv{4}{2}
= \vv{2}{22}
\end{align*}
Det vi er ute etter er at $AB$ skal være slik at
\[
(AB)\V{v} = A(B\V{v})
\]
for alle vektorer~$\V{v}$.  Spesielt må vi da ha:
\[
(AB) \vv{1}{0}
= A \left( B \vv{1}{0} \right)
= \vv{2}{22}
\]
Men det å gange med vektoren
\[
\vv{1}{0}
\]
er det samme som å plukke ut første kolonne av matrisen, så vi har nå
funnet ut at første kolonne i matrisen~$AB$ må være:
\[
\vv{2}{22}
\]
På samme måte finner vi andre kolonne i~$AB$:
\begin{align*}
B \vv{0}{1} &= \vv{3}{1}
\\
A \left( B \vv{0}{1} \right)
&=
\begin{bmatrix}
3 & -5 \\
2 &  7
\end{bmatrix}
\vv{3}{1}
= \vv{4}{13}
\\
(AB) \vv{0}{1} &= A \left( B \vv{0}{1} \right) = \vv{4}{13}
\end{align*}
Dette betyr at andre kolonne i~$AB$ må være:
\[
\vv{4}{13}
\]
Dermed kommer vi frem til at produktet av $A$ og~$B$ er:
\[
AB =
\begin{bmatrix}
 2 &  4 \\
22 & 13
\end{bmatrix}
\qedhere
\]
% % TODO eksempel med 3x3 e.l.?
% 5 0 -2
% 3 1 4
%
% 2 1
% 1 0
% 2 4
%
% 2
% 1
% 2
%
% 1
% 0
% 4
%
% 6  -3
% 15 19
\end{ex}

La oss nå generalisere det vi gjorde i eksempelet.  Foreløpig ser vi
på generelle $2 \times 2$-matriser, og så tar vi det helt generelle
tilfellet, med matriser av vilkårlig størrelse, etterpå.

La $A$ og~$B$ være to $2 \times 2$-matriser, og la
\[
\V{e}_1 = \vv{1}{0}
\quad\text{og}\quad
\V{e}_2 = \vv{0}{1}
\]
være de to spesielle vektorene vi brukte i eksempelet over.  La
$\V{b}_1$ og~$\V{b}_2$ være kolonnene i~$B$, slik at vi har:
\[
B = \begin{bmatrix} \V{b}_1 & \V{b}_2 \end{bmatrix}
\]
På samme måte som i eksempelet får vi nå:
\begin{align*}
(AB) \V{e}_1
&= A (B \V{e}_1)
 = A \V{b}_1 \\
(AB) \V{e}_2
&= A (B \V{e}_2)
 = A \V{b}_2 \\
\end{align*}
Dette betyr at første kolonne i matrisen~$AB$ må være~$A \V{b}_1$, og
andre kolonne må være~$A \V{b}_2$.  Vi får altså:
\[
AB = \begin{bmatrix} A \V{b}_1 & A \V{b}_2 \end{bmatrix}
\]
Hvis vi lar $\V{a}_1$ og~$\V{a}_2$ være radene i~$A$, så gir dette oss
at:
\[
AB =
\begin{bmatrix}
\V{a}_1 \V{b}_1 & \V{a}_1 \V{b}_2 \\
\V{a}_2 \V{b}_1 & \V{a}_2 \V{b}_2
\end{bmatrix}
\]
For å virkelig gjøre dette detaljert, kan vi skrive opp nøyaktig
hvordan vi finner hvert tall i~$AB$ ut fra hver enkelt av tallene i
$A$ og~$B$.  Hvis
\[
A =
\begin{bmatrix}
a_{11} & a_{12} \\
a_{21} & a_{22}
\end{bmatrix}
\quad\text{og}\quad
B =
\begin{bmatrix}
b_{11} & b_{12} \\
b_{21} & b_{22}
\end{bmatrix},
\]
så får vi:
\[
AB =
\begin{bmatrix}
a_{11} b_{11} + a_{12} b_{21} & a_{11} b_{12} + a_{12} b_{22} \\
a_{21} b_{11} + a_{22} b_{21} & a_{21} b_{12} + a_{21} b_{22}
\end{bmatrix}
\]
Merk hvordan dette siste uttrykket er bygd opp.  Når vi skal finne
tallet som skal stå på en bestemt posisjon i~$AB$, går vi bortover den
tilsvarende \emph{raden} i~$A$ og samtidig nedover den tilsvarende
\emph{kolonnen} i~$B$.  Vi ganger sammen tallene vi finner i~$A$ med
de vi finner i~$B$, og legger sammen disse produktene.

Alt det vi gjorde nå fungerer helt tilsvarende når vi går til større
matriser enn $2 \times 2$.  Men for at det skal gå an å gange sammen
to matriser $A$ og~$B$, må de være «kompatible» i størrelse.  Vi
finner produktet~$AB$ ved å kombinere rader fra~$A$ med kolonner
fra~$B$, og for at dette skal gå an, må lengden av radene i~$A$ være
lik lengden av kolonnene i~$B$.  Det vil si at bredden til~$A$ må være
lik høyden til~$B$.

Basert på det vi har gjort nå lager vi en generell definisjon av
matrisemultiplikasjon.

\begin{defn}
La $A$ være en $m \times n$-matrise med rader
$\V{a}_1$, $\V{a}_2$, \ldots, $\V{a}_m$,
og la $B$ være en $m \times p$-matrise med kolonner
$\V{b}_1$, $\V{b}_2$, \ldots, $\V{b}_p$:
\[
A = \vn{\V{a}}{n}
\qquad\qquad
B = \begin{bmatrix} \V{b}_1 & \V{b}_2 & \cdots & \V{b}_p \end{bmatrix}
\]
Da er produktet av $A$ og~$B$ en $m \times p$-matrise definert ved:
\begin{align*}
AB &=
\begin{bmatrix}
\V{a}_1 \V{b}_1 & \V{a}_1 \V{b}_2 & \cdots & \V{a}_1 \V{b}_p \\
\V{a}_2 \V{b}_1 & \V{a}_2 \V{b}_2 & \cdots & \V{a}_2 \V{b}_p \\
\vdots          & \vdots          & \ddots & \vdots          \\
\V{a}_m \V{b}_1 & \V{a}_m \V{b}_2 & \cdots & \V{a}_m \V{b}_p
\end{bmatrix}
% \\
% &= \begin{bmatrix} A \V{b}_1 & A \V{b}_2 & \cdots & A \V{b}_p \end{bmatrix}
\qedhere
\end{align*}
\end{defn}

Hvis vi lar $A$ og~$B$ være som i definisjonen, kan vi også skrive
produktet slik:
\[
AB = \begin{bmatrix} A \V{b}_1 & A \V{b}_2 & \cdots & A \V{b}_p \end{bmatrix}
\]

\begin{ex}
La $A$ og~$B$ og~$C$ være følgende matriser:
\[
A =
\begin{bmatrix}
5 & 0 & -2 \\
3 & 1 &  4
\end{bmatrix}
\qquad
B =
\begin{bmatrix}
2 & 1 \\
1 & 0 \\
2 & 4
\end{bmatrix}
\qquad
C =
\begin{bmatrix}
1 & 2 \\
3 & 4
\end{bmatrix}
\]
Siden $A$ er en $2 \times 3$-matrise og $B$ er en
$3 \times 2$-matrise, er $AB$ en $2 \times 2$-matrise.  Vi regner ut
denne matrisen ved å bruke definisjonen:
\begin{align*}
AB
&=
\begin{bmatrix}
5 \cdot 2 + 0 \cdot 1 + (-2) \cdot 2 &
5 \cdot 1 + 0 \cdot 0 + (-2) \cdot 4 \\
3 \cdot 2 + 1 \cdot 1 +    4 \cdot 2 &
3 \cdot 1 + 1 \cdot 0 +    4 \cdot 4
\end{bmatrix}
\\
&=
\begin{bmatrix}
 6 & -3 \\
15 & 19
\end{bmatrix}
\end{align*}
Hvis vi ganger sammen de samme to matrisene i motsatt rekkefølge, får
vi en $3 \times 3$-matrise:
\begin{align*}
BA
&=
\begin{bmatrix}
2 \cdot 5 + 1 \cdot 3 &
2 \cdot 0 + 1 \cdot 1 &
2 \cdot (-2) + 1 \cdot 4 \\
1 \cdot 5 + 0 \cdot 3 &
1 \cdot 0 + 0 \cdot 1 &
1 \cdot (-2) + 0 \cdot 4 \\
2 \cdot 5 + 4 \cdot 3 &
2 \cdot 0 + 4 \cdot 1 &
2 \cdot (-2) + 4 \cdot 4
\end{bmatrix}
\\
&=
\begin{bmatrix}
13 &  1 &  0 \\
 5 &  0 & -2 \\
22 &  4 & 12
\end{bmatrix}
\end{align*}

Vi kunne også regnet ut for eksempel $CA$ og~$BC$, men produktene $AC$
og~$CB$ er ikke definert.
\end{ex}

Vi kan merke oss følgende ting basert på dette eksempelet:
\begin{itemize}
\item Matrisemultiplikasjon er \emph{ikke kommutativt}.  Faktorenes rekkefølge spiller altså en rolle %TODO
\end{itemize}

\begin{thm}
\begin{align*}
(AB) \V{v} &= A (B \V{v}) \\
(c A) B &= c (AB) = A (cB) \\
A(BC) &= (AB)C
\end{align*}
\end{thm}


\section*{Identitetsmatriser}


\section*{Inverser}


\section*{Transponering}


\kapittelslutt
