% -*- TeX-master: "oving10"; -*-
\oppgaver{12}

\begin{oppgave}
Finn den adjungerte matrisen til
\[
\begin{bmatrix}
 1+i & 1-i\\
 2 & -i\\
 2-i & 1+i
\end{bmatrix}
\]
\end{oppgave}


\begin{losning}
\[
\begin{bmatrix}
 1-i & 2 & 2 + i \\
 1+i & i & 1-i
\end{bmatrix}
\]
\end{losning}


\begin{oppgave}
La
\[
\V u= \vv{2}{-5} \quad \text{og} \quad \V v=\vv{1}{2}
\]
Finn vektorprojeksjonen av $\V u$ på $\V v$. Tegn $\V u$, $\V v$ og vektorprojeksjonen i samme koordinatsystem. 
\end{oppgave}


\begin{losning}
Projeksjonen av $\u$ på~$\v$ er:
\[
\u_\v
= \frac{\v^* \u}{\v^* \v} \v
= \frac{1 \cdot 2 + 2 \cdot (-5)}{1^2 + 2^2} \vv{1}{2}
%= \frac{-8}{5} \vv{1}{2}
= \vv{-8/5}{-16/5}
\]
\begin{center}
\begin{tikzpicture}[scale=.6]
\draw[->] (-5.5,0) -- (5.5,0);
\draw[->] (0,-5.3) -- (0,3.5);
\foreach \x in {-5,-4,-3,-2,-1,1,2,3,4,5}
\draw (\x,.1) -- (\x,-.1);% node[anchor=north] {$\x$};
\foreach \y in {-5,-4,-3,-2,-1,1,2,3}
\draw (.1,\y) -- (-.1,\y);% node[anchor=east] {$\y$};
\draw[->] (0,0) -- (2,-5) node[anchor=west] {$\u$};
\draw[->] (0,0) -- (1,2) node[anchor=west] {$\v$};
\draw[->] (0,0) -- (-1.6,-3.2) node[anchor=north] {$\u_\v$};
\end{tikzpicture}
\end{center}
\end{losning}


\begin{oppgave}
Hvilke matriser representerer projeksjoner?
\\
\noindent
\begin{minipage}{0.14\textwidth}
\begin{punkt}
$
\begin{bmatrix}
 0 & 0\\
 1 & 1
\end{bmatrix}
$
\end{punkt}
\end{minipage}
\begin{minipage}{0.14\textwidth}
\begin{punkt}
$
\begin{bmatrix}
 0 & 0\\
 2 & 1
\end{bmatrix}
$
\end{punkt}
\end{minipage}
\begin{minipage}{0.14\textwidth}
\begin{punkt}
$
\begin{bmatrix}
 0 & 0\\
 0 & 1
\end{bmatrix}
$
\end{punkt}
\end{minipage}
\begin{minipage}{0.14\textwidth}
\begin{punkt}
$
\begin{bmatrix}
 0 & 0\\
 1 & 0
\end{bmatrix}
$
\end{punkt}
\end{minipage}
\begin{minipage}{0.14\textwidth}
\begin{punkt}
$
\begin{bmatrix}
 0 & 1\\
 1 & 0
\end{bmatrix}
$
\end{punkt}
\end{minipage}
\begin{minipage}{0.14\textwidth}
\begin{punkt}
$
\begin{bmatrix}
 1 & 0\\
 0 & 1
\end{bmatrix}
$
\end{punkt}
\end{minipage}
\end{oppgave}


\begin{losning}
For å finne ut om en matrise $P$ representerer en projeksjon, må vi
sjekke om $P^2 = P$.
\begin{punkt}
$
\begin{bmatrix}
 0 & 0\\
 1 & 1
\end{bmatrix}
^2
=
\begin{bmatrix}
 0 & 0\\
 1 & 1
\end{bmatrix}
$
\end{punkt}
\begin{punkt}
$
\begin{bmatrix}
 0 & 0\\
 2 & 1
\end{bmatrix}
^2
=
\begin{bmatrix}
 0 & 0\\
 2 & 1
\end{bmatrix}
$
\end{punkt}
\begin{punkt}
$
\begin{bmatrix}
 0 & 0\\
 0 & 1
\end{bmatrix}
^2
=
\begin{bmatrix}
 0 & 0\\
 0 & 1
\end{bmatrix}
$
\end{punkt}
\begin{punkt}
$
\begin{bmatrix}
 0 & 0\\
 1 & 0
\end{bmatrix}
^2
=
\begin{bmatrix}
 0 & 0\\
 0 & 0
\end{bmatrix}
\ne
\begin{bmatrix}
 0 & 0\\
 1 & 0
\end{bmatrix}
$
\end{punkt}
\begin{punkt}
$
\begin{bmatrix}
 0 & 1\\
 1 & 0
\end{bmatrix}
^2
=
\begin{bmatrix}
 1 & 0\\
 0 & 1
\end{bmatrix}
\ne
\begin{bmatrix}
 0 & 1\\
 1 & 0
\end{bmatrix}
$
\end{punkt}
\begin{punkt}
$
\begin{bmatrix}
 1 & 0\\
 0 & 1
\end{bmatrix}
^2
=
\begin{bmatrix}
 1 & 0\\
 0 & 1
\end{bmatrix}
$
\end{punkt}
\medskip
\noindent
Konklusjon: Matrisene i del~\textbf{a)}, \textbf{b)}, \textbf{c)}
og~\textbf{f)} representerer projeksjoner; matrisene i del~\textbf{d)}
og~\textbf{e)} representerer ikke projeksjoner.
\end{losning}



\begin{oppgave}
Hvilke av vektorene er ortogonale med hverandre?
\begin{punkt}
$\vvv{2}{-5}{1}$ og $\vvv{1}{2}{0}$ \\[4pt]
\end{punkt}

\begin{punkt}
$\vvv{1}{0}{-1}$, $\vvv{1}{\sqrt{2}}{1}$ og $\vvv{1}{-\sqrt{2}}{1}$ \\[4pt]
\end{punkt}

\begin{punkt}
$\vvv{i}{0}{1}$, $\vvv{i}{1}{0}$ og $\vvv{1}{i}{i}$ \\[4pt]
\end{punkt}

%\begin{punkt}
%$\vvv{1}{0}{0}$, $\vvv{1+i}{2+i}{3+i}$ og $\vvv{3+i}{2+i}{1+i}$ \\[4pt]
%\end{punkt}
\end{oppgave}


\begin{losning}
\begin{punkt}
Indreproduktet av de to vektorene er:
\[
\vvv{2}{-5}{1}^*
\vvv{1}{2}{0}
= 2 \cdot 1 + (-5) \cdot 2 + 1 \cdot 0 = -8
\]
Siden indreproduktet ikke er~$0$, er vektorene ikke ortogonale.
\end{punkt}

\begin{punkt}
Vi ser på indreproduktet av hvert par av vektorer:
\begin{align*}
\vvv{1}{0}{-1}^* \vvv{1}{\sqrt{2}}{1}  &= 0 \\
\vvv{1}{0}{-1}^* \vvv{1}{-\sqrt{2}}{1} &= 0 \\
\vvv{1}{\sqrt{2}}{1}^* \vvv{1}{-\sqrt{2}}{1} &= 0
\end{align*}
Alle de tre vektorene er ortogonale til hverandre.
\end{punkt}

\begin{punkt}
Vi ser på indreproduktet av hvert par av vektorer:
\begin{align*}
\vvv{i}{0}{1}^* \vvv{i}{1}{0}
&= \begin{bmatrix} -i & 0 & 1 \end{bmatrix} \vvv{i}{1}{0}
 = 1 \\
\vvv{i}{0}{1}^* \vvv{1}{i}{i}
&= \begin{bmatrix} -i & 0 & 1 \end{bmatrix} \vvv{1}{i}{i}
 = 0 \\
\vvv{i}{1}{0}^* \vvv{1}{i}{i}
&= \begin{bmatrix} -i & 1 & 0 \end{bmatrix} \vvv{1}{i}{i}
 = 0
\end{align*}
Vektorene $(i,0,1)$ og $(i,1,0)$ er altså ikke ortogonale,
men vektoren $(1,i,i)$ er ortogonal til hver av de to andre.
\end{punkt}

% \begin{punkt}
% Vi ser på indreproduktet av hvert par av vektorer:
% \begin{align*}
% \vvv{1}{0}{0}^* \vvv{1+i}{2+i}{3+i} &= 1 + i \\
% \vvv{1}{0}{0}^* \vvv{3+i}{2+i}{1+i} &= 3 + i \\
% \vvv{1+i}{2+i}{3+i}^* \vvv{3+i}{2+i}{1+i} &= 11 + 12i
% \end{align*}
% Ingen av de tre vektorene er ortogonale til hverandre.
% \end{punkt}
\end{losning}


\begin{oppgave}
Beregn $P_{\V u}$, $I-P_{\V u}$, $P_{\V u}\V v$ og $\V v-P_{\V u}\V v$ når \\[2pt]
\begin{punkt}
$\V{u}=\vvv{2}{-5}{1}$ og $\V{v}=\vvv{1}{2}{0}$\\[4pt]
\end{punkt}

\begin{punkt}
$\V{u}=\vvv{1}{0}{-1}$ og $\V{v}=\vvv{1}{\sqrt{2}}{1}$\\[4pt]
\end{punkt}

\begin{punkt}
$\V{u}=\vvv{i}{0}{1}$ og $\V{v}=\vvv{i}{1}{0}$\\[4pt]
\end{punkt}

%\begin{punkt}
%$\V{u}=\vvv{1+i}{2+i}{3+i}$ og $\V{v}=\vvv{3+i}{2+i}{1+i}$
%\end{punkt}
\end{oppgave}


\begin{losning}
\begin{punkt}
Vi får:
\begin{align*}
P_{\u}
&= \frac{1}{\u^* \u} \cdot (\u \u^*) \\
&= \frac{1}{2^2 + (-5)^2 + 1^2}
\begin{bmatrix}
 2 \cdot 2 &  2 \cdot (-5) &  2 \cdot 1 \\
-5 \cdot 2 & -5 \cdot (-5) & -5 \cdot 1 \\
 1 \cdot 2 &  1 \cdot (-5) &  1 \cdot 1
\end{bmatrix}
\\
&=
\begin{bmatrix}
2/15 & -1/3 & 1/15 \\
-1/3 &  5/6 & -1/6 \\
1/15 & -1/6 & 1/30
\end{bmatrix}
\\
P_{\u^\perp}
&= I_3 - P_\u
=
\begin{bmatrix}
13/15 & 1/3 & -1/15 \\
 1/3  & 1/6 &  1/6  \\
-1/15 & 1/6 & 29/30
\end{bmatrix}
\\
P_\u \v
&=
\begin{bmatrix}
13/15 & 1/3 & -1/15 \\
 1/3  & 1/6 &  1/6  \\
-1/15 & 1/6 & 29/30
\end{bmatrix}
\vvv{1}{2}{0}
= \vvv{-8/15}{4/3}{-4/15}
\\
P_{\u^\perp} \v
&=
\begin{bmatrix}
13/15 & 4/3 & 14/15 \\
 4/3  & 1/6 &  7/6  \\
14/15 & 7/6 & 29/30
\end{bmatrix}
\vvv{1}{2}{0}
= \vvv{23/15}{2/3}{4/15}
\end{align*}
\end{punkt}

\noindent
I de neste delene viser vi ikke all mellomregningen, men bare
resultatene.

\begin{punkt}
\vspace{-15pt}
\begin{align*}
P_\u
&=
\begin{bmatrix}
 1/2 & 0 & -1/2 \\
 0   & 0 & 0    \\
-1/2 & 0 &  1/2
\end{bmatrix}
&
P_\u \v &= \0
\\
P_{\u^\perp}
&=
\begin{bmatrix}
1/2 & 0 & 1/2 \\
0   & 1 & 0    \\
1/2 & 0 & 1/2
\end{bmatrix}
&
P_{\u^\perp} \v &= \v
\end{align*}
\end{punkt}

\begin{punkt}
$\V{u}=\vvv{i}{0}{1}$ og $\V{v}=\vvv{i}{1}{0}$\\[4pt]
\vspace{-15pt}
\begin{align*}
P_\u
&=
\begin{bmatrix}
1/2  & 0 & i/2 \\
0    & 0 & 0   \\
-i/2 & 0 & 1/2
\end{bmatrix}
&
P_\u \v &= \vvv{i/2}{0}{1/2}
\\
P_{\u^\perp}
&=
\begin{bmatrix}
1/2  & 0 & -i/2 \\
0    & 1 & 0    \\
i/2  & 0 & 1/2
\end{bmatrix}
&
P_{\u^\perp} \v &= \vvv{i/2}{1}{-1/2}
\end{align*}
\end{punkt}
\end{losning}


\begin{oppgave}
Vis at vektorene 
\[
\vvv{1}{-1}{1}, \quad \vvv{1}{2}{1} \text{og} \quad \vvv{-1}{0}{1}
\]
utgjør en ortogonal basis for $\R^3$, og finn koordinatene til punktet
\[
\vvv{1}{1}{1}
\]
i denne basisen.
\end{oppgave}


\begin{losning}
Vi kan sjekke at vektorene
\[
\b_1 = \vvv{1}{-1}{1}, \quad \b_2 = \vvv{1}{2}{1} \quad \text{og} \quad \b_3 = \vvv{-1}{0}{1}
\]
er lineært uavhengige på vanlig måte
(kombiner dem til en matrise, gausseliminer og sjekk at det blir
pivotelementer i alle kolonner).  Det betyr at de utgjør en basis
for~$\R^3$.  Videre sjekker vi at de er en ortogonal basis ved å
sjekke at hver av dem er ortogonal til begge de andre.

Når vi vet at vi har en ortogonal basis, kan vi finne koordinatene til
en vektor med hensyn på basisen ved å projisere vektoren ortogonalt på
hver basisvektor:
\begin{align*}
P_{\b_1} \vvv{1}{1}{1} &= \frac{1}{3} \b_1 \\
P_{\b_2} \vvv{1}{1}{1} &= \frac{2}{3} \b_2 \\
P_{\b_3} \vvv{1}{1}{1} &= 0 \b_3
\end{align*}
Koeffisientene til vektoren~$(1,1,1)$ med hensyn på
basisen~$(\b_1,\b_2,\b_3)$ er altså $(1/3,2/3,0)$.
\end{losning}


% TODO: denne løsningen har kommet bort fra oppgaven sin
% \begin{losning}
% \begin{punkt}
% Hint: Prikkproduktet er lineært.
% \end{punkt}
%
% \begin{punkt}
% Bruk $T$ på standardbasisen til $\mathbb{R}^3$ for å se at
% $$A=\frac{1}{30}\begin{bmatrix}
% 4 & -10 & 2\\
% -10 & 25 & -5\\
% 2 & -5 & 1
% \end{bmatrix}.$$
% \end{punkt}
%
% \begin{punkt}
% Bildet er linjen utspent av $\V{u}$. Kjernen er alt som sendes til null når vi prikker med $\V{u}$, dvs. det ortogonale komplementet. Fra dette ser vi at $T$ ikke er surjektiv eller injektiv.
% \end{punkt}
%
% \begin{punkt}
% Hint: En vektor $\V{x}$ kan skrives som en entydig sum $\V{x}=\text{proj}_{\V{u}^{\perp}}\V{x}+\text{proj}_{\V{u}}\V{x}$, hvor siste ledd er $T(\V{x})$ fra \textbf{a)}. Dermed blir projeksjonen ned på det ortogonale komplementet
% $$\text{proj}_{\V{u}^{\perp}}\V{x}=\V{x}-T(\V{x}).$$ Nå gjør vi som i løsningen til $\textbf{b)}$: sett inn for $\V{x}=\V{e}_i$, $i=1,2,3$, og bruk dette som kolonnevektorer.
% \end{punkt}
% \end{losning}

\ifx\inkludert\undefined
\ifthenelse{\boolean{vis-oppgaver}}{\newpage}{}
\fi

\begin{oppgave}
Finn en ortonormal basis for rommet utspent av% \\[2pt]
\begin{punkt}
$\vvv{2}{-5}{1}$ og $\vvv{1}{2}{0}$ \\[4pt]
\end{punkt}

\begin{punkt}
$\vvv{1}{0}{-1}$, $\vvv{1}{\sqrt{2}}{1}$ og $\vvv{1}{-\sqrt{2}}{1}$\\[4pt]
\end{punkt}

\begin{punkt}
$\vvv{i}{0}{1}$, $\vvv{i}{1}{0}$ og $\vvv{1}{i}{i}$\\[4pt]
\end{punkt}

%\begin{punkt}
%$\vvv{1}{0}{0}$, $\vvv{1+i}{2+i}{3+i}$ og $\vvv{3+i}{2+i}{1+i}$\\[4pt]
%\end{punkt}
\end{oppgave}


\begin{losning}
\begin{punkt}
Vi bruker Gram--Schmidt-ortogonalisering, og får:
\begin{align*}
\u_1 &= \vvv{2}{-5}{1} \\
\u_2 &= \vvv{1}{2}{0} - P_{\u_1} \vvv{1}{2}{0}
%      = \vvv{1}{2}{0} - \frac{\vvv{1}{2}{0}^* \vvv{1}{2}{0}}{\vvv{1}{2}{0}^* \vvv{1}{2}{0}} \vvv{2}{-5}{1}
      = \vvv{23/15}{2/3}{4/15}
\end{align*}
Da er $(\u_1,\u_2)$ en ortogonal basis.  Vi finner en ortonormal basis
ved å dele hver basisvektor på lengden sin:
\begin{align*}
\hat\u_1
&= \frac{1}{\|\u_1\|} \u_1
 = \frac{1}{\sqrt{30}} \u_1
 = \vvv{2/\sqrt{30}}{-5/\sqrt{30}}{1/\sqrt{30}} \\
\hat\u_2
&= \frac{1}{\|\u_2\|} \u_2
 = \frac{\sqrt{15}}{\sqrt{43}} \u_2
 = \vvv{23 \sqrt{15} / 15 \sqrt{43}}{2 \sqrt{15} / 3 \sqrt{43}}{4 \sqrt{15} / 15 \sqrt{43}}
% = \vvv{23/\sqrt{645}}{2\sqrt{5}/\sqrt{129}}{4/\sqrt{645}}
\end{align*}
Da er $(\hat\u_1, \hat\u_2)$ en ortonormal basis.
\end{punkt}

\begin{punkt}
Vektorene
\[
\vvv{1}{0}{-1},\quad
\vvv{1}{\sqrt{2}}{1},\quad
\vvv{1}{-\sqrt{2}}{1}
\]
er allerede ortogonale, så det eneste som gjenstår er å normalisere
dem.  Vi deler hver vektor på lengden sin og får:
\begin{align*}
\hat\u_1 &= \vvv{1/\sqrt{2}}{0}{-1/\sqrt{2}} \\
\hat\u_2 &= \vvv{1/2}{\sqrt{2}/2}{1/2} \\
\hat\u_3 &= \vvv{1/2}{-\sqrt{2}/2}{1/2}
\end{align*}
Da er $(\hat\u_1, \hat\u_2, \hat\u_3)$ en ortonormal basis.
\end{punkt}

\begin{punkt}
Vi husker at vi så de samme vektorene i oppgave~12.4. % TODO ordentlig referanse
Da fant vi ut at vektoren $(1,i,i)$ er ortogonal til hver av de to andre.
Det betyr at vi kan starte med å sette:
\begin{align*}
\u_1 &= \vvv{1}{i}{i} \\
\u_2 &= \vvv{i}{1}{0}
\end{align*}
Da er $\u_1$ og~$\u_2$ ortogonale til hverandre, og vi må bare
gjøre $(i,0,1)$ ortogonal til begge disse.  Vi setter:
\[
\u_3
 = \vvv{i}{0}{1} - P_{\u_1} \vvv{i}{0}{1} - P_{\u_2} \vvv{i}{0}{1}
 = \vvv{i/2}{-1/2}{1}
\]
Da er $(\u_1, \u_2, \u_3)$ en ortogonal basis.  Vi normaliserer:
\begin{align*}
\hat\u_1 &= \vvv{1/\sqrt{3}}{i/\sqrt{3}}{i/\sqrt{3}} \\
\hat\u_2 &= \vvv{i/\sqrt{2}}{1/\sqrt{2}}{0} \\
\hat\u_3 &= \vvv{i/\sqrt{6}}{-1/\sqrt{6}}{\sqrt{2}/\sqrt{3}}
\end{align*}
Da er $(\hat\u_1, \hat\u_2, \hat\u_3)$ en ortonormal basis.
\end{punkt}
\end{losning}


\begin{oppgave}
Finn vektoren 
\[
\vvv{i}{2+i}{1}
\]
sin komponent i rommet utspent av vektorene 
\[
\vvv{2}{-5}{1}\quad  \text{og} \quad \vvv{1}{2}{0}.
\]
\end{oppgave}


\begin{losning}
Vi bruker den ortogonale basisen $(\u_1, \u_2)$ som vi fant i oppgave~12.7~a),
og projiserer vektoren $(i,2+i,1)$ ned på hver basisvektor.
Da får vi følgende vektor:
\begin{align*}
&\!\!\!\!\!\!\!\!\!\!\!\!
P_{\u_1} \vvv{i}{2+i}{1} +
P_{\u_2} \vvv{i}{2+i}{1}
% 1/30 * ...
%   4i - 10(2+i) + 2 = -18 - 6i -> -3/5 - i/5
%   -10i + 25(2+i) - 5 = 45 + 15i -> 3/2 + i/2
%   2i - 5(2+i) + 1 = -9 - 3i -> -3/10 - i/10
% +
% 1/645 * ...
%   529i + 230(2+i) + 92 = 552 + 759i -> 184/215 + 253i/215
%   230i + 100(2+i) + 40 = 240 + 330i -> 16/43 + 22i/43
%   92i + 40(2+i) + 16   = 96 + 132i  -> 32/215 + 44i/215
\\
&= \vvv{-3/5 - i/5}{3/2 + i/2}{-3/10 - i/10} +
   \vvv{184/215 + 253i/215}{16/43 + 22i/43}{32/215 + 44i/215}
\\
&= \vvv{11/43 + 42i/43}{161/86 + 87i/86}{-13/86 + 9/86}
\end{align*}
\end{losning}


\begin{oppgave}
Bruk minste kvadraters metode på det overbestemte systemet \\
%\noindent
%\begin{minipage}{0.26\textwidth}
%\begin{punkt}
\textbf{a)}
$
\begin{amatrix}{2}
2 & 1 & -1 \\
-3 & 1 & -2 \\
 -1 & 1  & 1
\end{amatrix}
$
%\end{punkt}
%\end{minipage}
%\begin{minipage}{0.2\textwidth}
%\begin{punkt}
\hfill\textbf{b)}
$
\begin{amatrix}{3}
0 & 1 & 1  & 1-i\\
i & i & -1 & 1+i\\
 0 & i  & 0 & i\\
  0 & i  & 1 & 1
\end{amatrix}
$
%\end{punkt}
%\end{minipage}
\end{oppgave}


\begin{losning}
\begin{punkt}
La
\[
A =
\begin{bmatrix}
 2 & 1 \\
-3 & 1 \\
-1 & 1
\end{bmatrix}
\qquad\text{og}\qquad
\b = \vvv{-1}{-2}{1}
\]
være koeffisientmatrisen og høyresiden i likningssystemet vårt.  Vi
ganger hver av disse med den adjungerte av~$A$ på venstre side, og
får:
\begin{align*}
A^* A &=
\begin{bmatrix}
2 & -3 & -1 \\
1 &  1 &  1
\end{bmatrix}
\begin{bmatrix}
 2 & 1 \\
-3 & 1 \\
-1 & 1
\end{bmatrix}
=
\begin{bmatrix}
14 & -2 \\
-2 &  3
\end{bmatrix}
\\
A^* \b &=
\begin{bmatrix}
2 & -3 & -1 \\
1 &  1 &  1
\end{bmatrix}
\vvv{-1}{-2}{1}
=
\vv{3}{-2}
\end{align*}
Vi må løse likningssystemet
\[
A^* A \x = A^* \b.
\]
Dette systemet har følgende totalmatrise:
\[
\begin{amatrix}{2}
14 & -2 &  3 \\
-2 &  3 & -2
\end{amatrix}
\]
Når vi gausseliminerer denne, får vi:
%  0  19  -11
% -2   3   -2
% -
% -2   3   -2
%  0   1  -11/19
% -
% -2   0   -5/19
%  0   1  -11/19
% -
%  1   0    5/38
%  0   1  -11/19
\[
\begin{amatrix}{2}
1 & 0 &   5/38 \\
0 & 1 & -11/19
\end{amatrix}
\]
Minste kvadraters metode gir altså løsningen
\[
\vv{5/38}{-11/19}.
\]
\end{punkt}

\begin{punkt}
La
\[
A =
\begin{bmatrix}
0 & 1 &  1 \\
i & i & -1 \\
0 & i &  0 \\
0 & i &  1
\end{bmatrix}
\qquad\text{og}\qquad
\b = \vvvv{1-i}{1+i}{i}{1}
\]
være koeffisientmatrisen og høyresiden.  Vi får:
\begin{align*}
A^* A &=
\begin{bmatrix}
 0 & -i &  0 &  0 \\
-i & -i & -i & -i \\
 1 & -1 &  0 &  1
\end{bmatrix}
\begin{bmatrix}
0 & 1 &  1 \\
i & i & -1 \\
0 & i &  0 \\
0 & i &  1
\end{bmatrix}
\\
&=
\begin{bmatrix}
 1 & 1 &  i \\
 1 & 4 & -i \\
-i & i &  3
\end{bmatrix}
\\
A^* \b &=
\begin{bmatrix}
 0 & -i &  0 &  0 \\
-i & -i & -i & -i \\
 1 & -1 &  0 &  1
\end{bmatrix}
\vvvv{1-i}{1+i}{i}{1}
\\
&=
\vvv{1-i}{1-3i}{1-2i}
\end{align*}
Det betyr at systemet
\[
A^* A \x = A^* \b
\]
har følgende totalmatrise:
\[
\begin{amatrix}{3}
 1 & 1 &  i & 1-i  \\
 1 & 4 & -i & 1-3i \\
-i & i &  3 & 1-2i
\end{amatrix}
\]
Vi gausseliminerer og får:
%  1  1   i   1-i
%  1  4  -i   1-3i
% -i  i   3   1-2i
%-
%  1   1    i   1-i
%  0   3  -2i   -2i
%  0  2i    2   2-i
%-
%  1   1    i     1-i
%  0   1  -2i/3   -2i/3
%  0  2i    2     2-i
%-
%  1   1    i     1-i
%  0   1  -2i/3   -2i/3
%  0   0    2/3   2/3-i
%-
%  1   1    i     1-i
%  0   1  -2i/3   -2i/3
%  0   0    1     -3i/2
%-
%  1   1    i     1-i
%  0   1    0    1-2i/3
%  0   0    1     -3i/2
%-
%  1   0    i      -i/3
%  0   1    0    1-2i/3
%  0   0    1     -3i/2
%-
%  1   0    0      -3/2 - i/3
%  0   1    0    1-2i/3
%  0   0    1     -3i/2
\[
\begin{amatrix}{3}
1 & 0 & 0 & -3/2 - i/3 \\
0 & 1 & 0 & 1-2i/3 \\
0 & 0 & 1 & -3i/2
\end{amatrix}
\]
Minste kvadraters metode gir altså løsningen
\[
\vvv{-3/2 - i/3}{1-2i/3}{-3i/2}.
\]
\end{punkt}
\end{losning}



% \begin{losning}

% \begin{punkt}
% Det er flere løsninger. Du kan sjekke at svaret ditt i) ligger i spanet til $\V{u}$ og~$\V{v}$, ii) er ortogonale (prikkprodukt lik null) og iii) er normalisert (har lengde lik en). Alle svar som tilfredstiller i)-iii) er korrekt.
% \end{punkt}
% \begin{punkt}
% Vektorrommet $V$ er et plan i $\mathbb{R}^3$. Vektorene $\V{u}$ og $\V{v}$ ligger i planet og peker i ulike retninger (er lineært uavhengige). Vektorene du fant i $\textbf{a)}$ vil også ligge i $V$, men ha lik lengde og 90 grader mellom hverandre.
% \end{punkt}

% \end{losning}

% \begin{losning}
% \end{losning}

%\begin{oppgave}
%Gi en forklaring på hvorfor minste kvadraters metode er en applikasjon av ortogonale projeksjoner.
%\end{oppgave}
%
%\begin{losning}
%Vi formulerer problemet vi ønsker å løse som en likning $A\V{x}=\V{b}$. Det er ikke sikkert denne likningen har en løsning. Derfor tar vi den ortogonale projeksjonen av $\V{b}$ ned på kolonnerommet til $A$, $\text{proj}_{\Col A}\V{b}$. Dette er den vektoren som er nærmest $\V{b}$ i $\Col A$, og vi finner derfor den beste approksimasjonen ved å løse likningen $A\V{x}=\text{proj}_{\Col A}\V{b}$. Merk at denne likningen har løsning!
%\end{losning}

\begin{oppgave}
Vi skal finne polynomer som passer til punktene
$$\vv{0}{1}, \vv{1}{2}, \vv{2}{3},\vv{3}{5},\vv{4}{7}.$$

\begin{punkt}
Det finnes et entydig fjerdegradspolynom  som går gjennom alle punktene. Sett opp et ligningssystem for koeffisientene til dette polynomet. 
\end{punkt}

\begin{punkt}
Det finnes ingen annengradspolynomer som reiser gjennom alle punktene. Bruk minste kvadraters metode til å finne koeffisientene til det annengradspolynomet som passer best.
\end{punkt}
\end{oppgave}


\begin{losning}
\begin{punkt}
Vi vil finne et fjerdegradspolynom
\[
p(x) = a_4 x^4 + a_3 x^3 + a_2 x^2 + a_1 x + a_0
\]
som oppfyller:
\begin{align*}
p(0) &= 1 &
p(3) &= 5 \\
p(1) &= 2 &
p(4) &= 7 \\
p(2) &= 3
\end{align*}
Det betyr at koeffisientene $a_4$, $a_3$, \ldots, $a_0$ må oppfylle
følgende likninger:
\[
\systeme[a_4a_3a_2a_1a_0]{
a_0 = 1,
a_4 + a_3 + a_2 + a_1 + a_0 = 2,
16 a_4 + 8 a_3 + 4 a_2 + 2 a_1 + a_0 = 3,
81 a_4 + 27 a_3 + 9 a_2 + 3 a_1 + a_0 = 5,
256 a_4 + 64 a_3 + 16 a_2 + 4 a_1 + a_0 = 7
}
\]
\end{punkt}

\begin{punkt}
Hvis det fantes et annengradspolynom
\[
p(x) = a_2 x^2 + a_1 x + a_0
\]
som gikk gjennom alle punktene, ville koeffisientene oppfylt følgende
likningssystem:
\[
\systeme[a_2a_1a_0]{
a_0 = 1,
a_2 + a_1 + a_0 = 2,
4 a_2 + 2 a_1 + a_0 = 3,
9 a_2 + 3 a_1 + a_0 = 5,
16 a_2 + 4 a_1 + a_0 = 7
}
\]
Vi bruker minste kvadraters metode på dette systemet.
La
\[
A =
\begin{bmatrix}
 0 & 0 & 1 \\
 1 & 1 & 1 \\
 4 & 2 & 1 \\
 9 & 3 & 1 \\
16 & 4 & 1
\end{bmatrix}
\qquad\text{og}\qquad
\b = \vvvvv{1}{2}{3}{5}{7}
\]
være koeffisientmatrisen og høyresiden.  Vi får:
\begin{align*}
A^* A &=
\begin{bmatrix}
354 & 100 & 30 \\
100 &  30 & 10 \\
 30 &  10 &  5
\end{bmatrix}
\\
A^* \b &= \vvv{171}{51}{18}
\end{align*}
Løsningen av likningssystemet $A^* A \x = A^* \b$ blir:
\[
\x = \vvv{3/14}{9/14}{36/35}
\]
Annengradspolynomet som passer best til punktene er altså:
\[
p(x) = \frac{3}{14} x^2 + \frac{9}{14} x + \frac{36}{35}
\]
\end{punkt}
\end{losning}


% \begin{losning}
% \begin{punkt}
% $
% \begin{bmatrix}
% 0 & 0 & 1\\
% 1 & 1 & 1\\
% 4 & 2 & 1\\
% 9 & 3 & 1\\
% 16 & 4 & 1
% \end{bmatrix}\vvv{a}{b}{c}=\begin{bmatrix}
% 1\\
% 2\\
% 3\\
% 5\\
% 7
% \end{bmatrix}$
% \end{punkt}

% \begin{punkt}
% $a=\frac{3}{14}$, $b=\frac{9}{14}$ og $c=\frac{36}{35}$.
% \end{punkt}

% \begin{punkt}
% Nei.
% \end{punkt}

% \end{losning}


\begin{oppgave}
Nå skal vi finne polynomer som passer til datasettet 
$$\vv{0}{1}, \vv{1}{2}, \vv{2}{9},\vv{3}{28},\vv{4}{65}.$$ 
\begin{punkt}
Finn polynomet $p(x)=ax^3+bx^2+cx+d$ med best tilpasning.
\end{punkt}

\begin{punkt}
Ser du noe artig?
\end{punkt}
\end{oppgave}


\begin{losning}
\begin{punkt}
%Finn polynomet $p(x)=ax^3+bx^2+cx+d$ med best tilpasning.
Vi bruker minste kvadraters metode på likningssystemet
\[
\systeme{
d = 1,
a + b + c + d = 2,
8 a + 4 b + 2 c + d = 9,
27 a + 9 b + 3 c + d = 28,
64 a + 16 b + 4 c + d = 65
}
\]
og får løsningen
\[
\vvvv{a}{b}{c}{d} = \vvvv{1}{0}{0}{1}
\]
Polynomet som passer best til punktene er altså:
\[
p(x) = x^3 + 1
\]
\end{punkt}

\begin{punkt}
%Ser du noe artig?
Polynomet $p$ som vi fant i del~a) passer faktisk eksakt til punktene.
\end{punkt}
\end{losning}



% \begin{losning}

% \begin{punkt}
% $
% \begin{bmatrix}
% 0 & 0 & 0 & 1\\
% 1 & 1 & 1 & 1\\
% 8 & 4 & 2 & 1\\
% 27 & 9 & 3 & 1\\
% 64 & 16 & 4 & 1
% \end{bmatrix}\begin{bmatrix}
% a\\
% b\\
% c\\
% d
% \end{bmatrix}=\begin{bmatrix}
% 0\\
% 2\\
% 9\\
% 28\\
% 65
% \end{bmatrix}$
% \end{punkt}

% \begin{punkt}
% Polynomet $x^3+1$ går gjennom punktene. Derfor er en løsning $a=1$, $b=0$, $c=0$ og~$d=1$.
% \end{punkt}

% \begin{punkt}
% Ja.
% \end{punkt}

% \end{losning}


%\begin{oppgave}
%Er minste kvadraters løsning til et system $A\V{x}=\V{b}$ entydig?
%\end{oppgave}
%
%\begin{losning}
%Nei, ikke generelt.
%
%\noindent
%Forklaring: Kolonnene til $A\tr A$ kan være lineært avhengige slik at $A\tr A \V{x}=A\tr \V{b}$ ikke har entydig løsning. Prøv gjerne å finne et eksempel!
%\end{losning}


\begin{oppgave}
Anta at $\V{u}_1$, \ldots, $\V{u}_n$ er en ortonormal basis for $\mathbb{R}^n$. Vis at den inverse matrisen til $A=\begin{bmatrix}
\V{u}_1 & \cdots & \V{u_n}
\end{bmatrix}$ er gitt ved $A^{-1}=A\tr$.
\end{oppgave}


\begin{losning}
Siden $(\u_1, \u_2, \ldots, \u_n)$ er en ortonormal basis,
har vi
\begin{align*}
\u_k\tr \u_k &= 1
\ \ \text{for hver $k$, og} \\
\u_k\tr \u_l &= 0
\ \ \text{for $k \ne l$.}
\end{align*}
Det vil si at
\begin{align*}
A\tr A
&=
\begin{bmatrix}
\u_1\tr \\ \u_2\tr \\ \vdots \\ \u_n\tr
\end{bmatrix}
\begin{bmatrix}
\u_1 & \u_2 & \cdots & \u_n
\end{bmatrix}
\\
&=
\begin{bmatrix}
\u_1\tr \u_1 & \u_1\tr \u_2 & \u_1\tr \u_3 & \cdots & \u_1\tr \u_n \\
\u_2\tr \u_1 & \u_2\tr \u_2 & \u_2\tr \u_3 & \cdots & \u_2\tr \u_n \\
\u_3\tr \u_1 & \u_3\tr \u_2 & \u_3\tr \u_3 & \cdots & \u_3\tr \u_n \\
\vdots       & \vdots       & \vdots       & \ddots & \vdots       \\
\u_n\tr \u_1 & \u_n\tr \u_2 & \u_n\tr \u_3 & \cdots & \u_n\tr \u_n
\end{bmatrix}
\\
&=
\begin{bmatrix}
1 & 0 & 0 & \cdots & 0 \\
0 & 1 & 0 & \cdots & 0 \\
0 & 0 & 1 & \cdots & 0 \\
\vdots & \vdots & \vdots & \ddots & \vdots \\
0 & 0 & 0 & \cdots & 1
\end{bmatrix}
\\
&= I_n
\end{align*}
Dette vil si at $A\tr$ er inversen til~$A$.

(Hvorfor holdt det å regne ut $A\tr A$?  Må vi ikke også sjekke at
$A A\tr$ blir~$I_n$?  Husk at vi i slutten av kapittel~4 viste at
dersom $AB = I_n$ for to $n \times n$-matriser $A$ og~$B$, så er også
$BA = I_n$.  Så når vi har sjekket at $A\tr A = I_n$, så følger det at
$A A\tr$ også må være $I_n$.)
\end{losning}


\begin{oppgave}
Vis at en samling vektorer som er parvis ortogonale, og forskjellige fra null, er lineært uavhengige.
\end{oppgave}


\begin{losning}
La $\v_1$, $\v_2$, \ldots, $\v_n$ være en samling med vektorer som er
parvis ortogonale.  Da har vi
\[
\v_k^* \v_l = 0
\]
for alle $k$ og~$l$ slik at $k \ne l$.

Vi vil vise at vektorene er lineært uavhengige, så vi ser på likningen
\[
\v_1 x_1 + \v_2 x_2 + \cdots + \v_n x_n = 0.
\]
Hvis vi ganger denne til venstre med $\v_1^*$, får vi:
\[
\v_1^* \v_1 x_1 + \v_1^* \v_2 x_2 + \cdots + \v_1^* \v_n x_n = 0,
\]
som vi kan forenkle til
\[
\|\v_1\|^2 x_1 = 0.
\]
Siden $\v_1$ ikke er nullvektoren, er lengden $\|\v_1\|$ ulik~$0$, så
dette betyr at
\[
x_1 = 0.
\]
På samme måte kan vi gange likningen med $\v_2^*$, og $\v_3^*$, og så
videre, og da får vi
\[
x_2 = 0,\quad
x_3 = 0,\quad\ldots,\quad
x_n = 0.
\]
Det vil si at eneste løsning av likningen
\[
\v_1 x_1 + \v_2 x_2 + \cdots + \v_n x_n = 0.
\]
er den trivielle løsningen, og dermed er vektorene
\[
\v_1,\ \v_2,\ \ldots,\ \v_n
\]
lineært uavhengige.
\end{losning}


\begin{oppgave}
Vis at $\V v^* \V w=\overline{\V w^* \V v}$.
\end{oppgave}


\begin{losning}
La $\v$ og~$\w$ være to vilkårlige vektorer i~$\C^n$:
\[
\v = \vn{v}{n}
\quad\text{og}\quad
\w = \vn{w}{n}
\]
Da har vi:
\begin{align*}
\v^* \w
&=
\begin{bmatrix} \overline{v_1} & \overline{v_2} & \cdots & \overline{v_n} \end{bmatrix}
\vn{w}{n}
\\
&=
\overline{v_1} w_1 + 
\overline{v_2} w_2 + 
\cdots
\overline{v_n} w_n
\end{align*}
På samme måte får vi:
\[
\w^* \v =
\overline{w_1} v_1 + 
\overline{w_2} v_2 + 
\cdots
\overline{w_n} v_n,
\]
Dermed:
\[
\overline{\w^* \v} =
w_1 \overline{v_1} + 
w_2 \overline{v_2} + 
\cdots
w_n \overline{v_n}
= \v^* \w
\]
\end{losning}


\begin{oppgave}
Vis at $P_{\V v}\V w$ og $\V w-P_{\V v}\V w$ er ortogonale.
\end{oppgave}


\begin{losning}
Vi tar først noen merknader om matrisen~$P_\v$.

Vi har
\[
P_\v
= \frac{\v \v^*}{\v^* \v}
= \frac{1}{\v^* \v} \v \v^*
\]
Siden $\v^* \v$ er et reelt tall, har vi:
\[
P_\v^*
= \frac{1}{\v^* \v} (\v \v^*)^*
= \frac{1}{\v^* \v} (\v^*)^* \v^*
= \frac{1}{\v^* \v} \v \v^*
= P_\v
\]
Matrisen $P_\v$ er altså sin egen adjungerte.  Videre har vi:
\begin{align*}
P_\v^2
&= \left(\frac{1}{\v^* \v}\right)^2 (\v \v^*) (\v \v^*) \\
&= \frac{1}{(\v^* \v)^2} \v (\v^* \v) \v^* \\
&= \frac{1}{(\v^* \v)^2} (\v^* \v) \v \v^* \\
&= \frac{1}{\v^* \v} \v \v^* \\
&= P_\v
\end{align*}
Det å opphøye matrisen $P_\v$ i andre gir altså oss igjen bare den
samme matrisen tilbake.

Nå går vi løs på å vise at vektorene $P_\v \w$ og $\w - P_\v \w$ er
ortogonale.  Ved å bruke at $P_\v^* = P_\v$ og at $P_\v^2 = P_\v$ får
vi:
\begin{align*}
(P_\v \w)^* (\w - P_\v \w)
&= \w^* P_\v^* (\w - P_\v \w) \\
&= \w^* P_\v^* \w - \w^* P_\v^* P_\v \w \\
&= \w^* P_\v \w - \w^* P_\v P_\v \w \\
&= \w^* P_\v \w - \w^* P_\v \w \\
&= 0
\end{align*}
Det vil si at $P_\v \w$ og $\w - P_\v \w$ er ortogonale.
\end{losning}


\begin{oppgave}
Vis at $P_{\V v}\V w$ og $\V w-P_{\V v}\V w$ spenner ut det samme rommet som $\V v$ og $\V w$.
\end{oppgave}


\begin{losning}
Vi vet at $P_\v \w$ er parallell med $\v$ (det er jo hele meningen med
å projisere).  Mer presist: Vi vet at vi har
\[
P_\v \w = \frac{\v^* \w}{\v^* \v} \v
\]
der $(\v^* \w)/(\v^* \v)$ bare er et tall.  Hvis vi kaller dette
tallet~$a$, har vi
\[
P_\v \w = a \v
\qquad\text{og}\qquad
\v = \frac{1}{a} P_\v \w.
\]
Vi ser at vi kan skrive $\v$ og~$\w$ som lineærkombinasjoner av
$P_\v \w$ og $\w - P_\v \w$:
\begin{align*}
\v &= \frac{1}{a} P_\v \w \\
\w &= P_\v \w + (\w - P_\v \w)
\end{align*}
Vi kan også skrive $P_\v \w$ og $\w - P_\v \w$ som lineærkombinasjoner
av $\v$ og~$\w$:
\begin{align*}
P_\v \w &= a \v \\
\w - P_\v \w &= \w - a \v
\end{align*}
Dette betyr at rommet utspent av $P_\v \w$ og $\w - P_\v \w$ er det
samme som rommet utspent av $\v$ og~$\w$.
\end{losning}


% \begin{losning}
% \begin{punkt}
% Dette er oppgave \textbf{10. b)} kapittel 4 (Matriser).
% \end{punkt}
% \begin{punkt}
% Anta at $\V{v}_1,\dots,\V{v}_n$ er parvis ortogonale; $\V{v}_i\boldsymbol{\cdot}\V{v}_j=0$ for $i\neq j$. Vi må vise at likningen $$x_1\V{v}_1+\dots+a_n\V{v}_n=\V{0}$$ kun har triviell løsning (definisjonen på lineær uavhengighet). Prikk likningen med $\V{v}_1$ for å se at $x_1\|\V{v}_1\|^2=0$. Ettersom $\V{v}_1\neq \V{0}$ må $x_1=0$. Repeter denne prosessen for $i=2,\dots,n$ for å se at de resterende $x_i$-ene også må være lik null.
% \end{punkt}
% \end{losning}

%\begin{oppgave} I denne oppgaven skal du utlede din egen metode.
%\begin{punkt}
%La $\V{v}$ være en vektor i $\mathbb{R}^n$. Forklar hvordan du kan finne en ortogonal basis $\V{v},\V{v}_2,\dots \V{v}_n$.
%\end{punkt}
%\begin{punkt}
%Finn en ortogonal basis for $\mathbb{R}^3$ som inneholder $\vvv{1}{2}{3}$.
%\end{punkt}
%\end{oppgave}
%
%\begin{losning}
%\begin{punkt}
%La $V_1$ være det ortogonale komplementet til $\V{v}$. Ta en vilkårlig vektor $\V{v}_2$ i $V_1$. La $V_2$ være det ortogonale komplementet til $\{\V{v},\V{v}_2\}$. Ta en vilkårlig vektor $\V{v}_3$ i $V_2$. La $V_3$ være det ortogonale komplementet til $\{\V{v},\V{v}_2,\V{v}_3\}$. Ta en vilkårlig vektor $\V{v}_4$ i $V_3$. Fortsett denne prosessen til vektor $\V{v}_n$. Nå har du en kandidat $\B=(\V{v},\V{v}_2,\dots,\V{v}_n)$ til en basis. Vektorene er lineært uavhengige fordi de er ortogonale på grunn av måten de er valgt (ref tidligere oppgave). Derfor utgjør de en basis for $\mathbb{R}^n$ ($n$ lineært uavhengige vektorer spenner ut $\mathbb{R}^n$).
%\end{punkt}
%\begin{punkt}
%Det ortogonale komplementet til $\vvv{1}{2}{3}$ er alle vektorer $\vvv{x}{y}{z}$ slik at $$\vvv{1}{2}{3}\boldsymbol{\cdot }\vvv{x}{y}{z}=0,$$ eller ekvivalent
%$$x+2y+3z=0.$$ Det finnes mange løsninger\ldots Du kan f. eks ta $\V{v}_2=\vvv{1}{2}{-3}$. Nå må vi finne en vektor i det ortogonale komplementet til $\V{v}$ og~$\V{v}_2$. Dette kan -- på lik måte som ovenfor -- beskrives av ligningene
%$$x+2y+3z=0$$
%$$x+2y-3z=0,$$ eller ekvivalent matriseligningen $$\begin{bmatrix}
%1 & 2 & 3 \\
%1 & 2 & -3
%\end{bmatrix}\vvv{x}{y}{z}=\V{0}.$$ Det finnes mange løsninger\ldots Du kan f. eks ta $\V{v}_3=\vvv{2}{-1}{0}$. Dermed utgjør -- for eksempel -- vektorene 
%$$\vvv{1}{2}{3},\quad \vvv{1}{2}{-3},\quad \vvv{2}{-1}{0}$$ en ortogonal basis for $\mathbb{R}^3$ som inneholder $\vvv{1}{2}{3}$.
%\end{punkt}
%
%\end{losning}


%\begin{oppgave}
%Et vektorrom med et 'prikkprodukt', også kalt indreprodukt, kalles et indreproduktrom. Vanlig notasjon er $$\langle \V{v},\V{u} \rangle,$$ i stedet for $$\V{v}\boldsymbol{\cdot}\V{u}.$$ Dette produktet skal tilfredstille de viktige egenskapene til prikkproduktet:
%
%\noindent
%Symmetri: $\langle \V{v}, \V{u}\rangle =\langle \V{u}, \V{v}\rangle$.
%
%\noindent
%Linearitet: $\langle a\V{v}_1+b\V{v_2}, \V{u}\rangle =a\langle \V{v}_1, \V{u}\rangle+b\langle \V{v_2}, \V{u}\rangle$.
%
%\noindent
%Positiv-definitt: $\langle \V{v}, \V{v}\rangle \geq 0$, hvor vi har likhet hvis og bare hvis $\V{v}=0$.
%
%\begin{punkt}
%Hva burde definisjonen av det ortogonale komplementet til en vektor $\V{u}$ i et indreproduktrom være?
%\end{punkt}
%
%\begin{punkt}
%Hva burde definisjonen av projeksjonen på en vektor $\V{u}$ i et indreproduktrom være?
%\end{punkt}

%\begin{punkt}
%Vis at en vilkårlig vektor $\V{v}$ kan skrives entydig som en sum $\V{v}_{\V{u}}+\V{v}_{\V{u}^{\perp}}$ hvor første faktor er projeksjonen ned på $\V{u}$ og andre faktor er projeksjonen ned på det ortogonale komplementet til $\V{u}$.
%\end{punkt}
%
%\end{oppgave}
%\begin{losning}
%\begin{punkt}
%Alle vektorer $\V{v}$ slik at $\langle \V{v}, \V{u}\rangle=0$.
%\end{punkt}
%
%\begin{punkt}
%Projeksjonen av en vektor $\V{v}$ på $\V{u}$ er gitt ved $$\frac{\langle \V{v}, \V{u}\rangle}{\langle \V{u}, \V{u}\rangle} \V{u}.$$
%\end{punkt}
%\end{losning}
%
