% -*- TeX-master: "oving10"; -*-
\oppgaver{12}


\begin{oppgave}
Vis at det ortogonale komplementet til en vektor $\V{u}$ i $\R^n$
er kjernen til en lineærtransformasjon.

\noindent
Hint: Hvilken ligning tilfredstiller en vektor i det ortogonale komplementet?
\end{oppgave}

\begin{losning}
Vi kan skrive
\[
\V{u}=\begin{bmatrix}
u_1\\
u_2\\
\vdots\\
u_n\\
\end{bmatrix}.
\]
En vektor
\[
\V{x}=\begin{bmatrix}
x_1\\
x_2\\
\vdots\\
x_n\\
\end{bmatrix}
\]
i det ortogonale komplementet tilfredstiller
\[
u_1x_1+u_2x_2+\dots +u_nx_n=0,
\]
eller ekvivalent $\V{u}\tr \V{x}=0$.
Dette er altså kjernen til lineætransformasjonen
\[
\V{u}\tr:\mathbb{R}^n\rightarrow \mathbb{R}.
\]
\end{losning}


\begin{oppgave}
Gi en forklaring på hvorfor minste kvadraters metode er en applikasjon av ortogonale projeksjoner.
\end{oppgave}

\begin{losning}
Vi formulerer problemet vi ønsker å løse som en likning $A\V{x}=\V{b}$. Det er ikke sikkert denne likningen har en løsning. Derfor tar vi den ortogonale projeksjonen av $\V{b}$ ned på kolonnerommet til $A$, $\text{proj}_{\Col A}\V{b}$. Dette er den vektoren som er nærmest $\V{b}$ i $\Col A$, og vi finner derfor den beste approksimasjonen ved å løse likningen $A\V{x}=\text{proj}_{\Col A}\V{b}$. Merk at denne likningen har løsning!
\end{losning}

\begin{oppgave}
\begin{punkt}
Skriv ned matriselikningen som ville vært tilfredstilt dersom polynomet $p(x)=ax^2+bx+c$ gikk gjennom punktene
$$\vv{0}{1}, \vv{1}{2}, \vv{2}{3},\vv{3}{5},\vv{4}{7}.$$
\end{punkt}
\begin{punkt}
Finn koeffisienter $a$, $b$ og~$c$ som gir best tilpasning.
\end{punkt}

\begin{punkt}
Går polynomet du fant i $\textbf{b)}$ gjennom punktene i $\textbf{a)}$?
\end{punkt}

\end{oppgave}

\begin{losning}
\begin{punkt}
$
\begin{bmatrix}
0 & 0 & 1\\
1 & 1 & 1\\
4 & 2 & 1\\
9 & 3 & 1\\
16 & 4 & 1
\end{bmatrix}\vvv{a}{b}{c}=\begin{bmatrix}
1\\
2\\
3\\
5\\
7
\end{bmatrix}$
\end{punkt}

\begin{punkt}
$a=\frac{3}{14}$, $b=\frac{9}{14}$ og $c=\frac{36}{35}$.
\end{punkt}

\begin{punkt}
Nei.
\end{punkt}

\end{losning}


\begin{oppgave}

\begin{punkt}
Skriv ned matriselikningen som ville vært tilfredstilt dersom polynomet $p(x)=ax^3+bx^2+cx+d$ gikk gjennom punktene
$$\vv{0}{1}, \vv{1}{2}, \vv{2}{9},\vv{3}{28},\vv{4}{65}.$$ 
\end{punkt}


\begin{punkt}
Finn koeffisienter $a$, $b$, $c$ og~$d$ som gir best tilpasning.
\end{punkt}

\begin{punkt}
Går polynomet du fant i $\textbf{b)}$ gjennom punktene i $\textbf{a)}$?
\end{punkt}

\end{oppgave}

\begin{losning}

\begin{punkt}
$
\begin{bmatrix}
0 & 0 & 0 & 1\\
1 & 1 & 1 & 1\\
8 & 4 & 2 & 1\\
27 & 9 & 3 & 1\\
64 & 16 & 4 & 1
\end{bmatrix}\begin{bmatrix}
a\\
b\\
c\\
d
\end{bmatrix}=begin{bmatrix}
0\\
2\\
9\\
28\\
65
\end{bmatrix}$
\end{punkt}

\begin{punkt}
Polynomet $x^3+1$ går gjennom punktene. Derfor er en løsning $a=1$, $b=0$, $c=0$ og~$d=1$.
\end{punkt}

\begin{punkt}
Ja.
\end{punkt}

\end{losning}

\begin{oppgave}
Er minste kvadraters løsning til et system $A\V{x}=\V{b}$ entydig?
\end{oppgave}

\begin{losning}
Nei, ikke generelt.

\noindent
Forklaring: Kolonnene til $A^\tr A$ kan være lineært avhengige slik at $A^\tr A \V{x}=A\tr \V{b}$ ikke har entydig løsning. Prøv gjerne å finne et eksempel!
\end{losning}


