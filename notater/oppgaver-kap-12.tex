% -*- TeX-master: "oving10"; -*-
\oppgaver{12}

\begin{oppgave}
La $\V{u}=\vvv{2}{-5}{1}$ og la $T:\mathbb{R}^3\rightarrow \mathbb{R}^3$ være den ortogonale projeksjonen ned på underrommet utspent av $\V{u}$,
$$T(\V{x})=\frac{\V{x}\boldsymbol{\cdot} \V{u}}{\|\V{u} \|^2} \V{u}.$$

\begin{punkt}
Bruk definisjonen av en lineærtransformasjon for å vise at $T$ er en lineærtransformasjon.
\end{punkt}

\begin{punkt}
Finn matrisen $A$ slik at $T(\V{x})=A\V{x}$ for alle vektorer $\V{x}$.
\end{punkt}

\begin{punkt}
Gi geometriske argument for å avgjøre hva bildet til $T$ er, hva kjernen til $T$ er, og om $T$ er surjektiv og/eller injektiv.
\end{punkt}


\begin{punkt}
Finn en matrise som er den ortogonale projeksjonen ned på det ortogonale komplementet til $\V{u}$
\end{punkt}


\end{oppgave}

\begin{losning}

\begin{punkt}
Hint: Prikkproduktet er lineært.
\end{punkt}

\begin{punkt}
Bruk $T$ på standardbasisen til $\mathbb{R}^3$ for å se at
$$A=\frac{1}{30}\begin{bmatrix}
4 & -10 & 2\\
-10 & 25 & -5\\
2 & -5 & 1
\end{bmatrix}.$$
\end{punkt}

\begin{punkt}
Bildet er linjen utspent av $\V{u}$. Kjernen er alt som sendes til null når vi prikker med $\V{u}$, dvs. det ortogonale komplementet. Fra dette ser vi at $T$ ikke er surjektiv eller injektiv.
\end{punkt}

\begin{punkt}
Hint: En vektor $\V{x}$ kan skrives som en entydig sum $\V{x}=\text{proj}_{\V{u}^{\perp}}\V{x}+\text{proj}_{\V{u}}\V{x}$, hvor siste ledd er $T(\V{x})$ fra \textbf{a)}. Dermed blir projeksjonen ned på det ortogonale komplementet
$$\text{proj}_{\V{u}^{\perp}}\V{x}=\V{x}-T(\V{x}).$$ Nå gjør vi som i løsningen til $\textbf{b)}$: sett inn for $\V{x}=\V{e}_i$, $i=1,2,3$, og bruk dette som kolonnevektorer.
\end{punkt}

\end{losning}

\begin{oppgave}
Vis at det ortogonale komplementet til en vektor $\V{u}$ i $\R^n$
er kjernen til en lineærtransformasjon.

\noindent
Hint: Hvilken ligning tilfredstiller en vektor i det ortogonale komplementet?
\end{oppgave}

\begin{losning}
Vi kan skrive
\[
\V{u}=\begin{bmatrix}
u_1\\
u_2\\
\vdots\\
u_n\\
\end{bmatrix}.
\]
En vektor
\[
\V{x}=\begin{bmatrix}
x_1\\
x_2\\
\vdots\\
x_n\\
\end{bmatrix}
\]
i det ortogonale komplementet tilfredstiller
\[
u_1x_1+u_2x_2+\dots +u_nx_n=0,
\]
eller ekvivalent $\V{u}\tr \V{x}=0$.
Dette er altså kjernen til lineætransformasjonen
\[
\V{u}\tr:\mathbb{R}^n\rightarrow \mathbb{R}.
\]
\end{losning}
\begin{oppgave}
Anta at $\V{u}_1,\dots \V{u}_n$ er en ortonormal basis for $\mathbb{R}^n$. Vis at den inverse matrisen til $A=\begin{bmatrix}
\V{u}_1 & \dots & \V{u_n}
\end{bmatrix}$ er gitt ved $A^{-1}=A\tr$.
\end{oppgave}

\begin{losning}
Dette er oppgave \textbf{10. b)} kapittel 4 (Matriser).
\end{losning}

\begin{oppgave}
Gi en forklaring på hvorfor minste kvadraters metode er en applikasjon av ortogonale projeksjoner.
\end{oppgave}

\begin{losning}
Vi formulerer problemet vi ønsker å løse som en likning $A\V{x}=\V{b}$. Det er ikke sikkert denne likningen har en løsning. Derfor tar vi den ortogonale projeksjonen av $\V{b}$ ned på kolonnerommet til $A$, $\text{proj}_{\Col A}\V{b}$. Dette er den vektoren som er nærmest $\V{b}$ i $\Col A$, og vi finner derfor den beste approksimasjonen ved å løse likningen $A\V{x}=\text{proj}_{\Col A}\V{b}$. Merk at denne likningen har løsning!
\end{losning}

\begin{oppgave}
\begin{punkt}
Skriv ned matriselikningen som ville vært tilfredstilt dersom polynomet $p(x)=ax^2+bx+c$ gikk gjennom punktene
$$\vv{0}{1}, \vv{1}{2}, \vv{2}{3},\vv{3}{5},\vv{4}{7}.$$
\end{punkt}
\begin{punkt}
Finn koeffisienter $a$, $b$ og~$c$ som gir best tilpasning.
\end{punkt}

\begin{punkt}
Går polynomet du fant i $\textbf{b)}$ gjennom punktene i $\textbf{a)}$?
\end{punkt}

\end{oppgave}

\begin{losning}
\begin{punkt}
$
\begin{bmatrix}
0 & 0 & 1\\
1 & 1 & 1\\
4 & 2 & 1\\
9 & 3 & 1\\
16 & 4 & 1
\end{bmatrix}\vvv{a}{b}{c}=\begin{bmatrix}
1\\
2\\
3\\
5\\
7
\end{bmatrix}$
\end{punkt}

\begin{punkt}
$a=\frac{3}{14}$, $b=\frac{9}{14}$ og $c=\frac{36}{35}$.
\end{punkt}

\begin{punkt}
Nei.
\end{punkt}

\end{losning}


\begin{oppgave}

\begin{punkt}
Skriv ned matriselikningen som ville vært tilfredstilt dersom polynomet $p(x)=ax^3+bx^2+cx+d$ gikk gjennom punktene
$$\vv{0}{1}, \vv{1}{2}, \vv{2}{9},\vv{3}{28},\vv{4}{65}.$$ 
\end{punkt}


\begin{punkt}
Finn koeffisienter $a$, $b$, $c$ og~$d$ som gir best tilpasning.
\end{punkt}

\begin{punkt}
Går polynomet du fant i $\textbf{b)}$ gjennom punktene i $\textbf{a)}$?
\end{punkt}

\end{oppgave}

\begin{losning}

\begin{punkt}
$
\begin{bmatrix}
0 & 0 & 0 & 1\\
1 & 1 & 1 & 1\\
8 & 4 & 2 & 1\\
27 & 9 & 3 & 1\\
64 & 16 & 4 & 1
\end{bmatrix}\begin{bmatrix}
a\\
b\\
c\\
d
\end{bmatrix}=\begin{bmatrix}
0\\
2\\
9\\
28\\
65
\end{bmatrix}$
\end{punkt}

\begin{punkt}
Polynomet $x^3+1$ går gjennom punktene. Derfor er en løsning $a=1$, $b=0$, $c=0$ og~$d=1$.
\end{punkt}

\begin{punkt}
Ja.
\end{punkt}

\end{losning}

\begin{oppgave}
Er minste kvadraters løsning til et system $A\V{x}=\V{b}$ entydig?
\end{oppgave}

\begin{losning}
Nei, ikke generelt.

\noindent
Forklaring: Kolonnene til $A\tr A$ kan være lineært avhengige slik at $A\tr A \V{x}=A\tr \V{b}$ ikke har entydig løsning. Prøv gjerne å finne et eksempel!
\end{losning}


