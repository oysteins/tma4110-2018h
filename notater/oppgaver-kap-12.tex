% -*- TeX-master: "oving10"; -*-
\oppgaver{12}

\begin{oppgave}
Finn den adjungerte matrisen til
\[
\begin{bmatrix}
 1+i & 1-i\\
 2 & -i\\
 2-i & 1+i
\end{bmatrix}
\]

\end{oppgave}

\begin{oppgave}
La
\[
\V u= \vv{2}{-5} \quad \text{og} \quad \V v=\vv{1}{2}
\]
Finn vektorprojeksjonen av $\V u$ på $\V v$. Tegn $\V u$, $\V v$ og vektorprojeksjonen i samme koordinatsystem. 
\end{oppgave}

\begin{oppgave}
Hvilke matriser representerer projeksjoner?
\noindent
\begin{minipage}{0.14\textwidth}
\begin{punkt}
$
\begin{bmatrix}
 0 & 0\\
 1 & 1
\end{bmatrix}
$
\end{punkt}
\end{minipage}
\begin{minipage}{0.14\textwidth}
\begin{punkt}
$
\begin{bmatrix}
 0 & 0\\
 2 & 1
\end{bmatrix}
$
\end{punkt}
\end{minipage}
\begin{minipage}{0.14\textwidth}
\begin{punkt}
$
\begin{bmatrix}
 0 & 0\\
 0 & 1
\end{bmatrix}
$
\end{punkt}
\end{minipage}
\begin{minipage}{0.14\textwidth}
\begin{punkt}
$
\begin{bmatrix}
 0 & 0\\
 1 & 0
\end{bmatrix}
$
\end{punkt}
\end{minipage}
\begin{minipage}{0.14\textwidth}
\begin{punkt}
$
\begin{bmatrix}
 0 & 1\\
 1 & 0
\end{bmatrix}
$
\end{punkt}
\end{minipage}
\begin{minipage}{0.14\textwidth}
\begin{punkt}
$
\begin{bmatrix}
 1 & 0\\
 0 & 1
\end{bmatrix}
$
\end{punkt}
\end{minipage}
\end{oppgave}



\begin{oppgave}
Hvilke av vektorene er ortogonale med hverandre?
\begin{punkt}
$\vvv{2}{-5}{1}$ og $\vvv{1}{2}{0}$ \\[4pt]
\end{punkt}

\begin{punkt}
$\vvv{1}{0}{-1}$, $\vvv{1}{\sqrt{2}}{1}$ og $\vvv{1}{-\sqrt{2}}{1}$ \\[4pt]
\end{punkt}

\begin{punkt}
$\vvv{i}{0}{1}$, $\vvv{i}{1}{0}$ og $\vvv{1}{i}{i}$ \\[4pt]
\end{punkt}

\begin{punkt}
$\vvv{1}{0}{0}$, $\vvv{1+i}{2+i}{3+i}$ og $\vvv{3+i}{2+i}{1+i}$ \\[4pt]
\end{punkt}
\end{oppgave}

\begin{oppgave}
Beregn $P_{\V u}$, $P_{\V u^{\perp}}$, $P_{\V u}\V v$ og $P_{\V u^{\perp}}\V v$ når \\[2pt]
\begin{punkt}
$\V{u}=\vvv{2}{-5}{1}$ og $\V{v}=\vvv{1}{2}{0}$\\[4pt]
\end{punkt}

\begin{punkt}
$\V{u}=\vvv{1}{0}{-1}$ og $\V{v}=\vvv{1}{\sqrt{2}}{1}$\\[4pt]
\end{punkt}

\begin{punkt}
$\V{u}=\vvv{i}{0}{1}$ og $\V{v}=\vvv{i}{1}{0}$\\[4pt]
\end{punkt}

\begin{punkt}
$\V{u}=\vvv{1+i}{2+i}{3+i}$ og $\V{v}=\vvv{3+i}{2+i}{1+i}$
\end{punkt}
\end{oppgave}




\begin{losning}
\begin{punkt}
Hint: Prikkproduktet er lineært.
\end{punkt}

\begin{punkt}
Bruk $T$ på standardbasisen til $\mathbb{R}^3$ for å se at
$$A=\frac{1}{30}\begin{bmatrix}
4 & -10 & 2\\
-10 & 25 & -5\\
2 & -5 & 1
\end{bmatrix}.$$
\end{punkt}

\begin{punkt}
Bildet er linjen utspent av $\V{u}$. Kjernen er alt som sendes til null når vi prikker med $\V{u}$, dvs. det ortogonale komplementet. Fra dette ser vi at $T$ ikke er surjektiv eller injektiv.
\end{punkt}

\begin{punkt}
Hint: En vektor $\V{x}$ kan skrives som en entydig sum $\V{x}=\text{proj}_{\V{u}^{\perp}}\V{x}+\text{proj}_{\V{u}}\V{x}$, hvor siste ledd er $T(\V{x})$ fra \textbf{a)}. Dermed blir projeksjonen ned på det ortogonale komplementet
$$\text{proj}_{\V{u}^{\perp}}\V{x}=\V{x}-T(\V{x}).$$ Nå gjør vi som i løsningen til $\textbf{b)}$: sett inn for $\V{x}=\V{e}_i$, $i=1,2,3$, og bruk dette som kolonnevektorer.
\end{punkt}
\end{losning}


\begin{oppgave}
Finn en ortonormal basis for rommet utspent av \\[2pt]

\begin{punkt}
$\vvv{2}{-5}{1}$ og $\vvv{1}{2}{0}$ \\[4pt]
\end{punkt}

\begin{punkt}
$\vvv{1}{0}{-1}$, $\vvv{1}{\sqrt{2}}{1}$ og $\vvv{1}{-\sqrt{2}}{1}$\\[4pt]
\end{punkt}

\begin{punkt}
$\vvv{i}{0}{1}$, $\vvv{i}{1}{0}$ og $\vvv{1}{i}{i}$\\[4pt]
\end{punkt}

\begin{punkt}
$\vvv{1}{0}{0}$, $\vvv{1+i}{2+i}{3+i}$ og $\vvv{3+i}{2+i}{1+i}$\\[4pt]
\end{punkt}
\end{oppgave}

\begin{oppgave}
Finn vektoren 
\[
\vvv{i}{2+i}{1}
\]
sin komponent i rommet utspent av vektorene 
\[
\vvv{2}{-5}{1}\quad  \text{og} \quad \vvv{1}{2}{0}.
\]
\end{oppgave}

\begin{oppgave}
Bruk minste kvadraters metode på det overbestemte systemet \\
%\noindent
%\begin{minipage}{0.26\textwidth}
%\begin{punkt}
\textbf{a)}
$
\begin{amatrix}{2}
2 & 1 & -1 \\
-3 & 1 & -2 \\
 -1 & 1  & 1
\end{amatrix}
$
%\end{punkt}
%\end{minipage}
%\begin{minipage}{0.2\textwidth}
%\begin{punkt}
\hfill\textbf{b)}
$
\begin{amatrix}{3}
0 & 1 & 1  & 1-i\\
i & i & -1 & 1+i\\
 0 & i  & 0 & i\\
  0 & i  & 1 & 1
\end{amatrix}
$
%\end{punkt}
%\end{minipage}
\end{oppgave}

\begin{losning}

\begin{punkt}
Det er flere løsninger. Du kan sjekke at svaret ditt i) ligger i spanet til $\V{u}$ og~$\V{v}$, ii) er ortogonale (prikkprodukt lik null) og iii) er normalisert (har lengde lik en). Alle svar som tilfredstiller i)-iii) er korrekt.
\end{punkt}
\begin{punkt}
Vektorrommet $V$ er et plan i $\mathbb{R}^3$. Vektorene $\V{u}$ og $\V{v}$ ligger i planet og peker i ulike retninger (er lineært uavhengige). Vektorene du fant i $\textbf{a)}$ vil også ligge i $V$, men ha lik lengde og 90 grader mellom hverandre.
\end{punkt}

\end{losning}

\begin{losning}
\end{losning}

%\begin{oppgave}
%Gi en forklaring på hvorfor minste kvadraters metode er en applikasjon av ortogonale projeksjoner.
%\end{oppgave}
%
%\begin{losning}
%Vi formulerer problemet vi ønsker å løse som en likning $A\V{x}=\V{b}$. Det er ikke sikkert denne likningen har en løsning. Derfor tar vi den ortogonale projeksjonen av $\V{b}$ ned på kolonnerommet til $A$, $\text{proj}_{\Col A}\V{b}$. Dette er den vektoren som er nærmest $\V{b}$ i $\Col A$, og vi finner derfor den beste approksimasjonen ved å løse likningen $A\V{x}=\text{proj}_{\Col A}\V{b}$. Merk at denne likningen har løsning!
%\end{losning}

\begin{oppgave}
Vi skal finne polynomer som passer til punktene
$$\vv{0}{1}, \vv{1}{2}, \vv{2}{3},\vv{3}{5},\vv{4}{7}.$$

\begin{punkt}
Det finnes et entydig fjerdegradspolynom  som går gjennom alle punktene. Sett opp et ligningssystem for koeffisientene til dette polynomet. 
\end{punkt}

\begin{punkt}
Det finnes ingen annengradspolynomer som reiser gjennom alle punktene. Bruk minste kvadraters metode til å finne koeffisientene til det annengradspolynomet som passer best.
\end{punkt}


\end{oppgave}

\begin{losning}
\begin{punkt}
$
\begin{bmatrix}
0 & 0 & 1\\
1 & 1 & 1\\
4 & 2 & 1\\
9 & 3 & 1\\
16 & 4 & 1
\end{bmatrix}\vvv{a}{b}{c}=\begin{bmatrix}
1\\
2\\
3\\
5\\
7
\end{bmatrix}$
\end{punkt}

\begin{punkt}
$a=\frac{3}{14}$, $b=\frac{9}{14}$ og $c=\frac{36}{35}$.
\end{punkt}

\begin{punkt}
Nei.
\end{punkt}

\end{losning}


\begin{oppgave}
Nå skal vi finne polynomer som passer til datasettet 
$$\vv{0}{1}, \vv{1}{2}, \vv{2}{9},\vv{3}{28},\vv{4}{65}.$$ 
\begin{punkt}
Finn polynomet $p(x)=ax^3+bx^2+cx+d$ med best tilpasning.
\end{punkt}

\begin{punkt}
Ser du noe artig?
\end{punkt}

\end{oppgave}

\begin{losning}

\begin{punkt}
$
\begin{bmatrix}
0 & 0 & 0 & 1\\
1 & 1 & 1 & 1\\
8 & 4 & 2 & 1\\
27 & 9 & 3 & 1\\
64 & 16 & 4 & 1
\end{bmatrix}\begin{bmatrix}
a\\
b\\
c\\
d
\end{bmatrix}=\begin{bmatrix}
0\\
2\\
9\\
28\\
65
\end{bmatrix}$
\end{punkt}

\begin{punkt}
Polynomet $x^3+1$ går gjennom punktene. Derfor er en løsning $a=1$, $b=0$, $c=0$ og~$d=1$.
\end{punkt}

\begin{punkt}
Ja.
\end{punkt}

\end{losning}

%\begin{oppgave}
%Er minste kvadraters løsning til et system $A\V{x}=\V{b}$ entydig?
%\end{oppgave}
%
%\begin{losning}
%Nei, ikke generelt.
%
%\noindent
%Forklaring: Kolonnene til $A\tr A$ kan være lineært avhengige slik at $A\tr A \V{x}=A\tr \V{b}$ ikke har entydig løsning. Prøv gjerne å finne et eksempel!
%\end{losning}


\begin{oppgave}
Anta at $\V{u}_1,\dots \V{u}_n$ er en ortonormal basis for $\mathbb{R}^n$. Vis at den inverse matrisen til $A=\begin{bmatrix}
\V{u}_1 & \dots & \V{u_n}
\end{bmatrix}$ er gitt ved $A^{-1}=A\tr$.
\end{oppgave}

\begin{oppgave}
Vis at en samling vektorer som er parvis ortogonale, og forskjellige fra null, er lineært uavhengige.
\end{oppgave}


\begin{oppgave}
Vis at $\V v^* \V w=\overline{\V w^* \V v}$.
\end{oppgave}

\begin{oppgave}
Vis at $P_{\V v}\V w$ og $P_{\V v^{\perp}}\V w$ er ortogonale.
\end{oppgave}

\begin{oppgave}
Vis at $P_{\V v}\V w$ og $P_{\V v^{\perp}}\V w$ spenner ut det samme rommet som $\V v$ og $\V w$.
\end{oppgave}




\begin{losning}
\begin{punkt}
Dette er oppgave \textbf{10. b)} kapittel 4 (Matriser).
\end{punkt}
\begin{punkt}
Anta at $\V{v}_1,\dots,\V{v}_n$ er parvis ortogonale; $\V{v}_i\boldsymbol{\cdot}\V{v}_j=0$ for $i\neq j$. Vi må vise at likningen $$x_1\V{v}_1+\dots+a_n\V{v}_n=\V{0}$$ kun har triviell løsning (definisjonen på lineær uavhengighet). Prikk likningen med $\V{v}_1$ for å se at $x_1\|\V{v}_1\|^2=0$. Ettersom $\V{v}_1\neq \V{0}$ må $x_1=0$. Repeter denne prosessen for $i=2,\dots,n$ for å se at de resterende $x_i$-ene også må være lik null.
\end{punkt}
\end{losning}

%\begin{oppgave} I denne oppgaven skal du utlede din egen metode.
%\begin{punkt}
%La $\V{v}$ være en vektor i $\mathbb{R}^n$. Forklar hvordan du kan finne en ortogonal basis $\V{v},\V{v}_2,\dots \V{v}_n$.
%\end{punkt}
%\begin{punkt}
%Finn en ortogonal basis for $\mathbb{R}^3$ som inneholder $\vvv{1}{2}{3}$.
%\end{punkt}
%\end{oppgave}
%
%\begin{losning}
%\begin{punkt}
%La $V_1$ være det ortogonale komplementet til $\V{v}$. Ta en vilkårlig vektor $\V{v}_2$ i $V_1$. La $V_2$ være det ortogonale komplementet til $\{\V{v},\V{v}_2\}$. Ta en vilkårlig vektor $\V{v}_3$ i $V_2$. La $V_3$ være det ortogonale komplementet til $\{\V{v},\V{v}_2,\V{v}_3\}$. Ta en vilkårlig vektor $\V{v}_4$ i $V_3$. Fortsett denne prosessen til vektor $\V{v}_n$. Nå har du en kandidat $\B=(\V{v},\V{v}_2,\dots,\V{v}_n)$ til en basis. Vektorene er lineært uavhengige fordi de er ortogonale på grunn av måten de er valgt (ref tidligere oppgave). Derfor utgjør de en basis for $\mathbb{R}^n$ ($n$ lineært uavhengige vektorer spenner ut $\mathbb{R}^n$).
%\end{punkt}
%\begin{punkt}
%Det ortogonale komplementet til $\vvv{1}{2}{3}$ er alle vektorer $\vvv{x}{y}{z}$ slik at $$\vvv{1}{2}{3}\boldsymbol{\cdot }\vvv{x}{y}{z}=0,$$ eller ekvivalent
%$$x+2y+3z=0.$$ Det finnes mange løsninger\ldots Du kan f. eks ta $\V{v}_2=\vvv{1}{2}{-3}$. Nå må vi finne en vektor i det ortogonale komplementet til $\V{v}$ og~$\V{v}_2$. Dette kan -- på lik måte som ovenfor -- beskrives av ligningene
%$$x+2y+3z=0$$
%$$x+2y-3z=0,$$ eller ekvivalent matriseligningen $$\begin{bmatrix}
%1 & 2 & 3 \\
%1 & 2 & -3
%\end{bmatrix}\vvv{x}{y}{z}=\V{0}.$$ Det finnes mange løsninger\ldots Du kan f. eks ta $\V{v}_3=\vvv{2}{-1}{0}$. Dermed utgjør -- for eksempel -- vektorene 
%$$\vvv{1}{2}{3},\quad \vvv{1}{2}{-3},\quad \vvv{2}{-1}{0}$$ en ortogonal basis for $\mathbb{R}^3$ som inneholder $\vvv{1}{2}{3}$.
%\end{punkt}
%
%\end{losning}


%\begin{oppgave}
%Et vektorrom med et 'prikkprodukt', også kalt indreprodukt, kalles et indreproduktrom. Vanlig notasjon er $$\langle \V{v},\V{u} \rangle,$$ i stedet for $$\V{v}\boldsymbol{\cdot}\V{u}.$$ Dette produktet skal tilfredstille de viktige egenskapene til prikkproduktet:
%
%\noindent
%Symmetri: $\langle \V{v}, \V{u}\rangle =\langle \V{u}, \V{v}\rangle$.
%
%\noindent
%Linearitet: $\langle a\V{v}_1+b\V{v_2}, \V{u}\rangle =a\langle \V{v}_1, \V{u}\rangle+b\langle \V{v_2}, \V{u}\rangle$.
%
%\noindent
%Positiv-definitt: $\langle \V{v}, \V{v}\rangle \geq 0$, hvor vi har likhet hvis og bare hvis $\V{v}=0$.
%
%\begin{punkt}
%Hva burde definisjonen av det ortogonale komplementet til en vektor $\V{u}$ i et indreproduktrom være?
%\end{punkt}
%
%\begin{punkt}
%Hva burde definisjonen av projeksjonen på en vektor $\V{u}$ i et indreproduktrom være?
%\end{punkt}

%\begin{punkt}
%Vis at en vilkårlig vektor $\V{v}$ kan skrives entydig som en sum $\V{v}_{\V{u}}+\V{v}_{\V{u}^{\perp}}$ hvor første faktor er projeksjonen ned på $\V{u}$ og andre faktor er projeksjonen ned på det ortogonale komplementet til $\V{u}$.
%\end{punkt}
%
%\end{oppgave}
%\begin{losning}
%\begin{punkt}
%Alle vektorer $\V{v}$ slik at $\langle \V{v}, \V{u}\rangle=0$.
%\end{punkt}
%
%\begin{punkt}
%Projeksjonen av en vektor $\V{v}$ på $\V{u}$ er gitt ved $$\frac{\langle \V{v}, \V{u}\rangle}{\langle \V{u}, \V{u}\rangle} \V{u}.$$
%\end{punkt}
%\end{losning}
%
